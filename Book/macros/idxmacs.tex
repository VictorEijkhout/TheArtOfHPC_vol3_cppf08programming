% -*- latex -*-
%%%%%%%%%%%%%%%%%%%%%%%%%%%%%%%%%%%%%%%%%%%%%%%%%%%%%%%%%%%%%%%%
%%%%
%%%% This TeX file is part of the course
%%%% Introduction to Scientific Programming in C++/Fortran2003
%%%% copyright 2017-2023 Victor Eijkhout eijkhout@tacc.utexas.edu
%%%%
%%%% idxmacs.tex : macros for indexing
%%%%
%%%%%%%%%%%%%%%%%%%%%%%%%%%%%%%%%%%%%%%%%%%%%%%%%%%%%%%%%%%%%%%%

\usepackage[original]{imakeidx}
%\indexsetup{noclearpage,toclevel=chapter}

\makeindex
\makeindex[name=def]
\makeindex[name=cpp]
\makeindex[name=f90]
\makeindex[name=jul]
\def\idxset#1{\def\currentindex{#1}}
\idxset{def}

\newcommand{\indexterm}[1]{\emph{#1}\index{#1}}
\newcommand{\indextermunix}[1]{{\ttfamily\slshape #1}\index{#1@\texttt{#1}}}
\let\indexunix\indextermunix
\newcommand{\indexgdb}[1]{{\ttfamily\slshape #1}\index{gdb!#1@\texttt{#1}}}
\newcommand{\indextermdef}[1]{\emph{#1}\index{#1|textbf}}
\newcommand{\indextermdefp}[1]{\emph{#1s}\index{#1|textbf}}
\newcommand{\indextermp}[1]{\emph{#1s}\index{#1}}
\newcommand{\indextermsub}[2]{\emph{#1 #2}\index{#2!#1}}
\newcommand{\indextermsubdef}[2]{\emph{#1 #2}\index{#2!#1|textbf}}
\newcommand{\indextermsubp}[2]{\emph{#1 #2s}\index{#2!#1}}
\newcommand{\indextermbus}[2]{\emph{#1 #2}\index{#1!#2}}
\newcommand{\indextermbusp}[2]{\emph{#1 #2s}\index{#1!#2}}
\newcommand{\indextermbusdef}[2]{\emph{#1 #2}\index{#1!#2|textbf}}
\newcommand{\indextermstart}[1]{\emph{#1}\index{#1|(}}
\newcommand{\indextermend}[1]{\index{#1|)}}
\newcommand{\indexstart}[1]{\index{#1|(}}
\newcommand{\indexend}[1]{\index{#1|)}}

\newcommand{\indexpragma}[1]{%
  \lstinline{\##1}\index{\#pragma #1@{\texttt{\##1}}}%
  \index{#1@{\texttt{#1}}|see{\texttt{\#pragma #1}}}%
}

\def\idxprefix{}
\def\indexc   {\bgroup \idxset{cpp}\indextermttgroup} % these had #{
\def\indexcdef{\bgroup \idxset{cpp}\indextermttbgroup} % these had #{
\def\indexcstd{\bgroup \idxset{cpp}\def\idxprefix{std::}\indextermttgroup}
\def\indexcrng{\bgroup \idxset{cpp}\def\idxprefix{ranges::}\indextermttgroup}
\def\indexcdefstd
              {\bgroup \idxset{cpp}\def\idxprefix{std::}\indextermttbgroup}
\def\indexf   {\bgroup \idxset{f90}\indextermttgroup}
\def\indexfnote{\bgroup \idxset{f90}\indextermttgroup}
\def\indexj   {\bgroup \idxset{jul}\indextermttgroup}
% default
\def\indexg   {\bgroup \idxset{def}\indextermttgroup}

% regular formatting
\def\indextermtt#{\bgroup \innocentcharacters 
  \afterassignment\cmdtoindex\edef\indexedcmd}
\def\indextermheader#{\bgroup \innocentcharacters 
  \afterassignment\cmdtoindex\edef\indexedcmd}
\let\indexheader\indextermheader
\def\indexheaderdef#{\bgroup \innocentcharacters 
  \afterassignment\cmdtoindexb\edef\indexedcmd}
\def\indextermttgroup#{\innocentcharacters 
  \afterassignment\cmdtoindex\edef\indexedcmd}
\def\indextermttbgroup#{\innocentcharacters 
  \afterassignment\cmdtoindexb\edef\indexedcmd}

%% a language command to the current index
%% this assumes the macro \indexedcmd contains the command
\def\cmdtoindex{%
  \ttfamily
  \edef\idxtmp{%
    \noexpand\lstinline{\idxprefix\indexedcmd}%
    \noexpand\index[\currentindex]{%
      \indexedcmd@{\catcode95=12 \noexpand\texttt{\indexedcmd}}}%
  }%
  \idxtmp
  \egroup
}
\def\cmdtoindexb{%
  \ttfamily
  \edef\idxtmp{%
    \noexpand\lstinline{\indexedcmd}%
    \noexpand\index[\currentindex]{%
      \indexedcmd@{\catcode95=12 \noexpand\ttfamily \indexedcmd}|textbf}%
  }%
  \idxtmp
  \egroup
}

\def\indexfdef#{\bgroup \idxset{f90}\indextermttgroupbf}

% textbf formatting
\def\indextermttgroupbf#{\innocentcharacters 
  \afterassignment\cmdtoindexbf\edef\indexedcmd}
\def\indextermttdef#{\bgroup \innocentcharacters
  \afterassignment\cmdtoindexbf\edef\indexedcmd}

\def\cmdtoindexbf{%
  \expandafter\lstinline\expandafter$\indexedcmd$%
  \edef\idxtmp{%
    \noexpand\index[\currentindex]{%
      \indexedcmd@{\catcode95=12 \noexpand\texttt{\indexedcmd}}|textbf}%
  }%
  \idxtmp
  \egroup
}

\newcommand{\indextermttbus}[2]{\emph{\texttt{#1} #2}%
  \index{#1@\texttt{#1}!#2}}

%%
%% Fortran special case
%%
\let\indextermfort\indextermtt
\let\indextermfortdef\indextermttdef

\newcommand\indexac[1]{\emph{\ac{#1}}%
  \edef\tmp{\noexpand\index{%
    \expandafter\expandafter\expandafter
        \@secondoftwo\csname fn@#1\endcsname%
    @\acl{#1} (#1)}}\tmp}
\newcommand\indexacp[1]{\emph{\ac{#1}}%
  \edef\tmp{\noexpand\index{%
    \expandafter\expandafter\expandafter
        \@secondoftwo\csname fn@#1\endcsname%
    @\aclp{#1} (#1)}}\tmp}
\newcommand\indexacf[1]{\emph{\acf{#1}}%
  \edef\tmp{\noexpand\index{%
    \expandafter\expandafter\expandafter
        \@secondoftwo\csname fn@#1\endcsname
    @\acl{#1} (#1)}}\tmp}
\newcommand\indexacstart[1]{%
  \edef\tmp{\noexpand\index{%
    \expandafter\expandafter\expandafter
        \@secondoftwo\csname fn@#1\endcsname
    @\acl{#1} (#1)|(}}\tmp}
\newcommand\indexacend[1]{%
  \edef\tmp{\noexpand\index{%
    \expandafter\expandafter\expandafter
        \@secondoftwo\csname fn@#1\endcsname
    @\acl{#1} (#1)|)}}\tmp}
\newcommand\indexacdef[1]{\emph{\acf{#1}}%
  \edef\tmp{\noexpand\index{%
    \expandafter\expandafter\expandafter
        \@secondoftwo\csname fn@#1\endcsname%
    @\acl{#1} (#1)|textbf}}\tmp}

