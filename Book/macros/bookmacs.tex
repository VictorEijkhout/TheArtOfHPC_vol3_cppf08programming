% -*- latex -*-
%%%%%%%%%%%%%%%%%%%%%%%%%%%%%%%%%%%%%%%%%%%%%%%%%%%%%%%%%%%%%%%%
%%%%
%%%% This TeX file is part of the course
%%%% Introduction to Scientific Programming in C++/Fortran2003
%%%% copyright 2017-2023 Victor Eijkhout eijkhout@tacc.utexas.edu
%%%%
%%%% bookmacs.tex : macros for typsetting the source of the textbook
%%%%
%%%%%%%%%%%%%%%%%%%%%%%%%%%%%%%%%%%%%%%%%%%%%%%%%%%%%%%%%%%%%%%%

\newif\ifInBook \InBooktrue
\let\textbookref\ref
% slide blurb to exclude in book
\excludecomment{tldr}
\excludecomment{322}
\excludecomment{slideonly}
% code location wrt book
\newcommand\codedir{../code}
\newcommand\slidenewline{}\let\slidebreak\slidenewline

%%
%% refer to sections in the HPC book
%%
\usepackage{xr-hyper}
% vol 1
\externaldocument[HPSC-]{scicompbook}
\newcommand\HPSCref[2][section]{HPC book~\cite{ISTCwebpage}, #1~\ref{HPSC-#2}}
% vol 2
\externaldocument[PCSE-]{parcompbook}
\newcommand\PCSEref[2][section]{MPI/OpenMP book~\cite{PCSEwebpage}, #1~\ref{PCSE-#2}}
% vol 4
\externaldocument[CARP-]{scicomptutorials}
\newcommand\CARPref[2][section]{Tutorials book~\cite{CARPwebpage}, #1-\ref{CARP-#2}}
% local references;
% this is changed for the projects book.
\let\ISPref\ref

%%%%
%%%% Chapters (why are we doing this?)
%%%%
\newwrite\nx
\newwrite\chapterlist
\openout\chapterlist=chapternames.tex
\newcommand\CHAPTER[2]{
  \Level 0 {#1}\label{ch:#2}
  \def\chapshortname{#2}

  % start scooping up example codes used in this chapter
  \addchaptersource{header}%kludge because we have a bug for zero sources

  % input the chapter
  \SetBaseLevel 1 \input chapters/#2

  % include the sources used in this chapter
  \listchaptersources\chapshortname

  % add chapter to list of chapters
  \write\chapterlist{\chapshortname}
  \openout\nx=exercises/\chapshortname-nx.tex
  \write\nx{\arabic{excounter}}
  \closeout\nx

  \SetBaseLevel 0
}

% each chapter has a list of sources
\newcommand\addchaptersource[1]{
  \ifinlistcs{#1}{\chapshortname:sourcelist}
             {}
             {\listcsadd{\chapshortname:sourcelist}{#1}}
}
\newcounter{source}
\newcommand\listchaptersources[1]{
  \setcounter{source}{0}
  \renewcommand*\do[1]{\message{counting <<##1>>}\stepcounter{source}}%
%  \dolistcsloop{#1:sourcelist}
  \expandafter\ifnum\value{source}>0
    \immediate\message{Sources: \arabic{source}}
    %% \Level 0 {Sources used in this chapter}
    %% \renewcommand*\do[1]{\ListOneSource{##1}}
    %% \dolistcsloop{#1:sourcelist}
  \fi
}
\newcommand\ListOneSource[1]{
  \immediate\message{sourcing <<#1>>}
  \Level 1 {Listing of code #1}
  \label{lst:#1}
  \begingroup \footnotesize
  \immediate \write 18 { ./stripsource #1 }
  \verbatiminput{input.cut}
  \endgroup
  \par
}
\def\LSR{\LSR}
\def\ChapterSourceHeader#1\LSR{
  \def\test{#1\LSR}
  \ifx\test\LSR
  \else
    \Level 0 {Sources used in this chapter}
  \fi
}
\def\ListSourcesRecursively#1{
  \def\test{#1}
  \ifx\test\LSR
  \else
    % list the file
      \IfFileExists{../code/#1.\ISPcodeext}
                   {\Level 1 {Listing of code/#1}\label{lst:#1}
                     {\footnotesize
                       \lstinputlisting{../code/#1.\ISPcodeext}} % verbatiminput
                   }
                   {}
    \par
    % continue
    \expandafter\ListSourcesRecursively
  \fi
}

%%%%
%%%% Page layout
%%%%
\usepackage{geometry}
\addtolength{\textwidth}{.5in}
\addtolength{\textheight}{.5in}
\addtolength{\evensidemargin}{-.5in}

\usepackage{fancyhdr}
\pagestyle{fancy}\fancyhead{}\fancyfoot{}
% remove uppercase from fancy defs
\makeatletter
\def\chaptermark#1{\markboth {{\ifnum \c@secnumdepth>\m@ne
 \thechapter. \ \fi #1}}{}}
\def\sectionmark#1{\markright{{\ifnum \c@secnumdepth >\z@
 \thesection. \ \fi #1}}}
\makeatother
% now the fancy specs
%\fancyhead[LE]{\thepage \hskip.5\unitindent/\hskip.5\unitindent \leftmark}
%\fancyhead[RO]{\rightmark \hskip.5\unitindent/\hskip.5\unitindent \thepage}
\fancyhead[LE]{\leftmark}
\fancyfoot[LE]{\thepage}
\fancyhead[RO]{\rightmark}
\fancyfoot[RO]{\thepage}
\fancyfoot[RE]{\footnotesize\sl Introduction to Scientific Programming}
\fancyfoot[LO]{\footnotesize\sl Victor Eijkhout}
\setlength{\headheight}{14pt}
\addtolength{\topmargin}{-1.6pt}

%%%%
%%%% Outlining
%%%%
\usepackage{outliner}
\OutlineLevelStart0{\chapter{#1}}
\OutlineLevelStart1{\section{#1}}
\OutlineLevelCont1{\section{#1}}
\OutlineLevelStart2{\subsection{#1}}
\OutlineLevelStart3{\subsubsection{#1}}
\setcounter{secnumdepth}{4}
\OutlineLevelStart4{\paragraph{\bf #1}}

\newcommand\heading[1]{\paragraph*{\textbf{#1}}}

%%%%
%%%% Text and slide blocks
%%%%

\newcounter{framecounter}[chapter]
\newcounter{blcounter}[chapter]
\makeatletter
\renewcommand\theframecounter{\@arabic\c@chapter.\@arabic\c@framecounter}
\renewcommand\theblcounter{\@arabic\c@chapter.\@arabic\c@blcounter}
\makeatother

\newcommand\framenumber{\arabic{chapter}.\arabic{framecounter}}
\newcommand\blocknumber{\arabic{chapter}.\arabic{framecounter}}
%% block: write out a frame, and read back in
\generalcommentarg{block}
  {\refstepcounter{framecounter}%
    \begingroup
    \def\ProcessCutFile{}\par
    \def\PrepareCutFile{\immediate\write\CommentStream
      {\noexpand\begin{frameblock}{\CommentBlockName}}}
      %  \noexpand\nobreak}}%
    \def\FinalizeCutFile{\immediate\write\CommentStream
      {\noexpand\end{frameblock}}}%
    \edef\tmp{\def\noexpand\CommentCutFile
      {frames/\chapshortname-block-\arabic{framecounter}.tex}}\tmp
  }
  {\input{\CommentCutFile}
   \endgroup
  }
%% plain block: include without fancy formatting
\generalcommentarg{plainblock}
  {\refstepcounter{framecounter}%
    \begingroup
    \def\ProcessCutFile{}\par
    \def\PrepareCutFile{\immediate\write\CommentStream
      {\noexpand\begin{plainframeblock}{\CommentBlockName}}}
      %  \noexpand\nobreak}}%
    \def\FinalizeCutFile{\immediate\write\CommentStream
      {\noexpand\end{plainframeblock}}}%
    \edef\tmp{\def\noexpand\CommentCutFile
      {frames/\chapshortname-block-\arabic{framecounter}.tex}}\tmp
  }
  {\input{\CommentCutFile}
   \endgroup
  }
%% slide: write out a frame, and don't read back in
\generalcommentarg{slide}
  {\refstepcounter{framecounter}%
    \begingroup
    \def\ProcessCutFile{}\par
    \edef\tmp{\def\noexpand\CommentCutFile
      {frames/\chapshortname-block-\arabic{framecounter}.tex}}\tmp
    \def\PrepareCutFile{\immediate\write\CommentStream
      {\noexpand\begin{frameblock}{\CommentBlockName}
        \noexpand\par\noexpand\nobreak}}%
    \def\FinalizeCutFile{\immediate\write\CommentStream
      {\noexpand\end{frameblock}}}%
  }
  {\endgroup}

%%
%% in book, frame are rendered without title
%%
\newenvironment
    {frameblock}[2][keyword]
    {\par\hbox{}\smallskip
      \begin{mdframed}[style=blockstyle]
        \leavevmode \raggedright
    }
    {\par
      \end{mdframed}
      \smallskip
    }
% plain frames are inserted even without frame
\newenvironment
    {plainframeblock}[2][keyword]
    {\par\hbox{}\smallskip \begingroup\raggedright\leavevmode}
    {\par\endgroup\smallskip}

%%%%
%%%% More
%%%%

{\catcode`\^^I=13 \globaldefs=1
\newcommand\listing[2]{\begingroup\small\par\vspace{1ex}
  \catcode`\^^I=13 \def^^I{\leavevmode\hspace{40pt}}
  \noindent\fbox{#1}
  \verbatiminput{#2}\endgroup}
  \newcommand\codelisting[1]{\begingroup\small\par\vspace{1ex}
  \catcode`\^^I=13 \def^^I{\leavevmode\hspace{40pt}}
  \noindent\fbox{#1}
  \verbatiminput{#1}\endgroup}
}

\newcommand\inv{^{-1}}\newcommand\invt{^{-t}}
\newcommand\setspan[1]{[\![#1]\!]}
\newcommand\fp[2]{#1\cdot10^{#2}}
\newcommand\fxp[2]{\langle #1,#2\rangle}
\def\n#{\bgroup \catcode`\_=12 \catcode`\>=12 \catcode`\<=12 \catcode`\#=12
  \catcode`\&=12 \catcode`\^=12 \catcode`\~=12 \def\\{\char`\\}\relax
  \tt \let\next=}

\newcommand\prerequisite[1]{
  \begin{quote}
    \textsl{Before doing this section, make sure you study section \ref{#1}.}
  \end{quote}
}

\newcommand\diag{\mathop{\mathrm {diag}}}
\newcommand\argmin{\mathop{\mathrm {argmin}}}
\newcommand\defined{
  \mathrel{\lower 5pt \hbox{${\equiv\atop\mathrm{\scriptstyle D}}$}}}
\newcommand\lulubreak{\message{Hard page break!}\pagebreak\relax}

\newtheorem{remark}{Remark}
\expandafter\ifx\csname definition\endcsname\relax
    \newtheorem{definition}{Definition}
\fi
\expandafter\ifx\csname theorem\endcsname\relax
    \newtheorem{theorem}{Theorem}
\fi
\expandafter\ifx\csname lemma\endcsname\relax
    \newtheorem{lemma}{Lemma}
\fi
\expandafter\ifx\csname proof\endcsname\relax
 \newenvironment{proof}{\begin{quotation}\small\sl\noindent Proof.\ \ignorespaces}
     {\end{quotation}}
\fi
%% \newenvironment{highermath}
%%     {\bigskip\begin{quotation}\noindent\emph{MMM}}
%%     {\end{quotation}\bigskip\noindent\ignorespaces}

\def\innocentcharacters{%
  \catcode`\_=12 \catcode`\#=12
  \catcode`\>=12 \catcode`\<=12
  \catcode`\&=12 \catcode`\^=12 \catcode`\~=12
}


\newenvironment{exercises}{\begin{enumerate}}{\end{enumerate}}
\newcommand{\project}{\item[Project]}
\excludecomment{quiz}

\newcounter{excounter}[chapter]
\renewcommand\theexcounter{\@arabic\c@chapter.\@arabic\c@excounter}
%% \newcounter{revcounter}[chapter]
%% \renewcommand\therevcounter{\@arabic\c@chapter.\@arabic\c@revcounter}

\newcommand\exercisenumber{\arabic{chapter}.\arabic{excounter}}
\usepackage{mdframed,xcolor}
\mdfdefinestyle{exercisestyle}{%
    innertopmargin=2pt,
    innerbottommargin=2pt,
    innerrightmargin=2pt,
    innerleftmargin=2pt,
    backgroundcolor=green!20!white!80}
\mdfdefinestyle{reviewstyle}{%
    innertopmargin=2pt,
    innerbottommargin=2pt,
    innerrightmargin=2pt,
    innerleftmargin=2pt,
    backgroundcolor=blue!40!yellow!20!white!40}
\mdfdefinestyle{blockstyle}{%
    innertopmargin=2pt,
    innerbottommargin=2pt,
    innerrightmargin=2pt,
    innerleftmargin=2pt,
    backgroundcolor=yellow!20!green!20!white!60}
\makeatletter
\generalcomment{exercise}
  {\refstepcounter{excounter}%
   \begingroup\def\ProcessCutFile{}\par
   \edef\@currentlabel{\exercisenumber}%
   \edef\tmp{\def\noexpand\CommentCutFile
                 {exercises/\chapshortname-ex\arabic{excounter}.tex}}\tmp
   }
  {\par\hbox{}\smallskip
    \begin{mdframed}[style=exercisestyle]
   \leavevmode
   \hbox{\textbf{Exercise \exercisenumber.}\hspace{1em}}%
   \ignorespaces
   \input{\CommentCutFile}
    \end{mdframed}
    \smallskip
   \endgroup}
\makeatother

\newcounter{revcounter}[chapter]
\newcommand\reviewnumber{\arabic{chapter}.\arabic{revcounter}}
\generalcomment{review}
  {\refstepcounter{revcounter}%
   \begingroup\def\ProcessCutFile{}\par
   \edef\tmp{\def\noexpand\CommentCutFile
                 {exercises/\chapshortname-rev\arabic{revcounter}.tex}}\tmp
   }
  {\leavevmode\smallskip
    \begin{mdframed}[style=exercisestyle]
      \leavevmode
   \hbox{\textbf{Review} \reviewnumber.\hspace{1em}}%
   \ignorespaces
   \input{\CommentCutFile}
   \end{mdframed}
    \smallskip
   \endgroup}

\newif\ifIncludeAnswers
\input qa

\generalcomment{answer}
  {\begingroup
   \edef\tmp{\def\noexpand\CommentCutFile
                 {answers/\chapshortname-an\noexpand\arabic{excounter}.tex}}\tmp
   \def\ProcessCutFile{}}
  {\ifIncludeAnswers
    \par\smallskip
    \begin{mdframed}[backgroundcolor=red!30!white!70]%{quote}
      \leavevmode
      \begin{sl}\small
        \def\verbatim@startline{\verbatim@line{\leavevmode\relax}}%
        \hbox{\textbf Solution to exercise \arabic{chapter}.\arabic{excounter}.\hspace{1em}}%
        \ignorespaces\it
        \input{\CommentCutFile}
      \end{sl}
    \end{mdframed}
    \smallskip
    \fi
   \endgroup}

\newenvironment{exerciseB}[1]
  {\par
   {\bfseries Problem \arabic{chapter}.\arabic{excounter}}
   \parbox[t]{2in}{\slshape #1}%
   \par
  }
  {\par}


\makeatother

%%
%% Slack
%%
\newcommand\slackpoll[1]{}
%\newcommand\slackpollTF[1]{}
\def\slackpollTF#1{\begingroup\def\terminator{#1}
  \innocentcharacters
  \edef\tmp{\def\noexpand\slackpollTFp####1\terminator{\endgroup}}\tmp
  \slackpollTFp}
