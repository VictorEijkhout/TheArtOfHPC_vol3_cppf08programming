% -*- latex -*-
%%%%%%%%%%%%%%%%%%%%%%%%%%%%%%%%%%%%%%%%%%%%%%%%%%%%%%%%%%%%%%%%
%%%%
%%%% This TeX file is part of the course
%%%% Introduction to Scientific Programming in C++/Fortran2003
%%%% copyright 2017-2023 Victor Eijkhout eijkhout@tacc.utexas.edu
%%%%
%%%% macros for lecture slides
%%%%
%%%%%%%%%%%%%%%%%%%%%%%%%%%%%%%%%%%%%%%%%%%%%%%%%%%%%%%%%%%%%%%%

\newif\ifInBook \InBookfalse

\let\textbookref=\ref
\includecomment{322}   % overridden in cpp4c lecture
\excludecomment{cpp4c} % overridden in cpp4c lecture

% slide blurb only in slides
\includecomment{tldr}
% redefine code location wrt lectures
\newcommand\codedir{../../code}
\newcommand\slidenewline{\\}
% missing macro in beamer
\newdimen\unitindent
\unitindent=20pt
\let\slidebreak\slidenewline

\usepackage{xifthen}

%%%%
%%%% Indexing, disabled.
%%%%
\newcommand\indexterm[1]{{#1}}
\newcommand\indextermp[1]{{#1s}}
\newcommand\indextermdef[1]{{#1}}
\newcommand\indextermsub[2]{{#1 #2}}
\newcommand\indextermsubp[2]{{#1 #2s}}
\newcommand\indextermbus[2]{{#1 #2}}
\newcommand{\indextermsubdef}[2]{{#1 #2}}
\newcommand\indextermbusdef[2]{{#1 #2}}
\def\indextermtt{\bgroup \catcode`\_=12 \ttfamily \let\next=}
\let\indextermheader\indextermtt
\let\indextermunix\indextermtt
\let\indextermfort\indextermtt
\let\indexc\indextermtt
\def\indexcstd#1{\indexterm{std::#1}}
\let\indexcdef\indextermtt
\let\indexf\indextermtt
\let\indexfdef\indextermtt
\def\indextermttdef{\bgroup \catcode`\_=12 \ttfamily\slshape \let\next=}

\newcommand\inv{^{-1}}
\def\n#{\bgroup \innocentcharacters
  %\catcode`\_=12 \catcode`\>=12 \catcode`\<=12
  %\catcode`\&=12 \catcode`\^=12 \catcode`\~=12
  \def\\{\char`\\}\relax
  \tt \let\next=}

\makeatletter
\renewcommand{\verbatim@font}{\ttfamily\footnotesize}
\makeatother

%%%%
%%%% Outlines
%%%%
\usepackage{outliner}
\newcounter{sectionframe}
\OutlineLevelStart 0{\refstepcounter{sectionframe}
  \frame{\part{#1}\Large\bf #1}}
\OutlineLevelStart 1{\frame{\section*{#1}\large\bf #1}}
\def\sectionframe#{\Level 0 }
\usepackage{framed}
\colorlet{shadecolor}{blue!15}
\OutlineLevelStart 2{\subsection{#1}
  \frame{\begin{shaded}\large #1\end{shaded}}}

%%%%
%%%% Exercise and other slides
%%%%
\newcounter{excounter}
\newcommand\exerciseslide
    [1]{\frame{
        \refstepcounter{excounter}
        \frametitle{Exercise \arabic{excounter}}
        \input #1
}}
\newenvironment{exerciseframe}[1]
    {\refstepcounter{excounter}
      \begin{frame}[containsverbatim]{Exercise \arabic{excounter}: #1}}
    {\end{frame}}
\newcommand\optexerciseslide
    [1]{\frame{
        \refstepcounter{excounter}
        \frametitle{Optional exercise \arabic{excounter}}
        \input #1        
}}
\newcounter{revcounter}
\newcommand\reviewslide
    [1]{\frame{\frametitle{Review quiz
          \arabic{revcounter}}
        \input #1
        \stepcounter{revcounter}
}}
\newcommand\projectslide
    [1]{\frame{\frametitle{Project Exercise
          \arabic{excounter}}
        \input #1
        \stepcounter{excounter}
}}
\newcounter{slidecount}
\newenvironment
    {frameblock}[1]
    {\refstepcounter{slidecount}
      \begin{frame}[containsverbatim]
        {\hbox to \textwidth{\hbox to 20pt{\textrm{\arabic{slidecount}}.\ #1\hfil}}}}
    {\end{frame}}

%%%%
%%%% Acronyms
%%%%
\input acmacs

%%%%
%%%% Polling
%%%%
\def\innocentcharacters{%
  \catcode`\_=12 \catcode`\#=12
  \catcode`\>=12 \catcode`\<=12
  \catcode`\&=12 \catcode`\^=12 \catcode`\~=12
}

\newcommand\slackpoll[1]{\begingroup \par \tiny\texttt{/poll #1} \par \endgroup}
\def\slackpollTF#1{\begingroup \def\terminator{#1}
  \innocentcharacters \catcode`\{=12 \catcode`\}=12
  \edef\tmp{\def\noexpand\slackpollTFp####1\terminator
    {\noexpand\par
      \noexpand\tiny \noexpand\texttt{/poll "####1" "T" "F"}
      \noexpand\par \endgroup}}\tmp
  \slackpollTFp}

\newcommand\HPSCref[1]{see HPC book}

%%%%
%%%% title slide
%%%%
\input currentsemester
\newcommand\titleslide[1]{
  \title{#1}
  \frame{\titlepage}
  \date{\event}
}
