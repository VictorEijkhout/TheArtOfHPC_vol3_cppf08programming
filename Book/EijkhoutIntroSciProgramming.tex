% -*- latex -*-
%%%%%%%%%%%%%%%%%%%%%%%%%%%%%%%%%%%%%%%%%%%%%%%%%%%%%%%%%%%%%%%%
%%%%
%%%% This TeX file is part of the course
%%%% Introduction to Scientific Programming in C++11/Fortran2003
%%%% copyright 2017-2022 Victor Eijkhout eijkhout@tacc.utexas.edu
%%%%
%%%% ispbook.tex : main file of the textbook
%%%%
%%%%%%%%%%%%%%%%%%%%%%%%%%%%%%%%%%%%%%%%%%%%%%%%%%%%%%%%%%%%%%%%

\documentclass[11pt]{ispbook}

\usepackage{comment,pdflscape,verbatim}
\makeatletter
\def\verbatim@startline{\verbatim@line{\leavevmode\kern\unitindent\relax}}
\makeatother

\usepackage{amssymb}
\usepackage[fleqn]{amsmath}
\usepackage{array,graphicx,wrapfig,xifthen} % arydshln,
\usepackage{times}
\usepackage{booktabs,multirow,multicol}

\usepackage{wasysym}
\DeclareUnicodeCharacter{1F600}{\smiley}

\input acmacs
\input bookmacs
\input commonmacs
\input idxmacs
\input ispmacs
\input snippetmacs

\usepackage[pdftex,colorlinks]{hyperref}
\usepackage{etoolbox}
\usepackage[all]{hypcap}
\emergencystretch=1.5in

%%%%
%%%% Where is this course?
%%%%
\includecomment{tacc}
%
% omit material if a chapter is typeset by itself
%
\includecomment{inbook}

\makeindex

\def\revdate{2017--2022, formatted \today\\
  \small
  Book and slides download: \url{https://tinyurl.com/vle322course}\\
  Public repository: \url{https://bitbucket.org/VictorEijkhout/intro-programming-public}\\ [20pt]
  This book is published under the CC-BY 4.0 license.
}

\begin{document}

\title{Introduction to Scientific Programming in C++17/Fortran2008\\
  \small The Art of HPC, volume 3}
\author{Victor Eijkhout}
\expandafter\date\expandafter{\revdate}
\maketitle

\begin{multicols}{2}
  \tableofcontents
\end{multicols}

\part{Introduction}

\CHAPTER{Introduction}{intro}
\CHAPTER{Logistics}{warmup}
\lstset{language=C++,style=reviewcode}
\CHAPTER{Teachers guide}{teach}

\part{C++}
\label{part:cpp}
\idxset{cpp}
\lstset{language=C++,style=reviewcode}
\def\ISPcodeext{cxx}

\CHAPTER{Basic elements of C++}{elements}
\CHAPTER{Conditionals}{if}
\CHAPTER{Looping}{loop}
\CHAPTER{Functions}{function}
\CHAPTER{Scope}{scope}
%\CHAPTER{Structures}{struct}
\CHAPTER{Classes and objects}{object}
\CHAPTER{Arrays}{array}
\CHAPTER{Strings}{string}
\CHAPTER{Input/output}{io}
\CHAPTER{Lambda expressions}{lambda}
\CHAPTER{Iterators, Algorithms, Ranges}{range}
\CHAPTER{References}{address}
%\CHAPTER{Polymorphism}{poly}
%\CHAPTER{Memory}{memory}
\CHAPTER{Pointers}{pointer}
\CHAPTER{C-style pointers and arrays}{cpointer}
\CHAPTER{Const}{const}
\CHAPTER{Declarations and header files}{proto}
\CHAPTER{Namespaces}{namespace}
\CHAPTER{Preprocessor}{cpp}
\CHAPTER{Templates}{template}
\CHAPTER{Error handling}{error}
\CHAPTER{Standard Template Library}{stl}
\CHAPTER{Fine points of scalar types}{scalar}
\CHAPTER{Concurrency}{concur}
\CHAPTER{Obscure stuff}{obscure}
\CHAPTER{Graphics}{graphics}
\CHAPTER{C++ for C programmers}{ccpp}
\CHAPTER{C++ review questions}{cppreview}

%\CHAPTER{More exercises}{exercises}

\part{Fortran}
\label{part:f}
\lstset{language=[03]Fortran}
\idxset{f90}
\def\ISPcodeext{F90}

\CHAPTER{Basics of Fortran}{elementsf}
\CHAPTER{Conditionals}{iff}
\CHAPTER{Loop constructs}{loopf}
\CHAPTER{Procedures}{functionf}
\CHAPTER{Scope}{scopef}
\CHAPTER{String handling}{stringf}
\CHAPTER{Structures, eh, types}{structf}
\CHAPTER{Modules}{modulef}
\CHAPTER{Classes and objects}{classf}
\CHAPTER{Arrays}{arrayf}
\CHAPTER{Pointers}{pointf}
\CHAPTER{Input/output}{iof}
\CHAPTER{Leftover topics}{obscuref}
\CHAPTER{Fortran review questions}{freview}

%% \part{Julia (largely yet to be written)}
%% \label{part:j}
%% \lstset{language=Julia}
%% \idxset{julia}

\newcommand\jvp{python\index{julia!compared to pytyhon}}
\newcommand\jvc{C\index{julia!compared to C}}

%% \CHAPTER{Basics of Julia}{elementsj}
%% \CHAPTER{Conditionals}{ifj}
%% \CHAPTER{Looping}{loopj}
%% \CHAPTER{Functions}{functionj}
%% \CHAPTER{Arrays}{arrayj}
%% \CHAPTER{Structs}{structj}
%% \CHAPTER{More Julia}{morejulia}

\part{Exercises and projects}
\lstset{language=C++}
\idxset{default}
\def\ISPcodeext{cxx}

\CHAPTER{Style guide for project submissions}{projectstyle}

%% elementary projects
\CHAPTER{Prime numbers}{prime}
\CHAPTER{Geometry}{geom}
\CHAPTER{Zero finding}{zerofind}
\CHAPTER{Eight queens}{queens}

%% bigger projects
\CHAPTER{Infectious disease simulation}{infect}
\CHAPTER{Google PageRank}{google}
\CHAPTER{Redistricting}{gerry}
\CHAPTER{Amazon delivery truck scheduling}{amazon}
\CHAPTER{High performance linear algebra}{lapack}

%% not finished yet
\CHAPTER{Graph algorithms}{dijkstra}
\CHAPTER{Memory allocation}{malloc}
\CHAPTER{Ballistics calculations}{eniac.tex}
\CHAPTER{Cryptography}{crypt}
\CHAPTER{DNA Sequencing}{dna}

% one Fortran project
\lstset{language=Fortran,style=reviewcode}
\def\ISPcodeext{F90}
\CHAPTER{Climate change}{climate}
\CHAPTER{Desk Calculator Interpreter}{interpret}

\part{Advanced topics}

\lstset{language=C++}
\def\ISPcodeext{cxx}
\CHAPTER{External libraries}{external}
\CHAPTER{Programming strategies}{topdown}
\CHAPTER{Performance optimization}{walk}
\CHAPTER{Tiniest of introductions to algorithms and data structures}{algodata}
\CHAPTER{Provably correct programs}{proving}
\CHAPTER{Unit testing and Test-Driven Development}{tdd}
\CHAPTER{Debugging with gdb}{gdb}
\CHAPTER{Complexity}{efficiency}

\part{Index and such}

\addcontentsline{toc}{chapter}{General index of terms}

\begin{multicols*}{2}
  \printindex
\end{multicols*}

\Level 0 {Index of C++ keywords}

\begin{multicols*}{2}
  \printindex[cpp]
\end{multicols*}

\Level 0 {Index of Fortran keywords}

\begin{multicols*}{2}
  \printindex[f90]
\end{multicols*}

%% \Level 0 {Index of Julia keywords}

%% \begin{multicols*}{2}
%%   \printindex[jul]
%% \end{multicols*}

%% \Level 0 {remaining index}

%% \begin{multicols*}{2}
%%   \printindex[def]
%% \end{multicols*}

\Level 0 {Bibliography}

\bibliographystyle{plain}
\bibliography{vle}

\closeout\chapterlist
\end{document}
