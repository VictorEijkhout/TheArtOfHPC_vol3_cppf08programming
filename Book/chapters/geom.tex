%% -*- latex -*-
%%%%%%%%%%%%%%%%%%%%%%%%%%%%%%%%%%%%%%%%%%%%%%%%%%%%%%%%%%%%%%%%
%%%%
%%%% This TeX file is part of the course
%%%% Introduction to Scientific Programming in C++/Fortran2003
%%%% copyright 2017-2023 Victor Eijkhout eijkhout@tacc.utexas.edu
%%%%
%%%% geom.tex : exercises about geometry objects
%%%%
%%%%%%%%%%%%%%%%%%%%%%%%%%%%%%%%%%%%%%%%%%%%%%%%%%%%%%%%%%%%%%%%

In this set of exercises you will write a small `geometry' package:
code that manipulates points, lines, shapes.
These exercises mostly use the material of section~\ISPref{ch:object}.

\Level 0 {Basic functions}
\label{sec:geom-basic}

\begin{exercise}
  \label{ex:func-pointdistance}
  Write a function with (float or double) inputs $x,y$ that returns the distance of
  point $(x,y)$ to the origin.

  Test the following pairs: $1,0$; $0,1$; $1,1$; $3,4$.
\end{exercise}

\begin{exercise}
  \label{ex:pointrotate}
  Write a function with inputs $x,y,\theta$ that alters $x$ and~$y$
  corresponding to rotating the point $(x,y)$ over an angle~$\theta$.
  \[
  \begin{pmatrix}
    x'\\y'
  \end{pmatrix} =
  \begin{pmatrix}
    \cos\theta& -\sin\theta\\ \sin\theta&\cos\theta
  \end{pmatrix}
  \begin{pmatrix}
    x\\y
  \end{pmatrix}
  \]
  Your code should behave like:
  %
  \answerwithoutput{rotatedo2d}{geom}{rotate}
\end{exercise}

\Level 0 {Point class}
\label{ex:pointfunc}

\prerequisite{sec:object}

A class can contain elementary data. In this section you will make a
\n{Point} class that models Cartesian coordinates and functions
defined on coordinates.

\begin{exercise}
  \label{ex:geom:point}
  Make class \n{Point} with a constructor
\begin{lstlisting}
Point( float xcoordinate, float ycoordinate );
\end{lstlisting}
Write the following methods:
\begin{itemize}
\item \n{distance_to_origin} returns a \n{float}.
\item \n{angle} computes the angle of vector $(x,y)$ with the $x$-axis.
\end{itemize}
\end{exercise}

\begin{exercise}
  \label{ex:geom:pointpoint}
  Extend the \n{Point} class of the previous exercise with a method:
  \n{distance} that computes the distance between this point and
  another: if \n{p,q} are \n{Point} objects,
\begin{lstlisting}
p.distance(q) 
\end{lstlisting}
computes the distance between them.
\end{exercise}


\begin{exercise}
  \label{ex:geom:pointhalfway}
  Write a method \lstinline{halfway} that, given two \n{Point}
  objects~\n{p,q}, construct the \n{Point} halfway, that
  is,~$(p+q)/2$:
\begin{lstlisting}
Point p(1,2.2), q(3.4,5.6);
Point h = p.halfway(q);
\end{lstlisting}

  You can write this function directly, or you could write functions
  \n{Add} and \n{Scale} and combine these.\\
  (Later you will learn about operator overloading.)

  How would you print out a \lstinline{Point} to make sure
  you compute the halfway point correctly?
\end{exercise}

\begin{exercise}
  Make a default constructor for the point class:
\begin{lstlisting}
Point() { /* default code */ }
\end{lstlisting}
  which you can use as:
\begin{lstlisting}
Point p;
\end{lstlisting}
but which gives an indication that it is undefined:
%
\answerwithoutput{pointnan}{geom}{linearnan}
%
Hint: see section~\ISPref{sec:naninf}.
\end{exercise}

\begin{exercise}
  \label{ex:pointrotate-class}
  Revisit exercise~\textbookref{ex:pointrotate} using the \n{Point}
  class. Your code should now look like:
\begin{lstlisting}
newpoint = point.rotate(alpha);
\end{lstlisting}
\end{exercise}

\begin{exercise}
  Advanced. Can you make a \n{Point} class that can accommodate any
  number of space dimensions? Hint: use a \n{vector};
  section~\textbookref{sec:stdvector}. Can you make a constructor where you do
  not specify the space dimension explicitly?
\end{exercise}

\Level 0 {Using one class in another}
\label{sec:FuncHasPoint}
\label{sec:poly-rectangle}

\prerequisite{sec:hasa}

\begin{exercise}
  \label{ex:geom:line}
  Make a class \n{LinearFunction} with a constructor:
\begin{lstlisting}
LinearFunction( Point input_p1,Point input_p2 );
\end{lstlisting}
  and a member function
\begin{lstlisting}
float evaluate_at( float x );
\end{lstlisting}
  which you can use as:
\begin{lstlisting}
LinearFunction line(p1,p2);
cout << "Value at 4.0: " << line.evaluate_at(4.0) << endl;
\end{lstlisting}
\end{exercise}

\begin{exercise}
  \label{ex:geom:line2}
  Make a class \n{LinearFunction} with two constructors:
\begin{lstlisting}
LinearFunction( Point input_p2 );
LinearFunction( Point input_p1,Point input_p2 );
\end{lstlisting}
where the first stands for a line through the origin.\\
Implement again the \n{evaluate} function so that
\begin{lstlisting}
LinearFunction line(p1,p2);
cout << "Value at 4.0: " << line.evaluate_at(4.0) << endl;
\end{lstlisting}
\end{exercise}

\begin{exercise}
  \label{ex:pointrotate-matrix}
  Revisit exercises \textbookref{ex:pointrotate}
  and~\textbookref{ex:pointrotate-class}, introducing a \n{Matrix} class. Your
  code can now look like
\begin{lstlisting}
newpoint = point.apply(rotation_matrix);
\end{lstlisting}
  or
\begin{lstlisting}
newpoint = rotation_matrix.apply(point);
\end{lstlisting}
  Can you argue in favor of either one?
\end{exercise}

Suppose you want to write a \n{Rectangle} class, which could have methods such as
\n{float Rectangle::area()} or \n{bool Rectangle::contains(Point)}.
Since rectangle has four corners, you could store four \n{Point}
objects in each \n{Rectangle} object. However, there is redundancy
there: you only need three points to infer the fourth. Let's consider
the case of a rectangle with sides that are horizontal and vertical;
then you need only two points.

\begin{block}{Axi-parallel rectangle class}
  \label{ex:geom:rect}
  Intended API:
\begin{lstlisting}
float Rectangle::area();
\end{lstlisting}
It would be convenient to store width and height; for 
\begin{lstlisting}
bool Rectangle::contains(Point);  
\end{lstlisting}
it would be convenient to store bottomleft/topright points.
\end{block}

\begin{exercise}
  \label{ex:geom:rect2}
  \footnotesize
  \begin{enumerate}
  \item
    Make a class \n{Rectangle} (sides parallel to axes) with a constructor:
\begin{lstlisting}
Rectangle(Point botleft,float width,float height);
\end{lstlisting}
The logical implementation is to store these quantities.
Implement methods:
\begin{lstlisting}
float area(); float rightedge_x(); float topedge_y();
\end{lstlisting}
and write a main program to test these.
\item Add a second constructor
\begin{lstlisting}
Rectangle(Point botleft,Point topright);
\end{lstlisting}
Can you figure out how to use \indextermsub{member}{initializer} lists for the constructors?
\end{enumerate}
\end{exercise}

\begin{exercise}
  \label{ex:geom:rect3}
  Make a copy of your solution of the previous exercise,
  and redesign your class so that it
  stores two \n{Point} objects.
  Your main program should not change.
\end{exercise}


The previous exercise illustrates an important point: for well
designed classes you can change the implementation (for instance motivated
by efficiency) while the program that uses the class does not change.

\Level 0 {Is-a relationship}
\label{sec:geom-isa}

\prerequisite{sec:inheritance}

\begin{exercise}
  \label{ex:geom:square}
  Take your code where a \n{Rectangle} was defined from one point,
  width, and height.

  Make a class \n{Square} that inherits from \n{Rectangle}. It should
  have the function \n{area} defined, inherited from \n{Rectangle}.

  First ask yourself: what should the constructor of a \n{Square} look like?
\end{exercise}

\begin{exercise}
  \label{ex:geom:line3}
  Revisit the \n{LinearFunction} class.
  Add methods \n{slope} and \n{intercept}.

  Now generalize \n{LinearFunction} to \n{StraightLine}
  class. These two are almost the same except for vertical lines.
  The \n{slope} and \n{intercept} do not apply to vertical lines, so
  design \n{StraightLine} so that it stores the defining points
  internally. Let \n{LinearFunction} inherit.
\end{exercise}

\Level 0 {Pointers}

\prerequisite{sec:shared_ptr}

The following exercise is a little artificial.

\begin{exercise}
  \label{ex:dynrect}
  Make a \lstinline{DynRectangle} class, which is constructed
  from two shared-pointers-to-\lstinline{Point} objects:
  %
  \verbatimsnippet{dynrectmake}

  Calculate the area, scale the top-right point, and recalculate the area:
  %
  \snippetwithoutput{dynrectscale}{pointer}{dynrect}
  %
  \skeleton{pointrectangle}
\end{exercise}

\Level 0 {More stuff}

\prerequisite{sec:class-ref}

The \n{Rectangle} class stores at most one corner, but may be
convenient to sometimes have an array of all four corners.

\begin{exercise}
  \label{ex:geom:corners}
  Add a method
\begin{lstlisting}
const vector<Point> &corners()
\end{lstlisting}
to the \n{Rectangle} class. The result is an array of all four
corners, not in any order. Show by a compiler error that the array can
not be altered.
\end{exercise}

\prerequisite{sec:operatordef}

\begin{exercise}
  \label{ex:geom:halfwayop}
  Revisit exercise~\ref{ex:geom:pointhalfway}
  and replace the \lstinline{add} and \lstinline{scale} functions
  by overloaded operators.

  Hint: for the \lstinline{add} function you may need `\lstinline{this}'.
\end{exercise}
