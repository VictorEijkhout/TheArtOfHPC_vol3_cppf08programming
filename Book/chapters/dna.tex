% -*- latex -*-
%%%%%%%%%%%%%%%%%%%%%%%%%%%%%%%%%%%%%%%%%%%%%%%%%%%%%%%%%%%%%%%%
%%%%
%%%% This TeX file is part of the course
%%%% Introduction to Scientific Programming in C++/Fortran2003
%%%% copyright 2017-2023 Victor Eijkhout eijkhout@tacc.utexas.edu
%%%%
%%%% dna.tex : exercises about dna sequencing
%%%%
%%%%%%%%%%%%%%%%%%%%%%%%%%%%%%%%%%%%%%%%%%%%%%%%%%%%%%%%%%%%%%%%

In this set of exercises you will write mechanisms for DNA sequencing.

\Level 0 {Basic functions}

Refer to section~\textbookref{sec:tuple}.

First we set up some basic mechanisms.

\begin{exercise}
  \label{ex:basecomp}
  There are four bases, \n{A,C,G,T}, and each has a complement:
  $\n{A}\leftrightarrow\n{T}$, $\n{C}\leftrightarrow\n{G}$.
  Implement this through a \n{map}, and write a function
\begin{verbatim}
char BaseComplement(char);
\end{verbatim}
\end{exercise}

\begin{exercise}
  \label{ex:basecount}
  Write code to
  read a \indexterm{Fasta} file into a \n{string}.
  The first line, starting with~\n{>}, is a comment; all other lines
  should be concatenated into a single string denoting the genome.
  
  Read the virus genome in the file \n{lambda_virus.fa}.

  Count the four bases in the genome two different ways.  First use a
  \n{map}.  Time how long this takes. Then do the same thing using an
  array of length four, and a conditional statement.

  Bonus: try to come up with a faster way of counting. Use a vector of
  length~4, and find a way of computing the index directly from the
  letters \n{A,C,G,T}. Hint: research ASCII codes and possibly bit operations.
\end{exercise}

\Level 0 {De novo shotgun assembly}

One approach to generating a genome is to cut it to pieces,
and algorithmically glue them back together.
(It is much easier to sequence the bases of a short read
than of a very long genome.)

If we assume that we have enough reads that each
genome position is covered, we can look at
the overlaps by the reads.
One heuristic is then to find the \acf{SCS}.
%Shortest Common Superstring.

\Level 1 {Overlap layout consensus}

\begin{enumerate}
\item Make a graph where the reads are the vertices,
  and vertices are connected if they overlap; the amount of overlap
  is the edge weight.
\item The \ac{SCS} is then a \indextermsub{Hamiltonian}{path} through this graph
  --~this is already NP-complete.
\item Additionally, we optimize for maximum total overlap.\\
  $\Rightarrow$ \acf{TSP}, NP-hard.
\end{enumerate}

Rather than finding the optimal superset,
we can use a greedy algorithm,
where every time we find the read with maximal overlap.

Repeats are often a problem.
Another is spurious subgraphs from sequencing errors.

\Level 1 {De Bruijn graph assembly}

\Level 0 {`Read' matching}

A~`read' is a short fragment of DNA, that we want to match against a
genome. In this section you will explore algorithms for this type
of matching.

While here we mostly consider the context of genomics, such
algorithms have other applications.
For instance, searching for a word in a web page
is essentially the same problem.
Consequently, there is a considerable history of this topic.

\Level 1 {Naive matching}

We first explore a naive matching algorithm: for each location
in the genome, see if the read matches up.
\begin{verbatim}
ATACTGACCAAGAACGTGATTACTTCATGCAGCGTTACCAT
  ACCAAGAACGTG
    ^ mismatch

ATACTGACCAAGAACGTGATTACTTCATGCAGCGTTACCAT
      ACCAAGAACGTG
      total match
\end{verbatim}

\begin{exercise}
  \label{ex:readmatch}
  Code up the naive algorithm for matching a read. Test it on fake
  reads obtained by copying a substring from the
  genome. Use the genome in \n{phix.fa}.

  Now read the \indexterm{Fastq} file
  \n{ERR266411_1.first1000.fastq}. Fastq files contains groups of four
  lines: the second line in each group contains the reads.
  How many of these reads are matched to the genome?

  Reads are not necessarily a perfect match; in fact, each fourth line
  in the fastq file gives an indication of the `quality' of the
  corresponding read. How many matches do you get if you take a
  substring of the first 30 or so characters of each read?
\end{exercise}

\Level 1 {Boyer-Moore matching}

The \indexterm{Boyer-Moore} string matching algorithm~\cite{BoyerMoore:cacm77}
is much faster than naive matching,
since it uses two clever tricks to weed out comparisons that would
not give a match.

\textbf{Bad character rule}

In naive matching, we determined match locations left-to-right,
and then tried matching left-to-right.
In \ac{BM}, we still find match locations left-to-right, but
we do our matching right-to-left.

\begin{verbatim}
   vvvv match location
antidisestablishmentarianism
   blis
    ^bad character   
\end{verbatim}
The mismatch is an `l' in the pattern, which does not match a `d' in the text.
Since there is no `d' in the pattern at all,
we move the pattern completely past the mismatch:
\begin{verbatim}
   vvvv match location
antidisestablishmentarianism
       blis
\end{verbatim}
in fact, we move it further, to the first match on the first character of the pattern:
\begin{verbatim}
           vvvv match location
antidisestablishmentarianism
           blis
           ^ first character match
\end{verbatim}

The case where we have a mismatch, but the character in the text does
appear in the pattern is a little trickier:
we find the next occurrence of the mismatched character in the pattern,
and use that to determine the shift distance.
\begin{verbatim}
shoobeedoobeeboobah
edoobeeboob
       ^ mismatch
 ^ other occurrence of `d'
\end{verbatim}
Note that this can be a considerably smaller shift than in the previous case.
\begin{verbatim}
       v
shoobeedoobeeboobah
      edoobeeboob
       ^ match the bad character `d'
      ^ new location
\end{verbatim}

\begin{exercise}
  Discuss how efficient you expect this heuristic to be
  in the context of genomics versus text searching. (See above.)
\end{exercise}

\textbf{Good suffix rule}

The `good suffix' consists of the matched characters
after the bad character.
When moving the read,
we try to keep the good suffix intact:

\begin{verbatim}
desistrust
  listrest
        ^^good suffix
\end{verbatim}

\begin{verbatim}
desistrust
      listrest
        ^^next occurrence of suffix
\end{verbatim}
