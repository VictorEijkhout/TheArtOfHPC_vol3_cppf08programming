% -*- latex -*-
%%%%%%%%%%%%%%%%%%%%%%%%%%%%%%%%%%%%%%%%%%%%%%%%%%%%%%%%%%%%%%%%
%%%%
%%%% This TeX file is part of the course
%%%% Introduction to Scientific Programming in C++/Fortran2003
%%%% copyright 2017-2021 Victor Eijkhout eijkhout@tacc.utexas.edu
%%%%
%%%% functionf.tex : functions and such in Fortran
%%%%
%%%%%%%%%%%%%%%%%%%%%%%%%%%%%%%%%%%%%%%%%%%%%%%%%%%%%%%%%%%%%%%%

Programs can have subprograms: parts of code that for some reason you
want to separate from the main program. The term for these is
\indextermdef{procedure}.
While this is actually a keyword, you will not see it
until section~\ref{sec:f-polyfunction};
in this chapter we consider only \lstinline{subroutine} and \lstinline{function}.

If you structure your code in a single
file, this is the recommended structure:

\begin{block}{Procedures in contains clause}
  \label{sl:contains}
  Simplest way of defining procedures:\\
  in \lstinline{Contains} part of main program.
\begin{lstlisting}
Program foo
  < declarations>
  < executable statements >
  Contains
    < procedure definitions >
End Program foo
\end{lstlisting}
Two types of procedures: functions and subroutines. More later.
\end{block}

That is, procedures are placed after the main program statements,
separated by a \indexf{Contains} clause.

In general, these are the  placements of procedures:
\begin{itemize}
\item Internal: after the \indexf{Contains} clause of a program:
\begin{lstlisting}
Program foo
  ... stuff ...
Contains
  Subroutine bar()
  End Subroutine bar
End Program foo
\end{lstlisting}
This is the mode that we focus on in this chapter.
\item In a \indexf{Module}; see section~\ref{sec:modulef}.
\item Externally: the procedure is not internal to a \indexf{Program} or
  \indexf{Module}.
  This can happen in the case of 3rd party libraries, or
  code linked in from another language.
  In this case it's safest to declare the procedure through an
  \indexf{Interface} specification; section~\ref{sec:finterface}.
\end{itemize}

\Level 0 {Subroutines and functions}

Fortran has two types of procedures:
\begin{itemize}
\item Subroutines, which are somewhat like \n{void} functions in C++:
  they can be used to structure the code, and they can only return
  information to the calling environment through their parameters.
\item Functions, which are like C++ functions with a return value.
\end{itemize}

Both types have the same structure, which is roughly the same as of
the main program.
For subroutines:
\begin{lstlisting}
subroutine foo( <parameters> )
<variable declarations>
<executable statements>
end subroutine foo
\end{lstlisting}
and for functions:
\begin{lstlisting}
returntype function foo( <parameters> )
<variable declarations>
<executable statements>
end function foo
\end{lstlisting}
There is another syntax for declaring functions,
see section~\ref{sec:fresult}.

Exit from a procedure can happen two ways:
\begin{enumerate}
\item the flow of control reaches the end of the procedure body:
\begin{lstlisting}
subroutine foo()    
  statement1
  ..
  statementn
end subroutine foo
\end{lstlisting}
  or
\item execution is finished by an explicit \indexf{return}
  statement.
\begin{lstlisting}
subroutine foo()
  print *,"foo"
  if (something) return
  print *,"bar"
end subroutine foo
\end{lstlisting}
\end{enumerate}
The \indexf{return} statement is optional in the first case.
The \indexf{return} statement is different from C++ in that it does not
indicate a returned result value of a function.

\begin{exercise}
  Rewrite the above subroutine \n{foo} without a \indexf{return} statement.
\end{exercise}

A subroutine is invoked with a \indexf{call} statement:
\begin{lstlisting}
call foo()
\end{lstlisting}

\begin{slide}{Subroutines}
  \label{sl:subroutine}
\begin{lstlisting}
subroutine foo()
  implicit none
  print *,"foo"
  if (something) return
  print *,"bar"
end subroutine foo
\end{lstlisting}
\begin{itemize}
\item \lstinline{Subroutine} is like a \lstinline{void} function.
\item Same structure as main program.
\item Ends at the end, or when \indexf{return} is reached.
\item Note: \indexf{return} does not return anything:\\
  indicates return from the procedure.
\item Invoked with 
\begin{lstlisting}
  call foo()
\end{lstlisting}
\end{itemize}
\end{slide}

\begin{block}{Subroutine with argument}
  \label{sl:fsubr-arg}
  \snippetwithoutput{fprintone}{funcf}{printone}

  Arguments types are defined in the body, not the header
\end{block}

\begin{block}{Subroutine can change argument}
  \label{sl:fsubr-inout}
  \snippetwithoutput{faddone}{funcf}{addone}
  Parameters are always `by reference'!
\end{block}

Recursive functions in Fortran need to be explicitly declared as such,
with the \indexf{recursive} keyword.

\begin{block}{Recursion}
  \label{sl:funcf:recursion}
  Declare function as \indexf{Recursive}~\indexf{Function}
  %
  \snippetwithoutput{frecursf}{funcf}{fact}
  %
  Note the \indexf{result} clause. This prevents ambiguity.
\end{block}

\Level 0 {Return results}

While a \indexf{subroutine} can only return information through its parameters,
a~\indexterm{function} procedure returns an explicit result:
\begin{lstlisting}
logical function test(x)
  implicit none
  real :: x

  test = some_test_on(x)
  return ! optional, see above
end function test
\end{lstlisting}
You see that the result is not returned in the \indexf{return} statement,
but rather through assignment to the function name. The \indexf{return}
statement, as before, is optional and only indicates where the flow of
control ends.

A \indextermdef{function} in Fortran is a procedure that return a
result to its calling program, much like a non-void function in~C++

\begin{block}{Function definition and usage}
  \label{sl:ffunction-def}
  \begin{itemize}
  \item \lstinline$subroutine$ vs \lstinline$function$:\\
    compare \lstinline$void$ functions vs non-void in~C++.
  \item Function header:\\
    Return type, keyword \indexf{function}, name, parameters
  \item Function body has statements
  \item Result is returned by assigning to the function name
  \item Use: \n{y = f(x)}
  \end{itemize}
\end{block}

\begin{block}{Function example}
  \label{sl:ffunction-ex}
  \snippetwithoutput{fplusone}{funcf}{plusone}
  \begin{itemize}
  \item The function name is a variable
  \item \ldots~that you assign to.
  \end{itemize}
\end{block}

A function is not invoked with \indexf{call}, but rather through being used
in an expression:
\begin{lstlisting}
if (test(3.0) .and. something_else) ...
\end{lstlisting}
You now have the following cases to make the function known in the
main program:
\begin{itemize}
\item If the function is in a \indexf{contains} section, its type is known
  in the main program.
\item If the function is in a module (see section~\ref{sec:modulef}
  below), it becomes known through a \indexf{use} statement.
\end{itemize}

\begin{f77note}
  Without modules and \indexf{contains} sections, you need to declare the
  function type explitly in the calling program. The safe way is
  through using an \indexf{interface} specification.
\end{f77note}

\begin{exercise}
  \label{ex:freadpos}
  Write a program that asks the user for a positive number;
  non-positive
  input should be rejected.  Fill in the missing lines in this code
  fragment:
  \snippetwithoutput{readpos}{funcf}{readpos}
\end{exercise}

\Level 1 {The `result' keyword}
\label{sec:fresult}

Apart from assigning to the function name,
there is a second mechanism for returning a function result,
namely through the \indexf{Result} keyword.
%
\begin{lstlisting}
function some_function() result(x)
  implicit none
  real :: x
  !! stuff
  x = ! some computation
end function
\end{lstlisting}

You see that here
\begin{itemize}
\item the assignment to the name is missing,
\item the function name is not typed; but
\item instead there is a typed local variable that is marked to be the result.
\end{itemize}

\Level 1 {The `contains' clause}

\begin{block}{Why a `contains' clause?}
  \label{sl:whycontain}
  \begin{multicols}{2}
    \verbatimsnippet{nocontain}
    Warning only, crashes.
    \vfill\columnbreak
    \verbatimsnippet{wrongcontain}
    Error, does not compile
  \end{multicols}
\end{block}

\begin{block}{Why a `contains' clause, take 2}
  \label{sl:whycontain_type}
  \snippetwithoutput{nocontaintype}{funcf}{nocontaintype}
  At best compiler warning if all in the same file\\
  %% For future reference: if you see very small floating point numbers,
  %% maybe you have made this error.
\end{block}

\Level 0 {Arguments}

\begin{block}{Procedure arguments}
  \label{sl:farguments}
 Arguments are declared in procedure body:
\begin{lstlisting}
subroutine f(x,y,i)
  implicit none
  integer,intent(in) :: i
  real(4),intent(out) :: x
  real(8),intent(inout) :: y
  x = 5; y = y+6
end subroutine f
! and in the main program
call f(x,y,5)
\end{lstlisting}
declaring the `intent' is optional, but highly advisable.
\end{block}

\begin{block}{Parameter passing}
  \label{sl:fpassing}
  \begin{itemize}
  \item Everything is passed by reference.\\
    Don't worry about large objects being copied.
  \item Optional intent declarations:\\
    Use \indexf{in}, \indexf{out}, \indexf{inout} qualifiers to clarify semantics
    to compiler.
  \end{itemize}
\end{block}

\begin{block}{Fortran nomenclature}
  \label{sl:fortran-dummy}
  The term \indextermsub{dummy}{argument} is what Fortran calls the
  parameters in the procedure definition:
\begin{lstlisting}
subroutine f(x) ! `x' is dummy argument
\end{lstlisting}
  The arguments in the
  procedure call are the \indextermsub{actual}{argument}s:
\begin{lstlisting}
call f(x) ! `x' is actual argument
\end{lstlisting}
\end{block}

\begin{block}{Intent checking}
  \label{sl:fintent}
  Compiler checks your intent against your implementation. This code
  is not legal:

  \verbatimsnippet{arginwrong}
\end{block}

\begin{block}{Why intent checking?}
\label{sl:intentwhy}
Self-protection: if you state the intended behaviour of a routine, the
compiler can detect programming mistakes.

Allow compiler optimizations:

\begin{multicols}{2}
\begin{lstlisting}
x = f()
call ArgOut(x)
print *,x
\end{lstlisting}
Call to \n{f} removed
\vfill\columnbreak
\begin{lstlisting}
do i=1,1000
  x = ! something
  y1 = .... x ....
  call ArgIn(x)
  y2 = ! same expression as y1
\end{lstlisting}
\n{y2} is same as \n{y1} because \n{x} not changed
\end{multicols}
(May need further specifications, so this is not the prime justification.)
\end{block}

\begin{exercise}
  \label{ex:ffunc-sin-cos}
  Write a subroutine \n{trig} that takes a number~$\alpha$ as input
  and passes $\sin\alpha$ and $\cos\alpha$ back to the calling
  environment.
\end{exercise}

\Level 1 {Keyword and optional arguments}

The arguments in a procedure call can always be given with their
corresponding parameter name.
This is called a \indextermsub{keyword}{argument},
and it is sometimes useful to prevent confusion.
\begin{lstlisting}
! confusing:
call two_point( 1.1, 2.2, 3.3, 4.4 )
! better:  
call two_point( x1=1.1, x2=2.2, y1=3.3, y2=4.4 )
\end{lstlisting}
Arguments not given with a keyword are called
\indextermsub{positional}{argument}s.
You can mix positional and keyword arguments,
but if you give one argument by keyword, all subsequent ones
also need their keyword.

\begin{block}{Keyword arguments}
  \label{sl:funcf:keyword}
  \begin{itemize}
  \item Use the name of the \indextermsub{formal}{parameter} as
    keyword.
  \item Keyword arguments have to come last.
  \end{itemize}
  \snippetwithoutput{sayxykw}{funcf}{keyword}
\end{block}

A relation notion is that of \indextermsub{optional}{argument}s.
A~parameter can be marked \indexfdef{optional}, after which
it can be omitted from a procedure call.
\begin{itemize}
\item Optional parameters can be anywhere in the parameter list;
\item If you omit one optional parameter in the argument list,
  all subsequent arguments need to be given by keyword.
\item The procedure can test whether or not an optional argument was supplied
   with the function \indexf{Present}
\end{itemize}

\begin{block}{Optional arguments}
  \label{sl:funcf:optional}
  \begin{itemize}
  \item Extra specifier: \indexf{Optional}
  \item Presence of argument can be tested with \indexf{Present}
  \end{itemize}
\end{block}

\Level 0 {Types of procedures}

Procedures that are in the main program (or another type of program
unit), separated by a \indexf{contains} clause, are known as
\indextermsub{internal}{procedures}. This is as opposed to
\indextermsub{module}{procedures}.

There are also \indexterm{statement functions}, which are
single-statement functions, usually to identify commonly used
complicated expressions in a program unit. Presumably the compiler
will \indexterm{inline} them for efficiency.

The standard library functions, such as \lstinline+sqrt+,
can be declared as such in an \indexfdef{intrinsic} statement
\begin{lstlisting}
  Intrinsic :: sqrt,cmplx
\end{lstlisting}
but this is not necessary.

The \indexf{entry} statement is so bizarre that I refuse to discuss it.

\Level 0 {Local variable \texttt{save}-ing}
\label{sec:f-save-var}

Normally, local variables in a procedure act as if they
get created when the procedure is invoked,
and disappear again when its execution ends.
It is possible to retain the value of a variable
between invocations by giving it an attribute of \indexfdef{save}.
\begin{lstlisting}
subroutine whatever()
integer,save :: i
\end{lstlisting}
(This corresponds roughly to a \indexc{static} variable in~C++.)

Here is a major pitfall.
If you give a local variable an initialization value:
\begin{lstlisting}
subroutine whatever()
integer :: i = 5
\end{lstlisting}
then the variable implicitly gets a \indexf{save} attribute,
whether this is specified or not.
The initialization is only executed once,
probably at compile time,
and at the second procedure invocation
the saved value is used.

This may trip you up as the following example shows:

\begin{block}{Saved values}
  \label{sl:func-param-save}
  Local variable is initialized only once,\\
  second time it uses its retained value.
  
  \snippetwithoutput{unsafesave}{funcf}{save}
\end{block}


