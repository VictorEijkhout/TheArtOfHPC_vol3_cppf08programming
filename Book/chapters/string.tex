% -*- latex -*-
%%%%%%%%%%%%%%%%%%%%%%%%%%%%%%%%%%%%%%%%%%%%%%%%%%%%%%%%%%%%%%%%
%%%%
%%%% This TeX file is part of the course
%%%% Introduction to Scientific Programming in C++/Fortran2003
%%%% copyright 2017-2020 Victor Eijkhout eijkhout@tacc.utexas.edu
%%%%
%%%% string.tex : text strings
%%%%
%%%%%%%%%%%%%%%%%%%%%%%%%%%%%%%%%%%%%%%%%%%%%%%%%%%%%%%%%%%%%%%%

\Level 0 {Characters}
\label{sec:charint}

\begin{block}{Characters and ints}
  \label{sl:chardef}
  \begin{itemize}
  \item Type \indexcdef{char};
  \item represents `7-bit ASCII': printable and (some) unprintable
    characters.
  \item Single quotes: \n{char c = 'a'}
  \end{itemize}
\end{block}

\begin{block}{Char / int equivalence}
  \label{sl:int-char}
  Equivalent to (short) integer:
  %
  \snippetwithoutput{intcharxy}{string}{intchar}
  %
  Also: \n{'x'-'a'} is distance \n{a--x}
\end{block}

\begin{remark}
  The translation from \n{'x'} to ascii code,
  and in particular the letters having consecutive values,
  are not guaranteed by the standard.
\end{remark}

\begin{exercise}
  \label{ex:print-ichar}
  Write a program that accepts an integer $1\cdots26$ and prints the
  so-manieth letter of the alphabet.

  Extend your program so that if the input is negative, it prints the
  minus-so-manieth uppercase letter of the alphabet.
\end{exercise}

\Level 0 {Basic string stuff}
\label{sec:string}

\begin{block}{String declaration}
  \label{sl:string-declare}
\begin{lstlisting}
#include <string>
using std::string;

// .. and now you can use `string'
\end{lstlisting}
(Do not use the C legacy mechanisms.)
\end{block}

\begin{block}{String creation}
  \label{sl:string-create}
  A \indexterm{string} variable contains a string of characters.
\begin{lstlisting}
string txt;
\end{lstlisting}
You can initialize the string variable or assign it dynamically:
\begin{lstlisting}
string txt{"this is text"};
string moretxt("this is also text");
txt = "and now it is another text";
\end{lstlisting}
\end{block}

Normally, quotes indicate the start and end of a string. So what if
you want a string with quotes in it?

\begin{block}{Quotes in strings}
  \label{sl:string-quote}
  You can escape a quote, or indicate that the whole string is to be
  taken literally:
  %
  \snippetwithoutput{quotestring}{string}{quote}
\end{block}

\begin{block}{Concatenation}
  \label{sl:string-plus}
  Strings can be \emph{concatenated}\index{string!concatenation}:
  \snippetwithoutput{stringadd}{string}{stringadd}
\end{block}

\begin{block}{String indexing}
  \label{sl:string-vector}
  You can query the \emph{size}\index{string!size}:
  \snippetwithoutput{stringsize}{string}{stringsize}
  or use subscripts:
  \snippetwithoutput{stringsub}{string}{stringsub}
\end{block}

\begin{block}{Ranging over a string}
  \label{sl:string-index}
  Same as ranging over vectors.

  Range-based for:
  %
  \snippetwithoutput{stringrange}{string}{stringrange}

  Ranging by index:
  %
  \snippetwithoutput[stringrange]{stringindex}{string}{stringindex}

\end{block}

\begin{block}{Range with reference}
  \label{sl:string-index-ref}
  Range-based for makes a copy of the element\\
  You can also get a reference:
  %
  \snippetwithoutput[stringrange]{stringrangeset}{string}{stringrangeset}
\end{block}

\begin{block}{Iterating over a string}
\begin{lstlisting}
for ( auto c : some_string)
  // do something with the character 'c'
\end{lstlisting}
\end{block}

\begin{review}
  \label{q:string}
  True or false?
  \begin{enumerate}
  \item \verb+'0'+ is a valid value for a \n{char} variable
    \slackpollTF+single-quote 0 is a valid char+
  \item \verb+"0"+ is a valid value for a \n{char} variable
    \slackpollTF+double-quote 0 is a valid char+
  \item \verb+"0"+ is a valid value for a \n{string} variable
    \slackpollTF+double-quote 0 is a valid string+
  \item \verb/'a'+'b'/ is a valid value for a \n{char} variable
    \slackpollTF+adding single-quote chars is a valid char+
  %\item \verb/'a'+'b'/ is a valid value for a \n{string} variable
  \end{enumerate}
\end{review}

\begin{exercise}
  \label{ex:caesar}
  The oldest method of writing secret messages is the
  \indexterm{Caesar cipher}. You would take an integer~$s$ and rotate every character 
  of the text over that many positions:
  \[ s\equiv3\colon \hbox{"acdz" $\Rightarrow$ "dfgc"}. \]
  Write a program that accepts an integer and a string, and display
  the original string rotated over that many positions.
\end{exercise}
\begin{exercise}
  \label{ex:caesar-decrypt}
  (this continues exercise~\ref{ex:caesar})\\
  If you find a message encrypted with the Caesar cipher, can you
  decrypt it? Take your inspiration from the
  \indextermsub{Sherlock}{Holmes} story `The Adventure of the Dancing
  Men', where he uses the fact that `e' is the most common letter.

  Can you implement a more general letter permutation cipher, and
  break it with the `dancing men' approach?
\end{exercise}

\begin{block}{More vector methods}
  \label{sl:string-vector-methods}
  Other methods for the vector class apply: \n{insert}, \n{empty},
  \n{erase}, \n{push_back}, et cetera.

  \snippetwithoutput{stringpush}{string}{stringpush}

  Methods only for \n{string}: \n{find} and such.

  \url{http://en.cppreference.com/w/cpp/string/basic_string}
\end{block}

\begin{exercise}
  \label{ex:printdigits}
  Write a function to print out the digits of a number: \n{156} should
  print \n{one five six}.
  You need to convert a digit to a string first; can you think of more
  than one way to do that?
  % Use a vector or array of strings, containing the names of the digits.

  Start by writing a program that reads a single digit and prints its name.

  For the full program it is easiest to generate the  digits last-to-first.
  Then figure out how to print them reversed.
\end{exercise}

\begin{exercise}
  \label{ex:printnumber}
  Write a function to convert an integer to a string: the input
  \n{215} should give \n{two hundred fifteen}, et cetera.
\end{exercise}

\begin{exercise}
  Write a pattern matcher, where a period~\n{.} matches any one
  character, and \n{x*} matches any number of~`\n{x}' characters.

  For example:
  \begin{itemize}
  \item The string \n{abc} matches \n{a.c} but \n{abbc} doesn't.
  \item The string \n{abbc} matches \n{ab*c}, as does \n{ac}, but
    \n{abzbc} doesn't.
  \end{itemize}
\end{exercise}

\Level 0 {String streams}
\label{sec:stringstream}

You can concatenate string with the \lstinline{+} operator.
The less-less operator also does a sort of concatenation. It is
attractive because it does conversion from quantities to string.
Sometimes you may want a combination of these facilities: conversion
to string, with a string as result.

For this you can use a \indextermbus{string}{stream}
from the \indextermheader{sstream} header.

\begin{block}{String stream}
  \label{sl:sstream}
  Like cout (including conversion from quantity to string), but to
  object, not to screen.
\begin{itemize}
\item Use the \lstinline{<<} operator to build it up; then
\item use the \lstinline{str} method to extract the string.
\end{itemize}
  \lstset{style=snippetcode}
\begin{lstlisting}
#include <sstream>
stringstream s;
s << "text" << 1.5;
cout << s.str() << endl;
\end{lstlisting}
\end{block}

\Level 0 {Advanced topics}

\Level 1 {Raw string literals}

You can include characters such as quotes or backslashes in a string
by escaping them. This may get tiresome.
The \cppstandard{11} standard has a mechanism for 
\emph{raw string literal}\index{string!raw literal|textbf}s.

In its simplest form:
\snippetwithoutput{raw1}{string}{raw1}

The obvious question is now of course how to include the
closing-paren-quote sequence in a string.
For this, you can specify your own multi character delimiter:
\snippetwithoutput{raw2}{string}{raw2}

\Level 1 {String literal suffix}

A string literal \lstinline+"foo"+ is often compatible with a
\lstinline+std::string+ but it is not of that type.
Should you need that, you can add a suffix to the literal,
which is defined in the namespace \indexc{string_literals}:
%
\snippetwithoutput{stringsuffix}{string}{stringsuffix}

\Level 1 {Conversion to/from string}

\Level 2 {Converting to string}

There are various mechanisms for converting between strings and numbers.
\begin{itemize}
\item The C legacy mechanisms \indexc{sprintf} and \indexc{itoa}.
\item  \indexc{to_string}
\item Above you saw the \indexc{stringstream}; section~\ref{sec:stringstream}.
  Another use of this header follows below.
\item The \indexterm{Boost} library has a \indextermsub{lexical}{cast}.
\end{itemize}

Additionally, in \cppstandard{17} there is the
\indextermdef{charconv} header with \indexc{to_chars} and
\indexc{from_chars}. These are low level routines that do not
throw exceptions, do not allocate, and are each other's inverses. The
low level nature is for instance apparent in the fact that they work
on character buffers (not null-terminated). Thus, they can be used to
build more sophisticated tools on top of.

\Level 2 {Converting from string}

The \indexc{stringstream} object can be used to convert strings to numbers:
\begin{enumerate}
\item Initialize the string stream with a string;
\item Use a \lstinline{cin}-like syntax to set a numerical variable from the stream.
\end{enumerate}
\verbatimsnippet{strconvert2int}

\Level 1 {Unicode}

C++ strings are essentially vectors of characters. A~character is a
single byte. Unfortunately, in these interweb days there are more
characters around than fit in one byte. In particular, there is the
\indextermdef{Unicode} standard that covers millions of
characters. The way they are rendered is by an
\indextermsub{extendible}{encoding}, in
particular~\indextermdef{UTF8}. This means that sometimes a single
`character', or more correctly \indextermdef{glyph}, takes more than
one byte.

%Example:
%\snippetwithoutput{unismile}{string}{smile}

\Level 0 {C strings}
\label{sec:cstring}
\index{C!string|(}

In C a string is essentially an array of characters. C~arrays don't
store their length, but strings do have functions that implicitly or
explicitly rely on this knowledge, so they have a terminator
character: ASCII \indexc{NULL}. C~strings are called
\emph{null-terminated}\index{string!null-terminated} for this reason.

\index{C!string|)}
