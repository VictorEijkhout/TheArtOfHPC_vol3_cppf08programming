% -*- latex -*-
%%%%%%%%%%%%%%%%%%%%%%%%%%%%%%%%%%%%%%%%%%%%%%%%%%%%%%%%%%%%%%%%
%%%%
%%%% This TeX file is part of the course
%%%% Introduction to Scientific Programming in C++/Fortran2003
%%%% copyright 2017-2021 Victor Eijkhout eijkhout@tacc.utexas.edu
%%%%
%%%% error.tex : on error handling
%%%%
%%%%%%%%%%%%%%%%%%%%%%%%%%%%%%%%%%%%%%%%%%%%%%%%%%%%%%%%%%%%%%%%

\Level 0 {General discussion}

When you're programming, making errors is close to inevitable.
%
\emph{Syntax errors}\index{syntax!error}, violations of the grammar of
the language, will be caught by the compiler, and prevent generation
of an executable. In this section we will therefore talk about
%
\emph{runtime errors}\index{runtime error}: behaviour at runtime that
is other than intended.

Here are some sources of runtime errors
\begin{description}
\item[Array indexing] Using an index outside the array bounds may give
  a runtime error:
\begin{lstlisting}
vector<float> a(10);
for (int i=0; i<=10; i++)
  a.at(i) = x; // runtime error
\end{lstlisting}
or undefined behaviour:
\begin{lstlisting}
vector<float> a(10);
for (int i=0; i<=10; i++)
  a[i] = x;
\end{lstlisting}
See further section~\ref{sec:stdvector}.
\item[Null pointers] Using an uninitialized pointer is likely to crash
  your program:
\begin{lstlisting}
Object *x;
if (false) x = new Object;
x->method();
\end{lstlisting}
\item[Numerical errors] such as division by zero will not crash your
  program, so catching them takes some care.
\end{description}

Guarding against errors.
\begin{itemize}
\item Check preconditions.
\item Catch results.
\item Check postconditions.
\end{itemize}

Error reporting:
\begin{itemize}
\item Message
\item Total abort
\item Exception
\end{itemize}

\Level 0 {Mechanisms to support error handling and debugging}

\Level 1 {Assertions}

One way catch errors before they happen, is to sprinkle your code with
assertions: statements about things that have to be true. For
instance, if a function should mathematically always return a positive
result, it may be a good idea to check for that anyway. You do this by
using the \indexcdef{assert} command, which takes a boolean, and
will stop your program if the boolean is false:

\begin{block}{Using assertions}
 \label{sl:use-assert}
 Check on valid input parameters:
\begin{lstlisting}
#include <cassert>

// this function requires x<y
// it computes something positive
float f(x,y) {
  assert( x<y );
  return /* some result */;
}
\end{lstlisting}
Check on valid results:
\begin{lstlisting}
float positive_outcome = f(x,y);
assert( positive_outcome>0 );
\end{lstlisting}
\end{block}

There is also \indexc{static_assert}, which
checks compile-time conditions only.

Since checking an assertion is a minor computation, you may want to
disable it when you run your program in production by defining the
\indexcdef{NDEBUG} macro:
\begin{lstlisting}
#define NDEBUG
\end{lstlisting}
One way of doing that is passing it as a compiler flag:
\begin{verbatim}
icpc -DNDEBUG yourprog.cxx
\end{verbatim}

\begin{slide}{Assertions to catch logic errors}
  \label{sl:cpp-assert1}
  Sanity check on things `that you just know are true':
\begin{lstlisting}
#include <cassert>
...
assert( bool expression )
\end{lstlisting}
Example:
\begin{lstlisting}
x = sin(2.81);
y = x*x;
z = y * (1-y);
assert( z>=0. and z<=1. );
\end{lstlisting}
\end{slide}

\begin{slide}{Use assertions during development}
  \label{sl:cpp-assert2}
Assertions are disabled by
\begin{lstlisting}
#define NDEBUG
\end{lstlisting}
before the include.

You can pass this as compiler flag:\\
\n{icpc -DNDEBUG yourprog.cxx}
\end{slide}

As an example of using assertions, we can consider the iteration function
of the Collatz exercise~\ref{ex:collatz}.

\begin{block}{Example}
  \label{sl:collatzassert}
  \verbatimsnippet{collatzassert}
\end{block}

\begin{remark}
  If an assertion fails, your program
  will call \indexcstd{abort}.
  This is a less elegant exit than \indexcstd{exit}.
\end{remark}

\Level 1 {Exception handling}
\label{sec:exception}

Assertions are a little crude: they terminate your program,
and the only thing you can do is debug, rewrite, and rerun.
Some errors are of a type that you could possibly recover from them.
In that case, exception are a better idea,
since these can be handled inside the program.

\begin{block}{Exception structure}
  \label{sl:except-structure}
\begin{multicols}{2}
Code with problem:
\begin{lstlisting}
if ( /* some problem */ )
  throw(5); 
  /* or: throw("error"); */
\end{lstlisting}
\vfill\columnbreak
\begin{lstlisting}
try {
  /* code that can go wrong */
} catch (...) { // literally three dots!
  /* code to deal with the problem */
}
\end{lstlisting}
\end{multicols}
\end{block}

\Level 2 {Exception catching}

During the run of your program, an error condition may occur,
such as accessing a vector elements outside the bounds of that vector,
that will make your program stop.
You may see text on your screen
\begin{verbatim}
terminating with uncaught exception
\end{verbatim}
The operative word here is \indextermdef{exception}:
an exceptional situation that caused the normal program flow
to have been interrupted. We say that your program
\emph{throws} an \emph{exception}\index{exception!throwing}.

\snippetwithoutput{xoutbounds}{except}{boundthrow}

Now you know that there is an error in your program, but
you don't know where it occurs.
You can find out, but trying to
\emph{catch the exception}\index{exception!catching}.

\snippetwithoutput{xcatchbounds}{except}{catchbounds}

\Level 2 {Popular exceptions}

\begin{itemize}
\item \indexcstd{out_of_range}: usually caused by using \lstinline{at}
  with an invalid index.
\end{itemize}

\Level 2 {Throw your own exceptions}
\label{sec:except-throw}

\begin{block}{Exception throwing}
  \label{sl:exception-throw}
  \emph{Throwing} an \emph{exception} is one way of signalling an error or
  unexpected behaviour:
\begin{lstlisting}
void do_something() {
  if ( oops )
    throw(5);
}
\end{lstlisting}
\end{block}

\begin{block}{Catching an exception}
  \label{sl:exception-catch}
  It now becomes possible to detect this unexpected behaviour by
  \emph{catching}\index{exception!catch}
  the exception:
\begin{lstlisting}
try {
  do_something();
} catch (int i) {
  cout << "doing something failed: error=" << i << endl;
}
\end{lstlisting}
\end{block}

If you're doing the prime numbers project, you can now
do exercise~\ref{ex:primegenbreak}.

You can throw integers to indicate an error code, a string with an
actual error message. You could even make an error class:

\begin{block}{Exception classes}
  \label{sl:exception-class}
\begin{lstlisting}
class MyError {
public :
  int error_no; string error_msg;
  MyError( int i,string msg )
  : error_no(i),error_msg(msg) {};
}

throw( MyError(27,"oops");

try {
  // something
} catch ( MyError &m ) {
  cout << "My error with code=" << m.error_no
    << " msg=" << m.error_msg << endl;
}
\end{lstlisting}
You can use exception inheritance!
\end{block}

\begin{block}{Multiple catches}
  \label{sl:exception-catches}
  You can multiple \n{catch} statements to catch different types of
  errors:
\begin{lstlisting}
try {
  // something
} catch ( int i ) {
  // handle int exception
} catch ( std::string c ) {
  // handle string exception
}
\end{lstlisting}
\end{block}

\begin{block}{Catch any exception}
  \label{sl:exception-catchall}
  Catch exceptions without specifying the type:
\begin{lstlisting}
try {
  // something
} catch ( ... ) { // literally: three dots
  cout << "Something went wrong!" << endl;
}
\end{lstlisting}
\end{block}

\begin{exercise}
  \label{ex:throw-negroot}
  Define the function \[ f(x)=x^3-19x^2+79x+100 \] and evaluate
  $\sqrt{f(i)}$ for the integers $i=0\ldots20$.
  \begin{itemize}
  \item First write the program naively, and print out the root. Where is
    $f(i)$ negative? What does your program print?
  \item You see that floating point errors such as the root of a negative number
    do not make your program crash or
    something like that. Alter your program to throw an exception if
    $f(i)$ is negative, catch the exception, and print an error
    message.
  \item Alter your program to test the output of the \n{sqrt} call,
    rather than its input.
    Use the function \indexc{isnan}
\begin{lstlisting}
#include <cfenv>
using std::isnan;  
\end{lstlisting}
    and again throw an exception.
  \end{itemize}
\end{exercise}

\begin{block}{Exceptions in constructors}
  \label{sl:except-construct}
  A \indexterm{function try block} will catch exceptions,
    including in \indextermsub{member}{initializer} lists of constructors.
\begin{lstlisting}
f::f( int i ) 
  try : fbase(i) {
    // constructor body
  }
  catch (...) { // handle exception
  }
\end{lstlisting}
\end{block}

\begin{block}{More about exceptions}
  \label{sl:exception-more}
  \begin{itemize}
  \item Functions can define what exceptions they throw: 
\begin{lstlisting}
void func() throw( MyError, std::string );
void funk() throw();
\end{lstlisting}
\item Predefined exceptions: \indexc{bad_alloc},
  \indexc{bad_exception}, etc.
\item An exception handler can throw an exception; to rethrow the same
  exception use `\n{throw;}' without arguments.
\item Exceptions delete all stack data, but not \n{new} data. Also,
  destructors are called; section~\ref{sec:destructor}.
\item There is an implicit \n{try/except} block around your
  main. You can replace the handler for that. See the
  \indexc{exception} header file.
\item Keyword \indexc{noexcept}:
\begin{lstlisting}
void f() noexcept { ... };
\end{lstlisting}
\item There is no exception thrown when dereferencing a \n{nullptr}.
  \end{itemize}
\end{block}

\Level 1 {`Where does this error come from'}

The \ac{CPP} defines two macros, \indexc{__FILE__} and
\indexc{__LINE__} that give you respectively the current file name
and the current line number. You can use these to generate pretty
error messages such as
\begin{verbatim}
Overflow occurred in line 25 of file numerics.cxx
\end{verbatim}

The \cppstandard{20} standard will offer
\texttt{std::}\indexc{source_location} as a mechanism instead.

\Level 1 {Legacy mechanisms}

The traditional approach to error checking is for each routine to
return an integer parameter that indicates success or absence
thereof. 
Problems with this approach arise if it's used inconsistently, for
instance by a user forgetting to heed the return codes of a library.
Also, it requires that every level of the function calling hierarchy
needs to check return codes.

The \indexterm{PETSc} library uses this mechanism
consistently throughout, and to great effect.

Exceptions are a better mechanism, since
\begin{itemize}
\item they can not be ignored, and
\item they do not require handling on the levels of the calling
  hierarchy between where the exception is thrown and where it is caught.
\end{itemize}
And then there is the fact that memory management is automatic with
exceptions.

\Level 1 {Legacy C mechanisms}

The \indexc{errno} variable and the
\indexc{setjmp}~/ \indexc{longjmp} functions should not be
used. These functions for instance do not the memory management
advantages of exceptions.

\Level 0 {Tools}

Despite all your careful programming, your code may still compute the
wrong result or crash with strange errors. There are two tools that
may then be of assistance:
\begin{itemize}
\item \indextermtt{gdb} is the GNU interactive
  \indexterm{debugger}. With it, you can run your code step-by-step,
  inspecting variables along way, and detecting various conditions. It
  also allows you to inspect variables after your code throws an
  error.
\item \indextermtt{valgrind} is a memory testing tool. It can detect
  memory leaks
  (see section~\ref{sec:leak}),
  as well as the use of uninitialized data.
\end{itemize}
