% -*- latex -*-
%%%%%%%%%%%%%%%%%%%%%%%%%%%%%%%%%%%%%%%%%%%%%%%%%%%%%%%%%%%%%%%%
%%%%
%%%% This TeX file is part of the course
%%%% Introduction to Scientific Programming in C++/Fortran2003
%%%% copyright 2018 Victor Eijkhout eijkhout@tacc.utexas.edu
%%%%
%%%% climate.tex : climate change analysis
%%%%
%%%%%%%%%%%%%%%%%%%%%%%%%%%%%%%%%%%%%%%%%%%%%%%%%%%%%%%%%%%%%%%%

\begin{quotation}
  \noindent
  \textsl{The climate has changed and it is always changing.}
  \hfill\break\hbox{}\hfill Raj Shah, White House Principal Deputy Press Secretary\par
\end{quotation}

The statement that climate always changes is far from a rigorous
scientific claim.  We can attach a meaning to it, if we interpret it
as a statement about the statistical behaviour of the climate, in this
case as measured by average global temperature.  In this project you
will work with real temperature data, and do some simple analysis on
it. (The inspiration for this project came
from~\cite{ManRestrepo:climatechange}.)

Ideally, we would use data sets from various measuring stations around
the world. Fortran is then a great language because of its array
operations (see chapter~\textbookref{ch:arrayf}): you can process all
independent measurements in a single line. To keep things simple we
will use a single data file here that contains data for each month in
a time period 1880-2018. We will then use the individual months as
`pretend' independent measurements.

\Level 0 {Reading the data}

In the repository you find two text files
\begin{verbatim}
GLB.Ts+dSST.txt         GLB.Ts.txt
\end{verbatim}
that contain temperature deviations from the 1951--1980 average.
Deviations are given for each month of each year 1880--2018. These
data files and more can be found at
\url{https://data.giss.nasa.gov/gistemp/}.

\begin{exercise}
  Start by making a listing of the available years, and an array
  \n{monthly_deviation} of size $12\times\mathrm{nyears}$, where
  $\mathrm{nyears}$ is the number of full years in the file.
  Use formats and array notation.

  The text files contain lines that do not concern you. Do you filter
  them out in your program, or are you using a shell script? Hint: a
  judicious use of \n{grep} will make the Fortran code much easier.
\end{exercise}

\begin{comment}
\end{comment}

\begin{comment}
\begin{verbatim}
DATAFILE = GLB.Ts.txt
run_annual : annual
	@(    cat ${DATAFILE} | grep '^[12]' | grep -v 2018 | wc -l \
	   && cat ${DATAFILE} | grep '^[12]' | grep -v 2018 \
	 ) | ./annual
\end{verbatim}
\end{comment}

\Level 0 {Statistical hypothesis}

We assume that mr Shah was really saying that climate has a
`stationary distribution', meaning that highs and lows have a
probability distribution that is independent of time.
This means that in $n$ data points, each point has a chance
of~$1/n$ to be a record high. Since over $n+1$ years each year has a
chance of $1/(n+1)$, the $n+1$st year has a chance $1/(n+1)$ of being a
record.

We conclude that, as a function
of~$n$, the chance of a record high (or low, but let's stick with
highs) goes down as~$1/n$, and that the gap between successive highs
is approximately a linear function of the year\footnote {Technically,
  we are dealing with a uniform distribution of temperatures, which
  makes the maxima and minima have a beta-distribution.}.

This is something we can test.

\begin{exercise}
  Make an array \n{previous_record} of the same shape as
  \n{monthly_deviation}. This array records (for each month, which,
  remember, we treat like independent measurements) whether
  that year was a record, or, if not, when the previous record
  occurred:
  \[ \mathrm{PrevRec}(m,y) = 
  \begin{cases}
    y&\hbox{if
      $\mathrm{MonDev}(m,y) = \max_{m'} \bigl( \mathrm{MonDev}(m',y)\bigr)$
    }\\
    y'&\hbox{if
      $\mathrm{MonDev}(m,y) < \mathrm{MonDev}(m,y')$ }\\
      &\hbox{and
      $\mathrm{MonDev}(m,y') = \max_{m''<m'}\bigl( \mathrm{MonDev}(m'',y)\bigr)$
    }\\
  \end{cases}
  \]

  Again, use array notation. This is also a great place to use the
  \n{Where} clause.
\end{exercise}

\begin{exercise}
  Now take each month, and find the gaps between records. This gives
  you two arrays: \n{gapyears} for the years where a gap between
  record highs starts, and \n{gapsizes} for the length of that gap.

  This function, since it is applied individually to each month, uses
  no array notation.
\end{exercise}

The hypothesis is now that the gapsizes are a linear function of the
year, for instance measured as distance from the starting year. Of
course they are not exactly a linear function, but maybe we can fit a
linear function through it by \indexterm{linear regression}.

\begin{exercise}
  Copy the code from
  \url{http://www.aip.de/groups/soe/local/numres/bookfpdf/f15-2.pdf} and
  adapt for our purposes: find the best fit for the slope and
  intercept for a linear function describing the gaps between records.
\end{exercise}

You'll find that the gaps are decidedly not linearly increasing.
So is this negative result the end of the story, or can we do more?

\begin{exercise}
  Can you turn this exercise into a test of global warming? Can you
  interpret the deviations as the sum of a yearly increase in temperature
  plus a stationary distribution, rather than a stationary
  distribution by itself?
\end{exercise}
