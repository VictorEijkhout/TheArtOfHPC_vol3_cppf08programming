% -*- latex -*-
%%%%%%%%%%%%%%%%%%%%%%%%%%%%%%%%%%%%%%%%%%%%%%%%%%%%%%%%%%%%%%%%
%%%%
%%%% This TeX file is part of the course
%%%% Introduction to Scientific Programming in C++11/Fortran2003
%%%% copyright 2017-2023 Victor Eijkhout eijkhout@tacc.utexas.edu
%%%%
%%%% turtles-abstract.tex : abstract for the TSP project
%%%%
%%%%%%%%%%%%%%%%%%%%%%%%%%%%%%%%%%%%%%%%%%%%%%%%%%%%%%%%%%%%%%%%

Plastic floating in the ocean is more than just unsightly:
it endangers life in the ocean, such as fish and turtles.
While society is slowly moving away from using too much
plastic, single-use or otherwise,
some people think about cleaning up the ocean.

This project walks a student through a
coding a `cellular automaton' simulation
of turtles, plastic, and cleanup ships in the ocean.
This can also be considered a `predator-prey' model:
plastic `preys on' turtles, ships 'prey on' plastic.
With the full implementation, students can explore some scenarios:
how fast does garbage get cleanup, is there a danger of turtles dying out,
et cetera.

The project description explicitly targets an object-oriented formulation,
but coding in a regular procedural language is possible with
a little more work.

\newpage

\begin{tabular}{|l|p{5in}|}
  \hline
  Summary&Model a predator-prey scenario through a cellular automaton.
  \\
  Topics&Arrays, randomization, animation.
  \\
  Audience&Undergraduate or AP high school
  \\
  Difficulty&Moderate to high, depending on programming sophistication employed.
  \\
  Strengths&Visually appealing, opportunity for experimentation
  \\
  Weaknesses&Scenarios modelled may not always be realistic.
  \\
  Dependencies&No specific background in CS is needed.
  \\
  Variants&Introduce other predator or prey species; model the effects of wiping
  out one species on the ecological balance.
  \\
  \hline
\end{tabular}
