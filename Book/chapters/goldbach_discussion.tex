% -*- latex -*-
%%%%%%%%%%%%%%%%%%%%%%%%%%%%%%%%%%%%%%%%%%%%%%%%%%%%%%%%%%%%%%%%
%%%%
%%%% This TeX file is part of the course
%%%% Introduction to Scientific Programming in C++/Fortran2003
%%%% copyright 2017-2021 Victor Eijkhout eijkhout@tacc.utexas.edu
%%%%
%%%% goldbach_discussion.tex : discussion of the goldbach homework
%%%%
%%%%%%%%%%%%%%%%%%%%%%%%%%%%%%%%%%%%%%%%%%%%%%%%%%%%%%%%%%%%%%%%

\heading{best solution}

\begin{lstlisting}
  int main() {
    cout<< "Enter number of upper bound for Goldbach Conjecture: ";
    int evens;
    cin >> evens;
    for (int n = 4; n <= evens; n = n + 2) {
      PrimeGenerator sequence(0,2);
      while(true) {
        int prime1 = sequence.nextPrime();
        int difference = n - prime1;
        if (sequence.isPrime(difference)) {
          cout << n << "=" << prime1 << "+" << difference<<endl;
          break;
        }
      }
    }
  }
\end{lstlisting}

Roper nesting takes care of everything

\heading{unnecessary use of reset}

\begin{lstlisting}
  int main(){
    int even_numbers_bound;
    PrimeGenerator sequence = {0,1} ;
    cout << "Enter the bound for even numbers:" << endl;
    cin >> even_numbers_bound;
    for ( int e = 4; e <= even_numbers_bound; e += 2 ){
      while ( sequence.last_prime_number() <= e){
        int p = sequence.nextprime();
        if ( is_prime ( e - p ) and  p < e){
          cout << e << " = " << p << " + " << e - p << endl;
          break;
        };
      };
      sequence.reset();
    };
    return 0;
  }
\end{lstlisting}

This is what you get if you make the generator too global.

\heading{Too many generators}

\begin{lstlisting}
  int main() {
    cout << "Enter Upper Bound: " << endl;
    int bound;
    cin >> bound;
    for (int num = 4; num <= bound; num = num+2){
      int last = 2;
      Primegen seq (last);
      int q = num - seq.nextprime();
      while (is_prime(q) == false or q == 1){
        last++;
        Primegen seq (last);
        last = seq.nextprime();
        q = num - last;}
      if (q > last){
        int hold = q;
        q = last;
        last = hold;}
      cout << num << " = " <<  q << " + " << last << endl;}
  }
\end{lstlisting}

Unnecessary.

\heading{Another clean solution}

\begin{lstlisting}
    int main(){
      int lastEvenNumber;
      //read the upper bound
      cout << "Until which Even number you want to go?" << endl;
      cin>> lastEvenNumber;
      //this loop scans all the even numbers (e) up to the bound
      for(int e=4; e<= lastEvenNumber; e +=2){
        bool NoMatch = true;
        primegenerator sequence;//definition
        //iterate until the difference q is a prime number
        while (NoMatch){
          int p = sequence.nextPrime();//sequence of primes, generate next one
          //condition: if q is prime
          if (sequence.isPrime(e-p)){
            cout << e << "=" << p << "+" << (e-p) <<endl;
            NoMatch = false;
          }
        }
      }
    }
\end{lstlisting}

\heading{Some code duplication}

\begin{lstlisting}
  int main() {
    cout << "Upper bound up to which you want to verify Goldbach's conjecture: " << endl;
    int supremum; cin >> supremum;

    for (int var = 4; var <= supremum; var+=2) {
      primegenerator sequence;
      int number = sequence.nextprime();
      while(!is_prime(var-number)){number = sequence.nextprime();}
      cout << var << " is equal to " << number << " + " << var - number << endl;
    }
    return 0;
  }
\end{lstlisting}

Otherwise good.

\heading{Doubtful}

\begin{lstlisting}
  int main(){
    primegenerator Goldbach_conjecture;
    int up_bound;
    cout << "Input the upper bound of even numbers to be tested:" << endl;
    cin >> up_bound;
    for (int current_even_number=4; current_even_number <= up_bound; current_even_number+=2){
      Goldbach_conjecture.sum_of_two_primes_equal(current_even_number);
    }
    return 0;
  }


  where:

  class primegenerator{

    [....]
    void sum_of_two_primes_equal(int p_plus_q){
      p_is_prime = false;
      q_is_prime = false;
      for(p = 2; p < p_plus_q-1 && !(p_is_prime && q_is_prime); p++){
        q = p_plus_q - p;
        p_is_prime = primetest(p);
        q_is_prime = primetest(q);
        if(p_is_prime && q_is_prime){
          cout << p_plus_q << "=" << p << "+" << q << endl;
        }
      }
    }
\end{lstlisting}
