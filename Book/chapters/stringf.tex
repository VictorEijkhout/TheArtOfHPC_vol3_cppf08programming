% -*- latex -*-
%%%%%%%%%%%%%%%%%%%%%%%%%%%%%%%%%%%%%%%%%%%%%%%%%%%%%%%%%%%%%%%%
%%%%
%%%% This TeX file is part of the course
%%%% Introduction to Scientific Programming in C++/Fortran2003
%%%% copyright 2017--2021 Victor Eijkhout eijkhout@tacc.utexas.edu
%%%%
%%%% stringf.tex : string handling in Fortran
%%%%
%%%%%%%%%%%%%%%%%%%%%%%%%%%%%%%%%%%%%%%%%%%%%%%%%%%%%%%%%%%%%%%%

\Level 0 {String denotations}

A string can be enclosed in single or double quotes. That makes it
easier to have the other type in the string.

\verbatimsnippet{fquotes}
%\snippetwithoutput{fquotes}{stringf}{quote}

\Level 0 {Characters}

The datatype \indexf{Character} is used
both for characters and strings.
Therefore, see next section.

\Level 0 {Strings}
\label{sec:f-string-ops}

The length of a Fortran string is specified with the
\indexf{len} keyword when the string is created:

\begin{block}{Create string of length}
  \label{sl:f-char-string}
\begin{lstlisting}
character(len=50) :: mystring
mystring = "short string"
\end{lstlisting}
\end{block}

The \indexf{len} function also gives the length of the string,
but note that that is the length with which
it was allocated, not how much non-blank content you put in it.

\begin{block}{String length}
  \label{sl:f-string-length}
  String length, with~/ without trimming.
  \snippetwithoutput{fstrlen}{stringf}{strlen}
\end{block}

To get the more intuitive length of a string, that is, the location of
the last non-blank character, you need to \indexf{trim} the string.

%% Intrinsic functions: \indexf{LEN}\n{(string)},
%% \indexf{INDEX}\n{(substring,string)},
%% \indexf{CHAR}\n{(int)}, \indexf{ICHAR}\n{(char)},
%% \indexf{TRIM}\n{(string)}

Concatenation is done with a double slash:
\begin{block}{String concatenation}
  \label{sl:f-string-concat}
  \snippetwithoutput{fconcat}{stringf}{concat}
\end{block}

\Level 0 {Conversions}

Sometimes we want to convert between the string \n{123} and the number~$123$.
Let's start easy, by looking at characters and their \indexterm{ascii} codes.

\Level 1 {Character conversions}
\label{sec:f-ascci-char}

Given an integer, \indexf{Char} gives the character with that ascii code.
That can be a printable or an unprintable character:
\begin{block}{Conversion char to ascii}
  \label{sl:f-to-ascii}
  \snippetwithoutput{fascii}{stringf}{ascii}
  Note the last one!
\end{block}

In the other direction, \indexf{Iachar} gives the ascii code of a character.
\begin{block}{Ascii code of a character}
  \label{sl:f-ascii-code}
\begin{lstlisting}
character :: char
integer :: code

char = "x"
code = iachar(char)
print *,char," has code",code  
\end{lstlisting}
\end{block}

\begin{remark}
  There is also a function \indexf{Ichar}, but it returns the code
  in the native character set. In rare cases this can be something
  other than ascii.
\end{remark}

\begin{exercise}
  \label{ex:f-test-lower}
  Write a test to see if a character is lowercase:
  \snippetwithoutput{flowerp}{stringf}{lower}
  Similarly, write a test \lstinline{isdigit}.
\end{exercise}

\Level 1 {String conversions}

Converting between a number and a string relies on concept
from the  I/O chapter (chapter~\ref{ch:iof});
see section~\ref{sec:string-conversion}.

\Level 0 {Further notes}

In addition to the character definition with the \lstinline{len} specification,
there is
\begin{lstlisting}
character*80 :: str
character,dimension(80) :: str
character :: str(80
\end{lstlisting}
These should not be used.
