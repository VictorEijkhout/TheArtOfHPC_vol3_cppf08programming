% -*- latex -*-
%%%%%%%%%%%%%%%%%%%%%%%%%%%%%%%%%%%%%%%%%%%%%%%%%%%%%%%%%%%%%%%%
%%%%
%%%% This TeX file is part of the course
%%%% Introduction to Scientific Programming in C++/Fortran2003
%%%% copyright 2017-9 Victor Eijkhout eijkhout@tacc.utexas.edu
%%%%
%%%% loopf.tex : looping constructs in Fortran
%%%%
%%%%%%%%%%%%%%%%%%%%%%%%%%%%%%%%%%%%%%%%%%%%%%%%%%%%%%%%%%%%%%%%

\Level 0 {Loop types}

Fortran has the usual indexed and `while' loops. There are variants of the
basic loop, and both use the \indexf{do} keyword. The simplest loop has
a loop variable, an upper bound, and a lower bound.

\begin{block}{Indexed Do loops}
  \label{sl:doloop}
\begin{lstlisting}
integer :: i

do i=1,10
  ! code with i
end do
\end{lstlisting}

You can include a step size (which can be negative) as a third parameter:
\begin{multicols}{2}
By steps of~3:
\begin{lstlisting}
do i=1,10,3
  ! code with i
end do
\end{lstlisting}
\columnbreak
Counting down:
\begin{lstlisting}
do i=10,1,-1
  ! code with i
end do
\end{lstlisting}
\end{multicols}
\end{block}

The loop variable is defined outside the loop, so it will have a value
after the loop terminates.

%% \begin{block}{Non-integer loop variables}
%%   \label{sl:floopr}
%%   Used to be allowed-but-dangerous\\
%%   Now deleted language feature:
%% \snippetwithoutput{fdoreal}{loopf}{loopr}
%% \end{block}

\begin{block}{Semantic fine points}
  \label{sl:floop-semantics}
  \begin{itemize}
  \item Fortran loops determine the iteration count before execution;
    a~loop will run that many iterations, unless you \lstinline{Exit}.
  \item You are not allowed to alter the iteration variable.
  \item Non-integer loop variables used to be allowed, no longer.
  \end{itemize}
\end{block}

\begin{block}{While loop}
  \label{sl:whilef}
  The while loop has a pre-test:
\begin{lstlisting}
do while (i<1000)
  print *,i
  i = i*2
end do
\end{lstlisting}
\end{block}

%% You can label loops with a \indextermdef{construct name},
%% which improves readability, but see also below.
%% \begin{lstlisting}
%% outer: do i=1,10
%%   inner: do j=1,10
%%     ! something
%%   end do inner
%% end do outer
%% \end{lstlisting}

\Level 0 {Interruptions of the control flow}

For indeterminate looping, you can use the \indexf{while} test,
or leave out the loop parameter altogether.
In that case you need the \indextermtt{exit} statement to stop the iteration.

\begin{block}{Exit and cycle}
  \label{sl:loopexit}
  Loop without counter or while test:
\begin{lstlisting}
do
  call random_number(x)
  if (x>.9) exit
  print *,"Nine out of ten exes agree"
end do
\end{lstlisting}
 Compare to \n{break} in C++.

Skip rest of current iteration:
\begin{lstlisting}
do i=1,100
  if (isprime(i)) cycle
  ! do something with non-prime
end do
\end{lstlisting}
Compare to \n{continue} in C++.
\end{block}

\begin{block}{Labeled loops}
  \label{sl:looplabel}
  You can label loops\\
  useful with \lstinline{exit} statement:
  \verbatimsnippet{flooplabel}
\end{block}

The label needs to be on the same line as the \indexf{do}, and if you use a
label, you need to mention it on the \indexf{end do} line.

Cycle and exit can apply to multiple levels, if the do-statements are
labeled.

\begin{lstlisting}
outer : do i = 1,10
inner : do j = 1,10
    if (i+j>15) exit outer
    if (i==j) cycle inner
end do inner
end do outer
\end{lstlisting}

\Level 0 {Implied do-loops}
\label{sec:f-impdo}

There are do loops that you can write in a single line by an
expression and a loop header. In effect, such an
\indextermsub{implied}{do loop} becomes the sum of the indexed
expressions. This is useful
for I/O. For instance, iterate a simple expression:

\begin{block}{Implied do loops}
  \label{sl:implieddo}
  If you loop over a print statement, each print statement is on a new line;\\
  use an implied loop to print on one line.
\begin{lstlisting}
Print *,(2*i,i=1,20)
\end{lstlisting}
You can iterate multiple expressions:
\begin{lstlisting}
Print *,(2*i,2*i+1,i=1,20)
\end{lstlisting}
These loops can be nested:
\begin{lstlisting}
Print *,( (i*j,i=1,20), j=1,20 )
\end{lstlisting}
Also useful for \lstinline+Read+.
\end{block}

This construct is especially useful for printing arrays.

\begin{exercise}
  \label{ex:impl-triangle}
  Use the implied do-loop mechanism to print a triangle:
\begin{lstlisting}
  1
  2 2
  3 3 3
  4 4 4 4
\end{lstlisting}
  up to a number that is input.
\end{exercise}

\Level 0 {Obsolete loop statements}

Old versions of Fortran had other forms of the \indexf{o} statement,
which you may still encounter in codes.
As of \fstandard{2018}, \indexf{do} loops have to end
in \lstinline{end do} or \lstinline{continue}.
Shared termination is likewise a deleted feature.

Fortran has a \indexf{goto} statement.
While this was needed in the 1950 and 60s, nowadays
it is considered bad programming practice.
Most of its traditional uses can be covered
with the \lstinline{cycle} and \lstinline{exit} statements.
The \indexf{continue} statement,
usually used as the target of a \lstinline{goto},
is similarly rarely used anymore.

\Level 0 {Review questions}

\begin{exercise}
  \label{ex:floop-inf}
  What is the output of:
\begin{lstlisting}
do i=1,11,3
  print *,i
end do
\end{lstlisting}
What is the output of:
\begin{lstlisting}
do i=1,3,11
  print *,i
end do
\end{lstlisting}
\end{exercise}
