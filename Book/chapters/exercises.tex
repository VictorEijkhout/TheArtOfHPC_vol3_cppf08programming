\Level 0 {Practice}

The website
\url{http://www.codeforwin.in/2015/05/if-else-programming-practice.html}
lists the following exercises for conditional:

\begin{enumerate}
\item Write a C program to find maximum between two numbers.
\item  Write a C program to find maximum between three numbers.
\item  Write a C program to check whether a number is even or odd.
\item  Write a C program to check whether a year is leap year or not.
\item  Write a C program to check whether a number is negative, positive or
  zero.
\item  Write a C program to check whether a number is divisible by 5 and 11
  or not.
\item  Write a C program to count total number of notes in given amount.
\item  Write a C program to check whether a character is alphabet or not.
\item  Write a C program to input any alphabet and check whether it is
  vowel or consonant.
\item  Write a C program to input any character and check whether it is
  alphabet, digit or special character.
\item  Write a C program to check whether a character is uppercase or
  lowercase alphabet.
\item  Write a C program to input week number and print week day.
\item  Write a C program to input month number and print number of days in
  that month.
\item  Write a C program to input angles of a triangle and check whether
  triangle is valid or not.
\item  Write a C program to input all sides of a triangle and check whether
  triangle is valid or not.
\item  Write a C program to check whether the triangle is equilateral,
  isosceles or scalene triangle.
\item  Write a C program to find all roots of a quadratic equation.
\item  Write a C program to calculate profit or loss.
\end{enumerate}

\section{cplusplus}

\url{http://www.cplusplus.com/forum/articles/12974/}

Dungeon crawl.

\section{world best learning center}

\url{http://www.worldbestlearningcenter.com/index_files/cpp-tutorial-variables_datatypes_exercises.htm}




\endinput

\section{Hello world}

\begin{landscape}
  \begin{tabular}{|l|l|p{3in}|p{3in}|}
    \hline
    \multicolumn{4}{|c|}{Prime number exercises}\\
    \hline
    Topic&filename&description&notes\\
    \hline
    \hline
    Input/Ouput&hello&basic `hello world'&\n{cout}\\
    &hello34&hello and computation&arithmetic\\
    &hellopi&hello and computation&\\
    &helloin&looped `hello world' with user input&looping\\
    &helloinwhat&read line and count, loop&\\
  \end{tabular}
\end{landscape}

\section{Prime numbers}

Exercises for finding prime numbers; this exercises elementary
arithmetic, for and while loops, and hiding state in classes.

\begin{landscape}
  \begin{tabular}{|l|l|p{3in}|p{3in}|}
    \hline
    \multicolumn{4}{|c|}{Prime number exercises}\\
    \hline
    Topic&filename&description&notes\\
    \hline
    \hline
    Arithmetic&divisiontest
    &read two numbers, report whether one divides the other
    &requires arithmetic and I/O\\
    \hline
    Loops&primetest&read a number, loop to find divisors
    &for loop with break, do followup with stride~2\\
    &primes&find set number of primes&while loop\\
    \hline
    Functions&primetestfunction&isolate the divisor-finding loop
    into a function with \n{bool} result
    &functions; based on \n{primetest}\\
    \hline
    &primesbyfunction&primes finding loop, the counter is in the main
    program&
    based on \n{primes} and \n{primetestfunction}; maybe introduce
    \n{.h} files and separate compilation?\\
    \hline      
    &primesbyglobal&primes finding loop, the counter is in the file
    &followup: hide the counter in an include file\\
    \hline
    Classes&primesbyclass&hide the global counter in a class; use
    class methods
    &prepare with class examples; followup: hide class data with accessors\\
    \hline
    &primepairs&use of class methods&infinite loops and break\\
    &goldbach&test Goldbach conjecture&class with reset of generator\\
    \hline
  \end{tabular}
\end{landscape}

\section{Recursion}


\section{Geometry}

Interpolation and such.

\begin{landscape}
  \begin{tabular}{|l|l|p{3in}|p{3in}|}
    \hline
    \multicolumn{4}{|c|}{Prime number exercises}\\
    \hline
    Topic&filename&description&notes\\
    \hline
    \hline
    Objects&point&point object with function for distance to other
    point
    &\n{math.h}\\
    \hline
    &linear&object for linear function: object is not actually
    something physical
    &reference\\
    \hline
    &rectangle&define rectangle from two points or point and sizes
    &polymorphic constructor\\
    \hline
    &square&define square as special case of rectangle
    &inheritance\\
    \hline
  \end{tabular}
\end{landscape}

\begin{landscape}
  \begin{tabular}{|l|l|p{3in}|p{3in}|}
    \hline
    \multicolumn{4}{|c|}{Const exercises}\\
    \hline
    Topic&filename&description&notes\\
    \hline
    \hline
    &vector&&\\
    &privatepublic&&\\
    \hline
  \end{tabular}
\end{landscape}
