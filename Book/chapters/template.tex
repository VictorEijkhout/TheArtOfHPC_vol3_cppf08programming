% -*- latex -*-
%%%%%%%%%%%%%%%%%%%%%%%%%%%%%%%%%%%%%%%%%%%%%%%%%%%%%%%%%%%%%%%%
%%%%
%%%% This TeX file is part of the course
%%%% Introduction to Scientific Programming in C++/Fortran2003
%%%% copyright 2017-2023 Victor Eijkhout eijkhout@tacc.utexas.edu
%%%%
%%%% template.tex : on templates
%%%%
%%%%%%%%%%%%%%%%%%%%%%%%%%%%%%%%%%%%%%%%%%%%%%%%%%%%%%%%%%%%%%%%

You have seen in this course how
objects of type \lstinline+vector<string>+ and \lstinline+vector<float>+
are very similar:
have methods with the same names,
and these methods behave largely the same.
This angle-bracket notation is called `templating' and the
\lstinline+string+ or \lstinline+float+ is called the
\indextermbus{template}{parameter}.

If you go digging into the source of the C++ library,
you will find that, somewhere,
there is just a single definition of the \lstinline+vector+ class,
but with a new notation it gets the type \lstinline+string+ or \lstinline+float+
as template parameter.

To be precise, the templated class (or function)
is preceded by a line
\begin{lstlisting}
template< typename T >
class vector {
  ...
};
\end{lstlisting}

This mechanism is available for your own functions and classes too.

\begin{block}{Templated type name}
  \label{sl:template-gen}
  If you have multiple routines that do `the same' for multiple types,
  you want the type name to be a variable,
  a \indextermbusdef{template}{parameter}.
  Syntax:
\begin{lstlisting}
template <typename yourtypevariable>
// ... stuff with yourtypevariable ...
\end{lstlisting}
\end{block}

Historically \lstinline{typename} was \lstinline{class} but that's confusing.

\Level 0 {Templated functions}

We will start by taking a brief look at
templated functions.

\begin{block}{Example: function}
  \label{sl:template-fun}
  Definition:
\begin{lstlisting}
template<typename T>
void function(T var) { cout << var << end; }
\end{lstlisting}
Usage:
\begin{lstlisting}
int i; function(i);
double x; function(x);
\end{lstlisting}
and the code will behave as if you had defined \lstinline{function} twice,
once for \lstinline{int} and once for \lstinline{double}.
\end{block}

\begin{exercise}
  \label{ex:eps-template}
  Machine precision, or `machine epsilon', is sometimes defined as the
  smallest number~$\epsilon$ so that $1+\epsilon>1$ in computer
  arithmetic.

  Write a templated function \lstinline{epsilon} so that the following code
  prints out the values of the machine precision for the \lstinline{float} and
  \lstinline{double} type respectively:
  %
  \answerwithoutput{epsprint}{template}{eps}
\end{exercise}

\begin{exercise}
  If you're doing the zero-finding project,
  you can now do the exercises in section~\ref{sec:newton-template}.
\end{exercise}

\Level 0 {Templated classes}

The most common use of templates is probably to define templated
classes.
You have in fact seen this mechanism in action:

\begin{block}{Templated vector}
  \label{sl:template-vector}
  The \ac{STL} contains
  in effect
\begin{lstlisting}
template<typename T>
class vector {
private:
  T *vectordata; // internal data
public:
  T at(int i) { return vectordata[i]  };
  int size() { /* return size of data */ };
  // much more
}
\end{lstlisting}
\end{block}

Let's consider a worked out example.
We write a class \lstinline+Store+
that stored a single element of the type
of the template parameter:
%
\snippetwithoutput{storei5}{template}{example1i5}

The class definition looks pretty normal,
except that the type (\lstinline+int+ in the above example)
is parametrized:
%
\verbatimsnippet{store1class}

If we write methods that refer the templated type,
things get a little more complicated.
Let's say we want two methods \lstinline+copy+ and \lstinline+negative+
that return objects of the same templated type:
%
\snippetwithoutput{store1copy}{template}{example1f314}

The method definitions are fairly straightforward;
if you leave out the template parameter,
the \indextermbus{class}{name injection}
mechanism uses the same template value 
as for the class being defined:
%
\verbatimsnippet{store1method}

\Level 1 {Out-of-class method definitions}

If we separate the class signature and the method definitions
(for instance for separate compilation; see section~\ref{sec:class-header-file})
things get trickier.
The class signature is easy:
%
\verbatimsnippet{store2class}

The method definitions are more tricky.
Now the template parameter needs to be specified every single time
you mention the templated class,
except for the name of the constructor:
%
\verbatimsnippet{store2methods}

\Level 1 {Specific implementations}

Sometimes the template code works for a number of types (or values),
but not for all. In that case you can specify the instantiation for
specific types with empty angle brackets:
\begin{lstlisting}
template <typename T>
void f(T);
template <> 
void f(char c) { /* code with c */ };
template <>
void f(double d) { /* code with d */ };
\end{lstlisting}

\Level 1 {Templates and separate compilation}
\label{sec:templ-header}

The use of templates often makes
\emph{separate compilation}\index{templates!and separate compilation}
hard: in order to compile the
templated definitions the compiler needs to know with what types they
will be used.
For this reason, many libraries are \indextermdef{header-only}:
they have to be included in each file where they are used,
rather than compiled separately and linked.

In the common case where you can foresee with which types
a templated class will be instantiated, there is a way out.
Suppose you have a templated class and function:
\begin{lstlisting}
template <typename T>
class foo<T> {
};

template<typename T>
double f( T x );
\end{lstlisting}
and they will only be used (instantiated) with \lstinline{float} and
\lstinline{double}, then adding the following lines after the class definition
makes the file separately compilable:
\begin{lstlisting}
template class foo<float>;
template class foo<double>;

template double f(float);
template double f(double);
\end{lstlisting}

If the class is split into a header and implementation file,
these lines go in the implementation file.

\Level 0 {Example: polynomials over fields}
\label{sec:poly-template}

Any numerical application can be templated to
allow for computation in
\indextermsub{single}{precision} \lstinline{float}s,
and
\indextermsub{double}{precision} \lstinline{double}s.
However, we can often also generalize the computation
to other \indextermp{field}.
Consider as an example polynomials,
in both scalars and (square) matrices.

Let's start with a simple class for polynomials:
%
\verbatimsnippet{polydouble}
%
We store the polynomial coefficients, with the
zeroth-degree coefficient in location zero, et cetera.
The routine for evaluating a polynomial for a given input~$x$
is an implementation of \indexterm{Horner's scheme}:
\[ 5 x^2 +4x +3 \equiv ( ( 5 ) \cdot x + 4 ) \cdot x + 3 \]
(Note that the \lstinline{eval} method above
uses \indexc{rbegin}, \indexc{rend} to traverse
the coefficients in the correct order.)

For instance, the coefficients \lstinline{2,0,1} correspond
to the polynomial $2+0\cdot x+1\cdot x^2$:
%
\verbatimsnippet{polydoubleapply}

If we generalize the above to the case of matrices,
all polynomial coefficients, as well as the $x$ input and $y$ output,
are matrices.

The above code for evaluating a polynomial for a certain input
works just as well for matrices, as long as the multiplication and
addition operator are defined.
So let's say we have a class \lstinline{mat} and we have
\begin{lstlisting}
mat::mat operator+( const mat& other ) const;
void mat::operator+=( const mat& other );
mat::mat operator*( const mat& other ) const;
void mat::operator*=( const mat& other );
\end{lstlisting}

Now we redefine the \lstinline{polynomial} class,
templated over the scalar type:
%
\verbatimsnippet{polymatclass}
%
The code using polynomials stays the same, except that we
have to supply the scalar type as template parameter
whenever we create a polynomial object.
The above example of $p(x)=x^2+2$ becomes
for scalars:
%
\verbatimsnippet{temppolydouble}
%
and for matrices:
%
\verbatimsnippet{temppolymat}

You see that after the templated definition
the polynomial object is used entirely identically.

\Level 0 {Concepts}
\label{sec:cpp-concepts}

The \cppstandard{20} standard added the notion of \indexterm{concept}.

Templates can be too generic. For instance, one could write a
templated \lstinline{gcd} function
\begin{lstlisting}
template <typename T>
T gcd( T a, T b ) {
  if (b==0) return a;
  else return gcd(b,a%b);
}
\end{lstlisting}
which will work for various integral types. To prevent it being
applied to non-integral types, you can specify a \indextermdef{concept}
to the type:
\begin{lstlisting}
template<typename T>
concept bool Integral() {
  return std::is_integral<T>::value;
}
\end{lstlisting}
used as:
\begin{lstlisting}
template <typename T>
requires Integral<T>{}
T gcd( T a, T b ) { /* ... */ }
\end{lstlisting}
or
\begin{lstlisting}
template <Integral T>
T gcd( T a, T b ) { /* ... */ }
\end{lstlisting}
Abbreviated function templates:
\begin{lstlisting}
Integral auto gcd
    ( Integral auto a, Integral auto b )
  { /* ... */ }
\end{lstlisting}

\endinput

\Level 0 {Templating over non-types}

THESE EXAMPLES ARE NOT GOOD.

See:
\url{https://www.codeproject.com/Articles/257589/An-Idiots-Guide-to-Cplusplus-Templates-Part}

\begin{block}{Templating a value}
  Templating over integral types, not double.

  The templated quantity is a value:
  %
  \verbatimsnippet{itemplate}
\end{block}

\begin{exercise}
  Write a class that contains an array. The length of the array should
  be templated.
\end{exercise}
