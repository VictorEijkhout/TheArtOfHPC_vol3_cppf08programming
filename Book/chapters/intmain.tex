% -*- latex -*-
%%%%%%%%%%%%%%%%%%%%%%%%%%%%%%%%%%%%%%%%%%%%%%%%%%%%%%%%%%%%%%%%
%%%%
%%%% This TeX file is part of the course
%%%% Introduction to Scientific Programming in C++/Fortran2003
%%%% copyright 2017-2023 Victor Eijkhout eijkhout@tacc.utexas.edu
%%%%
%%%% intmain.tex : section about main program
%%%% NO LONGER USED
%%%%
%%%%%%%%%%%%%%%%%%%%%%%%%%%%%%%%%%%%%%%%%%%%%%%%%%%%%%%%%%%%%%%%

\label{sec:int-main}

The \indexc{main} program has to be of type \lstinline{int};
however, many compilers tolerate deviations from this,
for instance accepting \lstinline{void},
which is not language standard.

The arguments to main can be:
\begin{lstlisting}
int main() 
int main( int argc,char* argv[] ) 
int main( int argc,char **argv ) 
\end{lstlisting}

The \lstinline{argc/argv} variables contain the commandline
as a set of strings.
\begin{itemize}
\item \lstinline{argc} is the number of strings: the name of the program,
  and the number of space-separated arguments;
\item \lstinline{argv} contains the \indexterm{commandline arguments}
  as an array of strings.
\end{itemize}
You might be tempted to parse the commandline ad-hoc,
but there are dedicated libraries for this;
see section~\ref{sec:cxxopts}.

The returned \lstinline{int} can be specified several ways:
\begin{itemize}
\item If no \indexc{return} statement is given,
  implicitly \lstinline+return 0+ is performed.
  (This is also true in~\cstandard{99}.)
\item You can explicitly pass an integer to the operating system,
  which can then be queried in the 
  \emph{shell}\index{shell!inspect return code} as a \emph{return code}:
  \snippetwithoutput{returnone}{basic}{return}
\item For cleanliness, you can use the values
  \indexc{EXIT_SUCCESS} and \indexc{EXIT_FAILURE}
  which are defined in \indextermtt{cstdlib.h}.
\item You can also use the \indexc{exit} function:
\begin{lstlisting}
void exit(int);
\end{lstlisting}
\end{itemize}

\begin{slide}{Return statement}
  \label{sl:main-int}
  \begin{itemize}
  \item The language standard says that \lstinline{main} has to be of type
    \lstinline{int}; the \indextermbus{return}{statement} returns an int.
  \item Compilers are fairly tolerant of deviations from this.
  \item Usual interpretation: returning zero means success; anything else failure;
  \item This \indextermbus{return}{code} can be detected by the
    \emph{shell}\index{shell!inspect return code}
  \end{itemize}
  \snippetwithoutput{returnone}{basic}{return}
\end{slide}
