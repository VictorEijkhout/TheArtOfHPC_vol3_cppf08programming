%% -*- latex -*-
%%%%%%%%%%%%%%%%%%%%%%%%%%%%%%%%%%%%%%%%%%%%%%%%%%%%%%%%%%%%%%%%
%%%%
%%%% This TeX file is part of the course
%%%% Introduction to Scientific Programming in C++/Fortran2003
%%%% copyright 2017-2022 Victor Eijkhout eijkhout@tacc.utexas.edu
%%%%
%%%% interpret.tex : rpm interpreter project
%%%%
%%%%%%%%%%%%%%%%%%%%%%%%%%%%%%%%%%%%%%%%%%%%%%%%%%%%%%%%%%%%%%%%

In this set of exercises you will write a `desk calculator':
a small interactive calculator
that combines numerical and symbolic calculation.

These exercises mostly use the material of chapters
\textbookref{ch:structf}, \textbookref{ch:iof}, \textbookref{ch:stringf}.

\Level 0 {Named variables}

We start out by working with `named variables':
the \lstinline{namedvar} type associates a string with a variable:
%
\verbatimsnippet{fvardef}

A named variable has a value, and a string field that is the expression
that generated the variable.
When you create the variable, the expression can be anything.
%
\verbatimsnippet{fvarmake}

\begin{slide}{Define a named variable}
  \label{sl:f-named-var-def}
  Type definition:
  \verbatimsnippet{fvardef}
\end{slide}

Next we are going to do calculations with these type objects.
For instance, adding two objects
\begin{itemize}
\item
  adds their values, and
\item concatenates their \lstinline{expression} fields,
  giving the expression corresponding to the sum value.
\end{itemize}

Your first assignment is to write
\lstinline{varadd} and \lstinline{varmult} functions
that get the following program working
with the indicated output.
This uses string manipulation from sections
\ref{sec:f-string-ops} and~\ref{sec:string-conversion}.

\begin{exercise}
  \label{ex:f-named-var}
  The following main program should give the corresponding output:
  %
  \def\snippetcodefraction{.4}
  \def\snippetanswfraction{.6}
  %
  \snippetwithoutput{fvarcall}{structf}{varhandling} % {namedvar}
  %
  (To be clear: the two routines need to do both numeric and string
  `addition' and `multiplication'.)
  \skeleton{namedvar}  
\end{exercise}

\Level 0 {First modularization}

Let's organize the code so far by introducing modules;
see chapter~\ref{ch:modulef}.

\begin{exercise}
  \label{ex:f-named-var-mod}
  Create a module (suggested name: \lstinline{VarHandling}) and
  move the \lstinline{namedvar} type definition and
  the routines \lstinline{varadd}, \lstinline{varmult}
  into it.
\end{exercise}

\begin{exercise}
  \label{ex:f-char-mod}
  Also create a module (suggested name: \lstinline{InputHandling})
  that contains
  the routines \lstinline{islower}, \lstinline{isdigit}
  from the character exercises in chapter~\ref{ch:stringf}.
  You will also need an \lstinline{isop} routine to recognize
  arithmetic operations.
\end{exercise}

\Level 0 {Event loop and stack}

In our quest to write an interpreter, we will write an
`event loop': a loop that continually accepts single character inputs,
and processes them.
An input of \lstinline{"0"} will mean termination of the process.

\begin{exercise}
  \label{ex:f-var-event-loop}
  Write a loop that accepts character input,
  and only prints out what kind of character was encountered:
  a~lowercase character, a~digit, or a character denoting
  an arithmetic operation~\lstinline|+-*/|.

  \snippetwithoutput{finterchar}{structf}{interchar}

  Use the \lstinline{InputHandling} module introduced above.
\end{exercise}

\Level 1 {Stack}

Next, we are going to  store values in \lstinline{namedvar} types on a stack.
A \indextermdef{stack} is a data structure where new elements go on the top,
so we need to indicate with a \indextermbus{stack}{pointer}
that top element.
Equivalently, the stack pointer indicates how many elements there already are:

\begin{block}{Stack definition}
  \label{sl:f-var-stack-def}
  %\verbatimsnippet{fvardef}
  \verbatimsnippet{fvarstackstack}
\end{block}

Since we are using modules, let's keep the stack out of the main program
and put it in the appropriate module.

\begin{exercise}
  \label{ex:f-stack-in-mod}
  Add the \lstinline{stack} variable and the stack pointer
  to the \lstinline{VarHandling} module.
\end{exercise}

Since Fortran uses 1-based indexing,
a starting value of zero is correct.
For C/C++ it would have been~\n{-1}.

Next we will start implementing stack operations,
such as putting \lstinline{namedvar} objects on the stack.

\Level 1 {Stack operations}

We extend the above event loop,
which was only recognizing the input characters,
by actually incorporating actions.
That is, we repeatedly
\begin{enumerate}
\item read a character from the input;
\item \lstinline{0} causes the event loop to exit; otherwise:
\item if it is a digit, create a new \lstinline{namedvar} entry
  on the top of the stack, with that value both numerically as the
  \lstinline{value} field, and as string in the \lstinline{expression} field.

  You may be tempted to write the following in the main program:
  \verbatimsnippet{fstackpush}
  (You have already coded \lstinline{isdigit} in exercise~\ref{ex:f-test-lower}.)
  but a cleaner design uses a function call
  to a method in the \lstinline{VarHandling} module:
  \begin{multicols}{2}
    \verbatimsnippet{fstackpushcall}
    \columnbreak
    \verbatimsnippet{fstackpushdef}
  \end{multicols}
  Note that the \lstinline{stack_push} routine does not have the stack
  or stack pointer as arguments: since they are all in the same module,
  they are accessible as
  \emph{global variable}\index{variable!global!in Fortran module}.

  Finally,
\item if it is a letter indicating an operation $+$,~$-$,~$\times,/$,
  \begin{enumerate}
  \item take the two top entries from the stack, lowering the stack pointer;
  \item apply that operation to the operands; and
  \item push the result onto the stack.
  \end{enumerate}
\end{enumerate}

The auxiliary function \lstinline{stack_display} is a little tricky,
so you get that here.
This uses string formatting (section~\ref{sec:fformat})
and implied do loops (section~\ref{sec:f-impdo}):
Also, note that the \lstinline{stack} array and the \lstinline{stackpointer}
act like global variables.
%
\verbatimsnippet{fstackprintnum}

Let's add the various options to the event loop.

\begin{exercise}
  \label{ex:f-var-loop-dig-name}
  Make your event loop accept digits, creating a new entry:
  %
  \def\snippetcodefraction{.7}
  \def\snippetanswfraction{.9}
  \snippetwithoutput{fstackpushcall}{structf}{internum}
\end{exercise}

Next we integrate the operations:
if the \lstinline{input} character corresponds
to an arithmetic operator,
we call \lstinline{stack_op} with that character.
That routine in turn calls the appropriate operation
depending on what the character was.

\begin{exercise}
  \label{ex:f-var-loop-op}
  Add a clause to your event loop
  to handle characters that stand for arithmetic operations:
  %
  \def\snippetcodefraction{.7}
  \def\snippetanswfraction{.9}
  \snippetwithoutput{fstackopcall}{structf}{internumop}
  %
\end{exercise}

\Level 1 {Item duplication}

Finally, we may want to use a stack entry more than once,
so we need the functionality of
duplicating a stack entry.

For this we need to be able to refer to a stack entry,
so we add a single character label field:
the \lstinline{namedvar} type now stores
\begin{enumerate}
\item a single character id,
\item an integer value, and
\item its generating expression as string.
\end{enumerate}
%
\verbatimsnippet{fvarstackvar}

\begin{exercise}
  \label{ex:f-stack-extend-id}
  Add the \lstinline{id} field to the \lstinline{namedvar},
  and make sure your program still compiles and runs.
\end{exercise}

The event loop is now extended with an extra step.
If the input character is a lowercase letter,
it is used as the \lstinline{id} of a \lstinline{namedvar}
as follows.
\begin{itemize}
\item If there is already a stack entry with that \lstinline{id},
  it is duplicated on top of the stack;
\item otherwise, the \lstinline{id} of the stack top entry
  is set to this character.
\end{itemize}

Here is the relevant bit of the new \lstinline{stack_print} function:
%
\verbatimsnippet{fstackprint}

\begin{exercise}
  Write the missing function and its clause in the event loop:
  %
  \def\snippetcodefraction{.7}
  \def\snippetanswfraction{.9}
  \snippetwithoutput{fstackfind}{structf}{stackfind}
  %
  (What is in the \lstinline{else} part of this conditional?)
\end{exercise}

\Level 0 {Modularizing}

With the modules and the functions you have developed so far,
you have a very clean main program:
%
\verbatimsnippet{fvarmodmain}

You see that by moving the stack into the module,
neither the stack variable nor the stack pointer
are visible in the main program anymore.

%% \begin{exercise}
%%   Make the following input work:
%%   \def\snippetcodefraction{.6}
%%   \def\snippetanswfraction{.9}
%%   \snippetwithoutput{fvarmodfoo}{structf}{intermod}
%% \end{exercise}

But there is an important limitation to this design:
there is exactly one stack,
declared as a sort of global variable, accessible through a module.

%% An important note.
%% You may think that a Fortran module is similar to a C++ object,
%% but that's not true.
%% By moving the stack and its stack pointer into the module
%% you have made global variables: any main program or procedure
%% that uses the module gets access to the same stack variable.

Whether having global data is good practice is another matter.
In this case it's defensible: in a calculator app there will be
exactly one stack.

\Level 0 {Object orientation}

But maybe we do sometimes need more than one stack.
Let's bundle up the stack array and the stack pointer
in a new type:
%
\verbatimsnippet{fstackclass}

\begin{exercise}
  \label{ex:f-stack-push}
  Change the event loop so that it calls methods
  of the \lstinline{stackstruct} type,
  rather than functions that take the stack as input.

  For instance, the \lstinline{push} function is called as:
  %
  \verbatimsnippet{fstackpusho}
\end{exercise}


\Level 1 {Operator overloading}

The \lstinline{varadd} and similar arithmetic routines
use a function call for what we would like to write
as an arithmetic operation.

\begin{exercise}
  Use \indextermbus{operator}{overloading} in the \lstinline{varop} function:
  %
  \verbatimsnippet{fstackaddop}
  %
  et cetera.  
\end{exercise}
