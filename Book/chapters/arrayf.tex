% -*- latex -*-
%%%%%%%%%%%%%%%%%%%%%%%%%%%%%%%%%%%%%%%%%%%%%%%%%%%%%%%%%%%%%%%%
%%%%
%%%% This TeX file is part of the course
%%%% Introduction to Scientific Programming in C++/Fortran2003
%%%% copyright 2017-2023 Victor Eijkhout eijkhout@tacc.utexas.edu
%%%%
%%%% arrayf.tex : arrays in Fortran
%%%%
%%%%%%%%%%%%%%%%%%%%%%%%%%%%%%%%%%%%%%%%%%%%%%%%%%%%%%%%%%%%%%%%

Array handling in Fortran is similar to C++ in some ways, but there
are differences, such as that  Fortran indexing starts at~1, rather
than~0. More importantly, Fortran has better handling of
multi-dimensional arrays, and it is easier to manipulate whole arrays.

\Level 0 {Static arrays}

The preferred way for specifying an array size is:

\begin{block}{Fortran dimension}
  \label{sl:farray-dimension}
Creating arrays through \indexf{dimension}
keyword:

\begin{lstlisting}
real(8), dimension(100) :: x,y
\end{lstlisting}
One-dimensional arrays of size~100.

\begin{lstlisting}
integer,dimension(10,20) :: iarr
\end{lstlisting}
Two-dimensional array of size~$10\times 20$.

These arrays are statically defined, and only live inside their
program unit (subroutine, function, module).\\
Dynamic allocation later.
\end{block}

Such an array, with size explicitly indicated, is called a
\indextermsub{static}{array} or \indextermsub{automatic}{array}.
(See section~\ref{sec:allocatable} for
dynamic arrays.)

\begin{block}{1-based Indexing}
  \label{sl:farray-base1}
Array indexing in Fortran is 1-based by default:
\begin{lstlisting}
integer,parameter :: N=8
real(4),dimension(N) :: x
do i=1,N
  ... x(i) ...
\end{lstlisting}
Different from C/C++.

Note the use of \lstinline{parameter}:
\indexterm{compile-time constant}\\
Size needs to be known to the compiler.
\end{block}

\begin{block}{Lower bound}
  \label{sl:farray-lower}
Unlike C++, Fortran can specify the lower bound explicitly:
\begin{lstlisting}
real,dimension(-1:7) :: x
do i=-1,7
  ... x(i) ...
\end{lstlisting}

Preferred: use \indexf{lbound} and \indexf{ubound}\\
(see also \ref{sec:farray-query})
\snippetwithoutput{lubound}{arrayf}{lubound}
\end{block}

Such arrays, as in~C++, obey the scope: they disappear at the end of
the program or subprogram.

\Level 1 {Initialization}

There are various syntaxes for \indextermbus{array}{initialization},
including the use of
\emph{implicit do-loops}\index{do loop!implicit!and array initialization}:
%
\begin{block}{Array initialization}
  \label{sl:farray-init}
  Different syntaxes:
  \begin{itemize}
  \item Explicit:
    \verbatimsnippet{farrayinit1}
  \item Implicit do-loop:
    \verbatimsnippet{farrayinit2}
  \item Legacy syntax
    \verbatimsnippet{farrayinit3}
    (This is pre-\fstandard{2003}. Slashes were also used for some other deprecated constructs.)
  \end{itemize}
\end{block}

\Level 1 {Array sections}

Fortran is more sophisticated than C++ in how it can handle arrays as
a whole. For starters, you can assign one array to another:

\begin{lstlisting}
real*8, dimension(10) :: x,y
x = y
\end{lstlisting}

This obviously requires the arrays to have the same size.
You can assign subarrays,
or \indextermbus{array}{section}s\index{section|see{array, section}},
as long as they have the same shape. This uses a colon syntax.

\begin{slide}{Array sections example}
  \label{sl:farray-section}
  Use the colon notation to indicate ranges:
\begin{lstlisting}
real(4),dimension(4) :: y
real(4),dimension(5) :: x
x(1:4) = y
x(2:5) = x(1:4)
\end{lstlisting}
\end{slide}

\begin{block}{Array sections}
  \label{sl:farray-sections}
  \begin{itemize}
  \item \lstinline{A(:)} %\n{:}
    to get all indices,
  \item \lstinline{A(:n)} % \n{:n}
    to get indices up to~\n{n},
  \item \lstinline{A(n:)} % \n{n:}
    to get indices \n{n} and up.
  \item \lstinline{A(m:n)}
    indices in range \n{m,...,n}.
  \end{itemize}
\end{block}

\begin{block}{Use of sections}
  \label{sl:farray-sectionassign}
  Assignment from one section to another:
  \snippetwithoutput{fsectionassign}{arrayf}{sectionassign}
  Note: \\
  Format syntax will be discussed later:\\
    float number, 5~positions, 3~after decimal point.
\end{block}

\begin{exercise}
  \label{ex:sect-vs-loop}
  Code out the array assignment
\begin{lstlisting}
x(2:5) = x(1:4)
\end{lstlisting}
with an explicit indexed loop.
  Do you get the same output? Why? What conclusion do
  you draw about internal mechanisms used in array sections?
\end{exercise}

The above exercise illustrates a point about the
\emph{semantics of array operations}\index{array!operations, semantics of}:
an array statement behaves as if all inputs are gathered together
before any results are stored.
Conceptually, it is as if the right-hand side is assembled an copied
to some temporary locations before being written to the left-hand side.
In practice, this may require large temporary arrays
(and negatively affect performance by lessening \indexterm{locality})
so you hope that the compiler does something smarter.
However, the exercise showed that an array assignment can not trivially
be converted to a simple loop.

\begin{exercise}
  Can you formalize the sort of array statement for which a simple
  translation to a loop changes the semantics? (In compiler
  terminology this is called a \indexterm{dependence}.)
\end{exercise}

Array operations can be more sophistaced than assigning to a whole
array or a section of it.
For instance, you can use a stride:

\begin{block}{Strided sections}
  \label{sl:farray-strideassign}
  \lstinline+X(a:b:c)+ : stride~\lstinline{c}
  
  Analogous to: \lstinline{do i=a,b,c}

  Copy a contiguous array to a strided subset of another:
  %
  \snippetwithoutput{fsectionmg}{arrayf}{sectionmg}
\end{block}

You can even do arithmetic on array sections,
for instance adding them together.

\begin{exercise}
  \label{ex:farray-shift}
  Code $\forall_i\colon y_i=(x_i+x_{i+1})/2$:

  \includegraphics[scale=.3]{yx_average}

  \begin{itemize}
  \item First with a do loop; then
  \item in a single array assignment statement by using sections.
  \end{itemize}
  Initialize the array \n{x} with values that allow you to check the
  correctness of your code.
\end{exercise}

\Level 1 {Integer arrays as indices}

It's even possible to use a set of indices,
stored in an integer array,
to access arbitrary locations in an array.

\begin{block}{Index arrays}
  \label{sl:farray-indexarray}
  Indexed subset:
\begin{lstlisting}
integer,dimension(4) :: i = [2,3,5,7]
real(4),dimension(10) :: x
print *,x(i)
\end{lstlisting}
\end{block}

\Level 0 {Multi-dimensional}
\label{sec:farray-multid}

Arrays above had `\emph{rank}\index{array!rank}~one'. The rank is
defined as the number of indices you need to address the elements.
Mathematically this is not a rank but the dimension of the array,
but that word is already taken.
We will still use that word,
for instance talking about the first and second dimension of an array.

A rank-two array, or matrix, is defined like this:
\begin{block}{Multi-dimension arrays}
  \label{sl:farray-2d}
  Declaration and use with parentheses and comma\\
  (compare \lstinline{a[i][j]} in C++):
\begin{lstlisting}
real(8),dimension(20,30) :: array
array(i,j) = 5./2
\end{lstlisting}
\end{block}

A useful function is \indexfdef{reshape}.

\begin{block}{Reshaping array}
  \label{sl:finit2d}
  Reshape: convert 2D array to 1D (or vv)\\
  between arrays with the same number of elements.

  Example: 
  \begin{itemize}
  \item initialize as 1D,
  \item reshape to 2D
  \end{itemize}

  \snippetwithoutput{finit2d}{arrayf}{multi}
\end{block}

With multidimensional arrays we have to worry how they are stored in
memory. Are they stored row-by-row, or column-by-column? In Fortran
the latter choice, also known as \indextermdef{column-major} storage,
is used; see figure~\ref{fig:column-major}.

\begin{figure}[ht]
  \includegraphics[scale=.1]{arraybycol}
  \caption{Column-major storage in Fortran}
  \label{fig:column-major}
\end{figure}

To traverse the elements as they are stored in memory, you would need
the following code:
\begin{lstlisting}
do col=1,size(A,2)
  do row=1,size(A,1)
    .... A(row,col) .....
  end do
end do
\end{lstlisting}
This is sometimes described as `First index varies quickest'.
There are various performance-related reasons why such traversal
is better than with the loops exchanged.

\begin{slide}{Array layout}
  \label{sl:farray-layout}
  Sometimes you have to take into account how a multi-D array
  is laid out in (linear) memory:

  \includegraphics[scale=.1]{arraybycol}

  `First index varies quickest'
\end{slide}

\begin{exercise}
  Can you describe in words how memory elements are access if you
  would write
\begin{lstlisting}
do row=1,size(A,1)
  do col=1,size(A,2)
    .... A(row,col) .....
  end do
end do
\end{lstlisting}
?
\end{exercise}

You can make sections in multi-dimensional arrays: you need to
indicate a range in all dimensions.

\begin{block}{Array sections in multi-D}
  \label{sl:farray-sectiond}
\begin{lstlisting}
real(8),dimension(10) :: a,b
a(1:9) = b(2:10)
\end{lstlisting}
or
\begin{lstlisting}
logical,dimension(25,3) :: a
logical,dimension(25)   :: b
a(:,2) = b
\end{lstlisting}
You can also use strides.
\end{block}

\begin{block}{Array printing}
  \label{sl:farray-print}
  Fill array by rows, printing is by column:
  \[ \begin{pmatrix}1&2&\ldots&N\\ N+1&\ldots\\ &\ldots\\ &&&MN
  \end{pmatrix}
  \]
  %
  \scriptsize
  \snippetwithoutput{printarray}{arrayf}{printarray}
\end{block}

\Level 1 {Querying an array}
\label{sec:farray-query}

We have the following properties of an array:
\begin{itemize}
\item The bounds are the lower and upper bound in each dimension.
  For instance, after
\begin{lstlisting}
integer,dimension(-1:1,-2:2) :: symm
\end{lstlisting}
the array \n{symm} has a lower bound of~\n{-1} in the first dimension
and \n{-2} in the second. The functions \indexfdef{Lbound} and
\indexfdef{Ubound} give these bounds as array or scalar:
\begin{lstlisting}
array_of_lower = Lbound(symm)
upper_in_dim_2 = Ubound(symm,2)
\end{lstlisting}

\snippetwithoutput{fsection2}{arrayf}{fsection2}

\item The \emph{extent}\index{extent!of array dimension} is the number
  of elements in a certain dimension, and the
  \indexf{shape}\index{array!shape (Fortran)} is the array of extents.

\item The \indexfdef{size} is the number of elements, either for
  the whole array, or for a specified dimension.
\begin{lstlisting}
integer :: x(8), y(5,4)
size(x)
size(y,2)
\end{lstlisting}
\end{itemize}

\begin{slide}{Query functions}
  \label{sl:farray-query}
  \begin{itemize}
  \item Bounds: \indexf{lbound}, \indexf{ubound}
  \item \indexf{size}
  \item Can be used per dimension, or overall giving array of bounds/sizes.
  \end{itemize}
  \snippetwithoutput{farrayquery}{arrayf}{query}
\end{slide}

\Level 1 {Reshaping}

\indexf{RESHAPE}
\begin{lstlisting}
  array = RESHAPE( list,shape )
\end{lstlisting}
Example:
\verbatimsnippet{shape44}

\indexf{SPREAD}
\begin{lstlisting}
  array = SPREAD( old,dim,copies )
\end{lstlisting}

\Level 0 {Arrays to subroutines}

Subprogram needs to know the shape of an array, not the actual bounds:

\begin{block}{Pass array: main program}
  \label{sl:farray-pass1d-main}
  Passing array as one symbol:
  \snippetwithoutput{fpass1dmain}{arrayf}{arraypass1d}
\end{block}

\begin{block}{Pass array: subprogram}
  \label{sl:farray-pass1d-subr}
  Note declaration as \indexf{dimension(:)}\\
  actual size is queried
  \verbatimsnippet{fpass1dsubr}
\end{block}

The array inside the subroutine is known as a
\indextermsub{assumed-shape}{array} or
\indextermsub{automatic}{array}.

\Level 0 {Allocatable arrays}
\label{sec:allocatable}

Static arrays are fine at small sizes. However, 
there are two main arguments against using them at large sizes.
\begin{itemize}
\item Since the size is explicitly stated, it makes your program
  inflexible, requiring recompilation to run it with a different
  problem size.
\item Since they are allocated on the so-called \indexterm{stack},
  making them too large can lead to \indextermbus{stack}{overflow}.
\end{itemize}

A better strategy is to indicate the shape of the array, and use
\indexf{allocate} to specify
the size later, presumably in terms of run-time program parameters.

\begin{block}{Array allocation}
  \label{sl:farray-alloc}
\begin{lstlisting}
! static:
integer,parameter :: s=100
real(8), dimension(s) :: xs,ys

! dynamic
integer :: n
real(8), dimension(:), allocatable :: xd,yd
read *,n
allocate(xd(n), yd(n))
\end{lstlisting}
You can \indexf{deallocate} the array when you don't need the
space anymore.
\end{block}

If you are in danger of running out of memory, it can be a good idea
to add a \indexf{stat=ierror} clause to the \indexf{allocate} statement:
\begin{lstlisting}
integer :: ierr
allocate( x(n), stat=ierr )
if ( ierr/=0 ) ! report error
\end{lstlisting}

Has an array been allocated:
\begin{lstlisting}
Allocated( x ) ! returns logical
\end{lstlisting}

Allocatable arrays are automatically deallocated when they go out of
scope. This prevents the \indextermbus{memory}{leak} problems of~C++.

Explicit deallocate:
\begin{lstlisting}
deallocate(x)
\end{lstlisting}

\Level 1 {Returning an allocated array}

In an effort to keep the main program nice and abstract,
you may want to delegate the \indexf{allocate} statement
to a procedure.
In case of a \indexf{Subroutine}, you can pass the
(unallocated) array as a parameter.
But can you return it from a \indexf{Function}?

This requires the \indexf{Result} keyword
(section~\ref{sec:fresult}:
%
\verbatimsnippet{returnarrayresult}

\Level 0 {Array output}

The simple statement
\begin{lstlisting}
print *,A
\end{lstlisting}
will output the element of \lstinline{A} in memory order;
see section~\ref{sec:farray-multid}.

For a more sophisticated approach,
the main thing to know is that each call to a format statement
starts on a new line.
Thus
\begin{lstlisting}
print '(10f7.3)',A( ...stuff... )
\end{lstlisting}
will print 10 elements of the array on each line,
before starting a new line of output.

The way to specify the elements of the array
is to use implicit do-loops; section~\ref{sec:f-impdo}:
\begin{lstlisting}
print '(10f7.3)', (A(i),i=1,size(A))
\end{lstlisting}
or for multiple dimensions:
\begin{lstlisting}
do row=1,size(A,2)
  print '(10f7.3)', (A(i,row),i=1,size(A,1))
end do
\end{lstlisting}
What if, in this example, the rows are longer than 10 elements?
You can not parametrize the format,
but there is no harm in specifying more format
than there are array elements:

\Level 0 {Operating on an array}

\Level 1 {Arithmetic operations}

Between arrays of the same shape:
\begin{lstlisting}
A = B+C
D = D*E
\end{lstlisting}
(where the multiplication is by element).

\Level 1 {Intrinsic functions}

The following intrinsic functions are avaible for arrays:
\begin{block}{Array intrinsics}
  \label{sl:array-funcf}
  \begin{itemize}
  \item \indexf{Abs} creates the matrix of pointwise absolute values.
  \item \indexf{MaxLoc} returns the index of the maximum
    element.
  \item \indexf{MinLoc} returns the index of the minimum element.
  \item \indexf{MatMul} returns the matrix product of two matrices.
  \item \indexf{Dot_Product} returns the dot product of two
    arrays.
  \item \indexf{Transpose} returns the transpose of a matrix.
  \item \indexf{Cshift} rotates elements through an array.
  \end{itemize}
\end{block}

Reduction operations on the array itself:
\begin{itemize}
  \item \indexf{MaxVal} finds the maximum value in an array.
  \item \indexf{MinVal} finds the minimum value in an array.
  \item \indexf{Sum} returns the sum of all elements.
  \item \indexf{Product} return the product of all elements.
\end{itemize}

Reduction operations on a mask derived from the array:
\begin{itemize}
\item \indexf{All} finds if the mask is true for all elements
\item \indexf{Any} finds if the mask is true for any element
\item \indexf{Count} finds for how many elements the mask is true
\end{itemize}

\begin{block}{Multi-dimensional intrinsics}
  \label{sl:array-funcfmd}
  \begin{itemize}
  \item
    Functions such as \indexf{Sum} operate on a whole array by
    default.
  \item To
    restrict such a function to one subdimension add a keyword
    parameter \indexf{DIM}:
\begin{lstlisting}
s = Sum(A,DIM=1)
\end{lstlisting}
where the keyword is optional.
\item Likewise, the operation can be restricted to a \indexf{MASK}:
\begin{lstlisting}
s = Sum(A,MASK=B)
\end{lstlisting}
  \end{itemize}
\end{block}

\begin{block}{Row and column sums}
  \label{sl:farray-row-col-sum}
  \snippetwithoutput{frowcolsums}{arrayf}{rowcolsum}
\end{block}

\begin{exercise}
  \label{ex:fmatnorm}
  The 1-norm of a matrix is defined as the maximum of all sums of absolute
  values in any column:
  \[ \|A\|_1 = \max_j \sum_i |A_{ij}| \]
  while the infinity-norm is defined as the maximum row sum:
  \[ \|A\|_\infty = \max_i \sum_j |A_{ij}| \]
  Compute these norms using array functions as much as possible, that
  is, try to avoid using loops.
  
  For bonus points, write Fortran \lstinline{Function}s that compute
  these norms.
\end{exercise}

\begin{exercise}
  \label{ex:fmatmul}
  Compare implementations of the matrix-matrix product.
  \begin{enumerate}
  \item Write the regular \n{i,j,k} implementation, and store it as
    reference.
  \item Use the \indexf{DOT_PRODUCT} function, which eliminates the \n{k}
    index. How does the timing change? Print the maximum absolute
    distance between this and the reference result.
  \item Use the \indexf{MATMUL} function. Same questions.
  \item Bonus question: investigate the \n{j,k,i} and \n{i,k,j}
    variants. Write them both with array sections and individual array
    elements. Is there a difference in timing?
  \end{enumerate}
  Does the optimization level make a difference in timing?
\end{exercise}

\Level 1 {Restricting with \tt{where}}

If an array operation should not apply to all elements, you can
specify the ones it applies to with a \indexf{where} statement.

\begin{block}{Masked operations}
  \label{sl:farray-where}
\begin{lstlisting}
where ( A<0 ) B = 0
\end{lstlisting}

Full form:
\begin{lstlisting}
WHERE ( logical argument )
  sequence of array statements
ELSEWHERE
  sequence of array statements
END WHERE
\end{lstlisting}
\end{block}

\Level 1 {Global condition tests}

Reduction of a test on all array elements:
\indexf{all}
\begin{lstlisting}
REAL(8),dimension(N,N) :: A
LOGICAL :: positive,positive_row(N),positive_col(N)
positive = ALL( A>0)
positive_row = ALL( A>0,1 )
positive_col = ALL( A>0,2 )
\end{lstlisting}

\begin{exercise}
  Use array statements (that is, no loops) to fill a two-dimensional
  array~\n{A} with random numbers between zero and~one. Then fill two
  arrays \n{Abig} and \n{Asmall} with the elements of~\n{A} that are
  great than~$0.5$, or less than~$0.5$ respectively:
  \[ A_{\scriptstyle\mathrm{big}}(i,j) =
  \begin{cases}
    A(i,j)&\hbox{if $A(i,j)\geq 0.5$}\\ 0&\hbox{otherwise}
  \end{cases}
  \]
  \[ A_{\scriptstyle\mathrm{small}}(i,j) =
  \begin{cases}
    0&\hbox{if $A(i,j)\geq 0.5$}\\ A(i,j)&\hbox{otherwise}
  \end{cases}
  \]
  Using more array statements, add \n{Abig} and \n{Asmall}, and test
  whether the sum is close enough to~\n{A}.
\end{exercise}

Similar to \indexf{all}, there is a function~\indexf{any} that tests
if any array element satisfies the test.
\begin{lstlisting}
if ( Any( Abs(A-B)>
\end{lstlisting}

\Level 0 {Array operations}

\Level 1 {Loops without looping}

In addition to ordinary do-loops, Fortran has mechanisms that save you
typing, or can be more efficient in some circumstances.

\Level 2 {Slicing}

If your loop assigns to an array from another array,
  you can use section notation:
\begin{lstlisting}
a(:) = b(:)
c(1:n) = d(2:n+1)
\end{lstlisting}

\Level 2 {`forall' keyword}

The \indexf{forall} keyword also indicates an array assignment:
\begin{lstlisting}
forall (i=1:n)
  a(i) = b(i)
  c(i) = d(i+1)
end forall
\end{lstlisting}
You can tell that this is for arrays only, because the loop index has
to be part of the left-hand side of every assignment.

What happens if you apply \lstinline{forall} to a statement
with loop-carried dependencies?
Consider first the traditional loops

\begin{multicols}{2}
  \verbatimsnippet{forfor}
  \columnbreak
  \verbatimsnippet{forbac}
\end{multicols}

Can you predict the output? Now consider the following:

\snippetwithoutput{forallf}{arrayf}{forallf}
\snippetwithoutput{forallb}{arrayf}{forallb}

What does this tell you about the execution?

In other words, this mechanism is prone to misunderstanding and therefore now
deprecated.
It is not a parallel loop! For that, the following mechanism is preferred.

\Level 2 {Do concurrent}

\begin{block}{Do concurrent}
  \label{sl:farray-concurrent}

  The \indextermbus{do}{concurrent} is a true do-loop. With the
  \indexf{concurrent} keyword the user specifies that the
  iterations of a loop are independent, and can therefore possibly be
  done in parallel:
\begin{lstlisting}
do concurrent (i=1:n)
  a(i) = b(i)
  c(i) = d(i+1)
end do
\end{lstlisting}
(Do not use \indexf{for all})
\end{block}

\Level 1 {Loops without dependencies}

Here are some illustrations of simple array copying with the above
mechanisms.

\verbatimsnippet{blockrecur}
\begin{lstlisting}
Original     1   2   3   4   5   6   7   8   9  10
Recursive    0   2   4   6   8  10  12  14  16  18
\end{lstlisting}

\verbatimsnippet{blockslice}
\begin{lstlisting}
Original     1   2   3   4   5   6   7   8   9  10
Section      0   2   4   6   8  10  12  14  16  18
\end{lstlisting}

\verbatimsnippet{blockforall}
\begin{lstlisting}
Original     1   2   3   4   5   6   7   8   9  10
Forall       0   2   4   6   8  10  12  14  16  18
\end{lstlisting}

\verbatimsnippet{blockconc}
\begin{lstlisting}
Original     1   2   3   4   5   6   7   8   9  10
Concurrent   0   2   4   6   8  10  12  14  16  18
\end{lstlisting}

\begin{exercise}
  Create arrays \n{A,C} of length~\n{2N}, and \n{B}~of length~\n{N}.
  Now implement
  \[ B_i = (A_{2i}+A_{2i+1})/2,\quad i=1,\ldots N \]
  and
  \[ C_i = A_{i/2},\quad i=1,\ldots 2N \]
  using all four mechanisms. Make sure you get the same result every time.
\end{exercise}

\Level 1 {Loops with dependencies}

For parallel execution of a loop, all iterations have to be independent.
This is not the case if the loop has a \indexterm{recurrence}, and in
this case, the `do concurrent' mechanism is not appropriate.
%
Here are the above four constructs, but applied to a loop with a dependence.
%
\verbatimsnippet{slicerecur}
%
\begin{lstlisting}
Original   1   2   3   4   5   6   7   8   9  10
Recursiv   1   2   4   8  16  32  64 128 256 512
\end{lstlisting}

The slicing version of this:
%
\verbatimsnippet{sliceslice}
%
\begin{lstlisting}
Original   1   2   3   4   5   6   7   8   9  10
Section    1   2   4   6   8  10  12  14  16  18
\end{lstlisting}
%
acts as if the right-hand side is saved in a temporary array, and
subsequently assigned to the left-hand side.

Using `forall' is equivalent to slicing:
%
\verbatimsnippet{sliceforall}
%
\begin{lstlisting}
Original     1   2   3   4   5   6   7   8   9  10
Forall       1   2   4   6   8  10  12  14  16  18
\end{lstlisting}

On the other hand, `do concurrent' does not use temporaries, so it is
more like the sequential version:
%
\verbatimsnippet{sliceconc}
%
\begin{lstlisting}
Original     1   2   3   4   5   6   7   8   9  10
Concurrent   1   2   4   8  16  32  64 128 256 512
\end{lstlisting}
Note that the result does not have to be equal to the sequential
execution: the compiler is free to rearrange the iterations any way it
sees fit.

\Level 0 {Review questions}

\begin{exercise}
  \label{ex:farray-assign-section}
  Let the following declarations be given, and assume that all arrays
  are properly initialized:
\begin{lstlisting}
real                   :: x
real, dimension(10)    :: a, b
real, dimension(10,10) :: c, d
\end{lstlisting}

Comment on the following lines: are they legal, if so what do they do?
\begin{enumerate}
\item \n{a = b}
\item \n{a = x}
\item \n{a(1:10) = c(1:10)}
\end{enumerate}

How would you:
\begin{enumerate}
\item Set the second row of \n{c} to~\n{b}?
\item Set the second row of \n{c} to the elements of~\n{b}, last-to-first?
\end{enumerate}
\end{exercise}
