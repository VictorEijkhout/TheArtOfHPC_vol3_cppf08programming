% -*- latex -*-
%%%%%%%%%%%%%%%%%%%%%%%%%%%%%%%%%%%%%%%%%%%%%%%%%%%%%%%%%%%%%%%%
%%%%
%%%% This TeX file is part of the course
%%%% Introduction to Scientific Programming in C++/Fortran2003
%%%% copyright 2020 Victor Eijkhout eijkhout@tacc.utexas.edu
%%%%
%%%% structj.tex : about structures in Julia
%%%%
%%%%%%%%%%%%%%%%%%%%%%%%%%%%%%%%%%%%%%%%%%%%%%%%%%%%%%%%%%%%%%%%

\Level 0 {Why structures?}
\label{sec:structj}

You have seen the basic datatypes in section~\ref{sec:ctypes}. These
are enough to program whatever you want, but it would be nice if the
language had some datatypes that are more abstract, closer to the
terms in which you think about your application. For instance, if you
are programming something to do with geometry, you had rather talk
about points than explicitly having to manipulate their coordinates.

Structures are a
first way to define your own datatypes. A~\indextermttdef{struct}
acts like a datatype for which you choose the name. A~\lstinline$struct$
contains other datatypes; these can be elementary, or other structs.

\verbatimsnippet{structjuse}

The reason for using structures is to `raise the level of
abstraction': instead of talking about \lstinline{x,y}-values, you can
now talk about vectors and such. This makes your code look closer to
the application you are modeling.

\Level 0 {The basics of structures}

A structure behaves like a data type: you declare variables of the
structure type, and you use them in your program. The new aspect is
that you first need to define the structure type. This definition can
be done anywhere before you use it.

\verbatimsnippet{structjdef}

\Level 0 {Structures and functions}

You can use structures with functions, just as you could with
elementary datatypes.

Julia here uses \indexterm{multiple dispatch}
(which is a form of polymorphism)
where a function can receive additional definitions
for operating on user-defined types.

\Level 1 {Structures versus objects}

Multiple dispatch on
structures are Julia's alternative to \ac{OOP}:
rather than having methods inside a defined class,
functions are defined free-standing,
operating on newly defined types.

While there is not much difference between
\lstinline{obj.method()} and \lstinline{method(obj)},
things like inheritance are harder to realize.

\begin{comment}
\begin{block}{Functions on structures}
  \label{sl:struct-pass}
  You can pass a structure to a function:
  %\verbatimsnippet{structpass}
  \snippetwithoutput{structpass}{struct}{pointfun}
\end{block}

\begin{exercise}
  \label{ex:vecstruct-angle}
  \hbox{%
    \begin{minipage}[t]{.6\hsize}
      Write a vector class, and a function that, given such a  vector, returns the
      angle with the $x$-axis. (Hint: the \lstinline$atan$ function is in \lstinline$cmath$)
    \end{minipage}
    \includegraphics[scale=.4]{xy-atan}
    }

  \answerwithoutput{anglepi4}{struct}{pointangle}
\end{exercise}

The mechanisms of \indextermbus{parameter}{passing by value} and
\indextermbus{parameter}{passing by reference} apply to structures just as
to simple datatypes.

\begin{exercise}
  \label{ex:vecstruct-flip}
  Write a \lstinline$void$ function that has a \lstinline$struct vector$ parameter,
  and exchanges its coordinates:
  \[ \begin{pmatrix}2.5\\3.5\end{pmatrix} \rightarrow
    \begin{pmatrix}3.5\\2.5\end{pmatrix} \]
  \answerwithoutput{flip32}{struct}{pointflip}
\end{exercise}

\begin{exercise}
  \label{ex:vecstruct-scale}
  Write a function $y=f(x,a)$ that takes a \lstinline$struct vector$ and
  \lstinline$double$ parameter as input, and returns a vector that is the
  input multiplied by the scalar.
  \[ \begin{pmatrix}2.5\\3.5\end{pmatrix},3 \rightarrow
    \begin{pmatrix}7.5\\10.5\end{pmatrix} \]
\end{exercise}
\end{comment}

%% \begin{exercise}
%%   \label{ex:primestructj}
%%   If you are doing the prime project (chapter~\ref{ch:prime}) you can
%%   now do exercise~\ref{ex:prime:struct}.
%% \end{exercise}

\begin{comment}
\begin{exercise}
  \label{ex:vecstruct}
  Write a function \lstinline$inner_product$ that takes two \lstinline$vector$
  structures and computes the inner product.
\end{exercise}

\begin{block}{Denotations}
  \label{sl:struct-denote}
  You can use initializer lists as struct
  \emph{denotations}\index{struct!denotation}:
  %
  \snippetwithoutput[pointfun]{structdenote}{struct}{pointdenote}
  %\snippetwithoutput{structdenote}{struct}{pointdenote}
\end{block}

\begin{exercise}
  \label{ex:struct-denote}
  Take exercise \ref{ex:vecstruct} and rewrite it to use denotations.
\end{exercise}

\begin{block}{Returning structures}
  \label{sl:struct-return}
  You can return a structure from a function:
  %
  \snippetwithoutput{structreturn}{struct}{pointadd}

  (In case you're wondering about scopes and lifetimes here: the
  explanation is that the returned value is copied.)
\end{block}

\begin{exercise}
  \label{ex:matstruct}
  Write a $2\times 2$ matrix class (that is, a structure storing 4
  real numbers), and write a function \lstinline$multiply$
  that multiplies a matrix times a vector.

  Can you make a matrix structure that is based on the vector
  structure, for instance using vectors to store the matrix rows, and
  then using the inner product method to multiply matrices?
\end{exercise}

\end{comment}
