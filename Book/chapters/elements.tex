% -*- latex -*-
%%%%%%%%%%%%%%%%%%%%%%%%%%%%%%%%%%%%%%%%%%%%%%%%%%%%%%%%%%%%%%%%
%%%%
%%%% This TeX file is part of the course
%%%% Introduction to Scientific Programming in C++/Fortran2003
%%%% copyright 2017-2023 Victor Eijkhout eijkhout@tacc.utexas.edu
%%%%
%%%% elements.tex : basic language elements
%%%%
%%%%%%%%%%%%%%%%%%%%%%%%%%%%%%%%%%%%%%%%%%%%%%%%%%%%%%%%%%%%%%%%

\Level 0 {From the ground up: Compiling C++}

In this chapter and the next you are going to learn the C++
language. But first we need some externalia:
where do you get a program and how do you handle it?

\begin{block}{Two kinds of files}
  \label{sl:sourcebinary}
  In programming you have two kinds of files:
  \begin{itemize}
  \item \emph{source files}\index{file!source}, which are understandable
    to you, and which you create with an editor such as \n{vi} or
    \n{emacs}; and
  \item \emph{binary files}\index{file!binary}, which are understandable
    to the computer, and unreadable to you.
  \end{itemize}
  Your source files are translated to binary by a
  \indexterm{compiler}, which `compiles' your source file.
\end{block}

Let's look at an example:
\begin{verbatim}
icpc -o myprogram myprogram.cpp
\end{verbatim}
This means:
\begin{itemize}
\item you have a source code file \n{myprogram.cpp};
\item and you want an executable file as output called \n{myprogram},
\item and your compiler is the Intel compiler \indextermtt{icpc}.
  (If you want to use the C++ compiler of the
  \indexterm{GNU} project you specify \indextermtt{g++};
  the compiler of the clang project is \indextermtt{clang++}.)
\end{itemize}

Let's do an example.

\begin{exercise}{The edit-compile-run cycle}
\label{ex:compile-cycle}
  Make a file \n{zero.cc} with the following lines:
  %
  \verbatimsnippet{nullprogc}
  %
  and compile it. Intel compiler:
\begin{verbatim}
icpc -o zeroprogram zero.cc
\end{verbatim}
Run this program (it gives no output):
\begin{verbatim}
./zeroprogram
\end{verbatim}
\end{exercise}

\begin{slide}{Anatomy of the compile line}
  \label{sl:compile-line}
  \begin{itemize}
  \item \indextermtt{icpc} : compiler. Alternative: use \indextermtt{g++}
    or \indextermtt{clang++}
  \item \texttt{-o zeroprogram} : output into a binary name of your choosing
  \item \texttt{zero.cc} : your source file.
  \end{itemize}
\end{slide}

In the above program:
\begin{enumerate}
\item The first two lines are magic, for now. Always include them.
  Ok, if you want to know: the \lstinline+#include+ line
  is a \indexterm{preprocessor} (chapter~\ref{ch:cpp}) directive;
  it includes a \indexterm{header}
  into your program that makes certain functionality available.
\item The \lstinline{main} line indicates where the program starts; between
  its opening and closing brace will be the
  \indextermbus{program}{statements}.
\item The \lstinline{return} statement indicates successful completion of your program.
  (If you wonder about the details of this, see section~\ref{sec:int-main}.)
\end{enumerate}
If you followed the above instructions, 
and as you may have guessed,
you saw that this program produces absolutely no output
when you run it.

If you do \n{ls} in your current directory,
you see that you now have two files: the source file \n{zero.cc}
and the executable \n{zeroprogram}.
There is some freedom in how you choose these names.

\begin{block}{File names}
  \label{sl:file-ext}
  File names can have extensions: the part after the dot.
  (The part before the dot is completely up to you.)
  \begin{itemize}
  \item \n{program.cpp} or \n{program.cc} or \n{program.cxx}
    are typical extensions for
    C++ sources.
  \item \n{program.cpp} has a possible
    possible confusion with `C~PreProcessor',
    but it seems to be the standard, so we will use it in this course.
  \item Using \n{program} without extension usually indicates an \indexterm{executable}.
    (What happens if you leave the \n{-o myprogram} part off the compile line?)
  \end{itemize}
\end{block}

Let's make the program do something:
display a `hello world' message on your screen.
For now, just copy this line; the details
of what it all means will come later.

\begin{exercise}{Your first useful program}
  \label{ex:sayhello}
Add this line:
%
\verbatimsnippet{helloc}
%
(copying from the pdf file is dangerous! please type it yourself)

Compile and run again. What is the output? 
\end{exercise}

Test your knowledge of the file types involved in programming!

\begin{review}
  \label{q:compiler}
  True or false?
  \begin{enumerate}
  \item The programmer only writes source files, no binaries.
    \slackpollTF+A programmer writes sources, no binaries+
  \item The computer only executes binary files, no human-readable files.
    \slackpollTF+Computer executes binaries, no readable files+
  \end{enumerate}
\end{review}

\Level 1 {A quick word about unix commands}

The compile line
\begin{verbatim}
g++ -o myprogram myprogram.cpp
\end{verbatim}
can be thought of as consisting of three parts:
\begin{itemize}
\item The command \n{g++} that starts the line and determines what is
  going to happen;
\item The argument \n{myprogram.cpp} that ends the line is the main
  thing that the command works on; and
\item The option/value pair \n{-o myprogram}. Most Unix commands have
  various options that are, as the name indicates, optional. For
  instance you can tell the compiler to try very hard to make a fast program:
\begin{verbatim}
g++ -O3 -o myprogram myprogram.cpp
\end{verbatim}
  Options can appear in any order, so this last command is equivalent to
\begin{verbatim}
g++ -o myprogram -O3 myprogram.cpp
\end{verbatim}
\end{itemize}
Be careful not to mix up argument and option. If you type
\begin{verbatim}
g++ -o myprogram.cpp myprogram
\end{verbatim}
then Unix will reason: `\n{myprogram.cpp} is the output, so if that
file already exists (which, yes, it does) let's just empty it before
we do anything else'. And you have just lost your program.
Good thing that editors like
\indexterm{emacs} keep a backup copy of your file.

\Level 1 {Build environments}

Above we didn't say anything about how to make your sources files.
Also, while you saw a compile line, you may not need to issue those by hand.
It all depends on how sophisticated your build environment is.
Here we give some possibilities,
but this course leaves the choice up to you.

\Level 2 {Commandline and editor}

The traditional way of developing software is by using an editor
--~such as \indexterm{emacs}, \indexterm{vim}, \indexterm{nano}~--
and typing compile commands on the commandline of Unix
or some other \acf{OS}.

If you are proficient in Unix, this is not a bad strategy.
You can make life easier by using \indexunix{Make} or \indexunix{CMake}
but that's mostly for more complicated programs than you will
run into in the first number of chapters of this course.

\Level 2 {Integrated build system}

There some really nice programs with a graphical user interface
that make software development less painful
if you're not such a Unix-head.
They help you with editing your file,
and compiling is a press-on-a-button.

Specific to Apple, there is \indexterm{XCode},
and commercialy from Microsoft there is \indexterm{Visual Studio}.
The latter has a free (and limited) version \indexterm{VSCode}.
Another commercial product widely in use is \indexterm{CLion}.

Finally, there is the open source \indexterm{Eclipse} software.

\Level 1 {C++ is a moving target}

The C++ language has gone through a number of standards. (This is
described in some detail in section~\ref{sec:cpp-standards}.) In this
course we focus on recent standards: 
\cppstandard{17}, and to some amount \cppstandard{20}.
Unfortunately, your compiler will assume an earlier standard by default,
so constructs taught here may be marked as ungrammatical.

You can tell your compiler to use a modern standard:
\begin{verbatim}
icpc -std=c++17 [other options]
\end{verbatim}
but to save yourself a lot of typing, you can define
\begin{verbatim}
alias icpc='icpc -std=c++17'
\end{verbatim}
in your shell startup files.
\begin{tacc}
On the class \n{isp} machine this alias has been defined by default.
\end{tacc}

\begin{slide}{C++ versions}
  \label{sl:cpp-version}
  \begin{itemize}
  \item
    The compiler by default uses \n{C++98}. 
  \item This course explains \n{C++17} and \n{C++20}.
    You need tell your compiler about this.
  \item On \n{isp.tacc.utexas.edu} `icpc' uses this by default.
  \item On your own machine you need to do
\begin{verbatim}
  g++ -std=c++17 [other options]
\end{verbatim}
or
\begin{verbatim}
  alias g++='g++ -std=c++17'
\end{verbatim}
(in your \indextermtt{.bashrc} file)
  \end{itemize}
\end{slide}

\Level 0 {Statements}
\label{sec:statements}

Each programming language has its own (very precise!) rules for what
can go in a source file. Globally we can say that a program contains
instructions for the computer to execute, and these instructions take
the form of a bunch of `statements'. Here are some of the rules on
statements; you will learn them in more detail as you go through this
book.

\begin{block}{Program statements}
  \label{sl:cstatement}
  \begin{itemize}
  \item
    A program contains statements, each terminated by a semicolon.
  \item `Curly braces' can enclose multiple statements.
  \item A statement can correspond to some action when the program is
    executed.
  \item Some statements are definitions, of data or of possible actions.
  \item Comments are `Note to self', short:
\begin{lstlisting}
cout << "Hello world" << '\n'; // say hi!
\end{lstlisting}
and arbitrary:
\begin{lstlisting}
cout << /* we are now going
           to say hello
         */ "Hello!" << /* with newline: */ '\n';
\end{lstlisting}
  \end{itemize}
\end{block}

In the examples so far you see output statements terminated as:
\begin{lstlisting}
  cout << something << "\n";
\end{lstlisting}
where the `backslash-n' stands for a \indexterm{newline}.

\begin{exercise}
  Remove the newline from your print statement(s).
  Compile and run. What do you observe?
\end{exercise}

Sometimes you will also see:
\begin{lstlisting}
// at the top of the program:
using std::endl;

// among the statements:
cout << something << endl;
\end{lstlisting}
which has the same behavior of issuing a newline.
The distinction will not be important to you for now;
see the discussion in section~\ref{sec:to-endl-or-not}
if you're curious.

\begin{exercise}
  \label{ex:hello-line}
  Take the `hello world' program you wrote above, and duplicate the
  hello-line. Compile and run.

  Does it make a difference whether you have the two hellos on the
  same line of your file, or on different lines?

  Experiment with other changes to the layout of your source. Find at
  least one change that leads to a compiler error. Can you relate the
  message to the error?
\end{exercise}

Your program source can have several types of errors.
Distinguishing by when you notice them, we roughly distinguish them
as follows.
(For details on error handling, see chapter~\ref{ch:error}.)

\begin{block}{Error types}
  \label{sl:program-errors}
\begin{slideonly}
Your code may suffer from the following types of error:    
\end{slideonly}
  \begin{enumerate}
  \item \emph{Syntax}\index{error!syntax} or
    \emph{compile-time}\index{error!compile-time} errors: these arise
    if what you write is not according to the language specification.
    The compiler catches these errors, and it refuses to produce a
    \indextermsub{binary}{file}.
  \item \emph{Run-time}\index{error!run-time} errors: these arise if
    your code is syntactically correct, and the compiler has produced
    an executable, but the program does not behave the way you
    intended or foresaw. Examples are divide-by-zero or indexing
    outside the bounds of an array.
  \item Design errors: your program does not do what you think it does.
  \end{enumerate}
\end{block}

\begin{review}
  \label{q:compile}
  True or false?
  \begin{itemize}
  \item If your program compiles correctly, it is correct.
    \slackpollTF+If it compiles it is correct+
  \item If you run your program and you get the right output, it is correct.
    \slackpollTF+Program runs, no errors, therefore it is correct+
  \end{itemize}  
\end{review}

In the program you just wrote, the string you displayed
was completely up to you. Other elements, such as the \lstinline{cout}
keyword, are fixed parts of the language.
Most programs contains them.

\begin{block}{Fixed elements}
  \label{sl:fixedstuff}
  You see that certain parts of your program are inviolable:
  \begin{itemize}
  \item There are \indexterm{keywords} such as \lstinline{return} or \lstinline{cout}; you
    can not change their definition.
  \item Curly braces and parentheses need to be matched.
  \item There has to be a \lstinline{main} keyword.
  \item The \lstinline{iostream} and \lstinline{std} are usually needed.
  \end{itemize}
\end{block}

\Level 1 {Language vs library and about: using}
\label{sec:usingio}

The examples above had a line
\begin{lstlisting}
#include <iostream>
\end{lstlisting}
which allowed you to write \lstinline{cout} in your program
for output.
The \indexheader{iostream} is a \indextermdef{header},
and it adds \indextermsub{standard}{library}
functionality to the base language.

Functionality such as \indexc{cout} can be used in various ways:
\begin{multicols}{2}
  You can spell it out as \lstinline+std::cout+:
  \begin{lstlisting}
    #include <iostream>
    int main() {
      std::cout "hello\n";
      return 0;
      }
  \end{lstlisting}
  \columnbreak
  You can add a \lstinline{using} statement:
  \begin{lstlisting}
    #include <iostream>
    using std::cout;
    int main() {
      cout "hello\n";
      return 0;
      }
  \end{lstlisting}
\end{multicols}

Instead of having separate \lstinline{using} statements
for each library function,
you could also use a single line
\begin{lstlisting}
  using namespace std;
\end{lstlisting}
in your program.
While it is common to find this in examples online, it is frowned upon;
see section~\ref{sec:namespace-practice} for a discussion.

\begin{exercise}
  \label{ex:cout-what}
  Experiment with the \lstinline{cout} statement. Replace the string by a
  number or a mathematical expression. Can you guess how to print more
  than one thing, for instance:
  \begin{itemize}
  \item the string \n{One third is}, and
  \item the result of the computation~$1/3$,     
  \end{itemize}
  with the same \lstinline{cout} statement?
  Do you get anything unexpected?
\end{exercise}

\Level 0 {Variables}
\label{sec:variables}

A program could not do much without storing data: input data,
temporary data for intermediate results, and final results.
Data is stored in \emph{variables}\index{variable},  which have
\begin{itemize}
\item a name, so that you can refer to them,
\item a \indexterm{datatype}, and
\item a value.
\end{itemize}
Think of a variable as a labeled placed in memory.
\begin{itemize}
\item The variable is defined in a
  \indextermbus{variable}{declaration},
\item which can include an \indextermsub{variable}{initialization}.
\item After a variable is defined, and given a value, it can be used,
\item or given a (new) value in a \indextermbus{variable}{assignment}.
\end{itemize}

\begin{slide}{What's a variable?}
  \label{sl:declaration}
  Programs usually contain data, which is stored in a
  \indextermdef{variable}. A~variable has
  \begin{itemize}
  \item a \indexterm{datatype},
  \item a name, and
  \item a value.
  \end{itemize}
  These are defined in a \indextermbus{variable}{declaration} and/or
  \indextermbus{variable}{assignment}.
\end{slide}

\begin{block}{Example variable lifetime}
  \label{sl:varlife}
\begin{lstlisting}
int i,j; // declaration
i = 5;   // set a value
i = 6;   // set a new value
j = i+1; // use the value of i
i = 8; // change the value of i
       // but this doesn't affect j:
       // it is still 7.
\end{lstlisting}
\end{block}

\Level 1 {Variable declarations}

A variable is defined,
in a \indextermbus{variable}{declaration}.
This associates its name and its type,
and possibly an initial value; see section~\ref{sec:varinit}.

Let's first talk about what a variable name can be.

\begin{block}{Variable names}
  \label{sl:varname}
  \begin{itemize}
  \item
    A variable name has to start with a letter;
  \item a name can contains letters and  digits,
    but not most special characters, except for the underscore.
  \item For letters it matters
    whether you use upper or lowercase: the language is \indexterm{case sensitive}.
  \item Words such as \lstinline{main} or \lstinline{return} are \indexterm{reserved words}.
  \item Usually \lstinline{i} and \lstinline{j} are not the best variable names: use
    \lstinline{row} and \lstinline{column}, or other meaningful names, instead.
  \item While you can start a name with an underscore,
    there are some limitations on the use of the underscore:
    do not use two underscores in a row, and
    do not start a name with an underscore followed by a capital letter.
  \end{itemize}
\end{block}

%% One way you can associate a type with a variable name is through a
%% \indextermbusdef{variable}{declaration}.
%% Here you make explicitly connection between a name and a type.
%% (Later you'll see that the type of a variable can also be
%% determined through \indextermbus{type}{deduction}.)

Next, a \indextermbusdef{variable}{declaration}
states the type and the name of a variable.
If you have multiple variables of the same type,
you can combine the declarations.

\begin{block}{Declaration}
  \label{sl:declare-example}
  A variable declaration establishes the name
  and the type of a variable:
\begin{lstlisting}
int n_elements;
float value;
int row,col;
double re_part,im_part;
\end{lstlisting}
\end{block}

You can not redeclare a variable like this:
\begin{lstlisting}
int value=5;
cout << value << '\n';
float value=1.3;
cout << value << '\n';
\end{lstlisting}
but the rules for what \textbf{is} allowed are a little harder to state.
You'll see that later in chapter~\ref{ch:scope}.

\begin{block}{Where do declarations go?}
  \label{sl:declwhere}
  Declarations can go pretty much anywhere in your program, but they need
  to come before the first use of the variable.

  Note: it is legal to define a variable before the main program
  but such \indextermsubp{global}{variable} are usually not a good idea.
  Please only declare variables \emph{inside} main
  (or inside a function et cetera).
\end{block}

\begin{review}
  \label{q:varnames}
  Which of the following are legal variable names?
  \begin{enumerate}
  \item \n{mainprogram}
    \slackpollTF+Legal mainprogram?+
  \item \n{main}
    \slackpollTF+Legal main?+
  \item \n{Main}
    \slackpollTF+Legal Main?+
  \item \n{1forall}
    \slackpollTF+Legal 1forall?+
  \item \n{one4all}
    \slackpollTF+Legal one4all?+
  \item \n{one_for_all}
    \slackpollTF+Legal one_for_all?+
  \item \n{onefor\char`\{all\char`\}}
    \slackpollTF+Legal onefor{all}?+
  \end{enumerate}
\end{review}

\Level 1 {Initialization}
\label{sec:varinit}

It is a possible to give a variable a value right when it's
created. This is known as
\emph{initialization}\index{variable!initialization} and it's
different from creating the variable and later assigning to it
(section~\ref{c:assign}).

\begin{block}{Initialization syntax}
  \label{sl:init-var}
  There are (at least) two ways of initializing a variable
  \begin{lstlisting}
    int i = 5;
    int j{6};
  \end{lstlisting}
  Note that writing 
  \begin{lstlisting}
    int i;
    i = 7;
  \end{lstlisting}
  is not an initialization: it's a declaration followed by an
  assignment.
\end{block}

If you declare a variable but not initialize, you can not count on
its value being anything, in particular not zero.
While some compilers do this (some of the time),
such implicit initialization is also
often omitted for performance reasons.

\Level 1 {Assignments}
\label{c:assign}

Setting a variable
\begin{lstlisting}
i = 5;
\end{lstlisting}
means storing a value in the memory location. It is
not the same as defining a mathematical equality
\[ \hbox{let $i=5$}. \]

\begin{block}{Assignment}
  \label{sl:assign1}
  Once you have declared a variable, you need to establish a value. This is done in an
  \indextermdef{assignment} statement. After the above declarations, the
  following are legitimate assignments:
\begin{lstlisting}
n = 3;
x = 1.5 - n;
n1 = 7;
n2 = n1 * 3;
\end{lstlisting}
These are not math equations: the variable on the left hand side
gets the value of the expression on the right hand side.

You see that you can assign both a simple value or an
\indexterm{expression}.
\end{block}

You can set the value of a variable multiple times.
\begin{lstlisting}
  int i;
  i = 5;
  // do something with i
  i = 6;
  // do something with i
\end{lstlisting}
You can also update the value of a variable,
using its current value:
\begin{lstlisting}
  i = 2*i + 1;
\end{lstlisting}

\begin{block}{Special forms}
  \label{sl:special-assign}
  Certain assignments with the same variable in both the
  left and right hand sides can be simplified:
\begin{lstlisting}
x = x+2; y = y/3;
// can be written as
x += 2; y /= 3;
\end{lstlisting}
Integer add/subtract one:
\begin{lstlisting}
  i=i+1; j=j-1;
  // rewritten as:
  ++i; --j;
  // or
  i++; j--;
\end{lstlisting}
\end{block}

\begin{exercise}
  \label{q:assign}
  Which of the following are legal? If they are, what is their meaning?
  \begin{enumerate}
  \item \lstinline{n = n;}
    \slackpollTF+Legal 'n = n;'?+
  \item \lstinline{n = 2n;}
    \slackpollTF+Legal 'n = 2n;'?+
  \item \lstinline{n = n2;}
    \slackpollTF+Legal 'n = n2;'?+
  \item \lstinline{n = 2*k;}
    \slackpollTF+Legal 'n = 2*k;'?+
  \item \lstinline{n/2 = k;}
    \slackpollTF+Legal 'n/2 = k;'?+
  \item \lstinline{n /= k;}
    \slackpollTF+Legal 'n /= k;'?+
  \end{enumerate}
\end{exercise}

There are various levels of programming errors.
The following program uses the variable \lstinline{i}
without having given it a value.

\begin{exercise}
  \label{q:initvar}
\begin{lstlisting}
#include <iostream>
using std::cout;
int main() {
  int i;
  int j = i+1;
  cout << j << "\n";
  return 0;
}
\end{lstlisting}
What happens?
\begin{enumerate}
\item Compiler error
\item Output: \lstinline{1}
\item Output is undefined
\item Error message during running the program.
\end{enumerate}
\slackpoll{"What happens:" "Compiler error" "Output: 1" "Output undefined" "Runtime error"}
\end{exercise}

\Level 1 {Datatypes}
\label{sec:ctypes}

You have seen a couple of datatypes that variables can have.
We'll go into the issue of datatypes into a little more detail.

\begin{block}{Datatypes}
  \label{sl:datatypes}
  Variables come in different types;
  \begin{itemize}
  \item We call a variable of type
    \lstinline{int}, \lstinline{float}, \lstinline{double} a \indextermsub{numerical}{variable}.
  \item \emph{Complex numbers}\index{complex numbers} will be
    discussed later.
  \item 
    For characters: \lstinline{char}. Strings are complicated; see later.
  \item Truth values: \lstinline{bool}
  \item 
    You can make your own types. Later.
  \end{itemize}
\end{block}

For complex numbers see section~\ref{sec:stl-complex}.
For strings see chapter~\ref{ch:string}.

\Level 2 {Integers}

Mathematically, integers are a special case of real numbers.
In a computer, integers are stored very differently from
real numbers --~or technically, floating point numbers.

You might think that C++ integers are stored as binary number with a sign  bit,
but the truth is more subtle.
For now, know that 
within a certain range, approximately symmetric around zero,
all integer values can be represented.

\begin{exercise}{Integer range}
  \label{ex:int-range}
  These days, the default amount of storage for an \lstinline{int} is
  32 bits. After one bit is used for the sign, that leave 31 bits for
  the digits. What is the representable integer range?
\end{exercise}

The integer type in C++ is \indexcdef{int}:
\begin{lstlisting}
int my_integer;
my_integer = 5;
cout << my_integer << "\n";
\end{lstlisting}

For more integer types, see chapter~\ref{sec:scalar};
if you're wondering how large integers and such can get,
see section~\ref{sec:limits} in particular.

%%packtsnippet intbinoct

Integer constants can be represented on several \indextermp{base}.

\begin{block}{Integer constants}
  \label{sl:intvals}
  Integers are normally  written in decimal, and stored in 32 bits.
  If you need something else:
\begin{lstlisting}
int d = 42;
int o = 052; // start with zero
int x = 0x2a;
int X = 0X2A;
int b = 0b101010;
long ell = 42L;
\end{lstlisting}
\end{block}

Binary numbers are new to \cppstandard{17}.

%%packtsnippet end

\Level 2 {Floating point numbers}

\emph{Floating point number}\index{floating point number|textbf}
is the computer science-y name for scientific notation:
a number written like
\[ +6\cdot 022 \times 10^{23} \]
with:
\begin{itemize}
\item an optional sign;
\item an integral part;
\item a decimal point, or more generally \indextermdef{radix point}
  in other number bases;
\item a fractional part, also known as \indextermdef{mantissa}
  or \indextermdef{significand};
\item and an \indextermdef{exponent part}: base to some power.
\end{itemize}

Floating point numbers are by default of type \lstinline{double},
which standard for `double precision'. Double of what?
We will discuss that in section~\ref{sec:limits}.
For now, let's discuss only the matter of how they are represented.

Without further specification, a floating point literal is of type \lstinline{double}:
\begin{lstlisting}
1.5
1.5e+5
\end{lstlisting}
Use a suffix  \lstinline{1.5f} for type \lstinline{float} which stands for `single precision':
\begin{lstlisting}
1.5f
1.5e+5f
\end{lstlisting}
Use a suffix  \lstinline{1.5L} for \lstinline{long double}: quadruple precision.
\begin{lstlisting}
1.5L
1.5e+5L
\end{lstlisting}

\begin{slide}{Floating point constants}
  \label{sl:float-vars}
  \begin{itemize}
  \item Default: \lstinline{double}
  \item Float: \lstinline{3.14f} or \lstinline{3.14F}
  \item Long double: \lstinline{1.22l} or \n{1.22L}.
  \end{itemize}
\end{slide}

There is a way to give a 
hexadecimal representation of floating points number, but this is complicated.

\Level 3 {Storage sizes}

%%packtsnippet floattypes

The precise definitions of the floating point types are as follows.

The most used floating point types are:
\begin{itemize}
\item
  \indexc{float}, the IEEE 32-bit single precision type,
\item
  \indexc{double}, the IEEE 64-bit double precision type,
\item
  \indexc{long double}, the IEEE quadruple precision type.
\end{itemize}

The \cppstandard{23} standard added (optional)
\emph{fixed width floating-point}\index{floating-point!fixed width type}
in the \indexheader{stdfloat} header:
\begin{lstlisting}
float16_t // 16-bit half precision
float32_t // 32-bit single precision
float64_t // 64-bit double precision
float128_t // 128-bit double precision

bfloat16_t // `bfloat' half precision
\end{lstlisting}

Note that there are two 16-bit floating point types.
The \indexc{float16_t} is the type originally defined in the IEEE 754 standard;
the \indexc{bfloat16_t} is the IEEE 32-bit type with the last two bytes truncated.
This makes sense in \ac{ML} applications, both from a precision point of view,
and for increased bandwidth over using 32-bit floats.

\begin{remark}
  The half and quad precision types may be available in a language implementation,
  but without hardware support, which would make their execution very slow.
\end{remark}

%%packtsnippet end

\Level 3 {Limitations}

Floating point numbers are also referred to as `real numbers'
(in fact, in the Fortran language they are defined
with the keyword \lstinline{Real}),
but this is sloppy wording.
Since only a finite number of bits/digits is available,
only terminating fractions are representable.
For instance, since computer numbers are binary,
$1/2$~is representable but $1/3$~is not.

\begin{exercise}
  \label{ex:float-irrat}
  Can you think of a way that non-terminating fractions,
  including such numbers such as $\sqrt 2$,
  would still be representable?
\end{exercise}

\begin{itemize}
\item You can assign variables of one type to another, but this may
  lead to truncation (assigning a floating point number to an integer)
  or unexpected bits (assigning a single precision floating point
  number do a double precision).
\end{itemize}

Floating points numbers do not behave like mathematical numbers;
for extensive discussion, see \HPSCref{sec:real-numbers} and later.

\begin{block}{Warning: floating point arithmetic}
  \label{sl:float-arith}
  Floating point arithmetic is full of pitfalls.
  \begin{itemize}
  \item Don't count on \lstinline{3*(1./3)} being exactly~1.
  \item Not even associative.
  \end{itemize}
\end{block}

\begin{comment}
  The following exercise illustrates another point about computer numbers.

  \begin{exercise}
    \label{ex:macheps}
    Define 
    \begin{lstlisting}
      float one = 1.;
    \end{lstlisting}
    and
    \begin{enumerate}
    \item Read a \lstinline{float eps},
    \item Make a new variable that has the value \lstinline{one+eps}. Print
      this.
    \item Make another variable that is the previous one minus
      \lstinline{one}. Print the result again.
    \item Test your program with \lstinline{.001}, \lstinline{.0001}, \lstinline{.00001},
      \lstinline{000001}. Do you understand the result?
    \end{enumerate}
  \end{exercise}
\end{comment}

Complex numbers exist, see section~\ref{sec:stl-complex}.

\Level 2 {Boolean values}

\begin{block}{Truth values}
  \label{sl:bool-var}
  So far you have seen integer and real variables. There are also
  \indextermsub{boolean}{values} which represent truth values. There are
  only two values: \indexc{true} and \indexc{false}.
\begin{lstlisting}
bool found{false};
found = true;
\end{lstlisting}
\end{block}

\Level 0 {Input/Output, or I/O as we say}
\label{sec:io}

A program typically produces output. For now we will only display
output on the screen, but output to file is possible too.  Regarding
input, sometimes a program has all information for its computations,
but it is also possible to base the computation on user input.

\begin{block}{Terminal output}
  \label{sl:cout}
Terminal (console) output with \indexc{cout}:
\begin{lstlisting}
float x = 5;
cout << "Here is the root: " << sqrt(x) << '\n';
\end{lstlisting}
Note the newline character.\\
Alternatively: \lstinline{std::endl}, less efficient.
\end{block}

You can get input from the keyboard with \indexcdef{cin},
which accepts arbitrary strings, as long they don't have spaces.

\begin{block}{Terminal input}
  \label{sl:cin}
\begin{multicols}{2}
  \verbatimsnippet{cinspace}
  \vfill
  \columnbreak
      \begin{mdframed}[backgroundcolor=yellow!80!white!20]%{quote}
        \codesize
\begin{verbatim}
> ./cin
Your name?
Victor
age?
18
18 is a nice age, Victor
> ./cin
Your name?
THX 1138
age?
1138 is a nice age, THX
\end{verbatim}
      \end{mdframed}
\end{multicols}
\end{block}

For more flexible input, see section~\ref{sec:termin}.

\begin{slide}{Terminal input}
  \label{sl:cin-more}
  \indexc{cin} is limited.
  There is also \lstinline{getline}, which is more general.
\end{slide}

For fine-grained control over the output, see section~\ref{sec:iomanip}.
For other I/O related matters, such as file I/O, see chapter~\ref{ch:io}.

\Level 0 {Expressions}
\label{sec:expr}

The most basic step in computing is to form expressions such as sums,
products, logical conjunctions, string concatenations
from variables and constants.

Let's start by discussing constants:
numbers, truth values, strings.

\Level 1 {Numerical expressions}

Expressions in
programming languages for the most part look the way you would expect
them to.
\begin{itemize}
\item Mathematical operators: \n{+ - /} and \n{*}~for multiplication.
\item Integer modulus:~\n{5\char`\%2}
\item You can use parentheses: \n{5*(x+y)}. Use parentheses if you're
  not sure about the precedence rules for operators.
\item C++ does not have a power operator (Fortran does): `Power' and
  various mathematical functions are realized through library calls.
\end{itemize}

\begin{block}{Math library calls}
  \label{sl:cmath}
  Math functions are in \indexc{cmath}:
\begin{lstlisting}
#include <cmath>
.....
x = pow(3,.5);
\end{lstlisting}
For squaring, usually better to write \n{x*x} than \n{pow(x,2)}.
\end{block}

\begin{slide}{Arithmetic expressions}
  \label{sl:arith-expr}
  \begin{itemize}
  \item
    Expression looks pretty much like in math.\\
    With integers: \n{2+3}\\
    with reals: \n{3.2/7}
  \item Use parentheses to group \n{25.1*(37+42/3.)}
  \item Careful with types.
  \item There is no `power' operator: library functions.
  \item Modulus: \n{\char`\%}
  \end{itemize}
\end{slide}

\begin{exercise}
  \label{ex:cin-cout3np1}
  Write a program that :
  \begin{itemize}
  \item displays the message \n{Type a number},
  \item accepts an integer number from you (use~\lstinline{cin}),
  \item makes another variable that is three times that integer plus one,
  \item and then prints out the second variable.
  \end{itemize}
\end{exercise}

Fine points of integers and integer expression
are discussed in section~\ref{sec:more-ints}.

\Level 1 {Truth values}

In addition to numerical types, there are truth values,
\indexc{true} and \indexc{false}, with all the usual logical
operators defined on them.

\begin{block}{Boolean expressions}
  \label{sl:bool-expr}
  \begin{itemize}
  \item Relational operators: \n{== != < > <= >=}
  \item Boolean operators: \n{not, and, or} (oldstyle:  \n{! && ||});
  \end{itemize}
\end{block}

\Level 1 {Type conversions}

Since a variable has one type, and will always be of that type,
you may wonder what happens with
\begin{lstlisting}
float x = 1.5;
int i;
i = x;
\end{lstlisting}
or 
\begin{lstlisting}
int i = 6;
float x;
x = i;
\end{lstlisting}

\begin{itemize}
\item Assigning a floating point value to an integer truncates the
  latter.
\item Assigning an integer to a floating point variable fills it up
  with zeros after the decimal point.
\end{itemize}

\begin{exercise}
  \label{ex:float-convert}
  Try out the following:
  \begin{itemize}
  \item What happens when you assign a positive floating
    point value to an integer variable?

   What happens when you assign a negative floating
    point value to an integer variable?

    Does your compiler give warnings?
    Is there a way you can trick the compiler into not understanding what you are doing?
  \item What happens when you assign a \lstinline{float} to a \lstinline{double}?
    Try various numbers for the original float. 
    Print out the result, and if they look the same, see if the difference is actually zero.
  \end{itemize}
\end{exercise}

The rules for type conversion in expressions are not entirely
logical. Consider
\begin{lstlisting}
float x; int i=5,j=2;
x = i/j;
\end{lstlisting}
This will give~\n{2} and not~\n{2.5}, because \n{i/j} is an integer
expression and is therefore completely evaluated as such, giving~\n{2}
after truncation. The fact
that it is ultimately assigned to a floating point variable does not
cause it to be evaluated as a computation on floats.

You can force the expression to be computed in floating point numbers
by writing
\begin{lstlisting}
x = (1.*i)/j;
\end{lstlisting}
or any other mechanism that forces a conversion, without changing the
result.  Another mechanism is the \indexterm{cast}; this will be
discussed in section~\ref{sec:cast}.

\begin{slide}{Conversion and casting}
  \label{sl:convert-cast}
  Real to integer: round down:
\begin{lstlisting}
  double x,y; x = .... ; y = .... ;
  int i; i =  x+y:
\end{lstlisting}
Dangerous:
\begin{lstlisting}
  int i,j; i = ... ; j = ... ; 
  double x ; x = 1+i/j;
\end{lstlisting}
The fraction is executed as integer division.\\
For floating point result do:
\begin{lstlisting}
(1.*i)/j
\end{lstlisting}
\end{slide}

\begin{exercise}
  \label{ex:modulus}
  Write a program that asks for two integer numbers \n{n1,n2}.
  \begin{itemize}
  \item Assign the integer ratio $n_1/n_2$ to an integer variable.
  \item Can you use this variable to compute the modulus
    \[ n_1\mod n_2 \]
    (without using the \n{\char`\%} modulus operator!)\\
    Print out the value you get.
  \item Also print out the result from using the modulus
    operator:\n{\char`\%}.
  \item Investigate the behavior of your program for negative
    inputs. Do you get what you were expecting?
  \end{itemize}
\end{exercise}

\begin{exercise}
  \label{ex:C2F}
  Write two programs, one that reads a temperature in Centigrade and
  converts to Fahrenheit, and one that does the opposite conversion.
  \[ C = (F-32)\cdot 5/9,\qquad F = 9/5\,C+32 \]
  Check your program for the freezing and boiling point of water.\\
  (Do you know the temperature where Celsius and Fahrenheit are the
  same?)
  
  Can you use Unix pipes to make one accept the output of the other?
\end{exercise}

\begin{review}
  \label{q:vartypes}
  True of false?
  \begin{enumerate}
  \item Within a certain range, all integers are available as values of an
    integer variable.
  \item Within a certain range, all real numbers are available as values of a
    float variable.
  \item \verb-5(7+2)- is equivalent to~\n{45}.
  \item \verb+1--1+ is equivalent to zero.
  \item \verb-int i = 5/3.;- The variable \n{i} is~2.
  \item \verb-float x = 2/3;- The variable \n{x} is approximately~\n{0.6667}.
  \end{enumerate}
\end{review}

\Level 1 {Characters and strings}

In this course we are mostly concerned with numerical data,
but string and character data can be useful for purposes of output.

\Level 2 {Strings}

Strings, that is, strings of characters, are not a C++ built-in
datatype. Thus, they take some extra setup to use.
See chapter~\ref{ch:string} for a full discussion.

For characters there is the \lstinline{char} data type,
and for strings \lstinline{string},
If you want to use strings:

\begin{block}{Quick intro to strings}
  \label{sl:quick-string}
  \begin{itemize}
  \item Add the following at the top of your file:
\begin{lstlisting}
#include <string>
using std::string;
\end{lstlisting}
\item Declare string variables as
\begin{lstlisting}
string name;
\end{lstlisting}
\item And you can now \lstinline{cin} and \lstinline{cout} them.
  \end{itemize}
\end{block}

A character is enclosed in single quotes:
\begin{lstlisting}
'x'
\end{lstlisting}
while a general string is enclosed in double quotes:
\begin{lstlisting}
"The quick brown fox"
\end{lstlisting}

\begin{exercise}
  \label{ex:ask-for-name}
  Write a program that asks for the user's first name, uses
  \lstinline{cin} to read that, and prints
  something like \n{Hello, Susan!} in response.

  What happens if you enter first and last name?
\end{exercise}

\Level 0 {Advanced topics}

\Level 1 {The main program and the return statement}
%%packtsnippet intmain
\label{sec:int-main}

The \indexc{main} program has to be of type \lstinline{int};
however, many compilers tolerate deviations from this,
for instance accepting \lstinline{void},
which is not language standard.

The arguments to main can be:
\begin{lstlisting}
int main() 
int main( int argc,char* argv[] ) 
int main( int argc,char **argv ) 
\end{lstlisting}

The \lstinline{argc/argv} variables contain the commandline
as a set of strings.
\begin{itemize}
\item \lstinline{argc} is the number of strings: the name of the program,
  and the number of space-separated arguments;
\item \lstinline{argv} contains the \indexterm{commandline arguments}
  as an array of strings.
\end{itemize}
You might be tempted to parse the commandline yourself,
but there are dedicated libraries for this;
see section~\ref{sec:cxxopts}.

The returned \lstinline{int} can be specified several ways:
\begin{itemize}
\item If no \indexc{return} statement is given,
  implicitly \lstinline+return 0+ is performed.
  %% (This is also true in~\cstandard{99}.)
\item If you explicitly use \lstinline{return} with an integer value.
\item Instead of an explicit integer value you can use the values
  \indexc{EXIT_SUCCESS} and \indexc{EXIT_FAILURE}
  which are defined in \indextermtt{cstdlib}.
  In general, zero indicates success, while a nonzero value
  indicates failure.
\item You can also use the \indexc{exit} function:
\begin{lstlisting}
void exit(int);
\end{lstlisting}
\end{itemize}

The point of having a return code is that it 
is passed to the operating system as a \emph{return code},
which can then be queried in the 
\emph{shell}\index{shell!inspect return code}.

\begin{block}{Query return code}
  \renewcommand\snippetcodefraction{.4}
  \renewcommand\snippetanswfraction{.55}
  \snippetwithoutput{returnone}{basic}{return}
\end{block}

\begin{slide}{Return statement}
  \label{sl:main-int}
  \begin{itemize}
  \item The language standard says that \lstinline{main} has to be of type
    \lstinline{int}; the \indextermbus{return}{statement} returns an int.
  \item Compilers are fairly tolerant of deviations from this.
  \item Usual interpretation: returning zero means success; anything else failure;
  \item This \indextermbus{return}{code} can be detected by the
    \emph{shell}\index{shell!inspect return code}
  \end{itemize}
  \snippetwithoutput{returnone}{basic}{return}
\end{slide}

%%packtsnippet end
%% \input intmain

\Level 1 {Identifier names}
\label{sec:unicode}

Variable names, or more correctly: \indexterm{identifier}s,
have to start with a non-digit. To be precise, this can be
\begin{itemize}
\item
  a Latin letter, which is the most common case;
\item an \indexterm{underscore}, which is the convention for private members of a class, and other `internal' names; or 
\item \indexterm{Unicode} characters of class \texttt{XID\_Start}.
\end{itemize}
Any following character can be Unicode characters of class \texttt{XID\_Continue}.

On the topic of underscores, a leading \indextermsub{double}{underscore} should not be used
since such names are reserved for the compiler.

\begin{remark}
  General Unicode characters became allowed in \cppstandard{23}, but
  this convention was then applied retro-actively to earlier standards.
\end{remark}

\begin{comment}
  \Level 1 {Number values and undefined values}

  A computer allocates a fixed amount of space for integers and floating
  point numbers, typically 4 or 8 bytes. That means that not all numbers
  are representable.
  \begin{itemize}
  \item Using 4 bytes, that is 32 bits, we can represent $2^{32}$
    integers. Typically this is the range $-2^{31}\ldots 0 \ldots
    2^{31}-1$.
  \item Floating point numbers are represented by a sign bit, an
    exponent, and a number of significant digits.
    For 4-byte numbers, the number of significant (decimal) digits is
    around~6; for 8-byte numbers it is around 15.
  \end{itemize}

  If you compute a number that `fall in between the gaps' of the
  representable numbers, it gets truncated or rounded. The effects of
  this on your computation constitute its own field of numerical
  mathematics, called \indexterm{roundoff error analysis}.

  If a number goes outside the bounds of what is representable, it
  becomes either:
  \begin{itemize}
  \item \indextermtt{Inf}: infinity. This happens if you add or multiply
    enough large numbers together. There is of course also a value
    \n{-}\indextermtt{Inf}. Or:
  \item \indextermtt{NaN}: not a number. This happens if you subtract
    one \lstinline{Inf} from another, or do things such as taking the
    root of a negative number.
  \end{itemize}
  Your program will not be interrupted if a \lstinline{NaN} or \lstinline{Inf} is
  generated: the computation will merrily (and at full speed) progress
  with these numbers. See section~\ref{sec:limits} for detection of such quantities.

  Some people advocate filling uninitialized memory with such illegal
  values, to make it recognizable as such.
\end{comment}

\Level 0 {C differences}

\Level 1 {Boolean}

Traditionally, C~did not have a type for boolean values;
instead \lstinline{int} and \lstinline{short} was used,
where zero was false, and any nonzero value true.
In \cstandard{99} the type \indexc{_Bool} was introduced.
This only serves legibility: there are no true/false constants,
and variables of type \indexc{_Bool} still have to be treated as
integers in \indexc{printf}.

\verbatimsnippet{cbool}

The \indextermheader{stdbool.h} defines \lstinline{bool},
\lstinline{true}, and \lstinline{false} as aliases.

\Level 0 {Review questions}

\begin{review}
  \label{ex:cpp-mod}
What is the output of:
\begin{lstlisting}
int m=32, n=17;
cout << n%m << "\n";
\end{lstlisting}
\end{review}

\begin{review}
  \label{ex:cpp-cube}
  Given
\begin{lstlisting}
int n;
\end{lstlisting}
give an expression that
uses elementary mathematical operators to compute n-cubed: $n^3$.
Do you get the correct result for all~$n$? Explain.

How many elementary operations does the computer perform to compute
this result?

Can you now compute $n^6$, minimizing the number of operations the
computer performs?
\end{review}
