% -*- latex -*-
%%%%%%%%%%%%%%%%%%%%%%%%%%%%%%%%%%%%%%%%%%%%%%%%%%%%%%%%%%%%%%%%
%%%%
%%%% This TeX file is part of the course
%%%% Introduction to Scientific Programming in C++/Fortran2003
%%%% copyright 2017-2022 Victor Eijkhout eijkhout@tacc.utexas.edu
%%%%
%%%% walk.tex : random walk
%%%%
%%%%%%%%%%%%%%%%%%%%%%%%%%%%%%%%%%%%%%%%%%%%%%%%%%%%%%%%%%%%%%%%

This section discusses performance and code optimization issues
in the context of a random walk exercise.

\Level 0 {Problem statement}

In 1904, Sir Ronald Ross, the biologist who discovered that malaria was carried
by mosquitos, gave a lecture on
`The Logical Basis of the Sanitary Policy of Mosquito Reduction'.
In it, he considered the problem of how far a mosquito can fly,
and therefore, how far away you need to drain all pools that harbor them.

We can model a mosquito as flying some unit distance in each time period,
say, a day, of its life.
However, since mosquitos fly in random directions, it will
not cover a distance of $N$ in the $N$~days of its life.
So how far does it get, statistically?

Ros was only able to compute this for a one-dimensional mosquito,
that is, one that can only decide to go forward or backward along a line.
In that case, the mosquito will on average get $\sqrt N$ away from where it starts.

The more general problem was brought to mathematicians' attention in 1905
by Karl Pearson, and turned out to have been solved in 1880 by Lord Rayleigh.
This is now known as a \indextermdef{random walk} problem.
Somewhat surprisingly, in 2D a mosquito also is expected to travel~$\sqrt N$,
where $N$~is the number of time steps.

The general $d$-dimensional problem is a little harder, and the mosquito
travels a little less than~$\sqrt N$.
Let's code this in all generality.

\Level 0 {Coding}

Main program setup:

\snippetwithoutput{walkrun}{rand}{vec}

where the \lstinline{Mosquito} class stores its position:

\verbatimsnippet{mosqclass}

and the \lstinline{step} method updates this:

\verbatimsnippet{walkstep}

The random step method produces a random coordinate,
normalized to the unit circle.
There is a slight problem here:
if we generate a random coordinate in the unit cube,
and normalize it, it will be biased towards the corners of the cube.
Therefore, we iterate until we have a coordinate inside the unit circle,
and use that to be normalized:
%
\verbatimsnippet{walkrandcoord}
\verbatimsnippet{walkrandstep}

\begin{exercise}
  Take the basic code, and make a version based on
\begin{lstlisting}
template<int d>
class Mosquito { /* ... */
\end{lstlisting}
  How much does this simplify your code? Do you get any performance improvement?
  \skeleton{walk_vec}
\end{exercise}

\Level 1 {Optimization: save on allocation}

Probably the main problem with this implementation
is that each step creates multiple vectors.
This sort of memory management is relatively costly,
especially since very few operations are performed
on each of these.

So we move the creation of the vectors outside of the computational routines.
The random coordinates are now written into an array passed as parameter:
%
\verbatimsnippet{passrandcoord}

Likewise the random step:
%
\verbatimsnippet{passrandstep}

This process of passing the arrays in
stops at the \lstinline{step} method,
which we want to keep parameterless.
So we add an option \lstinline{cache} to the constructor
to store the step vector as well as the position:
%
\snippetwithoutput{mosqclasscache}{rand}{pass}
%\verbatimsnippet{mosqclasscache}

\verbatimsnippet{mosqclasscachestep}

\Level 1 {Caching in a static vector}

There is still a problem with the \lstinline{length} calculation.
Since there is no reduction operator for `sum of squares',
we need to create a temporary vector for the squares,
so that we can do a plus-reduction on it.

\begin{exercise}
  Explore options for this temporary. Discuss what's most elegant,
  and measure performance improvement.
  \begin{itemize}
  \item
    This temporary can be passed in as a parameter;
  \item
    it can be stored in a global variable;
  \item 
    or we can declare it \lstinline{static}.
  \item With the \cppstandard{20} standard,
    you could also use the \indextermheader{ranges} header.
  \end{itemize}
\end{exercise}

\Level 0 {Vector vs array}

In this simulation, we will mostly be working in 2D,
so instead of using \lstinline{vector},
we can use statically allocated \lstinline{array} objects.
This allows for the more elegantly functional interface:
%
\verbatimsnippet{walklenghtarr}

While above we have removed all unnecessary allocation,
we get an extra performance boost from optimizations
from the compiler knowing the length of the array.
Thus, instead of a loop of length two,
the compiler will probably replace this by
two explicit instructions.

\snippetwithoutput{walkrunarr}{rand}{arr}
