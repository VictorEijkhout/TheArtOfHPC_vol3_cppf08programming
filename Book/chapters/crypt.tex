% -*- latex -*-
%%%%%%%%%%%%%%%%%%%%%%%%%%%%%%%%%%%%%%%%%%%%%%%%%%%%%%%%%%%%%%%%
%%%%
%%%% This TeX file is part of the course
%%%% Introduction to Scientific Programming in C++/Fortran2003
%%%% copyright 2019 Victor Eijkhout eijkhout@tacc.utexas.edu
%%%%
%%%% crypt.tex : exercises for cryptography
%%%%
%%%%%%%%%%%%%%%%%%%%%%%%%%%%%%%%%%%%%%%%%%%%%%%%%%%%%%%%%%%%%%%%

\Level 0 {The basics}

While floating point numbers can span a rather large range
--~up to $10^{300}$ or so for double precision~--
integers have a much smaller one: up to about~$10^9$.
That is not enough to do cryptographic applications,
which deal in much larger numbers.
(Why can't we use floating point numbers?)

So the first step is to write classes
\lstinline{Integer} and \lstinline{Fraction}
that have no such limitations.
Use operator overloading to make simple expressions work:
\begin{lstlisting}
Integer big=2000000000; // two billion
big *= 1000000; bigger = big+1;
Integer one = bigger % big;
\end{lstlisting}

\begin{exercise}
  Code Farey sequences.

\begin{comment}
    A Farey sequence FN of degree N (or: the Farey fractions of degree N) is an ordered set of reduced
    fractions $\frac{p_i}{q_i}$
    with $p_i\leq q_i \leq N$ and
    \[ WRONG 0\leq i<|F_N|\colon p_i <p_j \forall 0\leq i<j<|FN|
    qi qj \]

    Use the class Rational from the previous exercise to write a function
    void Farey(int N)
    which calculates the Farey fractions up to degree N and prints the resulting Farey sequences up to
    degree N on the screen.

    Algorithm: The sequences can be computed recursively. The first sequence is given by
    F1 =􏰚0,1􏰛 11

    For a known sequence FN one can get FN+1 by inserting an additional fraction pi+pi+1 between two qi +qi+1
    consecutive entries pi and pi+1 if qi + qi+1 = N + 1 holds for the sum of denominators. qi qi+1
    Example: Determining F7 from F6 results in the following construction:

    F6 = 􏰚 0 , 1 , 1 , 1 , 1 , 2 , 1 , 3 , 2 , 3 , 4 , 5 , 1 􏰛 1654352534561 􏰞􏰝􏰜􏰟 􏰞􏰝􏰜􏰟 􏰞 􏰝􏰜 􏰟 􏰞􏰝􏰜􏰟 􏰞􏰝􏰜􏰟
    1 2 3and4 5 6 777777
    The new elements are:
    0+1=1 ; 1+1=2 ; 2+1=3 ; 1+3=4 ; 2+3=5 ; 5+1=6
    1+6 7 4+3 7 5+2 7 2+5 7 3+4 7 6+1 7
    The sorted sequence then is:
    F7 =􏰚0,1,1,1,1,2,1,2,3,1,4,3,2,5 3,4,5,6,1􏰛
    For checking:
    1765473572753745671
    There is a beautiful illustration of these fractions,
    The Farey sequences up to degree 6 F1 = 􏰚0,1􏰛
    11
    F2 = 􏰚0,1,1􏰛 121
    F3 = 􏰚0,1,1,2,1􏰛 13231
    F4 = 􏰚0,1,1,1,2,3,1􏰛 1432341
    F5 = 􏰚0,1,1,1,2,1,3,2,3,4,1􏰛 15435253451
    F6 = 􏰚0,1,1,1,1,2,1,3,2,3,4,5,1􏰛. 1654352534561
    the Ford circlesa:
\end{comment}

\end{exercise}

\Level 0 {Cryptography}

\url{https://simple.wikipedia.org/wiki/RSA_algorithm}

\url{https://simple.wikipedia.org/wiki/Exponentiation_by_squaring}

\Level 0 {Blockchain}

Implement a blockchain algorithm.
