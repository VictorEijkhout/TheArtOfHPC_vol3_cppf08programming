% -*- latex -*-
%%%%%%%%%%%%%%%%%%%%%%%%%%%%%%%%%%%%%%%%%%%%%%%%%%%%%%%%%%%%%%%%
%%%%
%%%% This TeX file is part of the course
%%%% Introduction to Scientific Programming in C++11/Fortran2003
%%%% copyright 2017-2020 Victor Eijkhout eijkhout@tacc.utexas.edu
%%%%
%%%% climate-abstract.tex : abstract for the climate change project
%%%%
%%%%%%%%%%%%%%%%%%%%%%%%%%%%%%%%%%%%%%%%%%%%%%%%%%%%%%%%%%%%%%%%

Climate change is a scientific fact.
But what science, exactly?

This project lets students do a statistical analysis
on observational climate data.
Here the statement that `climate always changes' is given
a testable form, and the students' code is used to actually
explore this form of the statement.

The project description targets a language with array syntax.
Engineering students can code this in Fortran, but
languages such as Matlab and Julia can also be used.

This project can be done by one or two undergraduate students,
or AP high schoolers
at the end of a first or semester programming course.

\newpage

\begin{tabular}{|l|p{5in}|}
  \hline
  Summary&Do a statistical analysis of observational climate data.
  \\
  Topics&Array manipulation
  \\
  Audience&Undergraduate or AP high school
  \\
  Difficulty&Moderate
  \\
  Prerequisites&Some statistics and curve fitting
  \\
  Strengths&Explores a question of great societal impact.
  \\ &Use of real-world data
  \\
  Weaknesses&The mathematical prerequisite can be limiting,
  or require extra time for preparation.
  \\
  Dependencies&Mathematical prerequisite: regression/curve fitting.
  \\
  \hline
\end{tabular}
