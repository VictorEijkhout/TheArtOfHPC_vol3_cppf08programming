% -*- latex -*-
%%%%%%%%%%%%%%%%%%%%%%%%%%%%%%%%%%%%%%%%%%%%%%%%%%%%%%%%%%%%%%%%
%%%%
%%%% This TeX file is part of the course
%%%% Introduction to Scientific Programming in C++/Fortran2003
%%%% copyright 2017-2023 Victor Eijkhout eijkhout@tacc.utexas.edu
%%%%
%%%% extern.tex : about external packages
%%%%
%%%%%%%%%%%%%%%%%%%%%%%%%%%%%%%%%%%%%%%%%%%%%%%%%%%%%%%%%%%%%%%%

%%packtsnippet libintro

If you have a C++ compiler,
you can write as much software as your want, by yourself.
However, some things that you may need for your work
have already been written by someone else.
How can you use their software?

\Level 0 {What are software libraries?}
\label{sec:makelib}

In this chapter you will learn about the use of
\emph{software libraries}\index{library!software}\index{software library|see{library, software}}:
software that is written not as a standalone package,
but in such a way that you can access its functionality
in your own program.

Software libraries can be enormous,
as is the case for scientific libraries,
which are often multi-person multi-year projects.
On the other hand, many of them
are fairly simple utilities written by a single programmer.
In the latter case you may have to worry about
future support of that software.

\Level 1 {Using an external library}

Using a software library typically means that
\begin{itemize}
\item your program has a line
\begin{lstlisting}
#include "fancylib.h"
\end{lstlisting}
\item and you compile and link as:
\begin{verbatim}
icpc -c yourprogram.cpp -I/usr/include/fancylib
icpc -o yourprogram yourprogram.o -L/usr/lib/fancylib -lfancy
\end{verbatim}
\end{itemize}
You will see specific examples below.

If you are now worried about having to do a lot of typing every time you compile,
\begin{itemize}
\item if you use an \ac{IDE}, you can typically add the library in the options,
  once and for all; or
\item you can use \indexterm{Make} for building your program.
  See the tutorial.
\end{itemize}

\begin{slide}{External libraries: usage}
  \label{sl:lib-use}
  Suppose the `fancy' library does what you need.
  \begin{enumerate}
  \item
    Include a header file
  \item Then use the functions defined there.
  \end{enumerate}

\begin{lstlisting}
#include "fancylib.h"

int main() {
  x = fancyfunction(y);
}
\end{lstlisting}
\end{slide}

\begin{slide}{External libraries: compile}
  \label{sl:lib-compile}
  \begin{enumerate}
  \item Compiler needs to know where the header is:
\begin{verbatim}
icpc -c yourprogram.cpp -I/usr/include/fancylib
\end{verbatim}
\item You may need to link a library file:
\begin{verbatim}
icpc -o yourprogram yourprogram.o \
    -L/usr/lib/fancylib -lfancy
\end{verbatim}
(not for `header only' libraries)
  \end{enumerate}
\end{slide}

\Level 1 {Obtaining and installing an external library}

Sometimes a software library is available through a \indexterm{package manager},
but we are going to do it the old-fashioned way:
downloading and installing it ourselves.

A popular location for finding downloadable software is \url{github.com}.
You can then choose whether to
\begin{itemize}
\item clone the repository, or
\item download everything in one file, typically with \n{.tgz} or \n{.tar.gz} extension;
  in that case you need to unpack it
\begin{verbatim}
tar fxz fancylib.tgz
\end{verbatim}
This usually gives you a directory with a name such as
\begin{verbatim}
fancylib-1.0.0
\end{verbatim}
containing the source and documentation of the library,
but not any binaries or machine-specific files.
\end{itemize}
Either way, from here on we assume that you have a directory containing the downloaded package.

There are two main types of installation:
\begin{itemize}
\item based on
  \indextermbus{GNU}{autotools}\index{autotools|see{GNU, autotools}},
  which you recognize by the presence of a \indexterm{configure} program;
\begin{verbatim}
cmake ## lots of options
make 
make install
\end{verbatim}
or
\item based on \indexterm{Cmake},
  which you recognize by the presence of \n{CMakeLists.txt} file:
\begin{verbatim}
configure ## lots of options
make 
make install
\end{verbatim}
\end{itemize}

\Level 2 {Cmake installation}

The easiest way to install a package using \n{cmake} is to create a build directory,
next to the source directory. The \n{cmake} command is issued from this directory,
and it references the source directory:
\begin{verbatim}
mkdir build
cd build
cmake ../fancylib-1.0.0
make 
make install
\end{verbatim}
Some people put the build directory inside the source directory,
but that is bad practice.

Apart from specifying the source location,
you can give more options to \n{cmake}.
The most common are
\begin{itemize}
\item specifying an install location, for instance because you don't have \indexterm{superuser}
  privileges on that machine; or
\item specifying the compiler, because Cmake will be default use the \indextermtt{gcc}
  compilers, but you may want the Intel compiler.
\end{itemize}
\begin{verbatim}
CC=icc CXX=icpc \
cmake \
    -D CMAKE_INSTALL_PREFIX:PATH=${HOME}/mylibs/fancy \
    ../fancylib-1.0.0
\end{verbatim}

%%packtsnippet end

%%packtsnippet cxxopts

\Level 0 {Options processing: \texttt{cxxopts}}
\label{sec:cxxoptlib}

Suppose you have a program that does something with a large array,
and you want to be able to change your mind about the array size.
You could
\begin{itemize}
\item You could recompile your program every time.
\item You could let your program parse \lstinline{argv}, and hope
  you remember precisely how your commandline options are to be interpreted.
\item You could use the \indextermtt{cxxopts} library.
  This is what we will be exploring now.
\end{itemize}

\Level 1 {Traditional commandline parsing}

Commandline options are available to the program through the
(optional)
\indexc{argv} and \indexc{argc} options of the \indexc{main} program.
The former is an array of strings, with the second the length:
\begin{lstlisting}
int main( int argc,char **argv ) { /* program */ };
\end{lstlisting}

For simple cases it would be feasible to parse such options yourself:
\snippetwithoutput{argcv}{args}{argcv}
but this 1. gets tedious quickly 2. is difficult to make robust.
Therefore, we will now see a library that makes such options handling
relatively easy.

\Level 1 {The \texttt{cxxopts} library}
\label{sec:cxxopts}

The \indextermtt{cxxopts} `commandline argument parser'
can be found at \url{https://github.com/jarro2783/cxxopts}.
After a \indexterm{Cmake} installation, it is a `header-only' library.

\begin{itemize}
\item Include the header
\begin{lstlisting}
#include "cxxopts.hpp"
\end{lstlisting}
which requires a compile option:
\begin{verbatim}
-I/path/to//cxxopts/installdir/include
\end{verbatim}
\item Declare an options object:
\begin{lstlisting}
cxxopts::Options options("programname", "Program description");
\end{lstlisting}

\item Add options:
\verbatimsnippet{cxxoptoptions}

\item Add array options
  \verbatimsnippet{cxxoptarray}

\item Add positional arguments:
\verbatimsnippet{cxxoptarguments}

\item Parse the options:
\begin{lstlisting}
auto result = options.parse(argc, argv);
\end{lstlisting}

\item Get help flag first:
  \verbatimsnippet{cxxopthelp}

\item Get result values
  for both options and arguments:
  \verbatimsnippet{cxxoptget}
\end{itemize}

Options can be specified the usual ways:
\begin{verbatim}
myprogram -n 10
myprogram --nsize 100
myprogram --nsize=1000
myprogram --array 12,13,14,15
\end{verbatim}

\begin{exercise}
  Incorporate this package into primality testing:
  exercise~\ref{ex:prime-opts}.
\end{exercise}

Options parsing can throw a
\lstinline{cxxopts::exceptions::option_has_no_value}
exception.

\Level 1 {Cmake integration}

The \indextermtt{cxxopts} package can be discovered in \indexterm{CMake}
through \indextermtt{pkgconfig}:
\begingroup\lstset{language=Bash}
\begin{lstlisting}
find_package( PkgConfig REQUIRED )
pkg_check_modules( CXXOPTS REQUIRED cxxopts )
target_include_directories( ${PROJECT_NAME} PUBLIC ${CXXOPTS_INCLUDE_DIRS} )
\end{lstlisting}
\endgroup

%%packtsnippet end

%%packtsnippet catchshort

\Level 0 {Catch2 unit testing}
\label{sec:catch2}

Test a simple function
\verbatimsnippet{catchfivefun}

Successful test:
\snippetwithoutput{catchtestyes}{catch}{require}

Unsuccessful test:
\snippetwithoutput{catchtestno}{catch}{requirerr}

Function that throws:
\verbatimsnippet{catchevenfun}

Test that it throws or not:
\snippetwithoutput{catchtesteven}{catch}{requireven}

Run the same test for a set of numbers:
\snippetwithoutput{catchtestgen}{catch}{requirgen}
How is this different from using a loop?
Using \lstinline{GENERATE} runs each value as a separate program.

Variants:
\begin{lstlisting}
int i = GENERATE( range(1,100) );
int i = GENERATE_COPY( range(1,n) );
\end{lstlisting}

%%packtsnippet end

\Level 0 {Range libraries}

The \indexheader{ranges} header underwent development
between \cppstandard{20} and \cppstandard{23},
so certain features may not yet be available
in your favorite compiler.
As a stop-gap measure you could install a 3rd party library
that offers the new functionality.

The \indextermdef{ranges-v3} library
(\url{https://github.com/ericniebler/range-v3})
is easily installed through CMake.
