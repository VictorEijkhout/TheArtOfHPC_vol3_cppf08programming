% -*- latex -*-
%%%%%%%%%%%%%%%%%%%%%%%%%%%%%%%%%%%%%%%%%%%%%%%%%%%%%%%%%%%%%%%%
%%%%
%%%% This TeX file is part of the course
%%%% Introduction to Scientific Programming in C++/Fortran2003
%%%% copyright 2017-2021 Victor Eijkhout eijkhout@tacc.utexas.edu
%%%%
%%%% pointf.tex : arrays in Fortran
%%%%
%%%%%%%%%%%%%%%%%%%%%%%%%%%%%%%%%%%%%%%%%%%%%%%%%%%%%%%%%%%%%%%%

Pointers in C/C++ are based on memory addresses; 
Fortran pointers on the other hand, are more
abstract.

\Level 0 {Basic pointer operations}

Fortran pointers are a little like C or C++ pointers,
and they are also different in many ways.
\begin{itemize}
\item Like C `star' pointers, and unlike C++ `smart' pointers, they can point at anything.
\item Unlike C pointers, you have to declare that an object can be pointed at.
\item Unlike any sort of pointer in C/C++, but like C++ references,
  they act as a sort of alias: there is no explicit dereferencing.
\end{itemize}
We will explore all this in detail.

\begin{slide}{Fortran Pointers}
  \label{sl:fpoint}
  \begin{itemize}
  \item A pointer is a variable that points at a variable of some type:\\
    elementary, or derived types. (but not pointers)
  \item You can access and change the value of a variable through a pointer
    that points at it.
  \item You can change what variable the pointer points at.
  \item A pointer acts like an alias: \\
    no explicit dereference needed.
  \end{itemize}
\end{slide}

Fortran pointers act like `aliases': using a pointer variable
is often the same as using the entity it points at.
The difference with actually using the variable, is
that you can decide what variable the pointer points at.

\begin{block}{Dereferencing}
  \label{sl:cpoint-vs-fpoint}
  Fortran pointers are often automatically
  \emph{dereferenced}\index{pointer!dereferencing}: if you print a
  pointer you print the variable it references, not some representation
  of the pointer.
  \snippetwithoutput{pointsetprint}{pointerf}{basicp}
\end{block}

Pointers are defined in a variable declaration
that specifies the type, with the \indexfdef{pointer} attribute.
Examples: the definition
\begin{lstlisting}
real,pointer :: point_at_real
\end{lstlisting}
declares a pointer that can point at a real variable.
Without further specification, this pointer
does not point at anything yet,
so using it is undefined.

\begin{block}{Setting the pointer}
  \label{sl:fpoint-set}
  \begin{itemize}
  \item You have to declare that a variable is pointable:
\begin{lstlisting}
real,target :: x
\end{lstlisting}
\item Declare a pointer:
\begin{lstlisting}
real,pointer :: point_at_real
\end{lstlisting}
\item Set the pointer with \verb+=>+ notation (New! Note!):
\begin{lstlisting}
point_at_real => x
\end{lstlisting}
  \end{itemize}
\end{block}

\begin{block}{Using the pointer}
  \label{sl:fpoint-use}
Now using \n{point_at_real} is the same as using~\n{x}.
\begin{lstlisting}
print *,point_at_real ! will print the value of x
\end{lstlisting}
\end{block}

Pointers can not just point at anything: the thing pointed at needs to
be declared as \indexf{target}
\begin{lstlisting}
real,target :: x
\end{lstlisting}
and you use the \verb+=>+ operator to let a pointer point at a target:
\begin{lstlisting}
point_at_real => x
\end{lstlisting}

If you use a pointer, for instance to print it
\begin{lstlisting}
print *,point_at_real
\end{lstlisting}
it behaves as if you were using the value of what it points at.

\begin{block}{Pointer example}
  \label{sl:fpoint-ex}
  \snippetwithoutput{pointatreal}{pointerf}{realp}
  %
  \begin{enumerate}
  \item \lstinline+that_real+ points at~\n{x}, so the value of \n{x} is printed.
  \item \lstinline+that_real+ is reset to point at~\n{y}, so its value is printed.
  \item The value of \n{y} is changed,
    and since \lstinline+that_real+ still
    points at~\n{y}, this changed value is printed.
  \end{enumerate}
\end{block}

\Level 0 {Combining pointers}

What happens if you point a pointer at another pointer?
The concept of pointer-to-pointer from C/C++ does not exist:
instead, you two pointers pointing at the same thing.

If you have two pointers
\begin{lstlisting}
real,pointer :: point_at_real,also_point
\end{lstlisting}
you can make the target of the one to also be the target of the other:
\begin{lstlisting}
point_at_real => x
also_point => point_at_real  
\end{lstlisting}
Note that the second pointer is also assigned with the \lstinline+=>+ symbol.
This is not a pointer to a pointer: it assigns the target of the
right-hand side to be the target of the left-hand side.

\begin{slide}{Assign pointer from other pointer}
  \label{sl:fpoint-copy}
\begin{lstlisting}
real,pointer :: point_at_real,also_point
point_at_real => x
also_point => point_at_real  
\end{lstlisting}
  Now you have two pointers that point at~\n{x}.

  \textbf{Very important to use the \n{=>}, otherwise strange
    memory errors}
\end{slide}

\begin{block}{Assignment subtleties}
  \label{sl:fpointassign}
  What happens if you want to write \lstinline+p2=>p1+\\
  but you write \lstinline+p2=p1+?\\
  The second one is legal, but has different meaning:
  \begin{multicols}{2}
    Assign underlying variables:
    \verbatimsnippet{passignequals}
    \columnbreak
    Crash because \lstinline{p2} pointer unassociated:
    \verbatimsnippet{passignwrong}
  \end{multicols}
\end{block}

\begin{exercise}
  \label{ex:fpoint-fun}
  Write a routine that accepts an array and a pointer, and on return
  has that pointer pointing at the largest array element:
  %
  \answerwithoutput{arraypointmain}{pointerf}{arpointf}
  \skeleton{arpointf}
\end{exercise}

\Level 0 {Pointer status}

A pointer can be in three states:
\begin{enumerate}
\item a pointer is undefined when it is first created,
\item it can be null, if explicitly set so,
\item or it can be associated if it has been pointed at something.
\end{enumerate}

As a common sense strategy, do not worry about the undefined state:
in the example in section~\ref{sec:linkf} pointer are quickly made null.

\begin{block}{Pointer status}
  \label{sl:fpoint-stat}
  \begin{itemize}
  \item \indexf{Nullify}: zero a pointer
  \item \indexf{Associated}: test whether assigned
  \end{itemize}  
  \snippetwithoutput{pointfstatus}{pointerf}{statusp}
\end{block}

You can also specifically test
\begin{itemize}
\item \lstinline+associated(p,x)+:
  whether the pointer is associated with the variable, or
\item \lstinline+associated(p1,p2)+:
  whether two pointers are associated with the same target.
\end{itemize}

\begin{block}{Pointer allocation}
  \label{sl:fpoint-alloc}
  If you want a pointer to point at something,\\
  but you don't need a variable for that something:
  %
  \snippetwithoutput{fptralloc}{pointerf}{allocptr}

  (Compare \lstinline+make_shared+ in C++)
\end{block}

\Level 0 {Pointers and arrays}

You can set a pointer to an array element or a whole array.
\begin{lstlisting}
real(8),dimension(10),target :: array
real(8),pointer              :: element_ptr
real(8),pointer,dimension(:) :: array_ptr

element_ptr => array(2)
array_ptr   => array
\end{lstlisting}
More surprising, you can set pointers to array sections:
\begin{lstlisting}
array_ptr => array(2:)
array_ptr => array(1:size(array):2)
\end{lstlisting}

In case you're wondering, this does not create temporary arrays, but
the compiler adds descriptions to the pointers, to translate code
automatically to strided indexing.

\begin{block}{Pointer allocation}
  \label{sl:fpoint-dynamic}
  You can use the \indexf{allocate} statement for pointers to arrays:
\begin{lstlisting}
Integer,Pointer,Dimension(:) :: array_point
Allocate( array_point(100) )
\end{lstlisting}  
  This is automatically deallocated when control leaves the scope.
  No memory leaks.
\end{block}

As an even more intersting  example of pointers to array sections,
let's consider the averaging operation
\[ x_{i,j} = ( x_{i+1,j}+x_{i-1,j}+x_{i,j+1}+x_{i,j-1} )/4. \]
We need pointers to the interior and its four offsets:
\verbatimsnippet{fpointinter}

The averaging operation is then an array statement:
\verbatimsnippet{fpointinteravg}

\Level 0 {Example: linked lists}
\label{sec:linkf}
\index{list!linked!in Fortran|(}

For pictures of linked lists, see section~\ref{sec:linklist}.

\begin{block}{Linked list}
  \label{sl:flink1}
  \begin{itemize}
  \item Linear data structure
  \item more flexible than array for insertion~/ deletion
  \item \ldots~but slower in access
  \end{itemize}
\end{block}

One of the standard examples of using pointers is the
\emph{linked list}. This is a dynamic one-dimensional structure
that is more flexible than an array. Dynamically extending an array
would require re-allocation, while in a list an element can be
inserted.

\begin{exercise}
  Using a linked list may be more flexible than using an array.
  On the other hand, accessing an element in a linked list is
  more expensive, both absolutely and as order-of-magnitude in the size
  of the list.

  Make this argument precise.
\end{exercise}

\Level 1 {Type definitions}

A list is based on a simple data structure, a node, which contains a
value field and a pointer to another node.
The list data structure itself only contains a pointer to the first node
in the list.

\begin{block}{Linked list datatypes}
  \label{sl:flinktypes}
  \begin{itemize}
  \item Node: value field, and pointer to next node.
  \item List: pointer to head node.
  \end{itemize}
  \verbatimsnippet{linklistftypes}
\end{block}

By way of example, we create a dynamic list of integers, sorted by
size.
To maintain the sortedness, we need to append or insert nodes,
as required.

\begin{block}{Sample main}
  \label{sl:flinkmain}
  Our main program will create three nodes,
  and append them to the end of the list:
  %
  \snippetwithoutput{linklistfmain}{pointerf}{listappend}
\end{block}

\Level 1 {Attach a node at the end}

First we write a routine \lstinline{attach} that takes
a node pointer and attaches it to the end of a list,
without any sorting.

\verbatimsnippet{listattachfproto}

We distinguish two cases: when the list is empty,
and when it is not.
Initially, the list is empty, meaning that the `head' pointer is
un-associated.
The first time we add an element to the list,
we assign the node as the head of the list:
%
\verbatimsnippet{listattachheadf}

\begin{slide}{List initialization}
  \label{sl:flinkinit}
  \verbatimsnippet{listattachfproto}
  First element becomes the list head:
  %
  \verbatimsnippet{listattachheadf}
\end{slide}

\begin{block}{Attaching a node}
  \label{sl:flink-attach}
  New element attached at the end.
  %  
  \verbatimsnippet{listattachfrecur}
\end{block}

\begin{exercise}
  \label{ex:flist-attach-while}
  Take the recursive code for attaching an element,
  and turn it into an iterative version,
  that is, use a \lstinline{while} loop
  that goes down the list till the end.
\end{exercise}


\Level 1 {Insert a node in sort order}

If we want to keep a list sorted, we need in many cases
to insert the new node at a location short of the end of the list.
This means that instead of iterating to the end,
we iterate to the first node that the new one needs to be attached to.

\begin{block}{Main for inserting}
  \label{sl:flink-insert-main}
  Almost the same as before,
  but now keep the list sorted:
  %
  \snippetwithoutput{linklistfinsert}{pointerf}{listinsert}
\end{block}

\begin{exercise}
  \label{ex:flink-insert}
  Copy the \lstinline{attach} routine to \lstinline{insert},
  and modify it so that inserting a node will keep the list ordered.

  \skeleton{listfappendalloc}
\end{exercise}

\begin{exercise}
  \label{ex:flink-insert-alloc}
  Modify your code from exercise~\ref{ex:flink-insert}
  so that the new node is not allocated in the main program.\\
  Instead, pass only the integer argument,
  and use \indexf{allocate} to create a new node when needed.
  %
  \verbatimsnippet{flistinsertalloc}
\end{exercise}

\begin{exercise}
  \label{ex:flist-print}
  Write a \lstinline{print} function for the linked list.\\
  For the simplest solution, print each element on a line by itself.

  More sophisticated: use the \indexf{Write} function and the
  \indexf{advance} keyword:
\begin{lstlisting}
write(*,'(i1",")',advance="no") current%value  
\end{lstlisting}
\end{exercise}

\begin{exercise}
  \label{ex:flist-length}
  Write a \lstinline{length} function for the linked list.\\
  Try it both with a loop, and recursively.
\end{exercise}

\index{list!linked!in Fortran|)}

