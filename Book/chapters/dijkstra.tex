% -*- latex -*-
%%%%%%%%%%%%%%%%%%%%%%%%%%%%%%%%%%%%%%%%%%%%%%%%%%%%%%%%%%%%%%%%
%%%%
%%%% This TeX file is part of the course
%%%% Introduction to Scientific Programming in C++/Fortran2003
%%%% copyright 2019-2022 Victor Eijkhout eijkhout@tacc.utexas.edu
%%%%
%%%% dijkstra.tex : exercises for graph algorithms
%%%%
%%%%%%%%%%%%%%%%%%%%%%%%%%%%%%%%%%%%%%%%%%%%%%%%%%%%%%%%%%%%%%%%

\Level 0 {Unweighted graphs}

\Level 1 {Auxiliaries}

We need a class \lstinline{Dag} for a \acf{DAG}.
It's probably a good idea to have a function
\begin{lstlisting}
vector<int> Dag::neighbors( int node );
\end{lstlisting}
that, given a node, returns a list of the neighbors of that node.

\Level 1 {Distance algorithm}

For unweighted graphs, the distance algorithm is fairly simple.
Inductively:
\begin{itemize}
\item Assume that we have a set of nodes reachable in at most~$n$ steps,
\item then their neighbors (that are not yet in this set)
  can be reached in~$n+1$ steps.
\end{itemize}

We can implement this in a variety of ways.

The following scheme is relatively easy to write in C++
because classes have a convenient copy constructor:
\begin{tabbing}
  let finished set $F=\{0\}$\\
  until \=no further updates:\\
  \> let current set $C\leftarrow F$\\
  \> for \=$n\in F$:\\
  \>\>if $n\not\in C$, add to $C$\\
  \> $F\leftarrow C$.
\end{tabbing}

\verbatimsnippet{graphdist1}

