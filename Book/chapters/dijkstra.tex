% -*- latex -*-
%%%%%%%%%%%%%%%%%%%%%%%%%%%%%%%%%%%%%%%%%%%%%%%%%%%%%%%%%%%%%%%%
%%%%
%%%% This TeX file is part of the course
%%%% Introduction to Scientific Programming in C++/Fortran2003
%%%% copyright 2019-2023 Victor Eijkhout eijkhout@tacc.utexas.edu
%%%%
%%%% dijkstra.tex : exercises for graph algorithms
%%%%
%%%%%%%%%%%%%%%%%%%%%%%%%%%%%%%%%%%%%%%%%%%%%%%%%%%%%%%%%%%%%%%%

In this project you will explore some common graph algorithms,
and their various possible implementations.
The main theme here will be that the common textbook exposition
of algorithms is not necessarily the best way to phrase them
computationally.

As background knowledge for this project,
you are encouraged to read \HPSCref[chapter]{app:graphalgorithms};
for an elementary tutorial on graphs, see \HPSCref[chapter]{app:graph}.

\Level 0 {Traditional algorithms}

We first implement the `textbook' formulations of two
\indexac{SSSP} algorithms: on unweighted and then on weighted graphs.
In the next section we will then consider formulations
that are in terms of linear algebra.

In order to develop the implementations, we start with some necessary preliminaries,

\Level 1 {Code preliminaries}

\Level 2 {Adjacency graph}

We need a class \lstinline{Dag} for a \acf{DAG}:
\verbatimsnippet{dagunweight}

It's probably a good idea to have a function
\verbatimsnippet{dagnodenbors}
that, given a node, returns a list of the neighbors of that node.

\begin{exercise}
  Finish the \lstinline{Dag} class. In particular,
  add a method to generate example graphs:
  \begin{itemize}
  \item For testing the `circular' graph is often useful:
    connect edges
    \[ 0\rightarrow 1\rightarrow \cdots \rightarrow N-1 \rightarrow 0. \]
  \item It may also be a good idea to have a graph with random edges.
  \end{itemize}
  Write a method that displays the graph.
\end{exercise}

\Level 2 {Node sets}

The classic formulation of \ac{SSSP} algorithms,
such as the \indextermbus{Dijkstra}{shortest path algorithm}
(see \HPSCref{sec:dijkstra})
uses sets of nodes that are gradually built up or depleted.

You could implement that as a vector:
\begin{lstlisting}
vector< int > set_of_nodes(nnodes);
for ( int inode=0; inode<nnodes; inode++)
  // mark inode as distance unknown:
  set_of_nodes.at(inode) = inf;
\end{lstlisting}
where you use some convention, such as negative distance,
to indicate that a node has been removed from the set.

However, C++ has an actual \indexc{set} container with methods
for adding an element, finding it, and removing it; see section~\textbookref{sec:stdset}.
This makes for a more direct expression of our algorithms.
In our case, we'd need a set of int/int or int/float pairs,
depending on the graph algorithm.
(It is also possible to use a \indexc{map}, using an int as lookup key,
and int or float as values.)

For the unweighted graph we only need a set of finished nodes,
and we insert node~0 as our starting point:
\verbatimsnippet{qlevsetup}

For Dijkstra's algorithm we need both a set of finished nodes,
and nodes that we are still working on.
We again set the starting node, and we set the distance
for all unprocessed nodes to infinity:
\verbatimsnippet{qdsetup}
(Why do we need that second set here, while it was not
necessary for the unweighted graph case?)

\begin{exercise}
  Write a code fragment that tests if a node is in the \lstinline{distances} set.
  \begin{itemize}
  \item You can of course write a loop for this.
    In that case know that iterating over a set gives you the key/value pairs.
    Use \indexterm{structured bindings}; section~\textbookref{sec:tuple}.
  \item But it's better to use an `algorithm', in the technical sense of
    `algorithms built into the standard library'. In this case, \lstinline{find}.
  \item \ldots~except that with \lstinline{find} you have to search for an exact
    key/value pair, and here you want to search: `is this node
    in the \lstinline{distances} set with whatever value'.
    Use the \indexc{find_if} algorithm; section~\textbookref{sec:stdset}.
  \end{itemize}
\end{exercise}

\Level 1 {Level set algorithm}

We start with a simple algorithm:
the \ac{SSSP} algorithm in an unweighted graph;
see \HPSCref{sec:sssp} for details.
Equivalently, we find level sets in the graph.

For unweighted graphs, the distance algorithm is fairly simple.
Inductively:
\begin{itemize}
\item Assume that we have a set of nodes reachable in at most~$n$ steps,
\item then their neighbors (that are not yet in this set)
  can be reached in~$n+1$ steps.
\end{itemize}

The algorithm outline is
\verbatimsnippet{qlevloop}

\begin{exercise}
  Finish the program that computes the \ac{SSSP} algorithm and test it.
\end{exercise}

This code has an obvious inefficiency:
for each level we iterate through all finished nodes,
even if all their neighbors may already have been processed.

\begin{exercise}
  Maintain a set of `current level' nodes,
  and only investigate these to find the next level.
  Time the two variants on some large graphs.
\end{exercise}

\Level 1 {Dijkstra's algorithm}

In Dijkstra's algorithm we maintain both a set of nodes
for which the shortest distance has been found,
and one for which we are still determining the shortest distance.
Note: a tentative shortest distance for a node may be updated
several times, since there may be multiple paths to that node.
The `shortest' path in terms of weights may not be the shortest
in number of edges traversed!

The main loop now looks something like:
\verbatimsnippet{qdloop}
(Note that we erase a record in the \lstinline{to_be_done} set,
and then re-insert the same key with a new value.
We could have done a simple update if we had used a \lstinline{map}
instead of a \lstinline{set}.)

The various places where you find nodes in the finished~/ unfinished sets
are up to you to implement.
You can use simple loops, or use \indexc{find_if}
to find the elements matching the node numbers.

\begin{exercise}
  Fill in the details of the above outline
  to realize Dijkstra's algorithm.
\end{exercise}

\Level 0 {Linear algebra formulation}

In this part of the project you will explore
how much you make graph algorithms look like
linear algebra.

\Level 1 {Code preliminaries}

\Level 2 {Data structures}

You need a matrix and a vector. The vector is easy:
\verbatimsnippet{adjvecdense}

For the matrix, use initially a dense matrix:
\verbatimsnippet{adjmatdense}
but later we will optimize that.

\begin{remark}
  In general it's not a good idea to store a matrix
  as a vector-of-vectors, but in this case we need to be able
  to return a matrix row, so it is convenient.
\end{remark}

\Level 2 {Matrix vector multiply}

Let's write a routine
\begin{lstlisting}
  Vector AdjacencyMatrix::leftmultiply( const Vector& left );
\end{lstlisting}
This is the simplest solution, but not necessarily the most efficient one,
as it creates a new vector object for each matrix-vector multiply.

As explained in the theory background, graph algorithms can be formulated
as matrix-vector multiplications with unusual add/multiply operations.
Thus, the core of the multiplication routine could look like
\verbatimsnippet{graphmultiply}

\Level 1 {Unweighted graphs}

\begin{exercise}
  Implement the \lstinline{add}~/ \lstinline{mult} routines
  to make the \ac{SSSP} algorithm on unweighted graphs work.
\end{exercise}

\Level 1 {Dijkstra's algorithm}

As an example, consider the following adjacency matrix:
\begin{verbatim}
 .  1 . .  5
 . .  1 . .
 . . .  1 .
 . . . .  1
  1 . . . .
\end{verbatim}
The shortest distance $0\rightarrow 4$ is~4,
but in the first step a larger distance of~5
is discovered.
Your algorithm should show an output similar to this
for the successive updates to the known shortest distances:
\begin{verbatim}
Input :  0 . . . .
step 0:  0 1 . . 5
step 1:  0 1 2 . 5
step 2:  0 1 2 3 5
step 3:  0 1 2 3 4
\end{verbatim}

\begin{exercise}
  Implement new versions of the  \lstinline{add}~/ \lstinline{mult} routines
  to make the matrix-vector multiplication correspond to Dijkstra's algorithm
  for \ac{SSSP} on weighted graphs.
\end{exercise}

\Level 1 {Sparse matrices}

The matrix data structure described above can be made
more compact by storing only nonzero elements.
Implement this.

\Level 1 {Further explorations}

How elegant can you make your code through operator overloading?

Can you code the all-pairs shortest path algorithm?

Can you extend the \ac{SSSP} algorithm to also generate the actual paths?

\Level 0 {Tests and reporting}

You now have two completely different implementations
of some graph algorithms.
Generate some large matrices and time the algorithms.

Discuss your findings, paying attention to amount of work performed and
amount of memory needed.
