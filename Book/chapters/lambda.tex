% -*- latex -*-
%%%%%%%%%%%%%%%%%%%%%%%%%%%%%%%%%%%%%%%%%%%%%%%%%%%%%%%%%%%%%%%%
%%%%
%%%% This TeX file is part of the course
%%%% Introduction to Scientific Programming in C++/Fortran2003
%%%% copyright 2017-2023 Victor Eijkhout eijkhout@tacc.utexas.edu
%%%%
%%%% lambda.tex : closures
%%%%
%%%%%%%%%%%%%%%%%%%%%%%%%%%%%%%%%%%%%%%%%%%%%%%%%%%%%%%%%%%%%%%%

The mechanism of \indextermbusp{lambda}{expression}
(first added in \cppstandard{11} and since expanded)
makes dynamic definition of functions possible.

\begin{block}{Introducing: lambda expressions}
  \label{sl:lambda-example}
  Traditional function usage:\\
  explicitly define a function and apply it:
\begin{lstlisting}
double sum(float x,float y) { return x+y; }
cout << sum( 1.2, 3.4 );
\end{lstlisting}
New:\\
apply the function recipe directly:
\snippetwithoutput{lambdaexp}{func}{lambdadirect}
\end{block}

This example is of course fairly pointless,
but it illustrates the syntax of a lambda expression:
\begin{lstlisting}
[capture] ( inputs ) -> outtype { definition };
[capture] ( inputs ) { definition };  
\end{lstlisting}

\begin{itemize}
\item The square brackets, in this case, but not in general, empty,
  are the \indextermdef{capture} part;
\item then follows the usual argument list;
\item with a stylized arrow you can indicate the return type,
  but this is optional if the compiler can figure it out by itself;
\item and finally the usual function body, include \lstinline{return} statement
  for non-void functions.
\end{itemize}

\begin{remark}
  Lambda expressions are sometimes called \indextermp{closure},
  but this term has a technical meaning in programming language theory,
  that only partly coincides with C++ lambda expressions.
\end{remark}

\begin{block}{Direct invocation, 1}
  \label{sl:lambda-iile}
  There are uses for `Immmediately Invoked Lambda Expression'.

  Example: different constructors.
  \lstset{numbers=left,numberstyle=\tiny}
\begin{multicols}{2}
    Does not work:
\begin{lstlisting}
if (foo)
  MyClass x(5,5);
else
  MyClass x(6);
\end{lstlisting}
\columnbreak
Solution:
\begin{lstlisting}
MyClass x = 
  [foo] () {
    if (foo)
      return MyClass(5,5);
    else
      return MyClass(6);
  }();
\end{lstlisting}
\end{multicols}
\end{block}

\begin{block}{Direct invocation, 2}
  \label{sl:lambda-omp}
  OpenMP:
  \lstset{numbers=left,numberstyle=\tiny}
\begin{lstlisting}
const int nthreads = [] () -> int {
  int nt;
#pragma omp parallel
#pragma omp master
  nt = omp_get_num_threads();
  return nt;
}();
\end{lstlisting}
\end{block}

For a slightly more useful example,
we can assign the lambda expression to a variable, and repeatedly apply it.

\begin{block}{Assign lambda expression to variable}
  \label{sl:lambdavar}
\def\snippetcodefraction{.67}
\def\snippetanswfraction{.3}  
\snippetwithoutput{lambdavar}{func}{lambdavar}
\begin{itemize}
\item This is a variable declaration.
\item Uses \lstinline+auto+ for technical reasons; see later.
\end{itemize}
Return type could have been omitted:
\begin{lstlisting}
    auto summing = 
    [] (float x,float y) { return x+y; };
\end{lstlisting}
\end{block}

\begin{slide}{Lambda syntax}
  \label{sl:lambda-syntax}
\begin{lstlisting}
[capture] ( inputs ) -> outtype { definition };
[capture] ( inputs ) { definition };  
\end{lstlisting}
  \begin{itemize}
  \item The square brackets are how you recognize a lambda;\\
    we will get to the `capture' later.
  \item Inputs: like function parameters
  \item Result type specification \lstinline+-> outtype+:\\
    can be omitted if compiler can deduce it;
  \item Definition: function body.
  \end{itemize}
\end{slide}

\begin{exercise}
  Do exercise~\ref{ex:newton-functions} of the zero finding project.
\end{exercise}

\Level 0 {Lambda expressions as function argument}
\label{sec:lambdaauto}

Above, when we assigned a lambda expression to a variable,
we used \lstinline{auto} for the type.
The reason for this is that each lambda expression gets its own
unique type, that is dynamically generated.
But that makes it hard to pass that variable to a function.

Suppose we want to pass a lambda expression to a function:
\begin{lstlisting}
int main() {
    apply_to_5( [] (int i) { cout << i+1; } );
\end{lstlisting}
What type do we use for the function parameter?
\begin{lstlisting}
void apply_to_5( /* what type are we giving? */ f ) {
    f(5);
}
\end{lstlisting}
Since the type of the lambda expression is dynamically generated,
we can not specify that type in the function header.

\begin{slide}{Lambdas as parameter: the problem}
  \label{sl:lambda-apply5}
  Lambdas have a type that is dynamically  generated,
  so you can not write a function
  that takes a lambda as argument, because you can't write the type.
  
  \lstset{numbers=left,numberstyle=\tiny}
\begin{lstlisting}
void apply_to_5( /* what? */ f ) {
    f(5);
}
int main() {
    apply_to_5
      ( [] (double x) { cout << x; } );
}
\end{lstlisting}
(Actually, this simple case does work with C~syntax,
but not for general lambdas)
\end{slide}

The way out is to use the \lstinline{functional} header:
\begin{lstlisting}
#include <functional>
using std::function;
\end{lstlisting}
With this, you can declare parameters by their signature.

\begin{slide}{Lambdas as parameter: the solution}
  \label{sl:lambda-function-parm}
\begin{lstlisting}
#include <functional>
using std::function;
\end{lstlisting}
With this, you can declare parameters by their signature\\
(that is, types of parameters and output):
%
\snippetwithoutput{lambdapass}{func}{lambdapass}
\end{slide}

In the following example we write a function \lstinline{apply_to_5}
which 
\begin{itemize}
\item takes a function \lstinline{f}, and
\item applies it to \lstinline{5}.
\end{itemize}
We call the \lstinline{apply_to_5} function
with a lambda expression as argument:

\snippetwithoutput{lambdapass}{func}{lambdapass}

\begin{exercise}
  Do exercise~\ref{ex:newton-function-ptr} of the zero-finding project.
\end{exercise}

\Level 1 {Lambda members of classes}

The fact that a lambda expression has a dynamically generated type
also makes it hard to store it in an object.
To do this we again use \lstinline{std::}\indexc{function}.

In the following example we make a class \lstinline{SelectedInts}
which takes a boolean function in the constructor:
an object will contain only those integers that satisfy the function.

\begin{block}{Lambda in object}
  \label{sl:lambda-class}
  %
  A set of integers, with a test on which ones can be admitted:
  \begin{multicols}{2}
    \verbatimsnippet{lambdaclass}
  \end{multicols}
\end{block}

We use the above class to construct an object as follows:
\begin{itemize}
\item we read an integer \lstinline{divisor},
\item and accept only those integers into our object
  that are divisible by that number.
\end{itemize}
For this we write a lambda expression \lstinline{is_divisible} that 
\begin{itemize}
\item captures the divisor, and then
\item takes an integer as (its only) argument,
\item returning whether that argument is divisible.
\end{itemize}

\begin{block}{Illustration}
  \label{sl:lambda-classed}
  \snippetwithoutput{lambdaclassed}{func}{lambdafun}
\end{block}

\Level 0 {Captures}

A \indextermdef{capture} is a way to `bake variables into' a function.
Let's say we want a function that increments its input,
and the increment amount is set when we define the function.

\begin{block}{Capture variable}
  \label{sl:capture-increment}
  Increment function:
  \begin{itemize}
  \item scalar in, scalar out;
  \item the increment amount has been fixed through the capture.
  \end{itemize}
  \def\snippetcodefraction{.67}
  \def\snippetanswfraction{.3}  
  \snippetwithoutput{lambdavalue}{func}{lambdavalue}
\end{block}

\begin{exercise}
  \label{ex:capture-factor}
  Write a program that 
  \begin{itemize}
  \item reads a \lstinline{float} factor;
  \item defines a function \lstinline{multiply} that
    multiplies its input by that factor.
  \end{itemize}
\end{exercise}

You can capture more than one variable.
Explicitly capturing variables is done with a comma-separated list.

\begin{block}{Capture more than one variable}
  \label{sl:capture-fraction}
  Example: multiply by a fraction.
\begin{lstlisting}
int d=2,n=3;
times_fraction = [d,n] (int i) ->int {
    return (i*d)/n;
}
\end{lstlisting}
\end{block}

\begin{exercise}
  \label{ex:capture-between}
  \begin{itemize}
  \item Set two variables 
    \begin{lstlisting}
      float low = .5, high = 1.5;
    \end{lstlisting}
  \item Define a function of one variable that tests
    whether that variable is between \lstinline{low,high}.\\
    (Hint: what is the signature of that function?
    What is/are input parameter(s) and what is the return result?)
  \end{itemize}
\end{exercise}

\begin{exercise}
  Do exercises \ref{ex:newton-capture-root} and~\ref{ex:newton-capture-diff}
  of the zero-finding project.
\end{exercise}

\Level 1 {Capture by reference}

Normally, captured variables are copied by value.

Attempting to change the captured variable doesn't even compile:
\begin{lstlisting}
auto f = // WRONG DOES NOT COMPILE
    [x] ( float &y ) -> void {
        x *= 2; y += x; };
\end{lstlisting}

If you do want to alter the captured parameter,
pass it by reference:
\snippetwithoutput{lambdareference}{func}{lambdareference}

\begin{slide}{Capture by value}
  \label{sl:lambda-val-val}
Normal capture is by value:
\snippetwithoutput{lambdavalue}{func}{lambdavalue}
\end{slide}

Capturing by reference can for instance be useful
if you are performing some sort of reduction.
The capture is then the reduction variable,
and the numbers to be reduced come in as function parameter
to the lambda expression.

In this example we count how many of the input values
test true under a certain function~\lstinline{f}:

\begin{block}{Capture by reference}
  \label{sl:capture-count}
  Capture a variable by reference so that
  you can update it:
\begin{lstlisting}
int count=0;
auto count_if_f = 
    [&count] (int i) {
      if (f(i)) count++; }
for ( int i : int_data )
  count_if_f(i);
cout << "We counted: " << count;
\end{lstlisting}
(See the \indexheader{algorithm} header%
\ifInBook ~section~\ref{sec:algorithm}\fi
.)
\end{block}

\Level 1 {Capturing `this'}

In addition to capturing specific variable,
whether by reference or not,
as you saw above, you can also capture the whole
environment of a lambda.
For this the following shorthands exist:
\begin{lstlisting}
  [=] () {} // capture everything by value
  [&] () {} // capture everything by reference
\end{lstlisting}
In the context of a class method, this means you are capturing \lstinline{this}.
As of \cppstandard{20}, implicit capture of \lstinline{this} by value
is deprecated.

\Level 0 {More}

\Level 1 {Making lambda stateful}

Let's consider the issue of lambda expressions and mutable state,
by which we mean no more than that a variable gets updated
multiple times.

A simple example is a doing a count reduction:
how many items satisfy some test.
In the following lambda expression, the item
is passed as an argument,
while the count is captured by reference.

\snippetwithoutput[printeach]{counteach}{stl}{counteach}

\begin{slide}{Capture by reference}
  \label{sl:lambda-count}
  Capture variables are normally by value,
  use ampersand for reference.
  This is often used in \lstinline{algorithm} header.
\snippetwithoutput[printeach]{counteach}{stl}{counteach}
\end{slide}

How about if that count is not really needed
in the calling environment of the lambda expression;
can we somehow make it internal?

Lambda expressions are normally stateless:

\snippetwithoutput{lambdanonmutable}{func}{nonmutable}

but with the \indexcdef{mutable} keyword you can
make them stateful:

\snippetwithoutput{lambdayesmutable}{func}{yesmutable}

Here is a nifty application: printing a list of numbers,
separated by commas, but without trailing comma:
%
\snippetwithoutput{mutablecomma}{func}{lambdaexch}

\Level 1 {Generic lambdas}
\label{sec:lambda-generic}

The \lstinline{auto} keyword can be used for
\indextermsubp{generic}{lambda}:
\begin{lstlisting}
auto compare = [] (auto a,auto b) { return a<b; };
\end{lstlisting}
Here the return type is clear, but the input types are generic.
This is much like using a templated function:
the compiler instantiates the expression with whatever types are needed.

\Level 1 {Algorithms}

The \indextermtt{algorithm} header (section~\ref{sec:algorithm})
contains a number of functions that
naturally use lambdas. For instance, \indexc{any_of} can test
whether any element of a \lstinline{vector} satisfies a condition.
You can then use a lambda to specify the \lstinline{bool} function
that tests the condition.

\Level 1 {C-style function pointers}

The C language had a --~somewhat confusing~-- notation
for function pointers.
If you need to interface with code that uses them,
it is possible to use lambda functions to an extent:
lambdas without captures can be converted to a function pointer.

\snippetwithoutput{lambdacptr}{func}{lambdacptr}
