% -*- latex -*-
%%%%%%%%%%%%%%%%%%%%%%%%%%%%%%%%%%%%%%%%%%%%%%%%%%%%%%%%%%%%%%%%
%%%%
%%%% This TeX file is part of the course
%%%% Introduction to Scientific Programming in C++/Fortran2003
%%%% copyright 2017 Victor Eijkhout eijkhout@tacc.utexas.edu
%%%%
%%%% morex.tex : more exercises
%%%%
%%%%%%%%%%%%%%%%%%%%%%%%%%%%%%%%%%%%%%%%%%%%%%%%%%%%%%%%%%%%%%%%

\Level 0 {List access}

\begin{exercise}
  Explore the efficiency of using an array versus a linked list.
  \begin{enumerate}
  \item Compare re-allocating the array versus adding elements to the
    linked list. Start with a simple case: add elements only at the
    end, and keep a pointer to the final element in the list.
  \item Investigate access times: retrieve many (as in: thousands if
    not more) elements from the array. Do this as follows: allocate an
    array of indexes, and repeatedly retrieve those list/array
    elements, for instance adding them together. Does the access time
    for the array go up if the number of elements gets large?
  \item Optimize allocation for the list: create an array of list
    nodes and use those. Does this make a difference in access times?
  \end{enumerate}
\end{exercise}
