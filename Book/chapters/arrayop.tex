% -*- latex -*-
%%%%%%%%%%%%%%%%%%%%%%%%%%%%%%%%%%%%%%%%%%%%%%%%%%%%%%%%%%%%%%%%
%%%%
%%%% This TeX file is part of the course
%%%% Introduction to Scientific Programming in C++/Fortran2003
%%%% copyright 2017 Victor Eijkhout eijkhout@tacc.utexas.edu
%%%%
%%%% arrayop.tex : arrays operations in Fortran
%%%%
%%%%%%%%%%%%%%%%%%%%%%%%%%%%%%%%%%%%%%%%%%%%%%%%%%%%%%%%%%%%%%%%

\Level 0 {Loops without looping}

In addition to ordinary do-loops, Fortran has mechanisms that save you
typing, or can be more efficient in some circumstances.
\begin{itemize}
\item Slicing: if your loop assigns to an array from another array,
  you can use section notation:
\begin{verbatim}
a(:) = b(:)
c(1:n) = d(2:n+1)
\end{verbatim}
\item The \indextermtt{forall} keyword also indicates an array assignment:
\begin{verbatim}
forall (i=1:n)
  a(i) = b(i)
  c(i) = d(i+1)
end forall
\end{verbatim}
You can tell that this is for arrays only, because the loop index has
to be part of the left-hand side of every assignment.

This mechanism is prone to misunderstanding and therefore now
deprecated.
It is not a parallel loop! For that, the following mechanism is preferred.
\item The \indexterm{do concurrent} is a true do-loop. With the
  \n{concurrent} keyword the user specifies that the
  iterations of a loop are independent, and can therefore possibly be
  done in parallel:
\begin{verbatim}
do concurrent (i=1:n)
  a(i) = b(i)
  c(i) = d(i+1)
end do
\end{verbatim}
\end{itemize}

\Level 1 {Loops without dependencies}

Here are some illustrations of simple array copying with the above
mechanisms.

\verbatimsnippet{blockrecur}
\begin{verbatim}
Original     1   2   3   4   5   6   7   8   9  10
Recursive    0   2   4   6   8  10  12  14  16  18
\end{verbatim}

\verbatimsnippet{blockslice}
\begin{verbatim}
Original     1   2   3   4   5   6   7   8   9  10
Sliced       0   2   4   6   8  10  12  14  16  18
\end{verbatim}

\verbatimsnippet{blockforall}
\begin{verbatim}
Original     1   2   3   4   5   6   7   8   9  10
Forall       0   2   4   6   8  10  12  14  16  18
\end{verbatim}

\verbatimsnippet{blockconc}
\begin{verbatim}
Original     1   2   3   4   5   6   7   8   9  10
Concurrent   0   2   4   6   8  10  12  14  16  18
\end{verbatim}

\begin{exercise}
  Create arrays \n{A,C} of length~\n{2N}, and \n{B}~of length~\n{N}.
  Now implement
  \[ B_i = (A_{2i}+A_{2i+1})/2,\quad i=1,\ldots N \]
  and
  \[ C_i = A_{i/2},\quad i=1,\ldots 2N \]
  using all four mechanisms. Make sure you get the same result every time.
\end{exercise}

\Level 1 {Loops with dependencies}

For parallel execution of a loop, all iterations have to be independent.
This is not the case if the loop has a \indexterm{recurrence}, and in
this case, the `do concurrent' mechanism is not appropriate.
%
Here are the above four constructs, but applied to a loop with a dependence.
%
\verbatimsnippet{slicerecur}
%
\begin{verbatim}
Original   1   2   3   4   5   6   7   8   9  10
Recursiv   1   2   4   8  16  32  64 128 256 512
\end{verbatim}

The slicing version of this:
%
\verbatimsnippet{sliceslice}
%
\begin{verbatim}
Original   1   2   3   4   5   6   7   8   9  10
Sliced     1   2   4   6   8  10  12  14  16  18
\end{verbatim}
%
acts as if the right-hand side is saved in a temporary array, and
subsequently assigned to the left-hand side.

Using `forall' is equivalent to slicing:
%
\verbatimsnippet{sliceforall}
%
\begin{verbatim}
Original     1   2   3   4   5   6   7   8   9  10
Forall       1   2   4   6   8  10  12  14  16  18
\end{verbatim}

On the other hand, `do concurrent' does not use temporaries, so it is
more like the sequential version:
%
\verbatimsnippet{sliceconc}
%
\begin{verbatim}
Original     1   2   3   4   5   6   7   8   9  10
Concurrent   1   2   4   8  16  32  64 128 256 512
\end{verbatim}
Note that the result does not have to be equal to the sequential
execution: the compiler is free to rearrange the iterations any way it
sees fit.
