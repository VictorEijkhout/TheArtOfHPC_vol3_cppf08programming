% -*- latex -*-
%%%%%%%%%%%%%%%%%%%%%%%%%%%%%%%%%%%%%%%%%%%%%%%%%%%%%%%%%%%%%%%%
%%%%
%%%% This TeX file is part of the course
%%%% Introduction to Scientific Programming in C++/Fortran2003
%%%% copyright 2017-2020 Victor Eijkhout eijkhout@tacc.utexas.edu
%%%%
%%%% classf.tex : object oriented programming in Fortran
%%%%
%%%%%%%%%%%%%%%%%%%%%%%%%%%%%%%%%%%%%%%%%%%%%%%%%%%%%%%%%%%%%%%%

\Level 0 {Classes}
\label{sec:objectf}

Fortran classes are based on \indexf{type} objects.
Some aspects are similar to C++.
For instance,
the same syntax is used for specifying data members and methods:
\begin{lstlisting}
print *,myobject%xfield
myobject%set_xfield(5.1)
\end{lstlisting}

Other aspects are a little different:
in C++ you can write in one class definition
all data and function members;
in Fortran 
data and functions are declared separately.

A big difference is in how function methods are defined:
the object itself becomes an extra parameter.
You will see the details later.

\begin{slide}{Classes and objects}
  \label{sl:fclass}
  Fortran classes are based on \indexf{type} objects.\\
  Similarities and differences with C++.
  \begin{itemize}
  \item
    Same \lstinline+%+
    syntax for specifying data members and methods.
  \item Data members and functions declared separately.
  \item Object itself as extra parameter.
  \end{itemize}
  All will become clear~\ldots
\end{slide}

First about how Fortran classes are organized.
A~class is a type definition inside a module,
with an extra clause indicating what function methods
are available for the type.

\begin{block}{Object is type with methods}
  \label{sl:fclass-type}
  You define a type as before, with its data members, but now the type
  has a \indexf{contains}\index{contains!for class functions} for the
  methods:
  %
  \footnotesize
  %\begin{multicols}{2}
  \verbatimsnippet{fmult1type}
  %\end{multicols}
\end{block}

As stated above, calling methods on an object
uses the same syntax as accessing its data members.

\begin{block}{Object methods}
  \label{sl:fclass-prog}
  Method call similar to C++
  %
  \footnotesize
  \snippetwithoutput{fmult1prog}{objectf}{mult1}
\end{block}

The method definition works slightly different from~C++,
but if you know python you'll see the similarity.
If a method is called with one argument;
\begin{lstlisting}
call obj%fun(arg)
\end{lstlisting}
the function has two parameters,
the first one being the object,
and the second one the parenthesized argument.

Additionally, the first parameter is of type
\lstinline+Type(obj)+, but in the method
it is declared as \lstinline+Class(obj)+.

\begin{block}{Method definition}
  \label{sl:fclass-method}
  Note the extra first parameter:\\
  which is a \lstinline+Type+ but declared
  here as \lstinline+Class+:
  %
  \verbatimsnippet{fmult1method}
\end{block}

%% this is redundant. are we using it anywhere?
\begin{slide}{Methods have object as argument}
  \label{sl:fclass-self}
  You define functions that accept the type as first argument, but
  instead of declaring the argument as \indexf{type}, you define it as
  \indexf{class}.

  The members of the class object have to be accessed through the
  \verb+%+~operator.
  %
  \verbatimsnippet{pointsetf}
\end{slide}

In summary:

\begin{block}{Class organization}
  \label{sl:fclass-org}
  \begin{itemize}
  \item A class is a \indexf{Type} with a \indexf{contains} clause\\
    followed by \indexf{procedure} declarations,
  \item \ldots~contained in a module.
  \item Actual methods go in the \indexf{contains} part of the module
  \item First argument of method is the object itself.
  \end{itemize}
\end{block}

\begin{block}{Point program}
  \label{sl:fpoint-program}
  \footnotesize
  
  \begin{multicols}{2}
    \verbatimsnippet{pointexmod}
    \columnbreak
    \verbatimsnippet{pointexmain}
  \end{multicols}  
\end{block}

\begin{block}{Similarities and differences}
  \label{sl:type-vs-class}
  \footnotesize
  \begin{tabular}{|l|ll|}
    \hline
    &C++&Fortran\\
    \hline
    Members&in the object&in the `type'\\
    Methods&in the object&interface: in the type\\
    &&implementation: in the module\\
    Constructor&default or explicit&none\\
    object itself&`this'&first argument\\
    Class members&global variable&accessed through first arg\\
    Object's methods&period&percent\\
    \hline
  \end{tabular}
\end{block}

\begin{exercise}
  \label{ex:fclass-point-distance}
  Take the point example program and add a distance function:
\begin{lstlisting}
  Type(Point) :: p1,p2
  ! ... initialize p1,p2
  dist = p1%distance(p2)
  ! ... print distance
\end{lstlisting}
\skeleton{pointexample}
\end{exercise}

\begin{exercise}
  \label{ex:fclass-translate}
  Write a method \n{add} for the \n{Point} type:
\begin{lstlisting}
  Type(Point) :: p1,p2,sum
  ! ... initialize p1,p2
  sum = p1%add(p2)
\end{lstlisting}
  What is the return type of the function \n{add}?
\end{exercise}

\Level 1 {Final procedures: destructors}

The Fortran equivalent of
\emph{destructors}\index{destructor!in Fortran|see{final procedure}}
is a \indextermsub{final}{procedure},
designated by the \indextermdef{final} keyword.

\verbatimsnippet{finaldecl}

A final procedure has a single argument of the type that it applies to:

\verbatimsnippet{finalproc}

The final procedure is invoked when a derived type object is deleted,
except at the conclusion of a program:

\verbatimsnippet{finaluse}

\Level 0 {Inheritance}

\begin{block}{More OOP}
  \label{sl:oopf}
Inheritance:
\begin{lstlisting}
type,extends(baseclas) :: derived_class
\end{lstlisting}
Pure virtual:
\begin{lstlisting}
type,abstract
\end{lstlisting}
\end{block}

\begin{block}{Further reading}
  \url{http://fortranwiki.org/fortran/show/Object-oriented+programming}
\end{block}

\begin{block}{Use modules!}
  \label{sl:fclass-module}
   It is of course best to put the type definition and method
   definitions in a module, so that you can \indexf{use} it.

   Mark methods as \indexf{private} so that they can only be used as part
   of the \indexf{type}:

   \verbatimsnippet{classmodule}
\end{block}

\Level 0 {Operator overloading}

For many physical quantities it makes sense to define an addition operator.
This makes it possible to write
\begin{lstlisting}
Type(X) :: x,y,z
! stuff
x = y+z
\end{lstlisting}

\begin{slide}{Define operators on classes}
  \label{sl:foverload0}
\begin{lstlisting}
Type(X) :: x,y,z

! function syntax:
x = y%add(z)

! operator syntax
x = y+z
\end{lstlisting}
Code looks closer to math formulas
\end{slide}

\begin{block}{Example class}
  \label{sl:foverload1}
  For purposes of exposition, let's make a very simple class:
  \verbatimsnippet{fscalarfield}
\end{block}

We define a couple of obvious methods:
\begin{block}{Methods just for testing}
  \label{sl:foverload2}
  \begin{multicols}{2}
    \verbatimsnippet{fscalarsetprint}
    \verbatimsnippet{fscalarsetprintuse}
  \end{multicols}
\end{block}

Before we can define the addition operator,
it is first necessary to define an addition function:
\verbatimsnippet{fscalaradd}

\begin{slide}{Addition function}
  \label{sl:foverload3}
  \verbatimsnippet{fscalaradd}
  Parameters need to be \lstinline+Intent(In)+
\end{slide}

This function needs to satisfy some conditions:
\begin{itemize}
\item The function needs to have two input parameters. Obviously.
\item The input parameters need to be declared \lstinline+Intent(In)+.
  This is a little less obvious, but it makes sense,
  because the arguments to the addition parameter are not really
  passed the normal way.
\end{itemize}

Turning the function into an operator is then pretty simple.
\begin{block}{Operator definition}
  \label{sl:field-op}
  Interface block:
  \verbatimsnippet{fscalarplusop}
\end{block}

\begin{exercise}
  \label{ex:field-op}
  Extend the above example program so that the type stores
  an array instead of a scalar.
  \snippetwithoutput{ffieldprog}{geomf}{field}
  \skeleton{scalar}
\end{exercise}

Similarly, we can rdefine the assignment operator;
see \url{https://dannyvanpoucke.be/oop-fortran-tut5-en/}.
This comes with some complications regarding
\indextermsub{shallow}{copy} and \indextermsub{deep}{copy}.
