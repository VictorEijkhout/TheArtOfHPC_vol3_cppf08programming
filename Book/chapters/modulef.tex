% -*- latex -*-
%%%%%%%%%%%%%%%%%%%%%%%%%%%%%%%%%%%%%%%%%%%%%%%%%%%%%%%%%%%%%%%%
%%%%
%%%% This TeX file is part of the course
%%%% Introduction to Scientific Programming in C++/Fortran2003
%%%% copyright 2017-2022 Victor Eijkhout eijkhout@tacc.utexas.edu
%%%%
%%%% modulef.tex : fortran modules
%%%%
%%%%%%%%%%%%%%%%%%%%%%%%%%%%%%%%%%%%%%%%%%%%%%%%%%%%%%%%%%%%%%%%

Fortran has a clean mechanism for importing data (including numeric
constants such as~$\pi$), functions, types
that are defined in another file.
This is done through modules, defined with the \indexf{module} keyword.

\begin{block}{Module definition}
  \label{sl:fmod-def}
  Modules look like a program, but without main\\
  (only `stuff to be used elsewhere'):
  %
  \verbatimsnippet{pointmod}% in structf
  %
  Note also the numeric constant.
\end{block}

\begin{block}{Example use}
  \label{sl:fmod-use}
  Module imported through \indexf{use} statement;\\
  placed before \indexf{implicit none}
  %
  \snippetwithoutput{pointmodprog}{structf}{typemod}
\end{block}

\begin{exercise}
  \label{ex:fmod-rect}
  Take exercise~\ref{ex:ftype-rect} and put all type definitions and
  all functions in a module.
\end{exercise}

\Level 0 {Modules for program modularization}

Modules are Fortran's mechanism for supporting
\indextermsub{separate}{compilation}: you can put your module in one
file, your main program in another, and compile them separately.

A module is a container for definitions of subprograms and types, and
for data such as constants and variables. A~module is not a 
structure or object: there is only one instance.

\begin{remark}
  The \lstinline{use} statement is somewhat similar to
  an \n{#include "stuff.h"} line in~C++.
  However, note that \cppstandard{20} has also adopted modules,
  as cleaner than preprocessor-based solutions.
\end{remark}

\Level 0 {Module definition}
\label{sec:modulef}

What do you use a module for?
\begin{itemize}
\item Type definitions: it is legal to have the same type definition
  in multiple program units, but this is not a good idea. 
  Write the definition just once in a module and make it available
  that way.
\item Function definitions: this makes the functions available in multiple
  sources files of the same program, or in multiple programs.
\item Define constants: for physics simulations, put all constants in
  one module and use that, rather than spelling out the constants each
  time.
\item Global variables: put variables in a module if they do not fit
  an obvious scope.
\end{itemize}
Any routines come after the \indexf{contains}\index{contains!in modules}
clause.

\begin{remark}
  Modules were introduced in \fstandard{90}.
  In earlier standards, information could be made globally available
  through \indexf{common} blocks. 
  Since modules are much cleaner than common blocks, do not use those anymore.
\end{remark}

A module is made available with the \indexf{use} keyword, which
needs to go before the \indexf{implicit none}.
%
\begin{block}{Module use syntax}
  \label{sl:moduleuse}
  \lstinline{Use} statement placed before \lstinline{Implicit}
  %
  \verbatimsnippet{moduleuse}
  %
  Also possible:
\begin{lstlisting}
Use mymodule, Only: func1,func2
Use mymodule, func1 => new_name1
\end{lstlisting}
\end{block}

If you compile a module, you will find a \n{.mod} file in your
directory. (This is little like a~\n{.h} file in~C++.)
If this file is not present, you can not \lstinline{use} the module in another
program unit, so you need to compile the file containing the module
first.

\begin{exercise}
  \label{ex:fpointmod}
  Write a module \lstinline{PointMod} that defines a type \lstinline{Point} and a
  function \lstinline{distance} to make this code work:
  %
  \verbatimsnippet{pointmainex}
  %
  Put the program and module in two separate files and compile thusly:
\begin{lstlisting}
ifort -g -c pointmod.F90
ifort -g -c pointmain.F90
ifort -g  -o pointmain pointmod.o pointmain.o 
\end{lstlisting}
\end{exercise}

\Level 0 {Separate compilation}

The exercises in this course are simple enough that you can
include any modules in the same file as your main program.
However, in realistic applications you will have
a separate files for modules, maybe even using one file per module.

\begin{block}{Separate compilation of modules}
  \label{sl:fcompile-mods}
  Suppose program is split over two files:\\
  \n{theprogram.F90} and \n{themodule.F90}.
  \begin{itemize}
  \item Compile the module: \n{ifort -c themodule.F90}; this gives
  \item an \indextermsub{object}{file} (extension:~\n{.o})
    that will be linked later, and
  \item a module file \n{modulename.mod}.
  \item Compile the main program:\\
    \n{ifort -c theprogram.F90} will read the \n{.mod} file; and finally
  \item Link the object files into an \indexterm{executable}:\\
    \n{ifort -o myprogram theprogram.o themodule.o}\\
    The compiler is used as \indexterm{linker}: there is no compiling
    in this step.
  \end{itemize}
  Important: the module needs to be compiled before any (sub)program
  that uses it.
\end{block}

The \fstandard{2008} standard introduced \indextermsub{sub}{module}s,
which can even further facilitate separate compilation.

\Level 0 {Access}

By default, all the contents of a module is usable by a subprogram
that uses it. However, a keyword \indexf{private} make module
contents available only inside the module.
You can make the default behavior explicit by using the
\indexf{public} keyword. Both \indexf{public} and \indexf{private} can be used as
attributes on definitions in the module.
There is a keyword \indexf{protected} for data members that
are public, but can not be altered by code outside the module.

\begin{multicols}{2}
  \verbatimsnippet{protectmod}
  \verbatimsnippet{protectmain}
\end{multicols}

\Level 0 {Polymorphism}
\label{sec:f-polyfunction}

\begin{lstlisting}
module somemodule

INTERFACE swap
MODULE PROCEDURE swapreal, swapint, swaplog, swappoint
END INTERFACE

  contains
  subroutine swapreal
  ...
  end subroutine swapreal
  subroutine swapint
  ...
  end subroutine swapint
\end{lstlisting}

\Level 0 {Operator overloading}

You can define operations such as \lstinline{+} or \lstinline{*} on types.

\begin{multicols}{2}
  \verbatimsnippet{ftypeadd}  
\end{multicols}

You can now make the code look nice and simple:
\snippetwithoutput{ftypeadddo}{typef}{plus}

Overloading includes the assignment operator:

\begin{lstlisting}
INTERFACE ASSIGNMENT (=) 
subroutine_interface_body
END INTERFACE
\end{lstlisting}

You can define new operators with a dot-notation:

\begin{lstlisting}
INTERFACE OPERATOR (.DIST.)
MODULE PROCEDURE calcdist
END INTERFACE
\end{lstlisting}

