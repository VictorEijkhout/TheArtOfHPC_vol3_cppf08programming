% -*- latex -*-
%%%%%%%%%%%%%%%%%%%%%%%%%%%%%%%%%%%%%%%%%%%%%%%%%%%%%%%%%%%%%%%%
%%%%
%%%% This TeX file is part of the course
%%%% Introduction to Scientific Programming in C++/Fortran2003
%%%% copyright 2023 Victor Eijkhout eijkhout@tacc.utexas.edu
%%%%
%%%% tools.tex : about support tools
%%%%
%%%%%%%%%%%%%%%%%%%%%%%%%%%%%%%%%%%%%%%%%%%%%%%%%%%%%%%%%%%%%%%%

\lstset{language=Bash}

Knowing how to write a program is not enough: you need a variety of tools
to be a good programmer, or to be a programmer at all.

\Level 0 {Editors and development environments}

Simple programs, such as most of the exercises in this book,
can be written using a simple editor.
Traditionally, programmers have used \indexunix{emacs}
or \indexunix{vi} (or one of its derivates such as \texttt{vim}).

More sophisticated are development environments such as
\indextermbus{Microsoft}{Visual Studio Code}
or \indexterm{CLion}.

\Level 0 {Compilers}

For the simple exercises in this book,
all you needed to know about a compiler
was the commandline
\begin{lstlisting}
icpx -o myprogram myprogram.cxx
\end{lstlisting}
(or whatever compiler name and program name you had).

There is much more to know about compilers;
see \CARPref[chapter]{tut:compile}.

\Level 0 {Build systems}

For programs that are more complicated than a single source file
(and even then\dots)
you are wise to use some form of build system.
\begin{itemize}
\item The simplest and oldest solution is \indexunix{Make};
  see \CARPref[chapter]{tut:gnumake}.
\item More modern, more powerful, yet also in a way more complicated,
  is \indexunix{CMake}; see \CARPref[chapter]{tut:cmake}.
\item Make and Cmake are often integrated into the above-mentioned
  development environments.
\end{itemize}

\Level 0 {Debuggers}

If your code misbehaves, having a good \indexterm{debugger}
can be a lifesaver.
The traditional debugger is \indexunix{gdb}
(see \CARPref[chapter]{tut:debug})
but, again, this is often integrated into build environments.

