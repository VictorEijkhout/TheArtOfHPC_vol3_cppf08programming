% -*- latex -*-
%%%%%%%%%%%%%%%%%%%%%%%%%%%%%%%%%%%%%%%%%%%%%%%%%%%%%%%%%%%%%%%%
%%%%
%%%% This TeX file is part of the course
%%%% Introduction to Scientific Programming in C++/Fortran2003
%%%% copyright 2017-2023 Victor Eijkhout eijkhout@tacc.utexas.edu
%%%%
%%%% infect.tex : exercises for infectious disease simulation
%%%%
%%%%%%%%%%%%%%%%%%%%%%%%%%%%%%%%%%%%%%%%%%%%%%%%%%%%%%%%%%%%%%%%

This section contains a sequence of exercises that builds up to a
somewhat realistic simulation of the spread of infectious
diseases. 

\Level 0 {Model design}

It is possible to model disease propagation statistically, but here we
will build an explicit simulation: we will maintain an explicit
description of all
the people in the population, and track for each of them their status.

We will use a simple model where a person can be:
\begin{itemize}
\item sick: when they are sick, they can infect other people;
\item susceptible: they are healthy, but can be infected;
\item recovered: they have been sick, but no longer carry the disease,
  and can not be infected for a second time;
\item vaccinated: they are healthy, do not carry the disease, and can
  not be infected.
\end{itemize}
In more complicated models a person could be infectious during only
part of their illness, or there could be secondary infections with
other diseases, et cetera. We keep it simple here:
any sick person can infect others while they are sick.

In the exercises below we will gradually develop a
model of how the disease spreads from an initial source of infection.
The program will then track the population from day to day,
running indefinitely until none of the population
is sick. Since there is no re-infection, the run will always end.
Later we will add mutation to the model which can extend the duration
of the epidemic.

\Level 1 {Other ways of modeling}

Instead of capturing every single person in code, a~`contact network'
model,
it is possible to use an
\ac{ODE} approach to disease modeling. You would then model the
percentage of infected persons by a single scalar, and derive
relations for that and other scalars~\cite{Anderson:population1979,KermackSIR}.

%% \url{http://mathworld.wolfram.com/Kermack-McKendrickModel.html}

This is known as a `compartmental model', where each of the three SIR states,
Susceptible, Infected, Removed,
is a compartment: a~section of the population.
Both the contact network and
the compartmental model capture part of the truth. In fact, they can
be combined. We can consider a country as a set of cities, where
people travel between any pair of cities. We then use a compartmental
model inside a city, and a contact network between cities.

In this project we will only use the network model.

\Level 0 {Coding}

The following sections are a step-by-step
development of the code for this project.
Later we will discuss running this code
as a scientific experiment.

\begin{remark}
  In various places you need a random number generator.
  You can use the C~language random number generator
  (section~\ISPref{sec:crand}),
  or the new \ac{STL} one in section~\ISPref{sec:stl:random}.
\end{remark}

\Level 1 {Person basics}

The most basic component of a disease simulation
is to infect a person with a disease,
and see the time development of that infection.
Thus you need a person class and a disease class.

Before you start coding, ask yourself
what behaviors these classes need to support.
\begin{itemize}
\item Person. The basic methods for a person are
  \begin{enumerate}
  \item Get infected;
  \item Get vaccinated; and
  \item Progress by one day.
  \end{enumerate}
  Furthermore you may want to query what the state of the person is:
  are they healthy, sick, recovered?
\item Disease. For now, a disease itself doesn't do much.
  (Later in the project you may want it to have a method \lstinline{mutate}.)
  However, you may want to query certain properties:
  \begin{enumerate}
  \item Chance of transmission; and
  \item Number of days a person stays sick when infected.
  \end{enumerate}
\end{itemize}

A test for a single person could have output along the following lines:
\begin{inplaceverbatim}
On day 10, Joe is susceptible
On day 11, Joe is susceptible
On day 12, Joe is susceptible
On day 13, Joe is susceptible
On day 14, Joe is sick (5 days to go)
On day 15, Joe is sick (4 days to go)
On day 16, Joe is sick (3 days to go)
On day 17, Joe is sick (2 days to go)
On day 18, Joe is sick (1 days to go)
On day 19, Joe is recovered
\end{inplaceverbatim}

\begin{exercise}
  \label{ex:infect:person}
  Write a \lstinline{Person} class with methods:
  \begin{itemize}
  \item \lstinline{status_string()} : returns a description of the
    person's state as a \lstinline{string};
  \item \lstinline{one_more_day()} : update the person's status to the next day;
  \item \lstinline{infect(s)} : infect a person with a disease,
    where the disease object
    \begin{lstlisting}
      Disease s(n);
    \end{lstlisting}
    is specified to run for $n$ days.
  \end{itemize}
\end{exercise}

Your main program could for instance look like:
\verbatimsnippet{infectpmain}
where the infection part has been left out.

Your main concern is how to model the internal state of a person.
This is actually two separate issues:
\begin{enumerate}
\item the state, and
\item if sick, how many days to go to recover.
\end{enumerate}

You can find a way to implement this with a single integer,
but it's better to use two.
Also, write enough support methods such as the \lstinline{is_recovered} test.

\Level 2 {Person tests}

It is easy to write code that seems to be the right thing,
but does not behave correctly in all cases.
So it is a good idea to subject your code to some systematic tests.

Make sure your \lstinline{Person} objects pass these tests:
\begin{itemize}
\item After being infected with a 100\% transmittable disease,
  they should register as sick.
\item If they are vaccinated or recovered,
  and they come in contact with such a disease,
  they stay in their original state.
\item If a disease has a transmission chance of~50\%,
  and a number of people come into contact with it,
  about half of them should get sick.
  This test maybe a little tricky to write.
\end{itemize}

Can you use the \indexterm{Catch2} unittesting framework?
See section~\ISPref{sec:catch2}.

\Level 1 {Interaction}

Next we model interactions between people:
one person is healthy, another is infected,
and when the two come into contact
the disease may be transferred.
\verbatimsnippet{flutransfer}

The disease has a certain probability of being transferred,
so you need to specify that probability.
You could let the declaration be:
\begin{lstlisting}
Disease flu( 5, 0.3 );
\end{lstlisting}
where the first parameter is the number of days an infection lasts,
and the second the transfer probability.

\begin{exercise}
  Add a transmission probability to the \lstinline{Disease} class,
  and add a \lstinline{touch} method to the \lstinline{Person} class.
  Design and run some tests.
\end{exercise}

\begin{exercise}
  Bonus: can you get the following disease specification to work?
  \verbatimsnippet{fluspecs}
  Why could you consider this better than the earlier suggested syntax?
\end{exercise}

\Level 2 {Interaction tests}

Adapt the above tests,
but now a person comes in contact with an infected person,
rather than directly with a disease.

\Level 1 {Population}

Next we need a \lstinline{Population} class,
where a population contains a \lstinline{vector}
consisting of \lstinline{Person} objects.
Initially we only infect one person, and there
is no transmission of the disease.

The \lstinline{Population} class should at least have the following methods:
\begin{itemize}
\item \lstinline{random_infection} to start out with an infected segment
  of the population;
\item \lstinline{random_vaccination} to start out with a number of
  vaccinated individuals.
\item counting functions \lstinline{count_infected} and \lstinline{count_vaccinated}.
\end{itemize}

To run a realistic simulation
you also need a \lstinline{one_more_day} method
that takes the population through a day. 
This is the heart of your code, and we will develop this
gradually in the next section.

\Level 2 {Population tests}

Most population testing will be done in the following section.
For now, make sure you pass the following tests:
\begin{itemize}
\item With a vaccination percentage of 100\%, everyone should indeed be vaccinated.
\end{itemize}

\Level 0 {Epidemic simulation}

To simulate the spread of a disease through the population,
we need an update method that progresses the population through one day:
\begin{itemize}
\item Sick people come into contact with a number of other members of the populace;
\item and everyone gets one day older, meaning mostly that sick people
  get one day closer to recovery.
\end{itemize}

We develop this in a couple of steps.

\Level 1 {No contact}

At first assume that people have no contact, so the disease ends
with the people it starts with.

The trace output should look something like:
\begin{inplaceverbatim}
Size of
population?
In step   1 #sick:    1 :  ? ? ? ? ? ? ? ? ? ? + ? ? ? ? ? ? ? ? ?
In step   2 #sick:    1 :  ? ? ? ? ? ? ? ? ? ? + ? ? ? ? ? ? ? ? ?
In step   3 #sick:    1 :  ? ? ? ? ? ? ? ? ? ? + ? ? ? ? ? ? ? ? ?
In step   4 #sick:    1 :  ? ? ? ? ? ? ? ? ? ? + ? ? ? ? ? ? ? ? ?
In step   5 #sick:    1 :  ? ? ? ? ? ? ? ? ? ? + ? ? ? ? ? ? ? ? ?
In step   6 #sick:    0 :  ? ? ? ? ? ? ? ? ? ? - ? ? ? ? ? ? ? ? ?
Disease ran its course by step 6
\end{inplaceverbatim}

\begin{remark}
  Such a display is good for a sanity check on your program behavior.
  If you include such displays
  in your writeup, make sure to use a monospace font, and don't use a population
  size that needs line wrapping. In further testing, you should use large populations,
  but do not include these displays.
\end{remark}

\begin{exercise}
  \label{ex:infect:notransfer}
  Program a population without infection.
  \begin{itemize}
  \item Write the \lstinline{Population} class. The constructor takes the number of people:
\begin{lstlisting}
Population population(npeople);  
\end{lstlisting}

  \item Write a method that infects a number of random people:
    \verbatimsnippet{popinitinfect}

  \item Write a method \lstinline{count_infected} that counts how many people are infected.
  \item Write an \lstinline{one_more_day} method that updates all persons in the population.
  \item Loop the \lstinline{one_more_day} method until no people are infected: the
    \lstinline{Population::one_more_day} method should apply \lstinline{Person::one_more_day} to
    all person in the population.
  \end{itemize}
\item Write a routine that displays the state of the popular, using
  for instance: \n{?}~for susceptible, \n{+}~for infected, \n{-}~for recovered.
\end{exercise}

\Level 2 {Tests}

Test that for the duration of the disease,
the number of infected people stays constant,
and that the sum of healthy and infected people stays equal to the population size.

\Level 1 {Contagion}

This past exercise was too simplistic: the original patient zero was
the only one who ever got sick.
Now let's incorporate contagion, and investigate the spread of the disease
from a single infected person.

We start with a very simple model of infection.

\begin{exercise}
  \label{ex:infect:1}
  Write a simulation where in each step the direct neighbors of an infected person
  can now get sick themselves.

    Run a number of simulations with population sizes and contagion
    probabilities. Are there cases where people escape getting sick?
\end{exercise}

\Level 2 {Tests}

Do some sanity tests:
\begin{itemize}
\item If one person is infected with a disease with $p=1$,
  the next day there should be 3 people sick.
  Unless the infected person is the first or last: then there are two.
\item If person~0 is infected, and $p=1$, the simulation
  should run for a number of days equal to the size of the population.
\item How is the previous case if $p=0.5$?
\end{itemize}

\Level 1 {Vaccination}

\begin{exercise}
  \label{ex:infect:2}
  Incorporate vaccination: read another number representing the
  percentage of people that has been vaccinated. Choose those members
  of the population randomly.

  Describe the effect of vaccinated people on the spread of the
  disease. Why is this model unrealistic?
\end{exercise}

\Level 1 {Spreading}
\label{sec:infect-spread}

To make the simulation more realistic, we let every sick person come
into contact with a fixed number of random people every day. This
gives us more or less the \indexterm{SIR model};
\url{https://en.wikipedia.org/wiki/Epidemic_model}.

Set the number of people that a person comes into contact with, per
day, to~6 or so. (You can also let this be an upper bound for a random
value, but that
does not essentially change the simulation.) You have already programmed the probability that a
person who comes in contact with an infected person gets sick themselves.
Again start the simulation with a single infected person.

\begin{exercise}
  \label{ex:infect:3}
  Code the random interactions. Now run a number of simulations varying
  \begin{itemize}
  \item The percentage of people vaccinated, and
  \item the chance the disease is transmitted on contact.
  \end{itemize}
  Record how long the disease runs through the population. With a
  fixed number of contacts and probability of transmission, how is
  this number of function of the percentage that is vaccinated?

  Report this function as a table or graph. Make sure you have enough
  data points for a meaningful conclusion. Use a realistic population size.
  You can also do multiple runs and report the average, to even out
  the effect of the random number generator.
\end{exercise}

\begin{exercise}
  Investigate the matter of `herd immunity': if enough people are
  vaccinated, then some people who are not vaccinated will still never get
  sick. Let's say you want to have the probability
  of being not vaccinated, yet never getting sick,
  to be over 95 percent.
  Investigate the percentage of vaccination that is needed
  for this as a function of the contagiousness of the disease.

  As in the previous exercise, make sure your data set is large enough.
\end{exercise}

\begin{remark}
  The screen output you used above is good for sanity checks on small problems.
  However, for realistic simulations you have to think what is a realistic population size.
  If your university campus is a population where random people are likely to meet each other,
  what would be a population size to model that? How about the city where you live?

  Likewise, if you test different vaccination rates, what granularity do you use?
  With increases of 5 or 10 percent you can print all results to you screen,
  but you may miss things. Don't be afraid to generate large amount of data
  and feed them directly to a graphing program.
\end{remark}

\Level 1 {Mutation}

The Covid years have shown how important mutations of an original virus can be.
Next, you can include mutation in your project. We model this as follows:
\begin{itemize}
\item Every so many transmissions, a~virus will mutate into a new variant.
\item A~person who has recovered from one variant is still susceptible to other variants.
\item For simplicity assume that each variant leaves a person sick the same number of days, and
\item Vaccination is all-or-nothing: one vaccine is enough to protect against all variant;
\item On the other hand, having recovered from one variant is not protection against others.
\end{itemize}

Implementation-wise speaking, we model this as follows.
First of all, we need a \lstinline{Disease} class,
so that we can infect a person with an explicit virus;
\verbatimsnippet{infectproto}

A \lstinline{Disease} object now carries the information such as
the chance of transmission, or how a long a person stays under the weather.
Modeling mutation is a little tricky. You could do it as follows:
\begin{itemize}
\item There is a global \lstinline{variants} counter for new virus variants,
  and a global \lstinline{transmissions} counter.
\item Every time a person infects another, the newly infected person gets
  a new \lstinline{Disease} object, with the current variant,
  and the transmissions counter is updated.
\item There is a parameter that determines after how many transmissions the disease mutates.
  If there is a mutation, the global \lstinline{variants} counter is updated,
  and from that point on, every infection is with the new variant.
  (Note: this is not very realistic. You are free to come up with a better model.)
\item A each \lstinline{Person} object has a vector of variants that they are recovered from;
  recovery from one variant only makes them immune from that specific variant,
  not from others.
\end{itemize}

\begin{exercise}
  Add mutation to your model. Experiment with the mutation rate:
  as the mutation rate increases, the disease should stay in the population longer.
  Does the relation with vaccination rate change that you observed before?
\end{exercise}

\Level 1 {Diseases without vaccine: Ebola and Covid-19}

\emph{This section is optional, for bonus points}

The project so far applies to diseases for which a vaccine is available,
such as MMR for measles, mumps and rubella.
The analysis becomes different if no vaccine exists, such
as is the case for \indexterm{Ebola} and \indexterm{Covid-19},
as of this writing.

Instead, you need to incorporate `social distancing' into your code:
people do not get in touch with random others anymore,
but only those in a very limited social circle.
Design a model distance function, and explore various settings.

The difference between Ebola and Covid-19 is how long an
infection can go unnoticed: the \indexterm{incubation period}.
With Ebola, infection is almost
immediately apparent, so such people are removed from the
general population and treated in a hospital.
For Covid-19, a person can be infected, and infect others,
for a number of days before they are sequestered from the population.

Add this parameter to your simulation and explore the behavior
of the disease as a function of it.

\Level 0 {Ethics}

The subject of infectious diseases and vaccination
is full of ethical questions.
The main one is \emph{The chances of something happening to me
are very small, so why shouldn't I bend the rules a little?}.
This reasoning is most often applied to vaccination,
where people for some reason or other refuse to get vaccinated.

Explore this question and others you may come up with:
it is clear that everyone bending the rules will have disastrous
consequences, but what if only a few people did this?

\Level 0 {Project writeup and submission}

\Level 1 {Program files}

In the course of this project you have written more than one main
program, but some code is shared between the multiple programs.
Organize your code with one file for each main program, and a single
`library' file with the class methods.

You can do this two ways:
\begin{enumerate}
\item You make a `library' file, say \n{infect_lib.cc},
  and your main programs each have a line
\begin{lstlisting}
#include "infect_lib.cc"
\end{lstlisting}
This is not the best solution, but it is acceptable for now.
\item The better solution requires you to use
  \indextermsub{separate}{compilation} for building the program, and you
  need a \indexterm{header} file.
  You would now have \n{infect_lib.cc} which is compiled separately,
  and \n{infect_lib.h} which is included both in the library file and the main program:
\begin{lstlisting}
#include "infect_lib.h"
\end{lstlisting}
\begin{inbook}
  See section~\ISPref{sec:hfile} for more information.
\end{inbook}
\end{enumerate}

Submit all source files with instructions on how to build all the main
programs. You can put these instructions in a file with a descriptive
name such as \n{README} or \n{INSTALL}, or you can use a
\indexterm{makefile}.

\Level 1 {Writeup}

In the writeup, describe the `experiments' you have performed and the
conclusions you draw from them. The exercises above give you a number
of questions to address.

For each main program, include some sample output, but note that this
is no substitute for writing out your conclusions in full sentences.

The exercises in section~\ISPref{sec:infect-spread}
ask you to explore the program behavior as a
function of one or more parameters. Include a table to report on the
behavior you found. You can use Matlab or Matplotlib in Python (or
even Excell) to plot your data, but that is not required.

\Level 0 {Bonus: mathematical analysis}

The SIR model can also be modeled through coupled difference or differential
equations. 
\begin{enumerate}
\item The number $S_i$ of susceptible people at time~$i$ decreases by a fraction
  \[ S_{i+1} = S_i(1-\lambda_i\,dt)\]
  where $\lambda_i$ is the product of the number of infected people
  and a constant that reflects the number of meetings and the infectiousness
  of the disease. We write:
  \[ S_{i+1} = S_i(1-\lambda I_i\,dt)\]
\item The number of infected people similarly increases by $\lambda S_iI_i$,
  but it also decreases by people recovering (or dying):
  \[ I_{i+1} = I_i(1+\lambda S_i\,dt-\gamma\,dt). \]
\item Finally, the number of `removed' people equals that last term:
  \[ R_{i+1} = R_i(1+\gamma I_i). \]
\end{enumerate}

\begin{exercise}
  Code this scheme. What is the effect of varying $dt$?
\end{exercise}

\begin{exercise}
  For the disease to become an epidemic, the number of newly infected
  has to be larger than the number of recovered. That is,
  \[ \lambda S_iI_i - \gamma I_i>0 \Leftrightarrow S_i>\gamma/\lambda. \]
  Can you observe this in your simulations?
\end{exercise}

The parameter $\gamma$ has a simple interpretation. Suppose that a person
stays ill for $\delta$ days before recovering.
If $I_t$ is relatively stable, that means every day the same number of people
get infected as recover, and therefore a $1/\delta$ fraction of people
recover each day. Thus, $\gamma$~is the reciprocal of the duration of the infection
in a given person.

\endinput

\begin{exercise}
  You can make the model more realistic by letting vaccination be
  only partly effective. For instance, 50\% of people got the flu
  vaccine, but it was only 40\% effective; 90\%~of people have the
  measles vaccine, and it is about 97\%
  effective. (\url{https://www.cdc.gov/nchs/fastats/immunize.htm}) How
  does your model function in this case? Keep in mind that different
  diseases have different degrees of infectiousness
  (\url{https://en.wikipedia.org/wiki/Basic_reproduction_number}).
\end{exercise}
