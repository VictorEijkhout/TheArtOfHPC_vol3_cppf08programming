% -*- latex -*-
%%%%%%%%%%%%%%%%%%%%%%%%%%%%%%%%%%%%%%%%%%%%%%%%%%%%%%%%%%%%%%%%
%%%%
%%%% This TeX file is part of the course
%%%% Introduction to Scientific Programming in C++11/Fortran2003
%%%% copyright 2017-2020 Victor Eijkhout eijkhout@tacc.utexas.edu
%%%%
%%%% google-abstract.tex : abstract for the redistricting project
%%%%
%%%%%%%%%%%%%%%%%%%%%%%%%%%%%%%%%%%%%%%%%%%%%%%%%%%%%%%%%%%%%%%%

The Google search engine uses an algorithm called `Pagerank'
to decide on the importance of web pages, that is,
how high to display them in search results.
This algorithm, in its original form, can be interpreted in two different ways.
Firstly, it emulates the behavior of a computer user clicking on links.
Secondly, it solves a mathematical linear algebra problem.

In this project, the student first builds up a simulated internet,
and models the behavior of a user randomly clicking on links.
Secondly, an implementation is made that is closer to linear
algebra concept. Both models are used to explore the Pagerank algorithm.

This project can be done by one or two undergraduate students,
or AP high schoolers
at the end of a first or semester programming course.

There is opportunity for learning some non-standard mathematics.

\newpage

\begin{tabular}{|l|p{5in}|}
  \hline
  Summary&Explore the Pagerank algorithm for ranking web search results.
  \\
  Topics&Pointers, arrays, matrices, classes\\
  &There is a possibility of exploring sparse matrices
  \\
  Audience&Undergraduate or AP high school
  \\
  Difficulty&Moderate to high
  \\
  Strengths&Explores a phenomenon that is in everyone's experience
  \\
  Weaknesses&The two parts (pointer-based, matrix-based) are independent
  of each other.
  \\
  Dependencies&No dependencies.
  \\
  Variants&The matrix approach can be extended to sparse matrices.
  \\
  \hline
\end{tabular}
