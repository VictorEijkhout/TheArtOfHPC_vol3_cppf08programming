% -*- latex -*-
%%%%%%%%%%%%%%%%%%%%%%%%%%%%%%%%%%%%%%%%%%%%%%%%%%%%%%%%%%%%%%%%
%%%%
%%%% This TeX file is part of the course
%%%% Introduction to Scientific Programming in C++/Fortran2003
%%%% copyright 2017-2023 Victor Eijkhout eijkhout@tacc.utexas.edu
%%%%
%%%% stl.tex : about the standard template library
%%%%
%%%%%%%%%%%%%%%%%%%%%%%%%%%%%%%%%%%%%%%%%%%%%%%%%%%%%%%%%%%%%%%%

The C++ language has a \indexterm{Standard Template Library} (STL),
which contains functionality that is considered standard, but that is
actually implemented in terms of already existing language
mechanisms. The STL is enormous, so we just highlight a couple of
parts.

You have already seen
\begin{itemize}
\item
  arrays (chapter~\ref{ch:array}),
\item strings (chapter~\ref{ch:string}),
\item streams (chapter~\ref{ch:io}).
\end{itemize}

Using a template class typically involves
\begin{lstlisting}
#include <something>
using std::function;
\end{lstlisting}
see section~\ref{sec:usename}.

\Level 0 {Complex numbers}
\label{sec:stl-complex}

\emph{Complex numbers}\index{complex numbers|textbf}
require the \indextermheader{complex} header.
The \indexcdef{complex} type uses templating to set the precision.
\begin{lstlisting}
#include <complex>
complex<float> f;
f.re = 1.; f.im = 2.;

complex<double> d(1.,3.);
\end{lstlisting}
Math operator like \n{+,*} are defined, as are math functions
such as \indexc{exp}.
Expressions involving a complex number and a simple scalar
are well-defined if the scalar is of the underlying type
of the complex number:
\begin{lstlisting}
complex<float> x;
x + 1.f; // Yes
x + x.;  // No, because `1.' is double
\end{lstlisting}

Imaginary unit number~$i$ through literals
\lstinline{i}, \lstinline{if} (float), \lstinline{il} (long):
\begin{lstlisting}
 using namespace std::complex_literals;
    std::complex<double> c = 1.0 + 1i;
\end{lstlisting}
Beware: \lstinline{1+1i} does not compile.

\begin{block}{Example usage}
  \label{sl:complexvec}
  \snippetwithoutput{cplxvec}{complex}{vec}
\end{block}

Support:
\begin{lstlisting}
std::complex<T> conj( const std::complex<T>& z );
std::complex<T> exp( const std::complex<T>& z );
\end{lstlisting}

\begin{slide}{Complex numbers}
  \label{sl-complex}
\begin{lstlisting}
#include <complex>

complex<float> f;
f.re = 1.; f.im = 2.;
complex<double> d(1.,3.);

using std::complex_literals::i;
std::complex<double> c = 1.0 + 1i;

conj(c); exp(c);
\end{lstlisting}
\end{slide}

\Level 1 {Complex support in C}

The C~language has had complex number support
since \cstandard{99} with the types
\begin{lstlisting}
  float _Complex
  double _Complex
  long double _Complex
\end{lstlisting}
The header \indextermheader{complex.h}
gives synonyms
\begin{lstlisting}
  float complex
  double complex
  long double complex
\end{lstlisting}
for these.

See for instance \url{https://en.cppreference.com/w/c/numeric/complex}
for details.

\Level 0 {Limits}
\label{sec:limits}

There used to be a header file \indextermtt{limits.h} that contained
macros such as \indexc{MAX_INT} and \indexc{MIN_INT}.
While this is still available,
the \ac{STL} offers a better solution in the
\indexc{numeric_limits} function
of the \indextermheader{numeric} header.

\begin{block}{Templated functions for limits}
  \label{sl:stl-limits}
  Use header file \indextermtt{limits}:
\begin{lstlisting}
#include <limits>
using std::numeric_limits;

cout << numeric_limits<long>::max();
\end{lstlisting}
\end{block}

\begin{block}{Limits of floating point values}
  \label{sl:float-limits}
  \begin{itemize}
  \item The largest number is given by \indexcdef{max};
    use \indexcdef{lowest} for `most negative'.
  \item The smallest denormal number is given by \indexcdef{denorm_min}.
  \item \indexcdef{min} is the smallest positive number
    that is not a denormal;
  \item There is an \indexcdef{epsilon} function for machine precision:
  \end{itemize}
  \snippetwithoutput{stllimitfloat}{stl}{eps}
\end{block}

\begin{block}{Some limit values}
  \label{sl:ieee-limits}
    \def\codesize{\ttfamily\scriptsize}
  \snippetwithoutput{stllimits}{stl}{limits}
\end{block}

\begin{exercise}
  \label{ex:big-factorial}
  Write a program to discover what the maximal~$n$ is so that~$n!$,
  that is, $n$-factorial, can be represented in an \n{int}, \n{long},
  or \n{long long}. Can you write this as a templated function?
\end{exercise}

Operations such as dividing by zero lead to floating point numbers
that do not have a valid value. For efficiency of computation, the
processor will compute with these as if they are any other floating
point number.

\Level 0 {Containers}

C++ has several types of \indextermdef{container}s.
You have already seen \indexcstd{vector}
(section~\ref{sec:stdvector})
and \indexcstd{array}
(section~\ref{sec:stdarray})
and strings (chapter~\ref{ch:string}).
Many containers have 
methods such as \lstinline{push_back} and \lstinline{insert} in common.

In this section we will look at a couple more containers.

\Level 1 {Maps: associative arrays}
\label{sec:map}

Arrays use an integer-valued index. Sometimes you may wish to use an
index that is not ordered, or for which the ordering is not relevant.
A~common example is looking up information by string, such as finding
the age of a person, given their name. This is sometimes called
`indexing by content', and the data structure that supports this is
formally known as an \indextermsub{associative}{array}.

In C++ this is implemented through a \indexcdef{map}:
\begin{lstlisting}
#include <map>
using std::map;
map<string,int> ages;
\end{lstlisting}
is set of
pairs where the first item (which is used for indexing) is of type
\lstinline{string}, and the second item (which is found) is of type \lstinline{int}.

A map is made by inserting the elements one-by-one:
\begin{lstlisting}
#include <map>
using std::make_pair;
ages.insert(make_pair("Alice",29));
ages["Bob"] = 32;
\end{lstlisting}

You can range over a map:
\begin{lstlisting}
for ( auto person : ages )
  cout << person.first << " has age " << person.second << endl;
\end{lstlisting}
A more elegant solution uses
\indexterm{structured binding} (section~\ref{sec:tuple}):
\begin{lstlisting}
for ( auto [person,age] : ages )
  cout << person << " has age " << age << endl;
\end{lstlisting}
(It is possible to use const-references here.)

Searching for a key gives either the iterator of the key/value pair,
or the \lstinline{end} iterator if not found:
%
\verbatimsnippet{intcountfind}

\begin{exercise}
  If you're doing the prime number project, you can now do
  the exercises in section~\ref{sec:prime-decomp}.
\end{exercise}

\Level 1 {Sets}
\label{sec:stdset}

The \indexcdef{set} container is like a \lstinline+map<SomeType,bool>+,
that is, it only says`this element is present'.
Like a mathematical set, in other words.

\begin{lstlisting}
#include <set>
std::set<int> my_ints;
my_ints.insert(5);
my_ints.empty(); // predicate
my_ints.size(); // obvious
\end{lstlisting}

You can iterate over a set:
\begin{lstlisting}
for ( auto x : my_ints )
  // do something  
\end{lstlisting}
This gives you the elements in no particular order.

You can search through a set with \indexc{find},
which results in an iterator.
If no match is found the \lstinline{end} iterator
is returned.
\begin{lstlisting}
auto itr = my_ints.find(6);
if ( itr==my_ints.end() )
  cout << "not found\n";
\end{lstlisting}
You can search on elements that satisfy a predicate
with \indexc{find_if}:
\begin{lstlisting}
auto res = find_i(my_ints.begin(),my_ints.end(),
        [] ( auto e ) { return e>37; } );
\end{lstlisting}

Sets are useful in many computer algorithms.
You often encounter idioms such as
\begin{lstlisting}
while (not done) {
  for ( x in unprocessed ) {
    if (something with x) {
      processed.add(x);
      unprocessed.remove(x);
    }
  }
  }
\end{lstlisting}
See for instance \HPSCref{sec:sssp}.

\Level 0 {Regular expression}

The header \indexc{regex} gives C++ the functionality for
\indexterm{regular expression} matching. For instance,
\indexcdef{regex_match} returns whether or not
a string matches an expression exactly:

\snippetwithoutput{regexname}{regexp}{regexp}

(Note that the regex matches substrings, but
\indexcdef{regex_match} only returns true for
a match on the whole string.

For finding substrings, use \indexcdef{regex_search}:
\begin{itemize}
\item the function itself evaluates to a \lstinline{bool};
\item there is an optional return parameter of type \indexterm{smatch}
  (`string match') with information about the match.
\end{itemize}
The \indexterm{smatch} object has these methods:
\begin{itemize}
\item \lstinline{smatch::position} states where the expression
  was matched,
\item while \lstinline{smatch::str} returns the string that
  was matched.
\item \lstinline{smatch::prefix} has the string preceding the match;
  with \lstinline{smatch::prefix().size()} you get the number of characters
  preceding the match, that is, the location of the match.
\end{itemize}

\snippetwithoutput{regsearch}{regexp}{search}

\Level 1 {Regular expression syntax}

C++ uses a variant of the \index{ECMA} International regular expression syntax.
\url{http://ecma-international.org/ecma-262/5.1/#sec-15.10}.
Consult that document for escape characters and more.

If your regular expression is getting too complicated with escape characters
and such, consider using the
\emph{raw string literal}\index{string!raw literal}
construct.

\Level 0 {Tuples and structured binding}
\label{sec:tuple}

Remember how in section~\ref{sec:pass-by-ref} we said that if you
wanted to return more than one value, you could not do that through a
return value, and had to use an \indextermsub{output}{parameter}?
Well, using the \ac{STL} there is a different solution.

You can make a \indextermdef{tuple} with \indexcdefstd{tuple}:
an entity that comprises several
components, possibly of different type, and which unlike a
\indexc{struct} you do not need to define beforehand.
(For tuples with exactly two elements, use \indexcdef{pair}.)

In \cppstandard{11} you would use the \indexc{get} method
to extract elements from a pair or tuple:

\lstset{style=reviewcode,language=C++}
\begin{block}{C++11 style tuples}
  \label{sl:tuple11}
\begin{lstlisting}
#include <tuple>

std::tuple<int,double,char> id = \
    std::make_tuple<int,double,char>( 3, 5.12, 'f' );
    // or:
    std::make_tuple( 3, 5.12, 'f' );
double result = std::get<1>(id);
std::get<0>(id) += 1;

// also:
std::pair<int,char> ic = make_pair( 24, 'd' );
\end{lstlisting}
Annoyance: all that `get'ting.
\end{block}

This does not look terribly elegant. Fortunately,
\cppstandard{17} can use denotations and the \lstinline{auto}
keyword to make this considerably shorter. Consider the case of a
function that returns a tuple. You could use \lstinline{auto} to deduce the
return type:
%
\verbatimsnippet{tuplemake}
%
but more interestingly, you can use a
\indextermbus{tuple}{denotation}:
%
\verbatimsnippet{tupledenote}

Here the return type of the function is indicated,
but the \lstinline{return} statements do not
construct this type explicitly.

\begin{slide}{Function returning tuple}
  \label{sl:tupleexplicit}
  How do you return two things of different types?
  \lstset{numbers=left,numberstyle=\tiny}
\begin{lstlisting}
#include <tuple>
using std::make_tuple, std::tuple;

tuple<bool,float> maybe_root1(float x) {
  if (x<0)
    return make_tuple<bool,float>(false,-1);
  else
    return make_tuple<bool,float>(true,sqrt(x));
};
  \end{lstlisting}
(not the best solution for the `root' code) 
\end{slide}

\begin{slide}{Returning tuple with type deduction}
  \label{sl:tuplefun}
  \lstset{numbers=left,numberstyle=\tiny}
  \begin{multicols}{2}
    Return type deduction:
    \verbatimsnippet{tuplemake}\columnbreak
    Alternative:
    \verbatimsnippet{tupledenote}
  \end{multicols}
  Note: use \lstinline{pair} for \lstinline{tuple} of two.
\end{slide}

\begin{block}{Catching a returned tuple}
  \label{sl:catch-tuple}
  The calling code is particularly elegant:
  %
  \snippetwithoutput{tupleauto}{stl}{tuple}

  This is known as \indextermdef{structured binding}.
\end{block}

The above examples,
with one component of the tuple being a \lstinline{bool},
are better formulated differently, as you'll see below.
An more appropriate use of structured binding is iterating over a map (section~\ref{sec:map}):
\begin{lstlisting}
for ( const auto &[key,value] : mymap ) ....
\end{lstlisting}

\Level 0 {Union-like stuff: tuples, optionals, variants, expected}

There are cases where you need a value that is one type or another,
for instance, a number if a computation succeeded, and an error
indicator if not.

The simplest solution is to have a function that returns both a bool
and a number:
%
\verbatimsnippet{rootorerror}

We will now consider some more idiomatically C++17 solutions to this.

\Level 1 {Tuples}

Using tuples or pairs (section~\ref{sec:tuple}) 
the solution to the above `a~number or an error' now becomes
a tuple of a \lstinline{bool} and a \lstinline{float}:
%
\verbatimsnippet{rootandvalid}

Note the \indextermbus{initializer}{statement}
in the conditional.

\Level 1 {Optional}
\label{sec:std-optional}

The most elegant solution to `a~number or an error' is to have a
single quantity that you can query whether it's valid.
For this, the
\cppstandard{17} standard
introduced the concept of a \indextermsub{nullable}{type}:
a~type that can somehow convey that it's empty.

Here we discuss \lstinline{std::}\indexc{optional}.

\begin{lstlisting}
#include <optional>
using std::optional;
\end{lstlisting}

\Level 2 {Creating an optional}

You can create an optional quantity with a function that returns
either a value of the indicated type, or \verb+{}+, which is a
synonym for \indexcstd{nullopt}.

\begin{block}{Create optional}
  \label{sl:opt-create}
  \begin{lstlisting}
    #include <optional>
    using std::optional;

    optional<float> f {
      if (something) return 3.14;
      else return {};
    }
  \end{lstlisting}
\end{block}

\Level 2 {Optional value}

You can test whether the optional quantity actually has a quantity with
the method \indexc{has_value},
in which case you can extract the quantity
with the method \indexc{value}:

\begin{block}{Optional value}
  \label{sl:opt-value}
  \begin{lstlisting}
    auto maybe_x = f();
    if (f.has_value())
    // do something with f.value();
  \end{lstlisting}

  Trying to take the \indexc{value} for something that doesn't have one
  leads to a \indexc{bad_optional_access} exception:
  \snippetwithoutput{badvalueaccess}{union}{optional}
\end{block}


There is a function \indexcdef{value_or} that gives the value, or
a default if the optional did not have a value.
\begin{lstlisting}
  optional<int> some_int;
  some_int.value_or(-1);
\end{lstlisting}

\begin{slide}{Optional results}
  \label{sl:optional-root}
  The most elegant solution to `a~number or an error' is to have a
  single quantity that you can query whether it's valid.
  %
\begin{lstlisting}
#include <optional>
\end{lstlisting}
  \lstset{numbers=left,numberstyle=\tiny}
  \verbatimsnippet{mayberootptr}
\end{slide}

\begin{exercise}
  If you are doing the prime number project,
  you can now do exercise~\ref{ex:primeoptfact}.
\end{exercise}

\begin{exercise}
  The \indexterm{eight queens} problem (chapter~\ref{ch:queens})
  can be elegantly solved using \lstinline+std::optional+.
  See also section~\ref{sec:8queens-tdd} for a \ac{TDD} approach.
\end{exercise}

\begin{remark}
  If you have an optional class object, you can assign that object
  with the \indexc{emplace} method:
  \verbatimsnippet{optionalconstruct}
\end{remark}

\Level 1 {Expected}

The \indexcstd{optional} of the previous section is great
for cases, such as the square root, where it is clear
what it means if the value does not exist.
However, if the non-existence comes from some sort of error,
you may want to ask what the reason for it is.

The \cppstandard{23} addition of \indexcstd{expected}
allows you to return a value, or provide more information
on why that error is not there.
\begin{block}{Expected}
\label{sl:expected}
Expect double, return info string if not:
\begin{multicols}{2}
\begin{lstlisting}
std::expected<double,string> 
      square_root( double x ) {
  auto result = sqrt(x);
  if (x<0)
  return std::unexpected("negative");
  else if (x<limits<double>::min())
  return std::unexpected("underflow");
  else return result;
}
\end{lstlisting}
\columnbreak
\begin{lstlisting}
auto root = square_root(x);
if (x)
cout << "Root=" << root.value() << '\n';
else if (root.error()==/* et cetera */ )
/* handle the problem */
\end{lstlisting}
\end{multicols}
\end{block}

\Level 1 {Variant}
\label{sec:stl-variant}

In~C, a~\indexc{union} is an entity that can be one of a number of types.
Similar to how C~arrays do not know their size,
a~\lstinline{union} does not know what type it is.
The C++ \indexcdef{variant} (\cppstandard{17})
does not suffer from these limitations.
The function \indexcdef{get_if} can retrieve a value by type.

For a first example we consider the square root example.

\begin{block}{Square root with variant}
  \label{sl:root-variant}
  \begin{multicols}{2}
    \verbatimsnippet{rootvariantdef}
    \columnbreak
    \verbatimsnippet{rootvariantuse}
  \end{multicols}
\end{block}

Showing some more possibilities with a variant of int, double, string:
\codesnippet{unionids}

We can use the \indexc{index} function to see what variant is used
(0,1,2 in this case)
and \indexc{get} the value accordingly:
\codesnippet{unionids_index}

Getting the wrong variant leads to a \indexc{bad_variant_access} exception:
\codesnippet{unionids_bad}

It is safer to use \indexc{get_if} which
takes a pointer to a variant,
and return a pointer if successful, and a null pointer if not:
\codesnippet{unionids_getif}

Note that the argument of \lstinline{get_if} is the address of the variant,
and the return result is also a pointer.

\begin{slide}{Variant methods}
  \label{sl:cpp-variant}
  \codesnippet{unionids}
  Get the index of what the variant contains:
  \codesnippet{unionids_index}
  
  \codesnippet{unionids_getif}
  (Takes pointer to variant, returns pointer to value)
\end{slide}

\begin{exercise}
  \label{ex:quad-roots}
  Write a routine that computes the roots of the quadratic equation
  \[ ax^2+bx+c=0. \]
  The routine should return two roots, or one root, or an indication
  that the equation has no solutions.
  \snippetwithoutput{quadraticloop}{union}{quadratic}
\end{exercise}

In this exercise you can return a boolean to indicate `no roots', but
a boolean can have two values, and only one has meaning. For such
cases there is \lstinline{std::}\indexcdef{monostate}.

\Level 2 {The same function on all variants}

Suppose you have a variant of some classes,
which all support an identically prototyped method:
\begin{lstlisting}
class x_type {
public: r_type method() { ... };
};
class y_type {
public: r_type method() { ... };
\end{lstlisting}
It is not directly possible to call this method on a variant:
\begin{lstlisting}
variant< x_type,y_type> xy;
// WRONG xy.method();
\end{lstlisting}

For a specific example:
%
\codesnippet{sizeruse}
%
where we have methods
\lstinline{stringer} that gives a string representation, and
\lstinline{sizer} that gives the `size' of the object.
The latter:
\verbatimsnippet{sizerclass}

The solution for this is \indexcdefstd{visit},
coming from the \lstinline+variant+ header.
This is used to apply an object (defined below) to the variant object:
%
\snippetwithoutput{sizervisit}{union}{visit}

The mechanism to realize this is to have an object
(here \lstinline{stringer} and \lstinline{sizer})
with an overloaded \lstinline+operator()+.

\Level 1 {Any}
\label{sec:stl-any}

While \lstinline{variant} can be any of a number of pre specified
types, \lstinline{std::}\indexcdef{any} can contain really any
type. Thus it is the equivalent of \lstinline{void*} in~C.

An \lstinline{any} object can be cast with \indexcdef{any_cast}:
\begin{lstlisting}
std::any a{12};
std::any_cast<int>(a); // succeeds
std::any_cast<string>(a); // fails
\end{lstlisting}

\input random

\Level 0 {Time}
\label{sec:chrono}

Header
\begin{lstlisting}
#include <chrono>
\end{lstlisting}

Convenient:
\begin{lstlisting}
using namespace std::chrono
\end{lstlisting}
but here we spell it all out.

\Level 1 {Time durations}

You can define durations with \indexcdef{seconds}:
%
\verbatimsnippet{seconddef}

\begin{remark}
  Use one of the following:
  \begin{lstlisting}
    std::chrono::seconds two{2};

    using std::chrono;
    seconds two{2};

    using namespace std::chrono;
    seconds two{2};
  \end{lstlisting}
\end{remark}

You can do arithmetic and comparisons on this type:
%
\snippetwithoutput{secondarith}{chrono}{basicsecond}
%
Note that while \indexc{seconds} takes an integer argument,
you can then multiply or divide it to get fractional values.

There is a duration \indexcdef{millisecond},
and you can convert seconds implicitly to milli,
but the other way around you need \indexcdef{duration_count}:
%
\verbatimsnippet{millisecond}

The full list of durations (with suffixes) is:
\indexc{hours} (\lstinline{1h}),
\indexc{minutes} (\lstinline{1min}),
\indexc{seconds} (\lstinline{1s}),
\indexc{milliseconds} (\lstinline{1ms}),
\indexc{microseconds} (\lstinline{1us}),
\indexc{nanoseconds} (\lstinline{1ns}).

\Level 1 {Time points}

A \indexterm{time point} can be considered as a duration from
some starting point, such as the start of the Unix \indexterm{epoch}:
the start of the year 1970.
\begin{lstlisting}
time_point<system_clock,seconds> tp{10'000s};
\end{lstlisting}
is \lstinline*2h+46min+40s* into 1970.

You make this explicit by calling the \lstinline+time_since_epoch+ method
on a time point, giving a duration.

\Level 1 {Clocks}

There are several clocks. The common supplied clocks
are
\begin{itemize}
\item \indexcdef{system_clock} for time points that have a relation to the calendar;
  and
\item \indexcdef{steady_clock} for precise measurements.
\end{itemize}

Usually, \indexcdef{high_resolution_clock} is a synonym
for either of these.

A clock has properties:
\begin{itemize}
\item \lstinline{duration}
\item \lstinline{rep}
\item \lstinline{period}
\item \lstinline+time_point+
\item \lstinline+is_steady+
\item and a method \lstinline+now()+.
\end{itemize}

As you saw above, a \lstinline+time_point+ is associated with a clock,
and time points of different clocks can not be compared or converted to each other.

\Level 2 {Duration measurement}

To time a segment of execution, use the \indexc{now} method of the clock,
before and after the segment.
Subtracting the time points gives a duration in nanoseconds,
which you can cast to anything else:
%
\snippetwithoutput{clocksleep}{chrono}{clock}

(The \indexc{sleep} function is not a \lstinline{chrono} function,
but comes from the \indextermheader{thread} header;
see section~\ref{sec:this-thread}.)

\Level 2 {Clock resolution}

The \indextermbus{clock}{resolution} can be found from the \indexc{period} property:
\begin{lstlisting}
auto
  num = myclock::period::num,
  den = myclock::period::den;
auto tick = static_cast<double>(num)/static_cast<double>(den);
\end{lstlisting}

Timing:
\begin{lstlisting}
auto start_time = myclock::now();
auto duration = myclock::now()-start_time;
auto microsec_duration =
    std::chrono::duration_cast<std::chrono::microseconds>(duration);
cout << "This took " << microsec_duration.count() << "usec" << endl;
\end{lstlisting}

Computing new time points:
\begin{lstlisting}
auto deadline = myclock.now() + std::chrono::seconds(10);
\end{lstlisting}

\begin{slide}{Chrono}
\label{sl:chrono}
\begin{lstlisting}
#include <chrono>

// several clocks
using myclock = std::chrono::high_resolution_clock;

// time and duration
auto start_time = myclock::now();
auto duration = myclock::now()-start_time;
auto microsec_duration =
    std::chrono::duration_cast<std::chrono::microseconds>
                (duration);
cout << "This took "
     << microsec_duration.count() << "usec\n"
\end{lstlisting}
\end{slide}

\Level 1 {C mechanisms not to use anymore}

Letting your process sleep: \indexc{sleep}

Time measurement: \indexc{getrusage}

\Level 0 {File system}

As of the \cppstandard{17} standard,there is a file system header,
\indexc{filesystem},
which includes things like a directory walker.
\begin{lstlisting}
#include <filesystem>
\end{lstlisting}

\Level 0 {Enum classes}
\label{sec:enum-class}

The C-style \indexc{enum} keyword introduced global names, so
\verbatimsnippet{enumcollide}
does not work.

In C++ the \indexc{enum class}
(or \indexc{enum struct})
was introduced, which makes the names into class members:
\verbatimsnippet{enumclass}

If such a class inherits from an integral type,
this does not mean it behaves like an integer;
for instance, you can not immediately ask if one is less than another.
Instead, it only determines the amount of space taken for an enum item.

To let it behave like an integral type,you need to cast it:
\verbatimsnippet{enumclassint}

If you only want a namespace-d enum:
\verbatimsnippet{enumspaced}

\Level 0 {Orderings and the `spaceship' operator}

It is possible to overload comparison operators
\lstinline|<,<=,==,>=,>,!=|
for classes; section~\ref{sec:operatordef}.
However, there is usually a lot of redundancy in doing so,
even if you express one operator in terms of another:
\begin{lstlisting}
  operator>=( const MyClass& other ) {
    return not (other<*this);
    };
\end{lstlisting}

The \cppstandard{20} \indextermtt{spaceship oeprator}
\lstinline|<=>|
can make life a lot easier.
This operator returns an ordering object,
which can be one of the six relations.

Let's illustrate with a simple example.
Each object of the class \lstinline{Record}
has a string and a unique number:
\verbatimsnippet{spaceshipclass1}

Now you want an ordering relation purely based on the record number.
\snippetwithoutput{spaceshipclass1test}{stl}{spacerecord}

For this we implement the spaceship operator,
using the fact that an ordering on integers exists:
\verbatimsnippet{spaceshipclass1order}

Note that the return type of this comparison is
\indexcstd{strong_ordering},
because in this case any two objects \lstinline{x,y}
always satisfy exactly one of
\begin{lstlisting}
  x<y  x==y  x>y
\end{lstlisting}

For a more complicated example, let's make a \lstinline{Coordinate} class,
where two coordinates compare as
\lstinline|<,=,>|
if all components satisfy that relation.
This is a partial ordering,
meaning that some coordinates \lstinline{x,y}
do not satisfy any of
\begin{lstlisting}
  x<y, x==y, x>y
\end{lstlisting}
\snippetwithoutput{spaceshipclass2test}{stl}{spacepartial}

In this case the spaceship operator returns a
\indexc{partial_ordering} result.
Also, we need to define equality separately.
\verbatimsnippet{spaceshipclass2order}
