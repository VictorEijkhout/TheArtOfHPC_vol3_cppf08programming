% -*- latex -*-
%%%%%%%%%%%%%%%%%%%%%%%%%%%%%%%%%%%%%%%%%%%%%%%%%%%%%%%%%%%%%%%%
%%%%
%%%% This TeX file is part of the course
%%%% Introduction to Scientific Programming in C++/Fortran2003
%%%% copyright 2017-2022 Victor Eijkhout eijkhout@tacc.utexas.edu
%%%%
%%%% stl.tex : about the standard template library
%%%%
%%%%%%%%%%%%%%%%%%%%%%%%%%%%%%%%%%%%%%%%%%%%%%%%%%%%%%%%%%%%%%%%

The C++ language has a \indexterm{Standard Template Library} (STL),
which contains functionality that is considered standard, but that is
actualy implemented in terms of already existing language
mechanisms. The STL is enormous, so we just highlight a couple of
parts.

You have already seen
\begin{itemize}
\item
  arrays (chapter~\ref{ch:array}),
\item strings (chapter~\ref{ch:string}),
\item streams (chapter~\ref{ch:io}).
\end{itemize}

Using a template class typically involves
\begin{lstlisting}
#include <something>
using std::function;
\end{lstlisting}
see section~\ref{sec:usename}.

\Level 0 {Complex numbers}
\label{sec:stl-complex}

\emph{Complex numbers}\index{complex numbers|textbf}
require the \indextermheader{complex} header.
The \indexc{complex} type uses templating to set the precision.
\begin{lstlisting}
#include <complex>
complex<float> f;
f.re = 1.; f.im = 2.;

complex<double> d(1.,3.);
\end{lstlisting}
Math operator like \n{+,*} are defined, as are math functions
such as \indexc{exp}.

Imaginary unit number~$i$ through literals
\lstinline{i}, \lstinline{if} (float), \lstinline{il} (long):
\begin{lstlisting}
 using namespace std::complex_literals;
    std::complex<double> c = 1.0 + 1i;
\end{lstlisting}
Beware: \lstinline{1+1i} does not compile.

\begin{block}{Example usage}
  \label{sl:complexvec}
  \snippetwithoutput{cplxvec}{complex}{vec}
\end{block}

Support:
\begin{lstlisting}
std::complex<T> conj( const std::complex<T>& z );
std::complex<T> exp( const std::complex<T>& z );
\end{lstlisting}

\begin{slide}{Complex numbers}
  \label{sl-complex}
\begin{lstlisting}
#include <complex>

complex<float> f;
f.re = 1.; f.im = 2.;
complex<double> d(1.,3.);

using std::complex_literals::i;
std::complex<double> c = 1.0 + 1i;

conj(c); exp(c);
\end{lstlisting}
\end{slide}

\Level 1 {Complex support in C}

The C~language has had complex number support
since \cstandard{99} with the types
\begin{lstlisting}
  float _Complex
  double _Complex
  long double _Complex
\end{lstlisting}
The header \indextermheader{complex.h}
gives synonyms
\begin{lstlisting}
  float complex
  double complex
  long double complex
\end{lstlisting}
for these.

See for instance \url{https://en.cppreference.com/w/c/numeric/complex}
for details.

\Level 0 {Containers}

C++ has several types of \indextermdef{container}s.
You have already seen \indexcstd{vector}
(section~\ref{sec:stdvector})
and \indexcstd{array}
(section~\ref{sec:stdarray})
and strings (chapter~\ref{ch:string}).
Many containers have 
methods such as \n{push_back} and \n{insert} in common.

In this section we will look at a couple more types.

\Level 1 {Maps: associative arrays}
\label{sec:map}

Arrays use an integer-valued index. Sometimes you may wish to use an
index that is not ordered, or for which the ordering is not relevant.
A~common example is looking up information by string, such as finding
the age of a person, given their name. This is sometimes called
`indexing by content', and the data structure that supports this is
formally known as an \indextermsub{associative}{array}.

In C++ this is implemented through a \indexcdef{map}:
\begin{lstlisting}
#include <map>
using std::map;
map<string,int> ages;
\end{lstlisting}
is set of
pairs where the first item (which is used for indexing) is of type
\n{string}, and the second item (which is found) is of type \n{int}.

A map is made by inserting the elements one-by-one:
\begin{lstlisting}
#include <map>
using std::make_pair;
ages.insert(make_pair("Alice",29));
ages["Bob"] = 32;
\end{lstlisting}

You can range over a map:
\begin{lstlisting}
for ( auto person : ages )
  cout << person.first << " has age " << person.second << endl;
\end{lstlisting}
A more elegant solution uses structured bindings (section~\ref{sec:tuple}):
\begin{lstlisting}
for ( auto [person,age] : ages )
  cout << person << " has age " << age << endl;
\end{lstlisting}

Searching for a key gives either the iterator of the key/value pair,
or the \lstinline{end} iterator if not found:
%
\verbatimsnippet{intcountfind}

\begin{exercise}
  If you're doing the prime number project, you can now do
  the exercises in section~\ref{sec:prime-decomp}.
\end{exercise}

\Level 0 {Regular expression}

The header \indexc{regex} gives C++ the functionality for
\indexterm{regular expression} matching. For instance,
\indexcdef{regex_match} returns whether or not
a string matches an expression exactly:

\snippetwithoutput{regexname}{regexp}{regexp}

(Note that the regex matches substrings, but
\indexcdef{regex_match} only returns true for
a match on the whole string.

For finding substrings, use \indexcdef{regex_search}:
\begin{itemize}
\item the function itself evaluates to a \lstinline{bool};
\item there is an optional return parameter of type \indexterm{smatch}
  (`string match') with information about the match.
\end{itemize}
The \indexterm{smatch} object has these methods:
\begin{itemize}
\item \lstinline{smatch::position} states where the expression
  was matched,
\item while \lstinline{smatch::str} returns the string that
  was matched.
\item \lstinline{smatch::prefix} has the string preceeding the match;
  with \lstinline{smatch::prefix().size()} you get the number of characters
  preceeding the match, that is, the location of the match.
\end{itemize}

\snippetwithoutput{regsearch}{regexp}{search}

\Level 1 {Regular expression syntax}

C++ uses a variant of the \index{ECMA} International regular expression syntax.
\url{http://ecma-international.org/ecma-262/5.1/#sec-15.10}.
Consult that document for escape characters and more.

If your regular expression is getting too complicated with escape characters
and such, consider using the
\emph{raw string literal}\index{string!raw literal}
construct.

\Level 0 {Tuples and structured bindings}
\label{sec:tuple}

Remember how in section~\ref{sec:pass-by-ref} we said that if you
wanted to return more than one value, you could not do that through a
return value, and had to use an \indextermsub{output}{parameter}?
Well, using the \ac{STL} there is a different solution.

You can make a \indextermdef{tuple} with \indexcdefstd{tuple}:
an entity that comprises several
components, possibly of different type, and which unlike a
\indexc{struct} you do not need to define beforehand.
(For tuples with exactly two elements, use \indexcdef{pair}.)

\lstset{style=reviewcode,language=C++}
\begin{block}{C++11 style tuples}
  \label{sl:tuple11}
\begin{lstlisting}
#include <tuple>

std::tuple<int,double,char> id = \
    std::make_tuple<int,double,char>( 3, 5.12, 'f' );
    // or:
    std::make_tuple( 3, 5.12, 'f' );
double result = std::get<1>(id);
std::get<0>(id) += 1;

// also:
std::pair<int,char> ic = 
  make_pair( 24, 'd' );
\end{lstlisting}
Annoyance: all that `get'ting.
\end{block}

This does not look terribly elegant. Fortunately,
\cppstandard{17} can use denotations and the \n{auto}
keyword to make this considerably shorter. Consider the case of a
function that returns a tuple. You could use \n{auto} to deduce the
return type:
%
\verbatimsnippet{tuplemake}
%
but more interestingly, you can use a
\indextermbus{tuple}{denotation}:
%
\verbatimsnippet{tupledenote}

\begin{slide}{Function returning tuple}
  \label{sl:tuplefun}
  \begin{multicols}{2}
    Return type deduction:
    \verbatimsnippet{tuplemake}\columnbreak
    Alternative:
    \verbatimsnippet{tupledenote}
  \end{multicols}
\end{slide}

\begin{block}{Catching a returned tuple}
  \label{sl:catch-tuple}
  The calling code is particularly elegant:
  %
  \snippetwithoutput{tupleauto}{stl}{tuple}

  This is known as \indexterm{structured binding}.
\end{block}

An interesting use of structured bindings is iterating over a map (section~\ref{sec:map}):
\begin{lstlisting}
for ( const auto &[key,value] : mymap ) ....
\end{lstlisting}

\Level 0 {Union-like stuff: tuples, optionals, variants}

There are cases where you need a value that is one type or another,
for instance, a number if a computation succeeded, and an error
indicator if not.

The simplest solution is to have a function that returns both a bool
and a number:
%
\verbatimsnippet{rootorerror}

We will now consider some more idiomatically C++17 solutions to this.

\Level 1 {Tuples}

Using tuples (section~\ref{sec:tuple}) 
the solution to the above `a~number or an error' now becomes:
%
\verbatimsnippet{rootandvalid}

\Level 1 {Optional}
\label{sec:std-optional}

The most elegant solution to `a~number or an error' is to have a
single quantity that you can query whether it's valid.
For this, the
\cppstandard{17} standard
introduced the concept of a \indextermsub{nullable}{type}:
a~type that can somehow convey that it's empty.

Here we discuss \lstinline{std::}\indexc{optional}.

\begin{lstlisting}
#include <optional>
using std::optional;
\end{lstlisting}

\begin{itemize}
\item You can create an optional quantity with a function that returns
  either a value of the indicated type, or \verb+{}+, which is a
  synonym for \indexcstd{nullopt}.
\begin{lstlisting}
optional<float> f {
  if (something) return 3.14;
  else return {};
}
\end{lstlisting}
\item You can test whether the optional quantity has a quantity with
  the method \indexc{has_value}, in which case you can extract the quantity
  with \indexc{value}:
\begin{lstlisting}
auto maybe_x = f();
if (f.has_value)
  // do something with f.value();
\end{lstlisting}
\end{itemize}

There is a function \indexcdef{value_or} that gives the value, or
a default if the optional did not have a value.

\begin{slide}{Optional results}
  \label{sl:optional-root}
  The most elegant solution to `a~number or an error' is to have a
  single quantity that you can query whether it's valid.
  %
\begin{lstlisting}
#include <optional>
\end{lstlisting}
  \verbatimsnippet{mayberootptr}
\end{slide}

\begin{exercise}
  If you are doing the prime number project,
  you can now do exercise~\ref{ex:primeoptfact}.
\end{exercise}

\begin{exercise}
  The \indexterm{eight queens} problem (chapter~\ref{ch:queens})
  can be elegantly solved using \lstinline+std::optional+.
  See also section~\ref{sec:8queens-tdd} for a \ac{TDD} approach.
\end{exercise}

\Level 1 {Variant}
\label{sec:stl-variant}

In~C, a~\indexc{union} is an entity that can be one of a number
of types. Just like that C~arrays do not know their size, a
\lstinline{union} does not know what type it is. The C++
\indexcdef{variant} does not suffer from these limitations. The
function \indexcdef{get_if} can retrieve a value by type.

For a first example we consider the square root example.

\begin{block}{Square root with variant}
  \label{sl:root-variant}
  \begin{multicols}{2}
    \verbatimsnippet{rootvariantdef}
    \columnbreak
    \verbatimsnippet{rootvariantuse}
  \end{multicols}
\end{block}

Showing some more possibilities with a variant of int, double, string:
\codesnippet{unionids}

We can use the \lstinline{index} function to see what variant is used
(0,1,2 in this case)
and \lstinline{get} the value accordingly:
\codesnippet{unionids_index}

Getting the wrong variant leads to a \lstinline{bad_variant_access} exception:
\codesnippet{unionids_bad}

It is safer to use \lstinline{get_if} which gives a pointer
if successful, and false if not:
\codesnippet{unionids_getif}

Note that this needs the address of the variant, and returns
something that you need to dereference.

\begin{slide}{More variant methods}
  \label{sl:cpp-variant}
  \codesnippet{unionids}
  \codesnippet{unionids_index}
  \codesnippet{unionids_getif}
\end{slide}

\begin{exercise}
  \label{ex:quad-roots}
  Write a routine that computes the roots of the quadratic equation
  \[ ax^2+bx+c=0. \]
  The routine should return two roots, or one root, or an indication
  that the equation has no solutions.
  \snippetwithoutput{quadraticloop}{union}{quadratic}
\end{exercise}

In this exercise you can return a boolean to indicate `no roots', but
a boolean can have two values, and only one has meaning. For such
cases there is \lstinline{std::}\indexcdef{monostate}.

\Level 2 {The same function on all variants}

Suppose you have a variant of some classes,
which all support an identically prototyped method:
\begin{lstlisting}
class x_type {
public: r_type method() { ... };
};
class y_type {
public: r_type method() { ... };
\end{lstlisting}
It is not directly possible to call this method on a variant:
\begin{lstlisting}
variant< x_type,y_type> xy;
// WRONG xy.method();
\end{lstlisting}

For a specific example:
%
\codesnippet{sizeruse}
%
where we have methods
\lstinline{stringer} that gives a string representation, and
\lstinline{sizer} that gives the `size' of the object.

The solution for this is \indexcdefstd{visit},
coming from the \lstinline+variant+ header.
This is used to apply an object (defined below) to the variant object:
%
\snippetwithoutput{sizervisit}{union}{visit}

The mechanism to realize this is to have an object
(here \lstinline{stringer} and \lstinline{sizer})
with an overloaded \lstinline+operator()+.
One implementation:
%
\codesnippet{sizerclass}

\Level 1 {Any}
\label{sec:stl-any}

While \lstinline{variant} can be any of a number of prespecified
types, \lstinline{std::}\indexcdef{any} can contain really any
type. Thus it is the equivalent of \lstinline{void*} in~C.

An \lstinline{any} object can be cast with \indexcdef{any_cast}:
\begin{lstlisting}
std::any a{12};
std::any_cast<int>(a); // succeeds
std::any_cast<string>(a); // fails
\end{lstlisting}

\input random

\Level 0 {Time}
\label{sec:chrono}

Header
\begin{lstlisting}
#include <chrono>
\end{lstlisting}

Convenient:
\begin{lstlisting}
using namespace std::chrono
\end{lstlisting}
but here we spell it all out.

\Level 1 {Time durations}

You can define durations with \indexcdef{second}:
%
\verbatimsnippet{seconddef}

You can do arithmetic and comparisons on this type:
%
\snippetwithoutput{secondarith}{chrono}{basicsecond}

There is a duration \indexcdef{millisecond},
and you can convert seconds implicitly to milli,
but the other way around you need \indexcdef{duration_count}:
%
\verbatimsnippet{millisecond}

The full list of durations (with suffixes) is:
\indexc{hours} (\lstinline{1h}),
\indexc{minutes} (\lstinline{1min}),
\indexc{seconds} (\lstinline{1s}),
\indexc{milliseconds} (\lstinline{1ms}),
\indexc{microseconds} (\lstinline{1us}),
\indexc{nanoseconds} (\lstinline{1ns}).

\Level 1 {Time points}

A \indexterm{time point} can be considered as a duration from
some starting point, such as the start of the Unix \indexterm{epoch}:
the start of the year 1970.
\begin{lstlisting}
time_point<system_clock,seconds> tp{10'000s};
\end{lstlisting}
is \lstinline*2h+46min+40s* into 1970.

You make this explicit by calling the \lstinline+time_since_epoch+ method
on a time point, giving a duration.

\Level 1 {Clocks}

There are several clocks. The common supplied clocks
are
\begin{itemize}
\item \indexcdef{system_clock} for time points that have a relation to the calendar;
  and
\item \indexcdef{steady_clock} for precise measurements.
\end{itemize}

Usually, \indexcdef{high_resolution_clock} is a synonym
for either of these.

A clock has properties:
\begin{itemize}
\item \lstinline{duration}
\item \lstinline{rep}
\item \lstinline{period}
\item \lstinline+time_point+
\item \lstinline+is_steady+
\item and a method \lstinline+now()+.
\end{itemize}

As you saw above, a \lstinline+time_point+ is associated with a clock,
and time points of different clocks can not be compared or converted to each other.

\Level 2 {Duration measurement}

To time a segment of execution, use the \indexc{now} method of the clock,
before and after the segment.
Subtracting the time points gives a duration in nanoseconds,
which you can cast to anything else:
%
\snippetwithoutput{clocksleep}{chrono}{clock}

(The sleep function is not a \lstinline{chrono} function,
but comes from the \indexc{thread} header;
see section~\ref{sec:this-thread}.)

\Level 2 {Clock resolution}

The \indextermbus{clock}{resolution} can be found from the \indexc{period} property:
\begin{lstlisting}
auto
  num = myclock::period::num,
  den = myclock::period::den;
auto tick = static_cast<double>(num)/static_cast<double>(den);
\end{lstlisting}

Timing:
\begin{lstlisting}
auto start_time = myclock::now();
auto duration = myclock::now()-start_time;
auto microsec_duration =
    std::chrono::duration_cast<std::chrono::microseconds>(duration);
cout << "This took " << microsec_duration.count() << "usec" << endl;
\end{lstlisting}

Computing new time points:
\begin{lstlisting}
auto deadline = myclock.now() + std::chrono::seconds(10);
\end{lstlisting}

\begin{slide}{Chrono}
\label{sl:chrono}
\begin{lstlisting}
#include <chrono>

// several clocks
using myclock = std::chrono::high_resolution_clock;

// time and duration
auto start_time = myclock::now();
auto duration = myclock::now()-start_time;
auto microsec_duration =
    std::chrono::duration_cast<std::chrono::microseconds>
                (duration);
cout << "This took "
     << microsec_duration.count() << "usec\n"
\end{lstlisting}
\end{slide}

\Level 1 {C mechanisms not to use anymore}

Letting your process sleep: \indexc{sleep}

Time measurement: \indexc{getrusage}

\Level 0 {File system}

As of the \cppstandard{17} standard,there is a file system header,
\indexc{filesystem},
which includes things like a directory walker.
\begin{lstlisting}
#include <filesystem>
\end{lstlisting}

\begin{taccnote}
  The filesystem header seems to be included and working
  only in the \indextermbus{Intel}{OneAPI compiler}.
\end{taccnote}

\Level 0 {Regular expressions}

\begin{block}{Example}
  \label{sl:regex-example}
  \snippetwithoutput{regexname}{regexp}{regexp}
\end{block}

\Level 0 {Enum classes}
\label{sec:enum-class}

The C-style \indexc{enum} keyword introduced global names, so
\verbatimsnippet{enumcollide}
does not work.

In C++ the \indexc{enum class}
(or \indexc{enum struct})
was introduced, which makes the names into class members:
\verbatimsnippet{enumclass}

Even if such a class inherits from an integral type,
you still need to cast it occasionally:
\verbatimsnippet{enumclassint}

If you only want a namespace-d enum:
\verbatimsnippet{enumspaced}

