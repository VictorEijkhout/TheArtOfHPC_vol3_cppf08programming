% -*- latex -*-
%%%%%%%%%%%%%%%%%%%%%%%%%%%%%%%%%%%%%%%%%%%%%%%%%%%%%%%%%%%%%%%%
%%%%
%%%% This TeX file is part of the course
%%%% Introduction to Scientific Programming in C++/Fortran2003
%%%% copyright 2017-2023 Victor Eijkhout eijkhout@tacc.utexas.edu
%%%%
%%%% prime.tex : exercises for prime number finding
%%%%
%%%%%%%%%%%%%%%%%%%%%%%%%%%%%%%%%%%%%%%%%%%%%%%%%%%%%%%%%%%%%%%%

In this chapter you will do a number of exercises regarding prime
numbers that build on each other. Each section lists the required
prerequisites. Conversely, the exercises here are also referenced from
the earlier chapters.

\Level 0 {Arithmetic}

\prerequisite{sec:expr}

\begin{exercise}
  \label{ex:prime:modvar}
  Read two numbers and print out their modulus.
  The modulus operator is \verb+x%y+.
  \begin{itemize}
  \item
    Can you also compute the modulus without the operator?
  \item What do you get for negative inputs, in both cases?
  \item Assign all your results to a variable before outputting them.
  \end{itemize}
\end{exercise}

\Level 0 {Conditionals}

\prerequisite{sec:if}

\begin{exercise}
  \label{ex:prime:divtest}
  Read two numbers and print a message stating
  whether the second is as divisor of the first:
  %
  \snippetwithoutput{divideask}{primes}{division}
\end{exercise}

\Level 0 {Looping}
\label{sec:prime-loop}

\prerequisite{sec:for}

\begin{exercise}
  \label{ex:prime:test}
  Read an integer and set a boolean variable to
  determine whether it is prime by testing for the
  smaller numbers if they divide that number. 

  Print a final message
\begin{verbatim}
Your number is prime
\end{verbatim}
or
\begin{verbatim}
Your number is not prime: it is divisible by ....
\end{verbatim}
where you report just one found factor.
\end{exercise}

Printing a message to the screen is hardly ever the point of a serious program.
In the previous exercise, let's therefore assume that the fact of primeness
(or non-primeness)
of a number
will be used in the rest of the program. So you want to store this conclusion.

\begin{exercise}
  \label{ex:prime:test2}
  Rewrite the previous exercise with a boolean variable to represent
  the primeness of the input number.
\end{exercise}

\begin{exercise}
  \label{ex:squaregrid}
  Read in an integer~$r$. If it is prime, print a message saying so.
  If it is not prime, find integers $p\leq q$ so that $r=p\cdot q$ and so
  that $p$ and $q$ are as close together as possible.
  For instance, for $r=30$ you should print out \lstinline{5,6}, rather than
  \lstinline{3,10}. You are allowed to use the function~\lstinline{sqrt}.
\end{exercise}

\Level 0 {Functions}

\prerequisite[chapter]{ch:function}

Above you wrote several lines of code to test whether a number was prime.
Now we'll turn this code into a function.

\begin{exercise}
  \label{ex:prime:func}
  Write a function \lstinline{is_prime} that has an integer parameter, and returns a boolean
  corresponding to whether the parameter was prime.
\begin{lstlisting}
int main() {
  bool isprime;
  isprime = is_prime(13);
\end{lstlisting}
Write a main program that reads the number in,
and prints the value of the boolean.
(How is the boolean rendered? See section~\ISPref{sec:boolalpha}.)

  Does your function have one or two \lstinline{return} statements?
  Can you imagine what the other possibility looks like?
  Do you have an argument for or against it?
\end{exercise}

\Level 0 {While loops}

\prerequisite{sec:loopuntil}

\begin{exercise}
  \label{ex:prime:while}
  Take your prime number testing function \lstinline{is_prime}, and use it to
  write a program that prints multiple primes:
  \begin{itemize}
  \item Read an integer \lstinline{how_many} from the input, indicating how
    many (successive) prime numbers should be printed.
  \item Print that many successive primes, each on a separate line.
  \item (Hint: keep a variable
    \lstinline{number_of_primes_found} that is increased whenever a new prime is found.)
  \end{itemize}
\end{exercise}

\Level 0 {Classes and objects}
\label{sec:prime-seq-class}

\prerequisite{sec:object}

\begin{comment}
  In exercise~\textbookref{ex:prime:struct} you made a structure that contains
  the data for a prime sequence, and you have separate functions that
  operate on that structure or on its members.
\end{comment}

\begin{exercise}
  \label{ex:prime:sequence}
  Write a class \lstinline{primegenerator} that contains:
  \begin{itemize}
  \item Methods  \lstinline{number_of_primes_found} and \lstinline{nextprime};
  \item Also write a function \lstinline{isprime} that does not need to be
    in the class.
  \end{itemize}

  Your main program should look as follows:
  %
  \verbatimsnippet{primesequencemain}
\end{exercise}

In the previous exercise you defined the \lstinline{primegenerator} class, and
you made one object of that class:
\begin{lstlisting}
primegenerator sequence;
\end{lstlisting}
But you can make multiple generators, that all have their own internal
data and are therefore independent of each other.

%%packtsnippet goldbachpqr
\begin{exercise}
  \label{ex:goldbach:conj}
  The \indexterm{Goldbach conjecture} says that every even number,
  from~4 on, is the sum of two primes $p+q$. Write a program to test this
  for the even numbers up to a bound that you read in.
  Use the \lstinline{primegenerator} class you developed in exercise~\textbookref{ex:prime:sequence}.

  This is a great exercise for a top-down approach!
  \begin{enumerate}
  \item
    Make an outer loop over the even numbers~$e$.
  \item For each~$e$, generate all primes~$p$.
  \item From $p+q=e$, it follows that $q=e-p$ is prime: test if
    that $q$ is prime.
  \end{enumerate}
  For each even number~\lstinline{e} then print \lstinline{e,p,q}, for instance:
  \begin{verbatim}
The number 10 is 3+7
  \end{verbatim}
  If multiple possibilities exist, only print the first one you find.
\end{exercise}

An interesting corollary of the Goldbach conjecture is that
each prime (start at~5) is equidistant between two other primes.

\begin{block}{A Goldbach corollary}
  \label{sl:goldbach-pqr}
  The \indexterm{Goldbach conjecture} says that every even number~$2n$
  (starting at~4), is the sum of two primes $p+q$: \[ 2n=p+q.\]
  Equivalently, every number~$n$ is equidistant from two primes:
  \[ n=\frac{p+q}2\qquad\hbox{or}\qquad q-n = n-p.\]
  In particular this holds for each prime number:
  \[ \forall_{r\,\mathrm{prime}}\exists_{p,q\,\mathrm{prime}}\colon
  \hbox{$r=(p+q)/2$ is prime}. \]
\end{block}

\begin{exercise}
  \label{ex:prime:goldbach-pqr}

  Write a program that tests this. You need at least one loop that
  tests all primes~$r$; for each $r$ you then need to find the primes
  $p,q$ that are equidistant to it. Do you use two generators for
  this, or is one enough? Do you need three, for~$p,q,r$?

  For each $r$ value, 
  when the program finds the $p,q$ values, print the $p,q,r$ triple and
  move on to the next~$r$.
\end{exercise}

%%packtsnippet end

\Level 1 {Exceptions}

\prerequisite{sec:exception}

\begin{exercise}
  \label{ex:primegenbreak}
  Revisit the prime generator class (exercise~\textbookref{ex:prime:sequence})
  and let it throw an exception once the candidate number is too large.
  (You can hardwire this maximum, or use a limit; section~\textbookref{sec:limits}.)

  \snippetwithoutput{primegenbreakuse}{primes}{genx}
\end{exercise}

\Level 1 {Prime number decomposition}
\label{sec:prime-decomp}

\prerequisite{sec:map}

Design a class \lstinline{Integer} which stores its value
as its prime number decomposition. For instance,
\[ 180=2^2\cdot 3^3\cdot 5 \qquad \Rightarrow \qquad \hbox{\lstinline{[ 2:2, 3:2, 5:1 ]}} \]
You can implement this decomposition itself as a \indexterm{vector},
(the $i$-th location stores the exponent of the $i$-th prime)
but let's use a \indexc{map} instead.

\begin{exercise}
  Write a constructor of an \lstinline{Integer} from an \lstinline{int},
  and methods \lstinline{as_int}~/ \lstinline{as_string} that convert the
  decomposition back to something classical. Start by assuming that each prime factor
  appears only once.
  %
  \answerwithoutput{decomp26}{primes}{decomposition26}
\end{exercise}

\begin{exercise}
  Extend the previous exercise to having multiplicity~$>1$ for the prime factors.
  %
  \answerwithoutput{decomp180}{primes}{decomposition180}
\end{exercise}

Implement addition and multiplication for \lstinline{Integer}s.

Implement a class \lstinline{Rational} for rational numbers, which
are implemented as two \lstinline{Integer} objects.
This class should have methods for addition and multiplication.
Write these through operator overloading if you've learned this.

Make sure you always divide out common factors in the numerator and denominator.

\Level 0 {Ranges}

\prerequisite{sec:ranges}

\begin{exercise}
  Write a range-based code that tests
  \[ \forall_{\hbox{\scriptsize prime $p$}}\colon
       \exists_{\hbox{\scriptsize prime $q$}}\colon
       q>p
  \]
\end{exercise}

%%packtsnippet goldbachrange
\begin{exercise}{Goldbach through ranges}
  \label{ex:goldbach-range20}
  Rewrite exercise~\ref{ex:prime:goldbach-pqr},
  using only range expressions, and no loops.
\end{exercise}
%%packtsnippet end

\begin{exercise}
  In the above Goldbach exercises you probably needed two
  prime number sequences, that, however, did not start at the same number.
  Can you make it so that your code reads
  \begin{lstlisting}
    all_of( primes_from(5) /* et cetera */
  \end{lstlisting}
\end{exercise}

\Level 0 {Other}

The following exercise requires \lstinline+std::optional+,
which you can learn about in section~\ISPref{sec:std-optional}.

\begin{exercise}
  \label{ex:primeoptfact}
  Write a function \lstinline+first_factor+ that optionally
  returns the smallest factor of a given input.
  \verbatimsnippet{optfactortest}
\end{exercise}

\Level 0 {Eratosthenes sieve}

The Eratosthenes sieve is an algorithm for prime numbers that
step-by-step filters out the multiples of any prime it finds.
\begin{enumerate}
\item Start with the integers from~2: $2,3,4,5,6,\ldots$
\item The first number, 2, is a prime: record it and remove all
  multiples, giving
  \[ 3,5,7,9.11,13,15,17\dots \]
\item The first remaining number, 3, is a prime: record it and remove
  all multiples, giving
  \[ 5,7,11,13,17,19,23,25,29\ldots \]
\item The first remaining number, 5, is a prime: record it and remove
  all multiples, giving
  \[ 7,11,13,17,19,23,29,\ldots \]
\end{enumerate}

\Level 1 {Arrays implementation}
\label{sec:arraysieve}

The sieve can be implemented with an array that stores all integers.

\begin{exercise}
  \label{ex:arraysieve}
  Read in an integer that denotes the largest number you want to test.
  Make an array of integers that long. Set the elements to the
  successive integers. Apply the sieve algorithm to find the prime numbers.
\end{exercise}

\Level 1 {Streams implementation}
\label{sec:streamsieve}

The disadvantage of using an array is that we need to allocate
an array. What's more, the size is determined by how many integers we
are going to test, not how many prime numbers we want to generate.
We are going to take the idea above of having a generator object, and
apply that to the sieve algorithm: we will now have multiple generator
objects, each taking the previous as input and erasing certain
multiples from it.

\begin{exercise}
  Write a \lstinline{stream} class that generates integers and use it through
  a pointer.
  %
  \answerwithoutput{streamints}{sieve}{ints}
\end{exercise}

Next, we need a stream that takes another stream as input, and filters
out values from it.

\begin{exercise}
  Write a class \lstinline{filtered_stream} with a constructor
\begin{lstlisting}
filtered_stream(int filter,shared_ptr<stream> input);
\end{lstlisting}
  that
  \begin{enumerate}
  \item Implements \lstinline{next}, giving filtered values,
  \item by calling the \lstinline{next} method of the input stream and
    filtering out values.
  \end{enumerate}
  %
  \answerwithoutput{streamodds}{sieve}{odds}
\end{exercise}

Now you can implement the Eratosthenes sieve by making a
\lstinline{filtered_stream} for each prime number.

\begin{exercise}
  Write a program that generates prime numbers as follows.
  \begin{itemize}
  \item Maintain a \lstinline{current} stream, that is initially the stream of
    prime numbers.
  \item Repeatedly:
    \begin{itemize}
    \item Record the first item from the current stream, which is a
      new prime number;
    \item and set \lstinline{current} to a new stream that takes \lstinline{current}
      as input, filtering out multiples of the prime number just found.
    \end{itemize}
  \end{itemize}
\end{exercise}

\Level 0 {Range implementation}

\prerequisite{sec:range-iter}

If we write the prime number definition
\[ 
\begin{array}{ll}
  D(n,d)&\equiv n|d=0\\
  P(n) &\equiv \forall_{d\leq\sqrt N}\colon \neg D(n,d)
\end{array}
\]
we see that this involves two streams
that we iterate over:
\begin{enumerate}
\item First there is the set of all $d$
  such that $d^2\leq n$; then
\item We have the set of booleans testing whether
  these $d$ values are divisors.
\end{enumerate}

\begin{exercise}
  Use the \indexc{iota} range view to generate all integers from 2 to infinity,
  and find a range view that cuts off the seqwuence at the last possible divisor.

  Then use the \indexc{all_of} or \indexc{any_of} rangified algorithms to test
  whether any of these potential divisors are actually a divisor, and therefore
  whether or not your number is prime.
\end{exercise}

\begin{exercise}
  Use the \indexc{filter} view to filter from
  an \indexc{iota} view those elements that are prime.
\end{exercise}

\begin{exercise}
  \label{ex:primerange}
  Make a \lstinline{primes} class that can be ranged:
  %
  \answerwithoutput{primerangecall}{primes}{range}
\end{exercise}

\Level 0 {User-friendliness}

Use the \indexc{cxxopts} package (section~\textbookref{sec:cxxoptlib})
to add commandline options to some primality programs.

\begin{exercise}
  \label{ex:prime-opts}
  Take your old prime number testing program, and add commandline options:
  \begin{itemize}
  \item the \n{-h} option should print out usage information;
  \item specifying a single int \n{--test 1001} should print out all
    primes under that number;
  \item specifying a set of ints \n{--tests 57,125,1001} should test
    primeness for those.
  \end{itemize}
\end{exercise}

\endinput

\begin{comment}
  \Level 0 {Structures}

  \prerequisite{textbookref{sec:struct}, \textbookref{sec:reference}}

  In the C programming language,
  a~\lstinline{struct} functions to bundle together a number of data item.
  The C++ \indexc{struct} does this, but, like a \lstinline{class},
  it can also contain methods.

  The exercises in this chapter
  only use data encapsulation as a preliminary to full-fledged
  classes with method encapsulation.
  You can do these exercises, or skip straight to
  section~\ref{sec:prime-seq-class}.

  \begin{exercise}
    \label{ex:prime:struct}
    Rewrite the exercise that found a predetermined number of primes,
    putting the \lstinline{number_of_primes_found} and
    \lstinline{last_number_tested} variables in a structure. Your main program should
    now look like:
    %
    \verbatimsnippet{primestructmain}
    %
    Hint: the variable \lstinline{last_number_tested} does not appear in the
    main program. Where does it get updated? Also, there is no update of
    \lstinline{number_of_primes_found} in the main program. Where do you think
    it would happen?
  \end{exercise}
\end{comment}

