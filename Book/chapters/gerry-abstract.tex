% -*- latex -*-
%%%%%%%%%%%%%%%%%%%%%%%%%%%%%%%%%%%%%%%%%%%%%%%%%%%%%%%%%%%%%%%%
%%%%
%%%% This TeX file is part of the course
%%%% Introduction to Scientific Programming in C++11/Fortran2003
%%%% copyright 2017-2020 Victor Eijkhout eijkhout@tacc.utexas.edu
%%%%
%%%% gerry-abstract.tex : abstract for the redistricting project
%%%%
%%%%%%%%%%%%%%%%%%%%%%%%%%%%%%%%%%%%%%%%%%%%%%%%%%%%%%%%%%%%%%%%

The US electoral system for the house of representatives
groups `census districts' into `electoral districts' that
supply a representative. Grouping the census districts
differently into electoral districts can greatly affect
the number of representatives the parties send to the house.

This project lets a student explore the phenomenon of redistricting,
and especially `gerrymandering': applying the redistricting in a way
that maximally benefits one party.
In particular, students will at first aim to maximize
the number of representatives in a state, from a party
that numerically is in the minority.
Students can then explore mitigating this phenomenon.

The project description explicitly targets an object-oriented
object, where creating and copying objects is simple.

This project can be done by one or two undergraduate students,
or AP high schoolers
at the end of a first or semester programming course.

\newpage

\begin{tabular}{|l|p{5in}|}
  \hline
  Summary&Develop a simple model of redistricting, and the potential unfairness
  of the process.
  \\
  Topics&Recursion, arrays, classes\\
  &There is a possibility of exploring memoization and dynamic programming
  \\
  Audience&Undergraduate or AP high school
  \\
  Difficulty&High
  \\
  Strengths&Explores a question of great societal impact.
  \\
  Weaknesses&Requires careful analysis of the problem.\\
  &No graphics output, so the interpretation of output requires some imagination
  \\
  Dependencies&Strictly speaking no dependencies.\\
  &Familiarity with the concepts of search and optimization may be helpful.
  \\
  Variants&Expansion beyond the simple model is possible but requires
  considerable programming.\\
  &Use of actual census data is very hard due to data formats used.
  \\
  \hline
\end{tabular}
