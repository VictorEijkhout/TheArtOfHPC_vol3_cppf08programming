% -*- latex -*-
%%%%%%%%%%%%%%%%%%%%%%%%%%%%%%%%%%%%%%%%%%%%%%%%%%%%%%%%%%%%%%%%
%%%%
%%%% This TeX file is part of the course
%%%% Introduction to Scientific Programming in C++/Fortran2003
%%%% copyright 2017-2020 Victor Eijkhout eijkhout@tacc.utexas.edu
%%%%
%%%% elementsj.tex : basic language elements in Julia
%%%%
%%%%%%%%%%%%%%%%%%%%%%%%%%%%%%%%%%%%%%%%%%%%%%%%%%%%%%%%%%%%%%%%

\Level 0 {Statements}
\label{sec:statementsj}

Each programming language has its own (very precise!) rules for what
can go in a source file. Globally we can say that a program contains
instructions for the computer to execute, and these instructions take
the form of a bunch of `statements'. Here are some of the rules on
statements; you will learn them in more detail as you go through this
book.

\begin{block}{Program statements}
  \label{sl:jstatement}
  \begin{itemize}
  \item
    A program contains statements. A~simple statement
    is one that fits on one line. No need for a semicolon:
    the line end terminates it.
\begin{lstlisting}
a = 1
\end{lstlisting}
  \item You can multiple statements on one line by using semicolons:
\begin{lstlisting}
a = 1; b = 2
\end{lstlisting}
  \item Compound statements, such as conditionals, take up mutiple lines;
\begin{lstlisting}
if a == 1
  println("It was one")
end
\end{lstlisting}
\item Comments are `Note to self', short:
\begin{lstlisting}
c = 3 # set c to three
\end{lstlisting}
  \end{itemize}
\end{block}

\begin{block}{Printing stuff}
  The basics of interacting with a program are through the output
  statements \indextermtt{print} and \indextermtt{println},
  and input \indextermtt{readline}.
  \begin{lstlisting}
a = 1; b = 2
print("a and b: ")
print(a)
print(" ")
println(b)
  \end{lstlisting}
or simpler:
  \begin{lstlisting}
println("a and b: $a $b")
  \end{lstlisting}
\end{block}


\Level 0 {Variables}
\label{sec:variablesj}

A program could not do much without storing data: input data,
temporary data for intermediate results, and final results.
Data is stored in \emph{variables}\index{variable},  which have
\begin{itemize}
\item a name, so that you can refer to them,
\item a \indexterm{datatype}, and
\item a value.
\end{itemize}
Think of a variable as a labeled placed in memory.

Variable names are more liberal than in other programming languages:
many \indexterm{Unicode} characters are allowed.

\Level 1 {Datatypes}
\label{sec:jtypes}

In Julia, like in Python, but unlike in~C++,
you do not need a variable declaration:
a variable is created the first time you assign something to it,
and its type is the type of the value you assign.

Variables have
\begin{itemize}
\item a name
\item a type
\item a value.
\end{itemize}
The type of the variable is determined by when a value is assigned to it.
This is like \jvp, and unlike~C.

\begin{block}{Numerical types}
  The main numerical types are \lstinline{Int64} and \lstinline{Float64},
  where the `64' indicates that in both cases these are 64-bit types.

  These are `concrete types', but they are both based on the `abstract
  type' \lstinline{Real}, which is a supertype of both; see later.
\end{block}

\begin{block}{Type of expression}
  Types can be determined by looking at the rhs expression,
  or explicitly forced:
  %
  \snippetwithoutput{scalarj}{basicj}{scalar}
\end{block}

\begin{block}{Subtypes and supertypes}
  Types have a hierarchy.
  
  The subtype relation is denoted with the \verb+<:+ relation:
\begin{lstlisting}
 Integer <: Number  
\end{lstlisting}
This can also be used in functions:
\begin{lstlisting}
function somecompute(x::T) where T <: Real
\end{lstlisting}
\end{block}

\Level 2 {Boolean values}

\begin{block}{Truth values}
  \label{sl:bool-varj}
  So far you have seen integer and real variables. There are also
  boolean values, type \lstinline{Bool}, which represent truth values. 
\begin{lstlisting}
primitive type Bool <: Integer 8 end    
\end{lstlisting}
  There are
  only two values: \indextermtt{true} and \indextermtt{false}.
\begin{lstlisting}
bool found{false};
found = true;
\end{lstlisting}
\end{block}

To create an array of booleans you can 
\begin{lstlisting}
alltrue = trues(n)
allfalse = falses(n)
\end{lstlisting}

\Level 2 {Characters and strings}

\begin{block}{Characters and strings}
  \begin{itemize}
  \item Characters, of type \lstinline{Char}, in single quotes
  \item Strings, of type \lstinline{String}, in double quotes.
  \end{itemize}
  In addition to the regular 128 \indexterm{ASCII} characters,
  \indexterm{Unicode} characters can be entered through a backslash notation:
  \verb+\gamma+. This follows \LaTeX{} naming conventions.
\end{block}


\Level 1 {Truth values}

In addition to numerical types, there are truth values,
\indextermtt{true} and \indextermtt{false}, with all the usual logical
operators defined on them.

\begin{block}{Boolean expressions}
  \label{sl:bool-exprj}
  \begin{itemize}
  \item Testing: \n{== != < > <= >=}
  \item Not, and, or:   \n{! && ||}
%%   \item Shortcut operators:
%% \begin{lstlisting}
%%   if ( x>=0 && sqrt(x)<5 ) {}
%% \end{lstlisting}
  \end{itemize}
\end{block}

\begin{block}{Bool-int conversion}
  As in other languages, booleans can be implicitly interpreted as
  integer, with \lstinline{true} corresponding to~1 and
  \lstinline{false} to~0. However, conversion the other way is not as
  generous.
\begin{lstlisting}
1+true
# evaluates to 2
2 || 3
# is a TypeError
\end{lstlisting}
\end{block}

Logical expressions in C++ are evaluated using
\emph{shortcut operators}\index{operator!shortcut}: you can write
\begin{lstlisting}
x>=0 && sqrt(x)<2
\end{lstlisting}
If \lstinline{x}~is negative, the second part will never be evaluated because
the `and' conjunction can already be concluded to be false.
Similarly, `or' conjunctions will only be evaluated until the first
true clause.

The `true' and `false' constants could strictly speaking be stored in
a single bit. C++~does not do that, but there are bit
operators that you can apply to, for instance, all the bits in an integer.

\begin{block}{Bit operators}
  \label{sl:bit-operj}
 Bitwise: \n{& | ^}
\end{block}

\Level 0 {Expressions}

Division \lstinline{2/3} gives a \lstinline{Float64} number.
You can cast to integer:
\begin{lstlisting}
half = Int64( (27+2)/2 )
\end{lstlisting}

\Level 1 {Terminal input}

\begin{block}{Terminal input}
  \label{sl:cinj}
  To make a program run dynamic, you can set starting values from
  keyboard input. For this, use \indextermtt{readline}, which takes
  keyboard input. To assign this input to a variable you need to parse
  that as the desired type, using the function \indextermtt{parse}.

  \verbatimsnippet{readi64j}
\end{block}

\Level 0 {Tuples}

\begin{lstlisting}
(a,b) = (1,2)
\end{lstlisting}
even
\begin{lstlisting}
(a,b) = (b,a)
\end{lstlisting}
