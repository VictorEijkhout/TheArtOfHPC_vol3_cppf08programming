% -*- latex -*-
%%%%%%%%%%%%%%%%%%%%%%%%%%%%%%%%%%%%%%%%%%%%%%%%%%%%%%%%%%%%%%%%
%%%%
%%%% This TeX file is part of the course
%%%% Introduction to Scientific Programming in C++/Fortran2003
%%%% copyright 2017-2023 Victor Eijkhout eijkhout@tacc.utexas.edu
%%%%
%%%% gdb.tex : about the GNU debugger
%%%%
%%%%%%%%%%%%%%%%%%%%%%%%%%%%%%%%%%%%%%%%%%%%%%%%%%%%%%%%%%%%%%%%

\Level 0 {A simple example}

The following program does not have any bugs;
we use it to show some of the basics of gdb.

\verbatimsnippet{gdbhello}

\Level 1 {Invoking the debugger}

After you compile your program, instead of running it the normal way,
you invoke \indextermtt{gdb}:
\begin{verbatim}
gdb myprogram
\end{verbatim}
That puts you in an environment,
recognizable by the \n{(gdb)} prompt:
\begin{verbatim}
GNU gdb (GDB) Red Hat Enterprise Linux 7.6.1-115.el7
[stuff]
(gdb)
\end{verbatim}
where you can do a controlled run of your program with the \indexgdb{run} command:
\begin{verbatim}
(gdb) run
Starting program: /home/eijkhout/gdb/hello
hello world 1
hello world 2
hello world 5
hello world 10
hello world 17
hello world 26
hello world 37
hello world 50
hello world 65
hello world 82
[Inferior 1 (process 30981) exited normally]
\end{verbatim}

\Level 0 {Example: integer overflow}

The following program shows \indextermbus{integer}{overflow}.
(We are using \indexc{short} to force this to happen soon.)

\snippetwithoutput{gdbshort}{gdb}{hello}

\Level 0 {More gdb}

\Level 1 {Run with commandline arguments}

This program is self-contained, but if you had a program that takes
\emph{commandline arguments}\index{gdb!run with commandline arguments}:
\begin{verbatim}
./myprogram 25
\end{verbatim}
you can supply those in gdb:
\begin{verbatim}
(gdb) run 25
\end{verbatim}

\Level 1 {Source listing and proper compilation}

Inside gdb, you can get a source listing with the \indexgdb{list} command.

Let's try our program again:
\begin{verbatim}
[] icpc -o hello hello.cpp
[] gdb hello
GNU gdb (GDB) Red Hat Enterprise Linux 7.6.1-115.el7
Copyright (C) 2013 Free Software Foundation, Inc.
Reading symbols from /home/eijkhout/gdb/hello...
    (no debugging symbols found)...
done.
(gdb) list
No symbol table is loaded.  Use the "file" command.
\end{verbatim}

See the repeated reference to `symbols'\index{symbol!debugging}?
You need to supply the \n{-g} compiler option for the \indextermbus{symbol}{table}
to be included in the binary:
\begin{verbatim}
[] icpc -g -o hello hello.cpp
[] gdb hello
GNU gdb (GDB) Red Hat Enterprise Linux 7.6.1-115.el7
[stuff]
Reading symbols from /home/eijkhout/gdb/hello...done.
(gdb) list
13      using std::cout;
14      using std::endl;
[et cetera]
\end{verbatim}
(If you hit return now, the list command is repeated and you get the next block of lines.
Doing \n{list -} gives you the block above where you currently are.)

\Level 1 {Stepping through the source}

Let's now make a more controlled run of the program.
In the source, we see that line~22 is the first executable one:
\begin{lstlisting}
20      int main() {
21
22        for (int i=0; i<10; i++) {
23          int ii;
24          ii = i*i;
...
\end{lstlisting}

We introduce a \indextermdef{breakpoint} with the \indexgdb{break} command:
\begin{verbatim}
(gdb) break 22
Breakpoint 1 at 0x400a03: file hello.cpp, line 22.
\end{verbatim}
(If your program is spread over multiple files,
you can specify the file name: \n{break otherfile.cpp:34}.)

Now if we run the program, it will stop at that line:
\begin{verbatim}
(gdb) run
Starting program: /home/eijkhout/gdb/hello

Breakpoint 1, main () at hello.cpp:22
22        for (int i=0; i<10; i++) {
\end{verbatim}
To be precise: the program is stopped in the state before it executes this line.

We can now use \indexgdb{cont} (for `continue') to let the program run on.
Since there are no further breakpoints, the program will run to completion.
This is not terribly useful, so let us change our minds about the location of the breakpoint:
it would be more useful if the execution stopped at the start of every iteration.

Recall that the breakpoint had a number of~1, so we use \indexgdb{delete} to remove it,
and we set a breakpoint inside the loop body instead,
and continue until we hit it.
\begin{lstlisting}
(gdb) delete 1
(gdb) break 23
Breakpoint 2 at 0x400a29: file hello.cpp, line 23.
(gdb) cont
Continuing.
Breakpoint 2, main () at hello.cpp:24
24          ii = i*i;
\end{lstlisting}
(Note that line 23 is not executable, so execution stops on the first line after that.)

Now if we continue, the program runs until the next break point:
\begin{lstlisting}
(gdb) cont
Continuing.
hello world 5

Breakpoint 1, main () at hello.cpp:24
24          ii = i*i;
\end{lstlisting}

To get to the next statement, we use \indexgdb{next}:
\begin{verbatim}
(gdb) next
25          ii++;
(gdb)
\end{verbatim}

Hitting return re-executes the previous command,
so we go to the next line:
\begin{verbatim}
(gdb)
26          say(ii);
(gdb)
hello world 2

Breakpoint 1, main () at hello.cpp:24
24          ii = i*i;
\end{verbatim}

You observe that the function call
\begin{enumerate}
\item is executed, as is clear from the \n{hello world 1} output, but
\item is not displayed in detail in the debugger.
\end{enumerate}
The conclusion is that \n{next} goes to the next executable statement
in the current subprogram, not into functions and such that get called from it.

If you want to go into the function~\n{say}, you need to use \indexgdb{step}:
\begin{verbatim}
(gdb) next
25          ii++;
(gdb) next
26          say(ii);
(gdb) step
say (n=10) at hello.cpp:17
17        cout << "hello world " << n << endl;
\end{verbatim}
The debugger reports the function name, and the names and values of the arguments.
Another `step' executes the current line and brings us to the end of the function,
and the next `step' puts us back in the main program:
\begin{verbatim}
(gdb)
hello world 10
18      }
(gdb)
main () at hello.cpp:24
24          ii = i*i;
\end{verbatim}

\Level 1 {Inspecting values}

When execution is stopped at a line
(remember, that means right before it is executed!)
you can inspect any values in that subprogram:
\begin{verbatim}
24          ii = i*i;
(gdb) print i
$1 = 4
\end{verbatim}
You can even let expressions be evaluated with local variables:
\begin{verbatim}
(gdb) print 2*i
$2 = 8
\end{verbatim}

You can combine this looking at values with breakpoints.
Say you want to know when the variable \n{ii} gets more than~40:
\begin{verbatim}
(gdb) break 26 if ii>40
Breakpoint 1 at 0x4009cd: file hello.cpp, line 26.
(gdb) run
Starting program: /home/eijkhout/intro-programming-private/code/gdb/hello
hello world 1
hello world 2
hello world 5
hello world 10
hello world 17
hello world 26
hello world 37

Breakpoint 1, main () at hello.cpp:26
26          say(ii);
Missing separate debuginfos, use: debuginfo-install glibc-2.17-292.el7.x86_64 libgcc-4.8.5-39.el7.x86_64 libstdc++-4.8.5-39.el7.x86_64
(gdb) print i
$1 = 7
\end{verbatim}

\Level 1 {A NaN example}

The following program:
\begin{lstlisting}
17      float root(float n)
18      {
19        float r;
20        float n1 = n-1.1;
21        r = sqrt(n1);
22        return r;
23      }
24
25      int main() {
26        float x=9,y;
27        for (int i=0; i<20; i++) {
28          y = root(x);
29          cout << "root: " << y << endl;
30          x -= 1.1;
31        }
32
33        return 0;
34      }
\end{lstlisting}
prints some numbers that are `not-a-number':
\begin{verbatim}
[] ./root
root: 2.81069
root: 2.60768
root: 2.38747
root: 2.14476
root: 1.87083
root: 1.54919
root: 1.14018
root: 0.447214
root: -nan
root: -nan
root: -nan
\end{verbatim}
Suppose you want to figure out why this happens.

The line that prints the `nan' is~29, so we want to set a breakpoint there,
and preferably a conditional breakpoint.
But how do you test on `nan'? This takes a little trick.
\begin{verbatim}
(gdb) break 29 if y!=y
Breakpoint 1 at 0x400ea6: file root.cpp, line 28.
(gdb) run
Starting program: /home/eijkhout/intro-programming-private/code/gdb/root
root: 2.81069
root: 2.60768
root: 2.38747
root: 2.14476
root: 1.87083
root: 1.54919
root: 1.14018
root: 0.447214

Breakpoint 1, main () at root.cpp:29
29          cout << "root: " << y << endl;
\end{verbatim}
We discover what iteration this happens:
\begin{verbatim}
(gdb) print i
$1 = 8
\end{verbatim}
so now we can rerun the program, and investigate that particular iteration:
\begin{verbatim}
(gdb) break 28 if i==8
Breakpoint 2 at 0x400eaf: file root.cpp, line 28.
(gdb) run
The program being debugged has been started already.
Start it from the beginning? (y or n) y

Starting program: /home/eijkhout/intro-programming-private/code/gdb/root
root: 2.81069
root: 2.60768
root: 2.38747
root: 2.14476
root: 1.87083
root: 1.54919
root: 1.14018
root: 0.447214

Breakpoint 2, main () at root.cpp:28
28          y = root(x);
\end{verbatim}
We now go into the \n{root} routine to see what is going wrong there:
\begin{verbatim}
(gdb) step
root (n=0.200000554) at root.cpp:20
20        float n1 = n-1.1;
(gdb)
21        r = sqrt(n1);
(gdb) print n
$2 = 0.200000554
(gdb) print n1
$3 = -0.89999944
(gdb) next
22        return r;
(gdb) print r
$4 = -nan(0x400000)
\end{verbatim}
And there we have the problem: our input \n{n} is used to compute
another number \n{n1} of which we compute the square root,
and sometimes this number gets negative.

\Level 1 {Assertions}

Instead of running a program and debugging it if you happen to spot a problem
(and note that this may not always be the case!)
you can also make your program more robust by including \indextermp{assertion}.
These are things that you know should be true,
from your knowledge of the problem you are solving.

For instance, in the previous example there was a square root function,
and you just `knew' that the input was always going to be positive.
So you edit your program as follows:
\begin{lstlisting}
// header to allow assertions:
#include <cassert>

float root(float n)
{
  float r;
  float n1 = n-1.1;
  assert(n1>=0); // NOTE!
  r = sqrt(n1);
  return r;
}
\end{lstlisting}
Now if you run your program, you get:
\begin{verbatim}
[] ./assert
root: 2.81069
root: 2.60768
root: 2.38747
root: 2.14476
root: 1.87083
root: 1.54919
root: 1.14018
root: 0.447214
assert: assert.cpp:22: float root(float): Assertion `n1>=0' failed.
Aborted (core dumped)
\end{verbatim}
What does this give you?
\begin{itemize}
\item It only tells you that an assertion failed, not with what values;
\item it does not give you a traceback or so; on the other hand
\item assertions can help you detect error conditions that you might
  otherwise have overlooked!
\end{itemize}
