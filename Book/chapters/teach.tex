% -*- latex -*-
%%%%%%%%%%%%%%%%%%%%%%%%%%%%%%%%%%%%%%%%%%%%%%%%%%%%%%%%%%%%%%%%
%%%%
%%%% This TeX file is part of the course
%%%% Introduction to Scientific Programming in C++/Fortran2003
%%%% copyright 2017-2023 Victor Eijkhout eijkhout@tacc.utexas.edu
%%%%
%%%% teach.tex : a teachers guide
%%%%
%%%%%%%%%%%%%%%%%%%%%%%%%%%%%%%%%%%%%%%%%%%%%%%%%%%%%%%%%%%%%%%%

This book was written for a one-semester introductory programming course at The
University of Texas at Austin, aimed primarily at students in the
physical and engineering sciences. Thus, examples and exercises are as
much as possible scientifically motivated. This target audience also
explains the inclusion of Fortran.

This book is not encyclopedic. Rather than teaching each topic in its
full glory,
the author has taken a `recommended practices' approach, where students learn
enough of each topic to become a competent programmer.
(Recommended by who you might ask?
The author freely admits being guided by his own taste.
However, he lets himself be informed by plenty of other current literature.)
This serves to
keep this book at a manageable length, and to minimize class lecture
time, emphasizing lab exercises instead.

Even then, there is more material here than can be covered and
practiced in one semester. If only C++ is taught, it is probably
possible to cover the whole of Part~\ref{part:cpp}; for the case where
both C++ and Fortran are taught, we have a suggested time line below.

\Level 0 {Justification}

The chapters of Part~\ref{part:cpp} and Part~\ref{part:f} are
presented in suggested teaching order. 
Here we briefly justify our (non-standard) sequencing of topics and
outline a timetable for material to be covered in one semester.
Most notably, Object-Oriented programming is covered
before arrays,
and pointers come very late, if at all.

There are several thoughts behind this. For one, dynamic arrays in~C++
are most easily realized through the \lstcstd{vector} mechanism, which
requires an understanding of classes. The same goes for
\lstcstd{string}.

Secondly, in the traditional approach, object-oriented techniques are
taught late in the course, after all the basic mechanisms, including
arrays. We consider OOP to be an important notion in program design,
and central to~C++, rather than an embellishment on the traditional
C~mechanisms, so we introduce it as early as possible.

Even more elementary, we emphasize range-based loops as much as
possible over indexed loops, since ranges are increasing in importance
in recent language versions.

\Level 1 {Algorithms}

Some of the programming exercises in this course
ask the student to reproduce algorithms
that exist in the \indexcstd{algorithm} header.
Thus this course could be open to the criticism that
students should learn their \ac{STL} algorithms,
rather than recreating them themselves.
(See section~\ref{sec:algorithm}.)

My defense would be that a programmer should know
more than what algorithms to pick from the standard library.
Students should understand the mechanisms behind these algorithms,
and be able to reproduce them, so that, when this is needed,
they can code variations of these algorithms.

\Level 0 {Time line for a C++/F03 course}

As remarked above, this book is based on a course that teaches both
C++ and Fortran2003. Here we give the time line used, including some of
the assigned exercises.

For a one semester course of slightly over three months, two months
would be spent on~C++ (see table~\ref{tab:c++plan}), after which a
month is enough to explain Fortran; see table~\ref{tab:f-plan}.
Remaining time will go to exams
and elective topics.

\newcounter{lesson}

\begin{table}[p]
  \setcounter{lesson}{0}
  \rightskip=0pt plus 1fil\relax
  \setbox0=\hbox{homework }\edef\colwidth{.7in}%{\the\wd0}
  % using array, which is incompatible somehow: >{\raggedright\arraybackslash}
  \begin{tabular}{lp{\colwidth}p{\colwidth}p{\colwidth}p{\colwidth}p{\colwidth}p{\colwidth}}
    \toprule % \hline
    &&\multicolumn{5}{c}{Exercises}\\
    \cmidrule{3-7}
    lesson&Topic&in-class&homework&prime&geom&infect\\
    \midrule % \hline
    %%
    \stepcounter{lesson}\arabic{lesson}
    & Statements and expressions
      &\ref{ex:cout-what}&\ref{ex:cin-cout3np1}~(T)
      &\ref{ex:prime:modvar}&&\\
    %%
    \stepcounter{lesson}\arabic{lesson}
    & Conditionals
      &\ref{ex:if2digits}~(S), \ref{ex:oddeven}&\ref{ex:fizzbuzz}~(T)&\ref{ex:prime:divtest}&&\\
    %%
    \stepcounter{lesson}\arabic{lesson}
    & Looping
      &\ref{ex:ij-print}~(S), \ref{ex:ij-triangle}&\ref{ex:ij-product}~(T)
      &\ref{ex:prime:test}, \ref{ex:prime:test2}&&\\
    %%
    \stepcounter{lesson}\arabic{lesson}
    & continue
        &&&&&\\
    %%
    \stepcounter{lesson}\arabic{lesson}
    & Functions
      &\ref{ex:bisect-extend}~(S), \ref{ex:swapbyvalue}&\ref{ex:div-remain}
      &\ref{ex:prime:func}, \ref{ex:prime:while}~(T)&&\\
    %%
    \stepcounter{lesson}\arabic{lesson}
    & continue
      &\ref{ex:recur-fib}&&&\ref{sec:geom-basic}&\\
    %%
    \stepcounter{lesson}\arabic{lesson}
    & I/O 
      &&\ref{ex:leadzero}&&&\\
    \stepcounter{lesson}\arabic{lesson}
    & Objects
      &&&\ref{ex:mult-two}~(S), \ref{ex:prime:sequence}~(T), \ref{ex:prime:goldbach-pqr}&
      \ref{ex:geom:point}&\ref{ex:infect:person}\\
    %%
    \stepcounter{lesson}\arabic{lesson}
    & continue
        &&&&&\\
    %%
    \stepcounter{lesson}\arabic{lesson}
    & has-a relation
        &&&&\ref{ex:geom:line}~(T), \ref{ex:geom:line2},
        \ref{ex:geom:rect}, \ref{ex:geom:rect2}&\ref{ex:infect:notransfer}\\
    %%
    \stepcounter{lesson}\arabic{lesson}
    & inheritance
        &&&&\ref{ex:geom:square}, \ref{ex:geom:line3}&\\
    %%
    \stepcounter{lesson}\arabic{lesson}
    & Vectors
       &\ref{ex:shortvectoralter}~(S), \ref{ex:range-for}
       &\ref{ex:pascal-ex}&\ref{ex:arraysieve}&&\ref{ex:infect:notransfer} and further\\
    %%
    \stepcounter{lesson}\arabic{lesson}
    & continue
        &&&&&\\
    %%
    \stepcounter{lesson}\arabic{lesson}
    & Strings
        &&&&&\\
    \bottomrule % \hline
  \end{tabular}
  \caption{Two-month lesson plan for C++; the annotation `(S)' indicates that a skeleton code is available; `(T)' indicates that
  a tester script is available.}
  \label{tab:c++plan}
\end{table}

\Level 1 {Advanced topics}

We also outline a `C++~101.5' course:
somewhere between beginning and truly advanced.
Here we assume that the student has had
about 8~lectures worth of C++,
covering
\begin{enumerate}
\item basic control structures,
\item simple functions including parameter passing by reference,
\item arrays through \lstinline+std::vector+.
\end{enumerate}
Based on this, the topics in table~\ref{tab:c++102} can be taught
in that order.

\begin{table}[p]
  \setcounter{lesson}{0}
  \setbox0=\hbox{homework }\edef\colwidth{1.1in}
  \begin{tabular}{l % lesson #
      >{\raggedright\arraybackslash}p{\colwidth} % topic
      >{\raggedright\arraybackslash}p{.3in} % chapter
      >{\raggedright\arraybackslash}p{\colwidth} % prerequisite
      >{\raggedright\arraybackslash}p{\colwidth} % in-class
      >{\raggedright\arraybackslash}p{\colwidth} % homework
    }
    \toprule % \hline
    &&&&\multicolumn{2}{c}{Exercises}\\
    \cmidrule{5-6}
    lesson&Topic&Book&Prerequisite&In-class&Homework\\
    \midrule % \hline
    %%
    \stepcounter{lesson}\arabic{lesson}
    & Welcome, accounts&&&&Essay, \n{coe_history}\\
    %%%%%%%%%%%%%%%%
    \cmidrule{1-4}
    \stepcounter{lesson}\arabic{lesson},\stepcounter{lesson}\arabic{lesson}
    & Unix, compilation, check prior knowledge
    &&
    &&     Collatz:~\ref{ex:collatz}; swap:~\ref{ex:swap}; vectors:~\ref{ex:even-index},
        \n{coe_catchup}\\
    %%
    \cmidrule{1-4}
    \stepcounter{lesson}\arabic{lesson}, \stepcounter{lesson}\arabic{lesson}
    & Test-driven development &\ref{ch:tdd}
    &     Separate compilation \ref{ch:proto}
    &     \ref{ex:catch-example}~(S), \ref{ex:bisect-coeff}--\ref{ex:bisect-outer}
    &     \ref{ex:bisect-find} \n{coe_bisection}\\
    %%
    \cmidrule{1-4}
    \stepcounter{lesson}\arabic{lesson}, \stepcounter{lesson}\arabic{lesson}
    & Objects
    &    \ref{ch:object}&
    &    \ref{ex:vectorclass-angle}~(S), \ref{ex:mult-two}~(S)
    &    \ref{ex:prime:sequence} \n{coe_primes}\\
    %%
    \cmidrule{1-4}
    \stepcounter{lesson}\arabic{lesson}
    & Class inclusion & \ref{sec:hasa}&
    &    \ref{ex:geom:line}, \ref{ex:geom:rect2}
    &    \\
    \stepcounter{lesson}\arabic{lesson}
    & Inheritance & \ref{sec:inheritance}&
    &    \ref{ex:geom:square}
    &    \ref{ex:goldbach:conj}, \ref{ex:prime:goldbach-pqr} \n{coe_goldbach}\\
    %%
    \cmidrule{1-4}
    \stepcounter{lesson}\arabic{lesson}
    & Vectors & \ref{ch:array} &
    &    \ref{ex:shortvectoralter}~(S),
         \ref{ex:vectornormalize}, \ref{ex:vectornormalize-function}\\
    \stepcounter{lesson}\arabic{lesson}
    & Vectors in classes &\ref{sec:class-w-vector}&
    &    \ref{ex:matrix-set-sum}~(S)
    &    \ref{ex:pascal-ex} \n{coe_pascal}\\
    %%
    \cmidrule{1-4}
    \stepcounter{lesson}\arabic{lesson}, \stepcounter{lesson}\arabic{lesson}
    & Lambda functions&\ref{ch:lambda}&
    &     \ref{ex:newton-functions}~(S), \ref{ex:newton-function-ptr}
    &     \ref{ex:newton-capture-root}, \ref{ex:newton-capture-diff} \n{coe_newton}\\
    %%
    \cmidrule{1-4}
    \stepcounter{lesson}\arabic{lesson},\stepcounter{lesson}\arabic{lesson}
    &    \acs{STL}, variant, optional
    &    \ref{sec:std-optional}& % no prerequisite
    &    \ref{ex:primeoptfact}
    &    Eight queens: \ref{ch:queens}\\
    %% & \n{chrono}&\ref{sec:chrono}&&&\\
    %%
    \cmidrule{1-4}
    \stepcounter{lesson}\arabic{lesson}
    & Pointers&\ref{ch:pointer}, \ref{sec:linklist}&
    &
    &     \ref{ex:listlist-insert-empty}--\ref{ex:linklist-insert-tail}\\
    \stepcounter{lesson}\arabic{lesson}
    & C pointers&\ref{ch:cpointer}
    &     &\\
    %%
    \cmidrule{1-4}
    \stepcounter{lesson}\arabic{lesson}
    & libraries and cmake\\
    & \n{cxxopts}&\ref{sec:makelib}&&\ref{sec:cxxoptlib}\\
    & \n{fmt}&&&\\
    & \n{random}&\ref{sec:stl:random}&&&\\
    & Exceptions&\ref{ch:error}&&\\
    & formal approaches\\
    \bottomrule % \hline
  \end{tabular}
  \caption{Advanced lessons for C++;
    the annotation `(S)' indicates that a skeleton code is available;
    `(T)' indicates that a tester script is available.}
  \label{tab:c++102}
\end{table}

\begin{table}[h]
  \setcounter{lesson}{0}
  \rightskip=0pt plus 1fil\relax
  \setbox0=\hbox{homework }\edef\colwidth{.7in}%{\the\wd0}
  % using array, which is incompatible somehow: >{\raggedright\arraybackslash}
  \begin{tabular}{lp{\colwidth}p{\colwidth}p{\colwidth}p{\colwidth}p{\colwidth}p{\colwidth}}
    \toprule % \hline
    lesson&Topic&Book&Slides&in-class&homework\\
    % &&\multicolumn{5}{c}{Exercises}\\
    % \cmidrule{3-7}
    \midrule % \hline
    %%
    \stepcounter{lesson}\arabic{lesson}
    & Statements and expressions&\ref{ch:elementsf}&4F\\
    & Conditionals&\ref{ch:iff}&5F\\
    %%
    \stepcounter{lesson}\arabic{lesson}
    & Looping&\ref{ch:loopf}&6F&&\ref{ex:impl-triangle}\\
    & Functions&\ref{ch:functionf}&7F\\
    %%
    \stepcounter{lesson}\arabic{lesson}
    & I/O &\ref{ch:iof}&8F\\
    & Arrays &\ref{ch:arrayf}&14F&\ref{ex:sect-vs-loop}, \ref{ex:farray-shift}, \ref{ex:fmatnorm}\\
    %%
    \stepcounter{lesson}\arabic{lesson}
    & Objects &\ref{ch:classf}&10F&\ref{ex:fclass-translate}\\
    & Modules&\ref{ch:modulef}&9F (?)&&\ref{ex:fpointmod}\\
    %%
    \stepcounter{lesson}\arabic{lesson}
    & Pointers&\ref{ch:pointf}&15F\\
    \bottomrule % \hline
  \end{tabular}
  \caption{Accelerated lesson plan for Fortran; the annotation `(S)' indicates that a skeleton code is available; `(T)' indicates that
  a tester script is available.}
  \label{tab:f-plan}
\end{table}

\Level 1 {Project-based teaching}

To an extent it is inevitable that students will do a number of
exercises that are not connected to any earlier or later ones.
However, to give some continuity,
this book contains  some programming
projects that students gradually build towards.

The following are simple projects that have a sequence
of exercises building on each other:

\begin{description}
\item[Prime] Prime number testing, culminating in prime number
  sequence objects, and testing a corollary of the Goldbach
  conjecture. Chapter~\ref{ch:prime}.
\item[Geom] Geometry related concepts; this is mostly an exercise in
  object-oriented programming. Chapter~\ref{ch:geom}.
\item[Root] Numerical zero-finding methods. Chapter~\ref{ch:zerofind}.
\end{description}

The following project are halfway serious research projects:

\begin{description}
\item[Infect] The spreading of infectious diseases; these are
  exercises in object-oriented design. Students can explore various
  real-life scenarios. Chapter~\ref{ch:infect}.
\item[Pagerank] The Google Pagerank algorithm. Students program a
  simulated internet, and explore pageranking, including `search
  engine optimization'. This exercise uses lots of
  pointers. Chapter~\ref{ch:google}.
\item[Gerrymandering] Redistricting through dynamic programming.
  Chapter~\ref{ch:gerry}.
\item[Scheduling] Exploration of concepts related to the
  \ac{MTSP}, modeling \indextermbus{Amazon}{Prime}.
  Chapter~\ref{ch:amazon}.
\item[Lapack] Exploration of high performance linear algebra.
  Chapter~\ref{ch:lapack}.
\end{description}

Rather than including the project exercises in the didactic sections,
each section of these projects list the prerequisite basic sections.

The project assignments give a fairly detailed step-by-step suggested
approach. This acknowledges the theory of Cognitive
Load~\cite{Ericson:parsons2017}.

Our projects are very much computation-based. A~more GUI-like approach
to project-based teaching is described
in~\cite{Chen:2012:project-cpp}.

\Level 1 {Choice: Fortran or advanced topics}

After two months of grounding in OOP programming in~C++, the Fortran
lectures and exercises reprise this sequence, letting the students do
the same exercises in Fortran that they did in~C++.  However, array
mechanisms in Fortran warrant a separate lecture.

\Level 0 {Scaffolding}

As much as possible, when introducing a new topic
we try to present a working code,
with the first exercise being a modification of this code.
At the root level of the repository is a directory \n{skeletons},
containing the example programs.
In the above tables,
exercises for which a skeleton code is given
are marked with~`(S)'.

\Level 0 {Grading guide}

Here are some general guidelines on things that should
count as negatives when grading.
These are all style points: a code using them may very well
give the right answer, but they go against what is
commonly considered good or clean code.

The general guiding principle should be:
\begin{quote}
  Code is primarily for humans to read, only secondarily
  for computers to execute.
  This means that in addition to being correct,
  the code
  has to convince the reader that the result is correct.
\end{quote}

As a corollary:
\begin{quote}
  Code should only be optimized when it is correct.
  Clever tricks detract from readability,
  and should only be applied when demonstrably needed.
\end{quote}

\lstset{language=C++}

\Level 1 {Code layout and naming}

Code should use proper indentation.
Incorrect indentation deceives the reader.

Non-obvious code segments should be commented,
but proper naming of variables and functions
goes a long way towards making this less urgent.

\Level 1 {Basic elements}

\begin{itemize}
\item
  Variables should have descriptive names.
  For instance, \lstinline{count} is not descriptive:
  half of all integers are used for counting.
  Count of what?
\item
  Do not use global variables:
\begin{lstlisting}
int i;
int main() {
  cin >> i;
}
\end{lstlisting}
\end{itemize}

%% \Level 1 {Conditionals}

\Level 1 {Looping}

The loop variable should be declared locally,
in the loop header,
if there is no overwhelming reason for declaring it global.
\begin{multicols}{2}
Local declaration:
\begin{lstlisting}
for ( int i=0; i<N; i++) {
  // something with i
}
\end{lstlisting}
\columnbreak
global:
\begin{lstlisting}
int i;
for ( i=0; i<N; i++) {
  // something with i
  if (something) break;
}
// look at i to see where the break was
\end{lstlisting}
\end{multicols}

Use range-based loops if the index is not strictly needed.

\Level 1 {Functions}

There is no preference whether to define a function entirely before the main,
or after the main with only its declaration before.

\begin{multicols}{2}
Defined before:
\begin{lstlisting}
bool f( double x ) { 
  return x>0;
}
int main() {
  cout << f(5.1);
}
\end{lstlisting}
\columnbreak
Only declaration given before:
\begin{lstlisting}
bool f(double x);
int main() {
  cout << f(5.1);
}
bool f( double x ) { 
  return x>0;
}
\end{lstlisting}
\end{multicols}

Only use C++-style parameter passing by value or reference:
do not use C-style \lstinline+int*+ parameters and such.

\Level 1 {Scope}

Variables should be declared in the most local scope that uses them.
Declaring all variables at the start of a subprogram is needed in Fortran,
but is to be discouraged in~C++

\begin{comment}
\Level 1 {Structures}

Do not declare a single global structure:
\begin{multicols}{2}
Good:
\begin{lstlisting}
struct Point { double x,y; } ;
int main() {
  Point mypoint;
  cin >> mypoint.x;
\end{lstlisting}
\columnbreak
Not good:
\begin{lstlisting}
struct Point { double x,y; } mypoint;
int main() {
  cin >> mypoint.x;
\end{lstlisting}
\end{multicols}
\end{comment}

\Level 1 {Classes}

\begin{itemize}
\item Do not declare data members \lstinline{public}.
\item Only write accessor functions if they are really needed.
\item Make sure method names are descriptive of what the method does.
\item The keyword \lstinline{this} is hardly ever needed.
  This usually means the student has been looking at stackoverflow too much.
\end{itemize}

\Level 1 {Arrays vs vectors}

\begin{itemize}
\item Do not use old-style C arrays.
\begin{lstlisting}
int indexes[5];    
\end{lstlisting}
\item Certainly never use \lstinline{malloc} or \lstinline{new}.
\item Iterator (\lstinline{begin}, \lstinline{erase}) are
  seldom used in this course, and should only be used if strictly needed,
  for instance with `algorithms'.
\end{itemize}

\Level 1 {Strings}

Do not use old-style C strings:
\begin{lstlisting}
char *words = "and another thing";
\end{lstlisting}

\Level 1 {Other remarks}

\Level 2 {Uninitialized variables}

Uninitialized variables can lead to undefined or indeterminate behavior.
Bugs, in other words.

\lstset{language=C++}
\begin{lstlisting}
  int i_solution;
  int j_solution;
  bool found_solution;
  for(int i = 0; i < 10; i++){
    for(int j = 0; j < 10; j++){
      if(i*j > n){
        i_solution = i;
        j_solution = j;
        found_solution = true;
        break;
      }

    }
    if(found_solution){break;}
  }
  cout << i_solution << "," << j_solution << endl;
\end{lstlisting}

\begin{verbatim}
%% icpc -o missinginit missinginit.cpp && echo 40 | ./missinginit
0, -917009232
6, 7
\end{verbatim}

This whole issue can be side stepped if the compiler or runtime system
can detect this issue.
Code structure will often prevent detection, 
but runtime detection is always possible, in principle.

For example, the Intel compiler can install a run-time check:

\begin{verbatim}
%% icpc -check=uninit -o missinginit missinginit.cpp && echo 40 | ./missinginit

Run-Time Check Failure: The variable 'found_solution' is being used in missinginit.cpp(22,5) without being initialized

Aborted (core dumped)
\end{verbatim}

\Level 2 {Clearing objects and vectors}
\label{teach:clear}

The following idiom is found regularly:
\begin{lstlisting}
vector<int> testvector;
for ( /* possibilities */ ) {
  testvector.push_back( something );
  if ( testvector.sometest() ) 
    // remember this as the best
  testvector.clear();
}
\end{lstlisting}
A similar idiom occurs with classes,
which students endow with a \lstinline+reset()+ method.

This is more elegantly done by declaring the testvector inside the loop:
the reset is then automatically taken care of.

The general principle here is that entities need to be declared as
local as possible.
