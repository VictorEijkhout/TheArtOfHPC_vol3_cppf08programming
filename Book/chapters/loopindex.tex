% -*- latex -*-
%%%%%%%%%%%%%%%%%%%%%%%%%%%%%%%%%%%%%%%%%%%%%%%%%%%%%%%%%%%%%%%%
%%%%
%%%% This TeX file is part of the course
%%%% Introduction to Scientific Programming in C++/Fortran2003
%%%% copyright 2017-2023 Victor Eijkhout eijkhout@tacc.utexas.edu
%%%%
%%%% loopindex.tex : loop index type
%%%% THIS FILE IS NO LONGER USED
%%%%
%%%%%%%%%%%%%%%%%%%%%%%%%%%%%%%%%%%%%%%%%%%%%%%%%%%%%%%%%%%%%%%%

In the indexed loops above we use \lstinline{int} as the type of the loop variable:
\begin{lstlisting}
  for ( int i=0; i<some_array.size(); ++i ) { /* stuff */ }
\end{lstlisting}
There is a problem with this: integers have a maximal value of
about 2~billion (see section~\ref{sec:limits} for details),
and containers such as \lstinline{vector} can have more elements than that.

If you absolute need that index variable,
give it a type of \indexc{size_t}:
\begin{lstlisting}
  for ( size_t i=0; i<some_array.size(); ++i ) { /* stuff */ }
\end{lstlisting}

Alternatively, do not use a loop index, but use a range-based loop,
or a range-based algorithm.

\begin{remark}
  Some compilers will indeed issue a warning on the first loop type,
  but that is not related to the choice of integer type.
  Instead, the warning will relate to the fact that in
  \lstinline|i<some_array.size()|
  you are comparing a signed to an unsigned quantity.
  That is considered an unsafe practice.
\end{remark}
