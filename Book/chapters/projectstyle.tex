% -*- latex -*-
%%%%%%%%%%%%%%%%%%%%%%%%%%%%%%%%%%%%%%%%%%%%%%%%%%%%%%%%%%%%%%%%
%%%%
%%%% This TeX file is part of the course
%%%% Introduction to Scientific Programming in C++/Fortran2003
%%%% copyright 2017-2022 Victor Eijkhout eijkhout@tacc.utexas.edu
%%%%
%%%% projectstyle.tex : style guide for submissions
%%%%
%%%%%%%%%%%%%%%%%%%%%%%%%%%%%%%%%%%%%%%%%%%%%%%%%%%%%%%%%%%%%%%%

\begin{quote}
  \textsl{The purpose of computing is insight, not numbers. (Richard Hamming)}
\end{quote}

Your project writeup is equally important as the code.
Here are
some common-sense guidelines for a good writeup. However, not all
parts may apply to your project.
Use your good judgement.

\Level 0 {General approach}

As a general rule, consider programming as an experimental science,
and your writeup as a report on some tests you have done: explain
the problem you're addressing, your strategy, your results.

Turn in a writeup in pdf form (Word and text documents are not acceptable)
that was generated from a text processing program such 
(preferably) \LaTeX\ (for a tutorial, see~\CARPref{tut:latex}).

\Level 0 {Style}

Your report should obey the rules of proper English.
\begin{itemize}
\item Observing correct spelling and grammar goes without saying.
\item Use full sentences.
\item Try to avoid verbiage that is disparaging
  or otherwise inadvisable.
  The academic \indexterm{XSEDE} has the following guidelines:
  \url{https://www.xsede.org/terminology};
  much longer and more extentsive, the
  \indextermbus{Google}{developer documentation style guide}~\cite{googlestyle}
  is also a great resource.
\end{itemize}

\Level 0 {Structure of your writeup}

Consider this project writeup an opportunity to practice writing a scientific article.

Start with the obvious stuff.
\begin{itemize}
\item Your writeup should have a title. Not `Project' or `parallel programming',
  but something like 'Parallelization of Chronosynclastic Enfundibula in MPI'.
\item Author and contact information. This differs per publication.
  Here it is: your name, EID, TACC username, and email.
\item Introductory section that is extremely high level: what is the problem,
  what did you do, what did you find.
\item Conclusion: what do your findings mean, what are limitations, opportunities
  for future extensions.
\item Bibliography.
\end{itemize}

\Level 1 {Introduction}

The reader of your document need not be familiar with the project
description, or even the problem it addresses.  Indicate what the
problem is, give theoretical background if appropriate, possibly
sketch a historic background, and describe in global terms how you set
out to solve the problem,
as well as a brief statement of your findings.

\Level 1 {Detailed presentation}

See section~\ref{sec:present-runs} below.

\Level 1 {Discussion and summary}

Separate the detailed presentation from a discussion at the end:
you need to have a short final section that summarizes your work and findings.
You can also discuss possible extensions of your work to cases not covered.

\Level 0 {Experiments}

You should not expect your program to run once and give you a final
answer to your research question.

Ask yourself: what parameters can be varied, and then vary them!
This allows you to generate graphs or multi-dimensional plots.

If you vary a parameter, think about what granularity you use.
Do ten data points suffice, or do you get insight from using~$10,000$?

Above all: computers are very fast, they do a billion operations per second.
So don't be shy in using long program runs.
Your program is not a calculator where a press on the button
immediately gives the answer: you should expect program runs
to take seconds, maybe minutes.

\Level 0 {Detailed presentation of your work}
\label{sec:present-runs}

The detailed presentation of your work is as combination of
code fragments, tables, graphs, and a description of these.

\Level 1 {Presentation of numerical results}

You can present results as graphs/diagrams or tables. The choice
depends on factors such as how many data points you have,
and whether there is an obvious relation to be seen in a graph.

Graphs can be generated any number of ways.
Kudos if you can figure out the \LaTeX\ \indextermtt{tikz} package,
but Matlab or Excel are acceptable too.
No screenshots though.

Number your graphs/tables and refer to the numbering in the text.
Give the graph a clear label and label the axes.

\Level 1 {Code}

Your report should describe in a global manner the algorithms you
developed, and you should include relevant code snippets. If you want
to include full listings, relegate that to an appendix:
code snippets in the text
should only be used to illustrate especially salient points.

Do not use screen shots of your code: at the very least use a
monospaced font such as the \n{verbatim} environment,
but using the \n{listings} package
(used in this book)
is very much recommended.



