% -*- latex -*-
%%%%%%%%%%%%%%%%%%%%%%%%%%%%%%%%%%%%%%%%%%%%%%%%%%%%%%%%%%%%%%%%
%%%%
%%%% This TeX file is part of the course
%%%% Introduction to Scientific Programming in C++/Fortran2003
%%%% copyright 2019-2023 Victor Eijkhout eijkhout@tacc.utexas.edu
%%%%
%%%% traffic.tex : exercises for traffic modeling
%%%%
%%%% Becky's model: https://github.com/cybergis/LargeScaleABM
%%%%
%%%%%%%%%%%%%%%%%%%%%%%%%%%%%%%%%%%%%%%%%%%%%%%%%%%%%%%%%%%%%%%%

This section contains a sequence of exercises that builds up to a
simulation of car traffic.

\Level 0 {Problem statement}

Car traffic is determined by individual behaviors:
drivers accelerating, braking, deciding to go one way or another.
From this, emergent phenomena arise,
most clearly of course traffic jams.

\Level 0 {Code design}

In this project we will explore the so-called \indexterm{actor model}:
the simulation consists of independent actors that react to each others' actions.

We need only two basic classes:
\begin{enumerate}
\item A \lstinline{Car} class, where objects have a location and a speed,
  and they can react to the car in front of them.
  (For simplicity we assume that drivers only look at the car immediately ahead of them.)
\item a \lstinline{Street} class, where each street is a container for a number of cars.
\end{enumerate}

There are various ways we can run a simulation
(with \indexterm{thread}s, driven by \indexterm{interrupt}s),
but for simplicity we consider discrete time steps.
This means that both the car and street class have a \lstinline{progress}
method that advanced the object by one time step.

\Level 1 {Cars}

The main thing you can ask of a car, is to move by a time step.
This means you need to have its location, speed, and direction.
It makes sense for a car to know its own speed and location,
but the direction derives from the street.
Thus, the street could update the cars as:
%
\verbatimsnippet{streetmovecars}
%
and the car would take the direction as an input.

\Level 1 {Street}

Let's start wit one-way streets.
A~street is then a collection of cars, but,
until they start passing each other,
they are actually ordered.
A~vector would be suitable for that.
However, a car can move to a different street,
or even be in two streets at once if it's on an intersection.
For that reason, model a street as
\begin{lstlisting}
class Street {
private:
  vector<shared_ptr<Car>> cars;
\end{lstlisting}

Once we get the simulation going,
cars are going to react to each other's behavior,
in particular the behavior of the car immediately in front of them.
There are various ways you can model that.
\begin{enumerate}
\item
  For instance, 
  the street can pass the information of one car to the next.
\item Possibly more elegant, each car has a pointer to the one
  before and after it.
\end{enumerate}

\Level 1 {Unit tests}

At this level you can do a number of unit tests to ensure that your code
behaves as intended.
\begin{enumerate}
\item Define a street, and test its unit vector.
\item Place a car in various locations, both on the street and
  beyond or the side of it. Write test methods for these cases,
  for instance
  \verbatimsnippet{streettestbeyond}
\item If you put a car on the street, its speed stays constant,
  and it leaves the street after a predictable amount of time:
  \verbatimsnippet{traffic1progress}
\end{enumerate}

