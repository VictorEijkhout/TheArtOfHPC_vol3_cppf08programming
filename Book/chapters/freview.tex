% -*- latex -*-
%%%%%%%%%%%%%%%%%%%%%%%%%%%%%%%%%%%%%%%%%%%%%%%%%%%%%%%%%%%%%%%%
%%%%
%%%% This TeX file is part of the course
%%%% Introduction to Scientific Programming in C++/Fortran2003
%%%% copyright 2017-9 Victor Eijkhout eijkhout@tacc.utexas.edu
%%%%
%%%% freview.tex : review questions over Fortran
%%%%
%%%%%%%%%%%%%%%%%%%%%%%%%%%%%%%%%%%%%%%%%%%%%%%%%%%%%%%%%%%%%%%%


\Level 0 {Fortran versus C++}

\begin{exercise}
  \label{ex:compare-language}
  For each of C++, Fortran, Python:
  \begin{itemize}
  \item Give an example of an application or application area that the
    language is suited for, and
  \item Give an example of an application or application area that the
    language is not so suited for.
  \end{itemize}
\end{exercise}

\Level 0 {Basics}

\begin{exercise}
  \label{ex:f-whatfor}
  \begin{itemize}
  \item What does the \lstinline{Parameter} keyword do? Give an
    example where you would use it.
  \item Why would you use a \lstinline{Module}?
  \item What is the use of the \lstinline{Intent} keyword?
  \end{itemize}
\end{exercise}

\Level 0 {Arrays}

\begin{exercise}
  \label{ex:ftemperature}
  You are looking at historical temperature data, say a table of the
  high and low temperature at January~1st of every year between 1920
  and now, so that is 100 years.

  Your program accepts data as follows:
\begin{lstlisting}
Integer :: year, high, low

!! code omitted

read *,year,high,low
\end{lstlisting}
  where the temperatures are rounded to the closest degree (Centigrade
  of Fahrenheit is up to you.)

Consider two scenarios. For both, 
  give the lines of code for 1.~the array in which you store the data,
  2.~the statement that inserts the values into the array.
  \begin{itemize}
  \item Store the raw temperature data.
  \item 
    Suppose you are interested in knowing how often certain high/low
    temperatures occurred. For instance, `how many years had a high
    temperature of~32F~/0~C'.
  \end{itemize}
\end{exercise}


\Level 0 {Subprograms}

\begin{exercise}
  \label{ex:flooppos}
  Write the missing procedure \n{pos_input} that
  \begin{itemize}
  \item reads a number from the user
  \item returns it
  \item and returns whether the number is positive
  \end{itemize}
  in such a way to make this code work:
  
  \answerwithoutput{flooppos}{funcf}{looppos}

  Give the function parameter(s) the right \lstinline{Intent} directive.

  Hint: is \n{pos_input} a \n{SUBROUTINE} or \n{FUNCTION}? If the
  latter, what is the type of the function result? How many parameters
  does it have otherwise? Where does the variable \n{user_input} get
  its value? So what is the type of the parameter(s) of the function?
\end{exercise}

