% -*- latex -*-
%%%%%%%%%%%%%%%%%%%%%%%%%%%%%%%%%%%%%%%%%%%%%%%%%%%%%%%%%%%%%%%%
%%%%
%%%% This TeX file is part of the course
%%%% Introduction to Scientific Programming in C++/Fortran2003
%%%% copyright 2017-2023 Victor Eijkhout eijkhout@tacc.utexas.edu
%%%%
%%%% proving.tex : theoretical aspects of programming
%%%%
%%%%%%%%%%%%%%%%%%%%%%%%%%%%%%%%%%%%%%%%%%%%%%%%%%%%%%%%%%%%%%%%

Programming often seems more an art, even a black one,
than a science. Still, people have tried
systematic approaches to program correctness.
One can distinguish between
\begin{itemize}
\item proving that a program is correct, or
\item writing a program so that it is guaranteed to be correct.
\end{itemize}
This distinction is only imaginary. A~more fruitful approach
is to let the proof drive the coding.
As E.W. Dijkstra pointed out
\begin{quotation}
  The only effective way to raise the confidence level of a program
  significantly is to give a convincing proof of its correctness. But
  one should not first make the program and then prove its correctness,
  because then the requirement of providing the proof would only
  increase the poor programmer’s burden. On the contrary: the programmer
  should let correctness proof and program grow hand in hand.
\end{quotation}

We will see a couple of examples of this.

\Level 0 {Loops as quantors}

%%packtsnippet quantor

Quite often, algorithms can be expressed mathematically.
In that case you should make your program look like the mathematics.
Here we consider one example,
pointing out the connection between for-loops
and mathematical quantors.

\Level 1 {Forall-quantor}

Consider a simple example: testing if a number is prime.
The predicate `isprime' can be expressed as:
\[
\mathop{\mathrm{isprime}}(n) \equiv
\forall_{2\leq f<n}\colon \neg\mathop{\mathrm{divides}}(f,n)
\]
You see that proving the original \lstinline{isprime} predicate
for some value~$n$ now involves 
\begin{enumerate}
\item a quantor over a new variable~$f$ --~we call this a `bound variable';
\item and a new predicate \lstinline{divides} that involves the original
  variable~$n$ and the variable $f$ that is bound by the quantor.
\end{enumerate}

We now spell out the `for all' quantor iteratively
as a loop where each iteration needs to be true.
That is, we do an `and' reduction on some iteration-dependent result.
\[ 
\neg\mathop{\mathrm{divides}}(2,n)
\cap \ldots \cap
\neg\mathop{\mathrm{divides}}(n-1,n)
\]
And this sequence of `and' conjunctions can be programmed:
\begin{lstlisting}
for (int f=2; f<n; f++)
  isprime = isprime && not divides(f,p)
\end{lstlisting}
You notice that the loop variable is the variable~$f$
that was introduced by the quantor.

Now our only worry is how to initialize \lstinline+isprime+.
The initial value corresponds to an `and' conjunction over
an empty set, which is true, so:
\begin{lstlisting}
bool isprime{true};
for (int f=2; f<n; f++)
  isprime = isprime && not divides(f,p)
\end{lstlisting}
\begin{exercise}
  Now that you have a loop that computes the right thing
  you can start worrying about performance.
  Can the loop be terminated prematurely in some cases?
  How would you program that?
\end{exercise}

\Level 1 {Thereis-quantor}

What if we had expressed primeness as:
\[
\mathop{\mathrm{isprime}}(n) \equiv
\neg\exists_{2\leq f<n}\colon \mathop{\mathrm{divides}}(f,n)
\]
To get a pure quantor, and not a negated one,  we write:
\[
\mathop{\mathrm{isnotprime}}(n) \equiv
\exists_{2\leq f<n}\colon \mathop{\mathrm{divides}}(f,n)
\]
Spelling out the exists-quantor as
\[
\mathop{\mathrm{isnotprime}}(n) \equiv
\mathop{\mathrm{divides}}(2,n)
\cup \ldots \cup
\mathop{\mathrm{divides}}(n-1,n)
\]
we see that we need a loop
where we test if any iteration satisfies a predicate.
That is, we do an `or'-reduction on the results of each iteration.
As before, the loop variable is the variable introduced by the quantor.
\begin{lstlisting}
for (int f=2; f<n; f++)
  isnotprime = isnotprime or divides(f,p)
bool isprime = not isnotprime;
\end{lstlisting}

Also as before, we take care to initialize the reduction variable correctly:
applying $\exists_{s\in S}P(s)$ over an empty set~$S$ is \lstinline{false}:
\begin{lstlisting}
bool isnotprime{false};
for (int f=2; f<n; f++)
  isnotprime = isnotprime or divides(f,p)
bool isprime = not isnotprime;
\end{lstlisting}

\begin{exercise}
  Same question as for the `forall' quantor:
  can this loop be terminated prematurely?
  How would you code that?
\end{exercise}

\Level 1 {Quantors through ranges}

We have the following correspondences:

  \begin{tabular}{cc}
    $\forall$&\n{all_of}\\ 
    $\exists$&\n{any_of}
  \end{tabular}

Let $S$ be a range and $P$ a predicate, then
\[ \forall_{n\in S}\colon P(n) \]
can be implemented as
\begin{lstlisting}
  all_of( S,[] { auto n } -> bool { return P(n); } );
\end{lstlisting}
Likewise,
\[ \exists_{n\in S}\colon P(n) \]
can be implemented as
\begin{lstlisting}
  any_of( S,[] { auto n } -> bool { return P(n); } );
\end{lstlisting}

In the following example we `prove' a simple statement over integers.
Using \lstinline+iota(1)+ to generate all integers
would lead to infinite runtime, so we truncate with a second parameter.

\begin{block}{Example}
  \label{sl:forall-n-greater}
  Example: \[ \forall_n\colon\exists_m\colon m>n \]
  \verbatimsnippet{allngreaterm}
\end{block}

Ranging over more complicated sets:
\[ \forall_{\scriptstyle \begin{array}{cc}n\in S \\ Q(n)\end{array}}\colon P(n) \]
can be implemented as
\begin{lstlisting}
all_of( S | filter( [] (auto n) -> bool { return Q(n) } ),
        [] { auto n } -> bool { return P(n); } );
\end{lstlisting}

\begin{exercise}
  \label{ex:forall-n-greater-even}
  Write range-based code that proves
  \[ \forall_n\colon \exists_{m>n}\colon \mathop{\mathrm{even}}(m) \]
\end{exercise}

%%packtsnippet end

If you're doing the prime numbers project,
you can now do exercise~\ref{ex:goldbach-range20}.

\Level 0 {Predicate proving}

For programs that have a clear loop structure you can take
an approach that is similar to doing a `proof by induction'.

Let us consider the Collatz conjecture again,
where for brevity we define
\[ c(\ell)=\hbox{ the length of the Collatz sequences, starting on $\ell$}. \]
Now we consider the Collatz conjecture as proving a predicate
\[ P(\ell_k,m_k,k) =
\begin{cases}
  \hbox{$\ell_k<k$ is the location of the longest sequence:}\\
  \hbox{$c(\ell_k)=m_k$: the length of sequence $\ell_k$ is $m_k$}\\
  \hbox{all other sequences $\ell<k$ are shorter}\\  
\end{cases}
\]
Formally:
\[ P(\ell_k,m_k,k) =
\left[
\begin{array}{cl}
  &\ell_k<k\\
  \wedge & c(\ell_k)=m_k \hbox{(only if $k>0$)} \\
  \wedge & \forall_{\ell<k}\colon c(\ell)\leq m_k\\
\end{array} \right.
\]
for $k=N$.

We develop the code that makes this predicate inductively true.
We start out with
\[ \ell_0=-1, \quad m_0=0 \Rightarrow P(\ell_0,m_0,0). \]
The inductive proof corresponds to a loop:
\begin{itemize}
\item we assume that at the start of the $k$-th iteration
  $P(\ell_k,m_k,k)$ is true;
\item the iteration body is such that at the end of the $k$-th iteration
  $P(\ell_{k+1},m_{k+1},k+1)$ is true;
\item this of course sets up the predicate at the start of the next iteration.
\end{itemize}

The loop structure is then:
\begin{lstlisting}[mathescape=true]
k=0;
$ \{ P(l_k,m_k,k) \} $
while ( k<N ) {
  $ \{ P(l_k,m_k,k) \} $
  update;
  $ \{ P(l_{k+1},m_{k+1},k+1) \} $
  k = k+1;
}
\end{lstlisting}
The update has to extend the predicate from $k$ to~$k+1$.
Let us consider the parts of it.

We need to establish
\[ \forall_{\ell<k+1}\colon c(\ell)\leq m_{k+1} \]
We split the range $\ell<k+1$ into $\ell<k$ and $\ell=k$:
\begin{itemize}
\item the first part
  \[ \forall_{\ell<k}\colon c(\ell)\leq m_{k+1} \]
  is true if $m_{k+1}\geq m_k$;
\item the part
  \[ \ell=k \colon c(\ell)\leq m_{k+1} \]
  states that $m_{k+1}\geq c(k)$.
\end{itemize}
Together we get that
\[ m_{k+1}\geq \max(m_k,c(k)) \]

Finally, the clause \[ c(\ell_{k+1})=m_{k+1} \]
can be satisfied:
\begin{itemize}
\item If $c(k)>m_k$, we need to set $m_{k+1}=c(k)$ and $\ell_{k+1}=k$.
\item (Strictly speaking, there is a possibility 
  $m_{k+1}>c(k)$. This is not possible, because we
  can not satisfy $m_{k+1}=c(\ell_k)$ for any~$k$.)
\item If $c(k)\leq m_k$, we need to set $m_{k+1}\geq m_k$.
  Again, $m_{k+1}>m_k$ can not be satisfied by any~$\ell_{k+1}$,
  so we conclude $m_{k+1}=m_k$.
\end{itemize}

\Level 0 {Flame}

Above, you saw a Dijkstra quote where he argues that testing is
insufficient to show correctness of a program.
So how does Dijkstra envision that correctness can be ensured?
That can be found in the second part of his quote:

\begin{block}{Dijkstra quote, part 2}
  \begin{quotation}
    The only effective way to raise the confidence level of a
    program significantly is to give a convincing proof of its
    correctness. But one should not first make the program and then
    prove its correctness, because then the requirement of providing
    the proof would only increase the poor programmer’s burden. On the
    contrary: the programmer should let correctness proof and program
    grow hand in hand.
  \end{quotation}
\end{block}

Let us develop this though, using matrix-vector multiplication
as a simple example:
we will derive the algorithm hand-in-hand with its correctness proof.

Many linear algebra algorithms are loop-based, and the foundation
to a correctness proof is the derivation of a \indextermbus{loop}{invariant}:
a predicate that is inductively shown to be true in each loop iteration,
thereby guaranteeing the correctness of the whole algorithm.

\Level 1 {Derivation of the common algorithm}

As a preliminary to deriving a loop invariant, we first consider
the result of the computation in a partitioned form.

\begin{block}{Matrix-vector product}
  \[ y = Ax \]
  Partitioned:
  \[
  \begin{pmatrix}
    y_T\\ y_B
  \end{pmatrix}
  =
  \begin{pmatrix}
    A_T\\ A_B
  \end{pmatrix}
  \begin{pmatrix}
    x 
  \end{pmatrix}
  \]
  Two equations:
  \[
  \begin{cases}
    y_T = A_T x\\ y_B = A_B x
  \end{cases}
  \]
\end{block}

The key to the inductive proof is
to take this partitioned form, and assume that
it is only partly satisfied.

\begin{block}{Inductive construction}
  \[
  \begin{pmatrix}
    y_T\\ y_B
  \end{pmatrix}
  =
  \begin{pmatrix}
    A_T\\ A_B
  \end{pmatrix}
  \begin{pmatrix}
    x 
  \end{pmatrix}
  \]
Assume only equation
  \[ y_T = A_T x \]
  is satisfied.
\end{block}

Now we are getting close to an inductive proof:
we consider the algorithm as aimed at increasing
the size of the block for which the predicate is true.
If this block equals the size of the problem,
we have correctness of the full result.

\begin{block}{Algorithm outline}
  \[
  \begin{pmatrix}
    y_T\\ y_B
  \end{pmatrix}
  =
  \begin{pmatrix}
    A_T\\ A_B
  \end{pmatrix}
  \begin{pmatrix}
    x 
  \end{pmatrix}
  \]
  \begin{tabbing}
    Whil\=e $T$ is not the whole system\\
    \> Predicate: $y_T = A_T x$ true\\
    \> Update: grow $T$ block by one\\
    \> Predicate: $y_T = A_T x$ true for new/bigger $T$ block\\
  \end{tabbing}
  Note initial and final condition.
\end{block}

Now we compare the true statement in one iteration,
and that in the next, and compare the two blocks
for which the predicate holds.

\begin{block}{Inductive step}
  \small
  Here is the big trick\\
  Before
  \[
  \begin{pmatrix}
    y_T\\ y_B
  \end{pmatrix}
  =
  \begin{pmatrix}
    A_T\\ A_B
  \end{pmatrix}
  \begin{pmatrix}
    x 
  \end{pmatrix}
  \]
  split:
  \[
  \begin{pmatrix}
    y_1\\ \cdots \\ y_2 \\ y_3
  \end{pmatrix}
  =
  \begin{pmatrix}
    A_1\\ \cdots \\ A_2 \\ A_3
  \end{pmatrix}
  \begin{pmatrix}
    x 
  \end{pmatrix}
  \]
  Then the update step, and \\ After
  \[
  \begin{pmatrix}
    y_1\\  y_2 \\ \cdots \\ y_3
  \end{pmatrix}
  =
  \begin{pmatrix}
    A_1 \\ A_2 \\ \cdots \\ A_3
  \end{pmatrix}
  \begin{pmatrix}
    x 
  \end{pmatrix}
  \]
  and unsplit
  \[
  \begin{pmatrix}
    y_T\\ y_B
  \end{pmatrix}
  =
  \begin{pmatrix}
    A_T\\ A_B
  \end{pmatrix}
  \begin{pmatrix}
    x 
  \end{pmatrix}
  \]
\end{block}

Comparing these two blocks gives us
the extra predicate that was made to be satisfied
by the iteration we consider:

\begin{block}{}
  Before the update:
  \[
  \begin{pmatrix}
    y_1\\ \cdots \\ y_2 \\ y_3
  \end{pmatrix}
  =
  \begin{pmatrix}
    A_1\\ \cdots \\ A_2 \\ A_3
  \end{pmatrix}
  \begin{pmatrix}
    x 
  \end{pmatrix}
  \]
  so \[ y_1 = A_1 x \] is true\\
  Then the update step, and \\ After
  \[
  \begin{pmatrix}
    y_1\\  y_2 \\ \cdots \\ y_3
  \end{pmatrix}
  =
  \begin{pmatrix}
    A_1 \\ A_2 \\ \cdots \\ A_3
  \end{pmatrix}
  \begin{pmatrix}
    x 
  \end{pmatrix}
  \]
  so \[
  \begin{cases}
    y_1 = A_1x & \hbox{we had this}\\
    y_2 = A_2x & \hbox{we need this}
  \end{cases}
  \]
\end{block}

This extra predicate can trivially be converted
to an elementary computation,
and so we find the full algorithm:

\begin{block}{Resulting algorithm}
  \begin{tabbing}
    Whil\=e $T$ is not the whole system\\
    \> Predicate: $y_T = A_T x$ true\\
    \> Update: $y_2=A_2x$ \\
    \> Predicate: $y_T = A_T x$ true for new/bigger $T$ block\\
  \end{tabbing}
\end{block}

\Level 2 {Derivation of the algorithm by columns}

In the previous section we derived the matrix-vector product,
expressed the usual way: each element of the output vector
is the inner product of a matrix row and the input vector.
You may think that this is a lot of to-do for not much result.

Consider then that we can derive other algorithms for the matrix-vector product,
using the same general principle.
Our basic assumption was that we split the matrix horizontally
in two block rows.
What would happen if we split the matrix vertically in two
block columns?

We divide the matrix into two block columns, and express the
result of the matrix-vector product on this form.

\begin{block}{Matrix-vector product, the other way around}
  \[ y = Ax \]
  Partitioned:
  \[
  \begin{pmatrix}
    y
  \end{pmatrix}
  =
  \begin{pmatrix}
    A_L & A_R
  \end{pmatrix}
  \begin{pmatrix}
    x_T \\ x_B
  \end{pmatrix}
  \]
  Equation:
  \[
  \begin{cases}
    y = A_L x_T + A_R x_B
  \end{cases}
  \]
\end{block}

Again we make a statement about a partially completed computation.

\begin{block}{Inductive construction}
  \[
  \begin{pmatrix}
    y
  \end{pmatrix}
  =
  \begin{pmatrix}
    A_L & A_R
  \end{pmatrix}
  \begin{pmatrix}
    x_T \\ x_B
  \end{pmatrix}
  \]
Assume 
  \[ y = A_L x_T \]
  is constructed, and grow the $T$ block.
\end{block}

Again we compare splitting the matrix in one iteration
and the next, and comparing the partial predicates:

\begin{block}{Inductive step}
  \small
  Before
  \[
  \begin{pmatrix}
    y
  \end{pmatrix}
  =
  \begin{pmatrix}
    A_L & A_R
  \end{pmatrix}
  \begin{pmatrix}
    x_T \\ x_B
  \end{pmatrix}
  \]
  split:
  \[
  \begin{pmatrix}
    y
  \end{pmatrix}
  =
  \begin{pmatrix}
    A_1 & \vdots & A_2 & A_3
  \end{pmatrix}
  \begin{pmatrix}
    x_1\\ \cdots \\ x_2 \\ x_3
  \end{pmatrix}
  \]
  Then the update step, and \\ After
  \[
  \begin{pmatrix}
    y
  \end{pmatrix}
  =
  \begin{pmatrix}
    A_1 & A_2  & \vdots & A_3
  \end{pmatrix}
  \begin{pmatrix}
    x_1 \\ x_2 \\ \cdots\\ x_3
  \end{pmatrix}
  \]
  and unsplit
  \[
  \begin{pmatrix}
    y
  \end{pmatrix}
  =
  \begin{pmatrix}
    A_L & A_R
  \end{pmatrix}
  \begin{pmatrix}
    x_T \\ x_B
  \end{pmatrix}
  \]
\end{block}

This gives us by `subtracting' one predicate from the other
the extra result that was derived by the single iteration,
and therefore the computation that was done in that iteration.

\begin{block}{Derivation of the update}
  \small
  Before the update:
  \[
  \begin{pmatrix}
    y
  \end{pmatrix}
  =
  \begin{pmatrix}
    A_1 & \vdots & A_2 & A_3
  \end{pmatrix}
  \begin{pmatrix}
    x_1\\ \cdots \\ x_2 \\ x_3
  \end{pmatrix}
  \]
  so \[ y = A_1 x_1 \] is true\\
  Then the update step, and \\ After
  \[
  \begin{pmatrix}
    y
  \end{pmatrix}
  =
  \begin{pmatrix}
    A_1 & A_2  & \vdots & A_3
  \end{pmatrix}
  \begin{pmatrix}
    x_1 \\ x_2 \\ \cdots\\ x_3
  \end{pmatrix}
  \]
  so \[ y = A_1x_1 + A_2x_2 \]
  in other words, we need
  \[ y \leftarrow y+A_2x_2 \]
\end{block}

As a result we obtain a second form of the matrix-vector product.

\begin{block}{Resulting algorithm}
  \begin{tabbing}
    Whil\=e $T$ is not the whole system\\
    \> Predicate: $y = A_L x_T$ true\\
    \> Update: $y \leftarrow y + A_2x_2$ \\
    \> Predicate: $y = A_L x_T$ true for new/bigger $T$ block\\
  \end{tabbing}
\end{block}

In quasi-Matlab notation we express both algorithms:

\begin{block}{Two algorithms}
  \begin{multicols}{2}
    \begin{tabbing}
      for \=$r=1,m $\\
      \> $y_r = A_{r,*}x_* $\\
    \end{tabbing}
    \columnbreak
    \begin{tabbing}
      $y\leftarrow 0$\\
      for \=$c=1,n$\\
      \> $y\leftarrow y+A_{*,c}x_c$
    \end{tabbing}
  \end{multicols}
\end{block}


