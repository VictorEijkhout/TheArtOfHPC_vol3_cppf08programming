% -*- latex -*-
%%%%%%%%%%%%%%%%%%%%%%%%%%%%%%%%%%%%%%%%%%%%%%%%%%%%%%%%%%%%%%%%
%%%%
%%%% This TeX file is part of the course
%%%% Introduction to Scientific Programming in C++/Fortran2003
%%%% copyright 2017-2023 Victor Eijkhout eijkhout@tacc.utexas.edu
%%%%
%%%% pointer.tex : about pointers
%%%%
%%%%%%%%%%%%%%%%%%%%%%%%%%%%%%%%%%%%%%%%%%%%%%%%%%%%%%%%%%%%%%%%


Pointers are an indirect way of associating an entity with a variable
name. Consider for instance the circular-sounding definition of a list:
\begin{quote}
  A list consists of nodes,
  where a node contains some information in its `head',
  and which has a `tail' that is a list.
\end{quote}

\begin{block}{Recursive data structures}
  \label{sl:recursive-node}
Naive code:
\begin{lstlisting}
class Node {
private:
  int value;
  Node tail;
  /* ... */
};
\end{lstlisting}
This does not work: would take infinite memory.

Indirect inclusion: only `point' to the tail:
\begin{lstlisting}
class Node {
private:
  int value;
  PointToNode tail;
  /* ... */
};
\end{lstlisting}
\end{block}

This chapter will explain C++ \indextermsub{smart}{pointer}s, and give some uses for them.

\Level 0 {The `arrow' notation}

\begin{block}{Members from pointer}
  \begin{itemize}
  \item If \lstinline{x} is object with member~\lstinline{y}:\\ \lstinline{x.y}
  \item If \lstinline{xx} is pointer to object with member~\lstinline{y}:\\ \lstinline{xx->y}
  \item In class methods \lstinline{this} is a pointer to the object, so:
\begin{lstlisting}
class x {
  int y;
  x(int v) {
    this->y = v; }
}
\end{lstlisting}
\item Arrow notation works with old-style pointers and new
  shared/unique pointers.
  \end{itemize}
\end{block}

\Level 0 {Making a shared pointer}
\label{sec:shared_ptr}

Smart pointers are used the same way as old-style pointers in~C.
If you have an
object \lstinline{Obj X} with a member \lstinline{y}, you access that with \lstinline{X.y}; if you
have a pointer~\lstinline{X} to such an object, you write \lstinline{X->y}.

So what is the type of this latter~\lstinline{X} and how did you create it?

\begin{block}{Use of a shared pointer}
  \label{sl:use-shared}
Same as C-pointer syntax:
\snippetwithoutput{pointxuse}{pointer}{pointx}
\end{block}

You may think that you first create an object, and then set a pointer to it,
the way it happens in many other languages, but smart pointers work differently:
you create the object and the pointer to it in one call.
\begin{lstlisting}
  make_shared< ClassName >( constructor arguments );
\end{lstlisting}
The resulting object is of type \lstinline+shared_ptr<ClassName>+,
but you can save yourself spelling that out, and use \lstinline{auto}
instead.

\begin{block}{Example: step 2, creating the pointer}
  \label{sl:make-shared}
  Allocation of object and pointer to it in one:
\begin{lstlisting}
auto X = make_shared<HasX>( /* args */ );

// or explicitly:

shared_ptr<HasX> X =
    make_shared<HasX>( /* constructor args */ );
\end{lstlisting}
\end{block}

\begin{block}{Example: step 1, we need a class}
  \label{sl:shared-ptr}
  Simple class that stores one number:
  \verbatimsnippet{hasxdef}
\end{block}

\begin{block}{Example: step 3: headers to include}
  \label{sl:pointer-header}
  Using smart pointers requires at the top of your file:
\begin{lstlisting}
#include <memory>
using std::shared_ptr;
using std::make_shared;

using std::unique_ptr;
using std::make_unique;
\end{lstlisting}
(unique pointers will not be discussed further here)
\end{block}

\begin{block}{Example: step 4: in use}
  \label{sl:pointwhy}
  Why do we use pointers?\\
  Pointers make it possible for two variables to own the same object.
  %%{pointxuse}{pointer}{pointx}
  \snippetwithoutput{twopoint}{pointer}{twopoint}
  What is the difference with
\begin{lstlisting}
HasX five(5);
HasX v1 = five;
HasX v2 = five;
\end{lstlisting}
?
\end{block}

\begin{block}{Pointers to arrays}
  \label{sl:shared-vector}
  The constructor syntax is a little involved for vectors:
\begin{lstlisting}
auto x = make_shared<vector<double>>(vector<double>{1.1,2.2});
\end{lstlisting}
\end{block}

\Level 1 {Pointers and arrays}

\begin{slide}{Shared pointers for arrays}
This offers a safer version of \lstinline{new}. But you should use a
\lstinline{vector} anyway:
\begin{lstlisting}
shared_ptr<myobject> obj_ptr
    = make_shared<myobject>(x);
    // = shared_ptr<myobject>( new myobject(x) );
obj_ptr->mymethod(1.1);
cout << obj_ptr->member << endl;

auto array = shared_ptr<double>( new double[100] );
// ILLEGAL: array[1] = 2.14;
array->at(2) = 3.15;
\end{lstlisting}
\end{slide}

The prototypical example for the use of pointers is in
linked lists and graph data structures.
See section~\ref{sec:linklist} for code and discussion.

\begin{exercise}
  If you are doing the geometry project (chapter~\ref{ch:geom})
  you can now do exercise~\ref{sec:dynrect}.
\end{exercise}

\Level 1 {Smart pointers versus address pointers}

\begin{block}{Pointers don't go with addresses}
  \label{sl:shareaddress}

  The oldstyle \lstinline{&y} address pointer can not be made smart:
  %
  \verbatimsnippet{shareaddress}
  %
  gives:
  {\small
\begin{verbatim}
address(56325,0x7fff977cc380) malloc: *** error for object
    0x7ffeeb9caf08: pointer being freed was not allocated
\end{verbatim}
}
\end{block}

Smart pointers are much better than old style pointers
\begin{lstlisting}
Obj *X;
*X = Obj( /* args */ );
\end{lstlisting}

There is a final way of creating a shared pointer where you cast an
old-style
\indexc{new} object to shared
pointer
\begin{lstlisting}
auto p = shared_ptr<Obj>( new Obj );
\end{lstlisting}
This is not the preferred mode of creation, but it can be useful in the case of
\indextermsub{weak}{pointer}s; section~\ref{sec:weak_ptr}.

\Level 0 {Memory leaks and garbage collection}
\label{sec:leak}

The big problem with C-style pointers is the chance of a
\indextermbus{memory}{leak}. If a pointer to a block of memory goes out of
scope, the block is not returned to the \ac{OS}, but it is no longer
accessible.
\begin{lstlisting}
// the variable `array' doesn't exist
{
  // attach memory to `array':
  double *array = new double[25];
  // do something with array
}
// the variable `array' does not exist anymore
// but the memory is still reserved.
\end{lstlisting}
Shared and unique
pointers do not have this problem: if they go out of scope, or are
overwritten, the destructor on the object is called, thereby releasing
any allocated memory.

An example.

\begin{block}{Illustration}
  \label{sl:construct-destruct-trace}
  We need a class with constructor and destructor tracing:
  \verbatimsnippet{thingcall}
\end{block}

Just to illustrate that the destructor gets called when the object
goes out of scope:

\begin{block}{Show constructor / destructor in action}
  \label{sl:thing-con-dest}
  \snippetwithoutput{shareptr0}{pointer}{ptr0}
\end{block}

Now illustrate that the destructor of the object is called
when the pointer no longer points to the object.
We do this by assigning \lstinline{nullptr} to the pointer.
(This is very different from \indexc{NULL} in~C:
the null pointer is actually an object with a type.)

\begin{block}{Illustration 1: pointer overwrite}
  \label{sl:shared-ptr-overwrite}
  Let's create a pointer and overwrite it:
  %
  \snippetwithoutput{shareptr1}{pointer}{ptr1}
\end{block}

However, if a pointer is copied, there are two pointers to the same
block of memory, and only when both disappear, or point elsewhere, is
the object deallocated.

\begin{block}{Illustration 2: pointer copy}
  \label{sl:shared-ptr-copy}
  \snippetwithoutput{shareptr2}{pointer}{ptr2}
  \begin{itemize}
  \item The object counts how many pointers there are:
  \item `reference counting'
  \item A pointed-to object is deallocated if no one points to it.
  \end{itemize}
\end{block}

\begin{remark}
  A more obscure source of memory leaks has to do with exceptions:
  \begin{lstlisting}
    void f() {
      double *x = new double[50];
      throw("something");
      delete x;
      }
  \end{lstlisting}
  Because of the exception
  (which can of course come from a nested function call)
  the \lstinline{delete} statement is never reached,
  and the allocated memory is leaked.
  Smart pointers solve this problem.
\end{remark}

\Level 0 {More about pointers}

\Level 1 {Get the pointed data}

Most of the time, accessing the target of the pointer through the
arrow notation is enough. However, if you actually want the object,
you can get it with \indexc{get}. Note that this does not give
you the pointed object, but a traditional pointer.

\begin{block}{Getting the underlying pointer}
  \label{sl:pointer-get}
\begin{lstlisting}
X->y;
// is the same as
X.get()->y;
// is the same as
( *X.get() ).y;
\end{lstlisting}

\snippetwithoutput{pointy}{pointer}{pointy}
\end{block}

\Level 0 {Advanced topics}

\Level 1 {Unique pointers}

Shared pointers are fairly easy to program, and they come with lots of
advantages, such as the automatic memory management. However, they
have more overhead than strictly necessary because they have a
\indexterm{reference count} mechanism to support the memory
management. Therefore, there exists a \indextermsub{unique}{pointer},
\indexc{unique_ptr}, for cases where an object will only ever be
`owned' by one pointer. In that case, you can use a C-style
\indextermsub{bare}{pointer} for non-owning references.

\Level 1 {Base and derived pointers}

Suppose you have base and derived classes:
\begin{lstlisting}
class A {};
class B : public A {};  
\end{lstlisting}
Just like you could assign a \lstinline{B} object to an \lstinline{A}
variable:
\begin{lstlisting}
B b_object;
A a_object = b_object;
\end{lstlisting}
is it possible to assign a \lstinline{B} pointer to an \lstinline{A}
pointer?

The following construct makes this possible:
\begin{lstlisting}
auto a_ptr = shared_pointer<A>( make_shared<B>() );
\end{lstlisting}

Note: this is better than
\begin{lstlisting}
auto a_ptr = shared_pointer<A>( new B() );
\end{lstlisting}
Again a reason we don't need \indexc{new} anymore!

\Level 1 {Shared pointer to `this'}

Inside an object method, the object is accessible as
\indexc{this}. This is a pointer in the classical sense. So what
if you want to refer to `this' but you need a shared pointer?

For instance, suppose you're writing a linked list code, and your
\lstinline{node} class has a method \lstinline{prepend_or_append} that gives a shared
pointer to the new head of the list.

\begin{slide}{Linked list code, old style}
  \label{sl:share-ptr-node}  
\begin{lstlisting}
node *node::prepend_or_append(node *other) {
  if (other->value>this->value) {
    this->tail = other;
    return this;
  } else {
    other->tail = this;
    return other;
  }
};
\end{lstlisting}
Can we do this with shared pointers?
\end{slide}

Your code would start something
like this, handling the case where the new node is appended to the current:
\begin{lstlisting}
shared_pointer<node> node::append
    ( shared_ptr<node> other ) {
  if (other->value>this->value) {
    this->tail = other;
\end{lstlisting}
But now you need to return this node, as a shared pointer. But
\lstinline{this} is a \lstinline{node*}, not a \lstinline{shared_ptr<node>}.

\begin{slide}{A problem with shared pointers}
  \label{sl:share-ptr-node-sh}
\begin{lstlisting}
shared_pointer<node> node::prepend_or_append
    ( shared_ptr<node> other ) {
  if (other->value>this->value) {
    this->tail = other;
\end{lstlisting}
So far so good. However, \lstinline{this} is a \lstinline{node*}, not a
\lstinline{shared_ptr<node>}, so
\begin{lstlisting}
    return this;
\end{lstlisting}
returns the wrong type.
\end{slide}

The solution here is that you can return
\begin{lstlisting}
    return this->shared_from_this();
\end{lstlisting}
if you have defined your node class to inherit from what probably
looks like magic:
\begin{lstlisting}
class node : public enable_shared_from_this<node>
\end{lstlisting}

Note that you can only return a \lstinline+shared_from_this+
if already a valid shared pointer to that object exists.

\begin{slide}{Solution: shared from this}
  \label{sl:share-ptr-node-from}
  It is possible to have a `shared pointer to this' if you
  define your node class with (warning, major magic alert):
\begin{lstlisting}
class node : public enable_shared_from_this<node> {
\end{lstlisting}
This allows you to write:
\begin{lstlisting}
    return this->shared_from_this();
\end{lstlisting}
\end{slide}

\Level 1 {Weak pointers}
\label{sec:weak_ptr}

In addition to shared and unique pointers, which own an object, there
is also \indexcdef{weak_ptr} which creates a
\indextermsubdef{weak}{pointer}. This pointer type does not own, but
at least it knows when it dangles.

\begin{lstlisting}
weak_ptr<R> wp;
shared_ptr<R> sp( new R );
sp->call_member();
wp = sp;
// access through new shared pointer:
auto p = wp.lock();
if (p) {
  p->call_member();
}
if (!wp.expired()) { // not thread-safe!
  wp.lock()->call_member();
}
\end{lstlisting}

There is a subtlety with weak pointers and shared pointers. The call
\begin{lstlisting}
auto sp = shared_ptr<Obj>( new Obj );
\end{lstlisting}
creates first the object, then the `control block' that counts owners.
On the other hand,
\begin{lstlisting}
auto sp = make_shared<Obj>();
\end{lstlisting}
does a single allocation for object and control block. However, if you
have a weak pointer to such a single object, it is possible that the
object is destructed, but not de-allocated. On the other hand, using
\begin{lstlisting}
auto sp  = shared_ptr<Obj>( new Obj );
\end{lstlisting}
creates the control block separately, so the pointed object can be
destructed and de-allocated. Having a weak pointer to it means that only
the control block needs to stick around.

\Level 1 {Null pointer}

In C there was a macro \indexterm{NULL} that, only by convention, was
used to indicate
\emph{null pointers}\index{pointer!null}:
pointers that do not point to anything.
C++ has the \indexc{nullptr}, which is an object of type
\lstinline{std::}\indexc{nullptr_t}.

There are some scenarios where this is useful, for instance, with
polymorphic functions:
\begin{lstlisting}
void f(int);
void f(int*);
\end{lstlisting}
Calling \lstinline{f(ptr)} where the point is \lstinline{NULL}, the first function is
called, whereas with \lstinline{nullptr} the second is called.

\begin{slide}{Null pointer}
  \label{sl:cpp-nullptr}
  C++ has the \indexc{nullptr}, which is an object of type
  \lstinline{std::}\indexc{nullptr_t}.

\begin{lstlisting}
void f(int);
void f(int*);
  f(NULL);    // calls the int version
  f(nullptr); // calls the ptr version
\end{lstlisting}
Note: dereferencing is undefined behavior; does not throw an exception.
\end{slide}

Unfortunately, \indextermttbus{dereferencing a}{nullptr} does not give an exception.

\Level 1 {Opaque pointer}

The need for \emph{opaque pointers}\index{pointer!opaque!in C++}
\lstinline{void*}
is a lot less in C++ than it was in~C. For
instance, contexts can often be modeled with captures in closures
(chapter~\ref{ch:lambda}). If you really need a pointer that does not
\textit{a~priori} knows what it points to, use \indexcstd{any},
which is usually smart enough to call destructors when needed.

\snippetwithoutput{opaqueany}{pointer}{any}

\begin{slide}{Opaque pointer}
  \label{sl:void-ptr}
  Use \lstinline{std::any} instead of void pointers.

  \snippetwithoutput{opaqueany}{pointer}{any}
\end{slide}

\Level 1 {Pointers to non-objects}

In the introduction to this chapter we argued that many of the uses
for pointers that existed in~C have gone away in C++, and the main one
left is the case where multiple objects share `ownership' of some
other object.

You can still make shared pointers to scalar data, for instance to an
array of scalars. You then get the advantage of the memory management,
but you do not get the \lstinline{size} function and such that you would have
if you'd used a \lstinline{vector} object.

Here is an example of a pointer to a solitary double:
%
\snippetwithoutput{ptrdouble}{pointer}{ptrdouble}
%
It is possible to initialize that double:
%
\snippetwithoutput[ptrdouble]{ptrdoubleinit}{pointer}{ptrdoubleinit}

\Level 0 {Smart pointers vs C pointers}

We remark that there is less need for pointers in C++
than there was in~C.

\begin{itemize}
\item To pass an argument
  \emph{by reference}\index{parameter!passing!by reference},
  use a \emph{reference}\index{reference!argument}.
  Section~\ref{sec:passing}.
\item Strings are done through \lstinline{std::string}, not character arrays;
  see~\ref{ch:string}.
\item Arrays can largely be done through \lstinline{std::vector}, rather than
  \indexc{malloc}; see~\ref{ch:array}.
\item Traversing arrays and vectors can be done with ranges;
  section~\ref{sec:arrayrange}.
\item Anything that obeys a scope should be created through a
  \indexterm{constructor}, rather than using \indexc{malloc}.
\end{itemize}

Legitimate needs:
\begin{itemize}
\item Linked lists and \acp{DAG}; see the example in section~\ref{sec:linklist}.
\item Objects on the heap.
\item Use \indexc{nullptr} as a signal.
\end{itemize}

\endinput

OLD TEXT FOR LINKED LISTS:

\begin{block}{Linked lists}
  \label{sl:linkedlistimpl}
  The prototypical example use of pointers is in linked lists. 
  Consider a class \lstinline{Node} with
  \begin{itemize}
  \item a data value to store, and
  \item a pointer to another \lstinline{Node}, or \lstinline{nullptr} if none.
  \end{itemize}

  %
  \begin{multicols}{2}
    Constructor sets the data value:
    \verbatimsnippet{simplenodeconstruct}
    \columnbreak
    Set next / test if there is a next:
    \verbatimsnippet{simplenodetest}
    \vfill\hbox{}
  \end{multicols}
\end{block}

\begin{block}{List usage}
\label{sl:linkedlistuse}
  Example use:
  %
  \answerwithoutput{linksimple1}{tree}{simple}
\end{block}

\begin{block}{Linked lists and recursion}
  \label{sl:linkedlistrecursive}
  Many operations on linked lists can be done recursively:
  %
  \verbatimsnippet{linklengthrecursive}
\end{block}

\begin{exercise}
  \label{ex:linkedsettail}
  Write a method \lstinline+set_tail+ that sets the tail of a node.
\begin{lstlisting}
Node one;
one.set_tail( two ); // what is the type of `two'?
cout << one.list_length() << endl; // prints 2
\end{lstlisting}
\end{exercise}

\begin{exercise}
  \label{ex:linkedlist1}
  Write a recursive \lstinline{append} method that appends a node to the end
  of a list:
  %
  \answerwithoutput{linkappend}{tree}{append}
\end{exercise}

\begin{exercise}
  \label{ex:linkedlist2}
  Write a recursive \lstinline{insert} method that inserts a node in a list, such that
  the list stays sorted:
  %
  \answerwithoutput{linkinsert}{tree}{insert}

  Assume that the new node always comes somewhere after the head node.
\end{exercise}

\begin{exercise}
  \label{ex:linkedlist3}
  For a more sophisticated approach to linked lists, do the exercises
  in section~\ref{sec:linklist}.
\end{exercise}

\begin{exercise}
  If you are doing the prime numbers project (chapter~\ref{ch:prime})
  you can now do exercise~\ref{sec:streamsieve}.
\end{exercise}

