% -*- latex -*-
%%%%%%%%%%%%%%%%%%%%%%%%%%%%%%%%%%%%%%%%%%%%%%%%%%%%%%%%%%%%%%%%
%%%%
%%%% This TeX file is part of the course
%%%% Introduction to Scientific Programming in C++/Fortran2003
%%%% copyright 2023 Victor Eijkhout eijkhout@tacc.utexas.edu
%%%%
%%%% turtles.tex : exercises for the Great Garbage Patch
%%%%
%%%%%%%%%%%%%%%%%%%%%%%%%%%%%%%%%%%%%%%%%%%%%%%%%%%%%%%%%%%%%%%%

This section contains a sequence of exercises that builds up to a
\indexterm{cellular automaton} simulation
of turtles in the ocean, garbage that is deadly to them,
and ships that clean it up.
To read more about this: \url{https://theoceancleanup.com/}.

Thanks to Ernesto Lima of TACC for the idea and initial code
of this exercise.

\Level 0 {Problem and model solution}

There is lots of plastic floating around in the ocean,
and that is harmful to fish and turtles and cetaceans.
Here you get to model the interaction between
\begin{itemize}
\item Plastic, randomly located;
\item Turtles that swim about; they breed slowly, and they die from
  ingesting plastic;
\item Ships that sweep the ocean to remove plastic debris.
\end{itemize}

The simulation method we use is that of a \indextermdef{cellular automaton}:
\begin{itemize}
\item We have a grid of cells;
\item Each cell has a `state' associated with it, namely,
  it can contain a ship, a turtle, or plastic, or be empty; and
\item On each next time step, the state of a cell is a simple
  function of the states of that cell and its immediate neighbors.
\end{itemize}

The purpose of this exercise is to simulate a number of time steps,
and explore the iteraction between parameters: with how much gargage will turtles
die out, how many ships are enough to protect the turtles.

\Level 0 {Program design}

The basic idea is to have an \lstinline{ocean} object that
is populated with turtles, trash, and ships.
Your simulation will let the ocean undergo a number of time steps:
\begin{lstlisting}
for (int t=0; t<time_steps; t++)
  ocean.update();
\end{lstlisting}
Ultimately your purpose is to investigate the development of the turtle population:
is it stable, does it die out?

While you can make a `hackish' solution to this problem,
partly you will be judged on your use of modern/clean C++ programming techniques.
A~number of suggestions are made below.

\Level 1 {Grid update}

Here is a point to be aware of.
Can you see what's wrong with with doing an update entirely in-place:
\begin{lstlisting}
for ( i )
  for ( j )
    cell(i,j) = f( cell(i,j), .... other cells ... );
\end{lstlisting}
?

\Level 0 {Testing}

It can be complicated to test this program for correctness.
The best you can do is to try out a number of scenarios.
For that it's best to make your program input flexible:
use the \n{cxxopts} package~\cite{cxxopts},
and drive your program from a shell script.

Here is a list of things you can test.
\begin{enumerate}
\item Start with only a number of ships;
  check that after 1000 time steps you still have the same number.
\item Likewise with turtles;
  if they don't breed and don't die, check that their number stays constant.
\item With only ships and trash, does it all get swept?
\item With only turtles and trash, do they all die off?
\end{enumerate}

It is harder to test that your turtles and ships don't `teleport' around,
but only move to contiguous cells. For that, use visual inspection;
see section~\ref{sec:vt100-animation}.

\begin{figure}[p]
  \footnotesize
\makeatletter
\def\verbatim@startline{\verbatim@line{\leavevmode\kern-25pt\relax}}
\makeatother
\begin{verbatim}
    0  1  2  3  4  5  6  7  8  9  0  1  2  3  4  5  6  7  8  9  0  1  2  3  4  5  6  7  8  9  0  1  2  3  4
0      o
1                           o                    o     o                       |        o
2                                                |
3         o              o        o                             o
4                                                               x              o                 o
5                                                                                       x
6                                                                        o                             |
7      o                    x                                o                    x
8                                       o        o                 o                                x
9                                    o                                                     o
0      o
1                                                            o
2      o                                                        o
3      o                                               |        o
4            o                       o                       o
5                                                                        o
6                                       |
7                              o        o     o                                            x
8                           |     o                                                                       x
9      o                       o     o           |
0                        o                    o  o              |
1                                 o                          o                    o                    o
2         o           o
3                                                                                                   x
4            o                                x  x
5                                                   |                          x
6                                       o                          x                       x
7                                             x           x                                            x
8         |              o  o                                            o                 x
9
0         x                       o                 o                                x
1
2                                          |                                         x
3         o                                      o
4                           o  |           x
Turtles: 55
Trash  : 21
Ships  : 12
\end{verbatim}
  \caption{Ascii art printout of a time step}
  \label{fig:turtle-patch}
\end{figure}

\Level 1 {Animated graphics}
\label{sec:vt100-animation}

The output of this program is a prime candidate for visualization.
In fact, some test (`make sure that turtles don't teleport')
are hard to do other than by looking at the output.
Start by making an ascii rendering of the ocean grid,
as in figure~\ref{fig:turtle-patch}.

It would be better to have some sort of animated output.
However, not all programming languages generate visual
output equally easily.
\begin{python}
  Python has easy access to good graphics libraries.
\end{python}
\begin{cxx}
  There are very powerful video/graphics libraries in C++,
  but these are also hard to use.
  There is a simpler way out.
\end{cxx}

For simple output such as this program yields,
you can make a simple low-budget animation.
Every \indextermbus{terminal}{emulator} under the sun
supports \indextermbus{VT100}{cursor control}
\footnote{\url{https://vt100.net/docs/vt100-ug/chapter3.html}}:
you can send certain magic output to your screen to
control cursor positioning.

In each time step you would
\begin{enumerate}
\item Send the cursor to the home position, by this magic output:
  \verbatimsnippet{vt100home}
\item Display your grid as in figure~\ref{fig:turtle-patch};
\item Sleep for a fraction of a second;
  see section~\textbookref{sec:this-thread}.
\end{enumerate}

\Level 0 {Modern programming techniques}

\Level 1 {Object oriented programming}

While there is only have one ocean, you should still
make an \lstinline{ocean} class, rather than
having a global array object.
All functions are then methods of the single object you create of that class.

\Level 1 {Data structure}
\label{sec:turtle-grid}

We lose very little generality by ignoring the depth of the ocean
and the shape of coastlines,
and model the ocean as a 2D grid.

If you write an indexing function \lstinline{cell(i,j)}
you can make your code largely independent of the actual data structure chosen.
Argue why \lstinline+vector<vector<int>>+ is not the best storage.
\begin{enumerate}
\item What do you use instead?
\item When you have a working code, can you show by timing that
  your choice is indeed superior?
\end{enumerate}

\Level 1 {Cell types}
\label{sec:turtle-cell}

Having `magic numbers' in your code ($0=$empty, $1=$turtle, et cetera)
is not elegant.
Make a \lstinline{enum} or \lstinline{enum class}
(see section~\textbookref{sec:enum-class})
so that you have names
for what's in your cells:
\verbatimsnippet{turtlename}

If you want to print out your ocean, it might be nice
if you can directly \lstinline{cout} your cells:
\verbatimsnippet{oceanprint}


\Level 1 {Ranging over the ocean}

It is easy enough to write a loop as
\begin{lstlisting}
for (int i=0; i<iSize; i++)
  for (int j=0; j<jSize; j++)
    ... cell(i,j) ...
\end{lstlisting}
However, it may not be a good idea to always sweep
over your domain so orderly. Can you implement this:
\begin{lstlisting}
for ( auto [i,j] : permuted_indices() ) {
    ... cell(i,j) ...
\end{lstlisting}
? See section~\textbookref{sec:tuple} about \indexterm{structured binding}.


Likewise, if you need to count how many pieces of trash there are
around a turtle, can you get this code to work:
\verbatimsnippet{countnbrs}

\Level 1 {Random numbers}

For the random movement of ships and turtles you need
a random number generator.
Do not use the old C generator, but the new
\indextermheader{random} one;
section~\textbookref{sec:stl:random}.

Try to find a solution so that you use exactly one generator
for all places where you need random numbers.
Hint: make the generator \lstinline{static} in your class.


\Level 0 {Explorations}

Instead of having the ships move randomly, can you
give them a preferential direction to the closest
garbage patch?
Does this improve the health of the turtle population?

Can you account for the relative size of ships and turtles
by having a ship occupy a $2\times2$ block in your grid?

So far you have let trash stay in place.
What if there are ocean currents?
Can you make the trash `sticky' so that trash particles
start moving as a patch if they touch?

Turtles eat sardines. (No they don't.)
What happens to the sardine population if turtles die out?
Can you come up with parameter values that correspond to
a stable ecology or a de-stabilized one?

\Level 1 {Code efficiency}

Investigate whether your implementation of the \lstinline{enum}
in section~\ref{sec:turtle-cell} has any effect on timing.
Parse the fine print of section~\textbookref{sec:enum-class}.


You may remark that ranging over a largely empty ocean
can be pretty inefficient.
You could contemplate keeping an `active list' of where the
turtles et cetera are located, and only looping over that.
How would you implement that?
Do you expect to see a difference in timing? Do you actually?

How is runtime affected by choosing a
vector-of-vectors implementation for the ocean;
see section~\ref{sec:turtle-grid}.
