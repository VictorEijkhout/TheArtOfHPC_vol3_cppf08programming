% -*- latex -*-
%%%%%%%%%%%%%%%%%%%%%%%%%%%%%%%%%%%%%%%%%%%%%%%%%%%%%%%%%%%%%%%%
%%%%
%%%% This TeX file is part of the course
%%%% Introduction to Scientific Programming in C++11/Fortran2003
%%%% copyright 2017-2020 Victor Eijkhout eijkhout@tacc.utexas.edu
%%%%
%%%% amazon-abstract.tex : abstract for the TSP project
%%%%
%%%%%%%%%%%%%%%%%%%%%%%%%%%%%%%%%%%%%%%%%%%%%%%%%%%%%%%%%%%%%%%%

Delivery trucks are all over town, getting as many packages to people
as quickly as possible. Is it possible to find the best route
through town? This is an important question for many companies,
such as Amazon. (The US Postal Service solves a different problem.)
In this simple form this is the \ac{TSP},
which has been studied extensively.
But Amazon has some liberties: packages can be spread out
over days (unless you have Amazon Prime), and there are multiple trucks
to divide the area over.

This project walks a student through a
heuristic solution of the \ac{TSP} and the `multiple TSP'.
With the full implementation, students can explore some scenarios.

The project description explicitly targets an object-oriented formulation;
this project is much harder in a non-OO language.

This project is preferably done by two students in close collaboration;
it will take about two weeks
at the end of a first or second semester programming course.

\newpage

\begin{tabular}{|l|p{5in}|}
  \hline
  Summary&Model the `Multiple Traveling Salesman Problem.'
  \\
  Topics&Recursion, arrays, classes
  \\
  Audience&Undergraduate or AP high school
  \\
  Difficulty&Moderate to high
  \\
  Strengths&Appealing, opportunity for experimentation
  \\
  Weaknesses&Requires many pieces to be in place before the code does something useful.
  \\
  &No graphics output, so the interpretation of output requires some imagination
  \\
  Dependencies&No specific background in CS (such as the \ac{TSP}) is needed.
  Familiarity with the notions of heuristics and search is welcome.
  \\
  Variants&Application slight different problem statements,
  such as HEB (big supermarket chain) curb-side delivery shoppers.
  \\
  \hline
\end{tabular}
