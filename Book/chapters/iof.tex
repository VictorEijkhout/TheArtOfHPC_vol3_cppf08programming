% -*- latex -*-
%%%%%%%%%%%%%%%%%%%%%%%%%%%%%%%%%%%%%%%%%%%%%%%%%%%%%%%%%%%%%%%%
%%%%
%%%% This TeX file is part of the course
%%%% Introduction to Scientific Programming in C++/Fortran2003
%%%% copyright 2017-2022 Victor Eijkhout eijkhout@tacc.utexas.edu
%%%%
%%%% iof.tex : input and output in Fortran
%%%%
%%%%%%%%%%%%%%%%%%%%%%%%%%%%%%%%%%%%%%%%%%%%%%%%%%%%%%%%%%%%%%%%

\Level 0 {Types of I/O}

Fortran can deal with input/output in ASCII format, which is called
\emph{formatted}\index{I/O!formatted} I/O, and binary, or
\emph{unformatted}\index{I/O!unformatted}~I/O.  Formatted I/O can use
default formatting, but you can specify detailed formatting
instructions, in the I/O statement itself, or separately in a
\indextermfort{Format} statement.

Fortran I/O can be described as \indextermsub{list-directed}{I/O}:
both input and output commands take a list of item, possibly with
formatting specified.

\begin{block}{I/O commands}
  \label{sl:fio-commands}
  \begin{itemize}
  \item \indextermfort{Print} simple output to terminal
  \item \indextermfort{Write} output to terminal or file (`unit')
  \item \indextermfort{Read} input from terminal or file
  \item \indextermfort{Open}, \indextermfort{Close} for files and
    streams
  \item \indextermfort{Format} format specification that can be used
    in multiple statements.
  \end{itemize}
\end{block}

\begin{block}{Formatted and unformatted I/O}
  \label{sl:fio-types}
  \begin{itemize}
  \item Formatted: ASCII output. This is good for reporting, but not
    for numeric data storage.
  \item Unformatted: binary output. Great for further processing of
    output data.
  \item Beware: binary data is machine-dependent. Use \indexterm{hdf5}
    for portable binary.
  \end{itemize}
\end{block}

\Level 0 {Print to terminal}

The simplest command for outputting data is \indextermtt{print}.
\begin{lstlisting}
print *,"The result is",result
\end{lstlisting}
In its easiest form, you use the star to indicate that you don't care
about formatting; after the comma you can then have any number of
comma-separated strings and variables.

\Level 1 {Print on one line}

The statement
\begin{lstlisting}
print *,item1,item2,item3
\end{lstlisting}
will print the items on one line, as far as the line length allows.

\begin{slide}{Simple print}
  \label{sl:fio-print}
  All on one line:
\begin{lstlisting}
print *,"The result is",result
print *,item1,item2,item3
\end{lstlisting}
\end{slide}

\begin{block}{Implicit do loops}
  \label{sl:print-implicit-loop}
  Parameterized printing with an \indextermsub{implicit}{do loop}:
\begin{lstlisting}
print *,( i*i,i=1,n)
\end{lstlisting}
All values will be printed on the same line.
\end{block}

\Level 1 {Printing arrays}

If you print a variable that is an array, all its elements will be
printed, in \indexterm{column-major} ordering if the array is
multi-dimensional.

You can also control the printing of an array by using an
\indextermsub{implicit}{do loop}:
\begin{lstlisting}
print *,( A(i),i=1,n)
\end{lstlisting}

\begin{slide}{Array printing}
  \label{sl:fprint-array}
  \begin{itemize}
  \item \lstinline{print *,A} prints whole array, column-major
  \item Implicit do loops:
\begin{lstlisting}
print *,( A(i,i),i=1,n)
\end{lstlisting}
  Can also be nested.
  \end{itemize}
\end{slide}

\Level 0 {Formatted I/O}
\label{sec:fformat}

The default formatting uses quite a few positions for what can be
small numbers. To indicate explicitly the formatting, for instance
limiting the number of positions used for a number, or the whole and
fractional part of a real number, you can use a format string.

For instance, you can use a letter followed by a digit to control the
formatting width:
%
\snippetwithoutput{i4}{iof}{i4}
%
(In the last line, the format specifier was not wide enough for the
data, so an
\emph{asterisk}\index{asterisk!in Fortran formatted I/O}
was used as output.)

Let's approach this semi-systematically.

\Level 1 {Format letters}

\Level 2 {Integers}

Integers can be set with \lstinline{i}$n$.
\begin{itemize}
\item If $n>0$, that many positions are used
  with the number right aligned; except
\item if the number does not fit in $n$ positions,
  it is rendered as asterisks.
\item To use precisely the required number of positions,
  use \lstinline{i0}.
\end{itemize}

\begin{block}{Integer formatting}
  \label{sl:f-format-i}
  \snippetwithoutput{i4}{iof}{i4}
\end{block}

\Level 2 {Strings}

Strings can be handled two ways: 
\begin{enumerate}
\item They can be literally part of the format string:
\begin{lstlisting}
print '(i2,"--",i2)', m,n
\end{lstlisting}
\item The can be formatted with the \lstinline{a}$n$ specifier:
\begin{lstlisting}
print '(a5,2)',somestring,someint
\end{lstlisting}
\item To use precisely the required number of positions,
  use \lstinline{a}
\begin{lstlisting}
print '(a,i0,a)', str1,int2,str3
\end{lstlisting}
\end{enumerate}

\begin{slide}{String formatting}
  \label{sl:f-format-f}
  String in the format spec:
\begin{lstlisting}
print '(i2,"--",i2)', m,n
\end{lstlisting}
Explicit \lstinline{a} specifier:
\begin{lstlisting}
print '(a5,i2)',somestring,someint
\end{lstlisting}
Use only the required space:
\begin{lstlisting}
print '(a,i0,a)', str1,int2,str3
\end{lstlisting}
\end{slide}

\Level 2 {Floating point}

\begin{itemize}
%% \item `\lstinline{a}$n$' specifies a string of $n$~characters. If the actual
%%   string is longer, it is truncated in the output.
%% \item `\lstinline{i}$n$' specifies an integer of up to~$n$ digits. If the actual
%%   number takes more digits, it is rendered with asterisks.
%%   To use the minimum number of digits, use~\lstinline{i}$0$.
\item `\lstinline{f}$m.n$' specifies a fixed point representation of a real
  number, with $m$~total positions (including the decimal point)
  and $n$~digits in the fractional part.
\item `\lstinline{e}$m.n$' Exponent representation.
%% \item Strings can go into the format:
%% \begin{lstlisting}
%% print '("Result:",3f5.3)',x,y,z
%% \end{lstlisting}
%% \item `\lstinline{x}' for a space, `\lstinline{/}'~for newline
\end{itemize}

\Level 2 {Other}

\begin{verbatim}
x : one space
x5 : five spaces

b : binary
o : octal
z : hex
\end{verbatim}

\Level 1 {Repeating and grouping}

If you want to display items of the same type, you can use a repeat
count:
%
\snippetwithoutput{ii}{iof}{ii}

You can mix variables of different types, as well as literal string,
by separating them with commas. And you can group them with
parentheses, and put a repeat count on those groups again:
%
\snippetwithoutput{ij}{iof}{ij}

\begin{slide}{Repeated formats}
  \label{sl:f-format-rep}
  \snippetwithoutput{ii}{iof}{ii}
  \snippetwithoutput{ij}{iof}{ij}
\end{slide}

Putting a number in front of a single specifier indicates that it is
to be repeated.

If the data takes more positions than the format specifier allows, a
string of asterisks is printed:
%
\snippetwithoutput{fasterisk}{fio}{asterisk}

If you find yourself using the same format a number of times, you can
give it a \indexterm{label}:
\begin{lstlisting}
  print 10,"result:",x,y,z
10 format('(a6,3f5.3)')
\end{lstlisting}

\url{https://www.obliquity.com/computer/fortran/format.html}

\begin{slide}{Formats}
  \label{sl:formats}
  \begin{itemize}
  \item
    Fine control of input/output.
  \item
    Direct use in print statement:  
\begin{lstlisting}
print '(a6,3f5.3)',"Result",x,y,z
print '("Result:",3f5.3)',x,y,z
\end{lstlisting}
\item Format statement:
  \end{itemize}
\begin{lstlisting}
  print 10,"result:",x,y,z
10 format('(a6,3f5.3)')
\end{lstlisting}
\end{slide}

\begin{slide}{Format specifiers}
  \label{sl:formatspec}
  \begin{itemize}
  \item `\lstinline{a}$n$' specifies a string of $n$~characters. If the actual
    string is longer, it is truncated in the output.
  \item `\lstinline{i}$n$' specifies an integer of up to~$n$ digits. If the actual
    number takes more digits, it is rendered with asterisks.
  \item `\lstinline{f}$m.n$' specifies a fixed point representation of a real
    number, with $m$~total positions (including the decimal point)
    and $n$~digits in the fractional part.
  \item `\lstinline{e}$m.n$' Exponent representation.
  \item Strings can go into the format:
\begin{lstlisting}
print '("Result:",3f5.3)',x,y,z
\end{lstlisting}
\item `\lstinline{x}' for a space, `\lstinline{/}'~for newline
  \end{itemize}
\end{slide}

\begin{block}{Format repetitions}
  \label{sl:fformat-rep}
\begin{lstlisting}
print '( 3i4 )', i1,i2,i3
print '( 3(i2,":",f7.4) )', i1,r1,i2,r2,i3,r2
\end{lstlisting}
\end{block}

\begin{block}{Repeats and line breaks}
  \label{sl:formatrepeat}
  \begin{itemize}
  \item If \lstinline{abc} is a format string, then \lstinline{10(abc)} gives 10
    repetitions. There is no line break.
  \item If there is more data than specified in the format, the format
    is reused in a new print statement. This causes line breaks.
  \item The \lstinline{/} (slash) specifier causes a line break.
  \item There may be a 80 character limit on output lines.
  \end{itemize}
\end{block}

\begin{exercise}
  \label{ex:f99}
  Use formatted I/O to print the number $0\cdots99$ as follows:
\begin{lstlisting}
 0  1  2  3  4  5  6  7  8  9
10 11 12 13 14 15 16 17 18 19
20 21 22 23 24 25 26 27 28 29
30 31 32 33 34 35 36 37 38 39
40 41 42 43 44 45 46 47 48 49
50 51 52 53 54 55 56 57 58 59
60 61 62 63 64 65 66 67 68 69
70 71 72 73 74 75 76 77 78 79
80 81 82 83 84 85 86 87 88 89
90 91 92 93 94 95 96 97 98 99
\end{lstlisting}
\end{exercise}

\Level 0 {File and stream I/O}

If you want to send output anywhere else than the terminal screen, you
need the \indextermfort{write} statement, which looks like:
\begin{lstlisting}
write (unit,format) data
\end{lstlisting}
where \indextermfort{format} and \indextermfort{data} are as described above. The new element
is the \indexterm{unit}, which is a numerical indication of an output
device, such as a file.

\Level 1 {Units}

For file I/O you write to a unit number,
which is associated with a file through an \indextermfort{open}
statement.

After you are done with the file, you \indextermfort{Close} it.

\begin{lstlisting}
Open(11)
\end{lstlisting}
will result in a file with a name typically \n{fort11}.
To give it a name of your choosing:
\begin{lstlisting}
Open(11,FILE="filename")
\end{lstlisting}
Many other options for error handling, new vs old file, etc.

After this, a \indextermfort{Write} statement can refer to the unit:
\begin{lstlisting}
Write (11,fmt) data
\end{lstlisting}
Again options for errors and such.

\Level 1 {Other write options}

By default, each \indexf{Write} statement,
like a \indexf{Print} statement,
writes to a new line (or `record' in Fortran terminology).
To prevent this,
\begin{lstlisting}
write(unit,fmt,ADVANCE="no") data
\end{lstlisting}
will not issue a newline.

\Level 0 {Conversion to/from string}
\label{sec:string-conversion}

Above, you saw
the commands \lstinline{Print}, \lstinline{Write},
and \lstinline{Read} in the context of output to/from
terminal and file. However, there is a second type of use
for \lstinline{Write} and \lstinline{Read} for string handling,
namely conversion between strings and numerical quantities.

\Level 2 {String to numeric}

To convert a string to numerical quantities,
  perform a \indexf{Read} operation:
\begin{lstlisting}
Read( some_string, some_format ) bunch, of, quantities
\end{lstlisting}
Example:
\begin{block}{Parse date string}
  \label{sl:f-date-string}
  \snippetwithoutput{fdatestr}{stringf}{date}
\end{block}

\Level 2 {Numeric to string}

Conversely, to construct a string from some quantities
  you would perform a \indexf{Write} operation:
\begin{lstlisting}
Write( some_string, some_format ) bunch, of, quantities
\end{lstlisting}

\begin{block}{Date quantities to string}
  \label{sl:f-date-slash}
  \snippetwithoutput{flongdate}{stringf}{slash}
\end{block}

\Level 0 {Unformatted output}
\label{sec:rawdataf}

So far we have looked at ASCII output, which is nice to look at for a
human , but is not the right medium to communicate data to another
program.
\begin{itemize}
\item ASCII output requires time-consuming conversion.
\item ASCII rendering leads to loss of precision.
\end{itemize}
Therefore, if you want to output data that is later to be read by a
program, it is best to use \indextermsub{binary}{output} or
\indextermsub{unformatted}{output}, sometimes also called
\indextermsub{raw}{output}.

\begin{block}{Unformatted output}
  Indicated by lack of format specification:
\begin{lstlisting}
write (*) data
\end{lstlisting}
  Note: may not be portable between machines.
\end{block}

\Level 0 {Print to printer}

In Fortran standards before \fstandard{2003},
  column~1 of the output had a special meaning,
  corresponding to \indexterm{line printer} tractor control.
  Notoriously, having a character here would
  move to a new page.
  While this feature has been removed from the standard,
  you may still see a black first column in your output,
  without specifying such.


