% -*- latex -*-
%%%%%%%%%%%%%%%%%%%%%%%%%%%%%%%%%%%%%%%%%%%%%%%%%%%%%%%%%%%%%%%%
%%%%
%%%% This TeX file is part of the course
%%%% Introduction to Scientific Programming in C++/Fortran2003
%%%% copyright 2017-9 Victor Eijkhout eijkhout@tacc.utexas.edu
%%%%
%%%% scope.tex : functions and such in Fortran
%%%%
%%%%%%%%%%%%%%%%%%%%%%%%%%%%%%%%%%%%%%%%%%%%%%%%%%%%%%%%%%%%%%%%

\Level 0 {Scope}

Fortran `has no curly brackets': you not easily create nested
scopes with local variables as in~C++.
For instance, the range between \indexf{do} and \indexf{end do} is not a scope.
This means that all variables have to be declared at the top of a
program or subprogram.

\Level 1 {Variables local to a program unit}

Variables declared in a subprogram have similar scope rules as in~C++:
\begin{itemize}
\item Their visibility is controlled by their textual scope:
\begin{lstlisting}
Subroutine Foo()
  integer :: i
  ! `i' can now be used
  call Bar()
  ! `i' still exists
End Subroutine Foo
Subroutine Bar() ! no parameters
  ! The `i' of Foo is unknown here
End Subroutine Bar
\end{lstlisting}
\item Their dynamic scope is the lifetime of the program unit in which
  they are declared:
\begin{lstlisting}
Subroutine Foo()
  call Bar()
  call Bar()
End Subroutine Foo
Subroutine Bar()
  Integer :: i
  ! `i' is created every time Bar is called
End Subroutine Bar
\end{lstlisting}
(That last example has a little subtlety; see section~\ref{sec:f-save-var}
for the \lstinline{Save} attribute on procedure variables.)
\end{itemize}

\Level 2 {Variables in a module}

Variables in a module (section~\ref{sec:modulef}) have a lifetime that
is independent of the calling hierarchy of program units: they are
\emph{static variables}\index{variable!static}.

\Level 2 {Other mechanisms for making static variables}

Before Fortran gained the facility for recursive functions, the data
of each function was placed in a statically determined location. This
meant that the second time you call a function, all variables still
have the value that they had last time. To force this behavior
in modern Fortran, you can add the \indexf{Save} specification
to a variable declaration.

Another mechanism for creating static data was the
\indexf{Common} block. This should not be used, since a
\indexf{Module} is a more elegant solution to the same problem.

\Level 1 {Variables in an internal procedure}

An \indextermsub{internal}{procedure} (that is, one placed in the
\indexf{Contains} part of a program unit) can receive arguments
from the containing program unit. It can also access directly any
variable declared in the containing program unit, through a process
called \indextermdef{host association}.

The rules for this are messy, especially when considering implicit
declaration of variables, so we advise against relying on it.
