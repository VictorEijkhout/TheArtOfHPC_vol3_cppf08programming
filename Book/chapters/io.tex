% -*- latex -*-
%%%%%%%%%%%%%%%%%%%%%%%%%%%%%%%%%%%%%%%%%%%%%%%%%%%%%%%%%%%%%%%%
%%%%
%%%% This TeX file is part of the course
%%%% Introduction to Scientific Programming in C++/Fortran2003
%%%% copyright 2017-2022 Victor Eijkhout eijkhout@tacc.utexas.edu
%%%%
%%%% io.tex : input and output
%%%%
%%%%%%%%%%%%%%%%%%%%%%%%%%%%%%%%%%%%%%%%%%%%%%%%%%%%%%%%%%%%%%%%

Most programs take some form of input, and produce some form of output.
In a beginning course such as this, the output is of more importance than
the input,
and what output there is only goes to the screen,
so that's what we start by focusing on.
We will also look at output to file, and input.

\Level 0 {Screen output}
\label{sec:iomanip}

In examples so far, you have used \lstinline{cout}
with its default formatting.
In this section we look at ways of customizing
the \lstinline{cout} output.
\begin{remark}
  Even after the material below you may find \lstinline{cout}
  not particularly elegant.
  In fact, if you've programmed in~C before,
  you may prefer the \indexc{printf} mechanism.
  The \cppstandard{20} standard
  has the \indextermheader{format} header, which
  is as powerful as \lstinline{printf}, but considerably more elegant.

  However, as of early 2022 this is not available in most compilers,
  so in section~\ref{sec:fmtlib}
  we will give examples from \indextermdef{fmtlib},
  the open source library that gave rise to \indexcstd{format}.
\end{remark}

\begin{block}{Formatted output}
  \label{sl:cformat}
  From \indextermheader{iostream}:
  \lstinline{cout} uses default formatting.

  Possible manipulation in \indextermheader{iomanip} header:
  pad a number, use limited precision, format as hex, etc.
\end{block}

\begin{slide}{Default unformatted output}
  \label{sl:unformat}
  \snippetwithoutput{cunformat}{io}{cunformat}
\end{slide}

Normally, output of numbers takes up precisely the space that it needs:
\snippetwithoutput{cunformat}{io}{cunformat}

We will now look at some examples of non-standard formatting.
You may for instance want to output several lines, with the numbers
in them nicely aligned.
The most common I/O manipulation is to set a uniform width,
that is, use the same number of positions for each number,
regardless how many they need.
\begin{itemize}
\item The \indexc{setw} specifies how many positions to use
  for the following number.
\item Note the singular in the previous sentence: the \indexc{setw}
  specifier applies only once.
\item By default, numbers are right-aligned in the space
  given for them, and if they require more positions,
  they overflow on the right.
\end{itemize}

\begin{block}{Reserve space}
  \label{sl:io-setw}
  You can specify the number of positions, and the output is right
  aligned in that space by default:
  \snippetwithoutput{formatwidth6}{io}{width}
\end{block}

\begin{block}{Padding character}
  \label{sl:io-fill}
  Normally, padding is done with spaces, but you can specify other characters:
  \snippetwithoutput{formatpad}{io}{formatpad}
Note: single quotes denote characters, double quotes denote strings.
\end{block}

\begin{block}{Left alignment}
  \label{sl:io-left}
  Instead of right alignment you can do left:
  \snippetwithoutput{formatleft}{io}{formatleft}
\end{block}

\begin{block}{Number base}
  \label{sl:io-base}
  \advance\textwidth -2in 

  Finally, you can print in different number bases than~10:

  \def\snippetcodefraction{.4}
  \def\snippetanswfraction{.6}
  \def\codesize{\ttfamily\tiny}
  \def\verbsize{\ttfamily\tiny}
  \snippetwithoutput{format16}{io}{format16}

\end{block}

\begin{exercise}
  \label{ex:leadzero}
  Make the first line in the above output align better with the other lines:
\begin{verbatim}
00 01 02 03 04 05 06 07 08 09 0a 0b 0c 0d 0e 0f 
10 11 12 13 14 15 16 17 18 19 1a 1b 1c 1d 1e 1f 
20 21 22 23 24 25 26 27 28 29 2a 2b 2c 2d 2e 2f 
etc
\end{verbatim}
\end{exercise}

\begin{block}{Hexadecimal}
  \label{sl:io-hex}
  Hex output is useful for addresses (chapter~\ref{sec:cderef}):
  %
  \snippetwithoutput{coutpoint}{pointer}{coutpoint}
  %
  Back to decimal:
\begin{lstlisting}
cout << hex << i << dec << j;
\end{lstlisting}
\end{block}

There is no standard modifier for outputting as binary.
However, you can use the \indextermheader{bitset} header
to print the bit pattern of an integer.

\begin{block}{Bitsets for ints}
  \snippetwithoutput{ibitset}{io}{bits}
\end{block}

\Level 1 {Floating point output}

The output of floating point numbers is more tricky.
\begin{itemize}
\item How many positions are used for the digits before the decimal point?
\item How many digits after the decimal point are printed?
\item Is scientific notation used?
\end{itemize}

For floating point numbers, the \indexc{setprecision} modifier
determines how many positions are used for the integral and fractional part
together. If the integral part takes more positions, scientific notation is used.

\begin{block}{Floating point precision}
  \label{sl:io-float}
  Use \n{setprecision} to set the number of digits before and after
  decimal point:
  %
  \snippetwithoutput{formatfloat}{io}{formatfloat}
  %
  This mode is a mix of fixed and floating point.
  See the \lstinline{scientific} option below
  for consistent use of floating point format.
\end{block}

With the \indexc{fixed} modifier, \indexc{setprecision}
applies to the fractional part.

\begin{block}{Fixed point precision}
  \label{sl:io-fix}
  Fixed precision applies to fractional part:
  %
  \snippetwithoutput{fixfrac}{io}{fix}
  %
  (Notice the rounding)
\end{block}

The \indexc{setw} modifier, for fixed point output,
applies to the total width of integral and fractional part,
plus the decimal point.

\begin{block}{Aligned fixed point output}
  \label{sl:io-align}
  Combine width and precision:
  %
  \snippetwithoutput{align}{io}{align}
\end{block}

\begin{exercise}
  \label{ex:fixedpout}
  Use integer output to print real numbers aligned on the
  decimal:
  %
  \answerwithoutput{qfcall}{io}{quasifix}
  %
  Use four spaces for both the integer and fractional part; test only
  with numbers that fit this format.
\end{exercise}

Above you saw that \indexc{setprecision} may give both fixed and floating point
output. To get strictly floating point `scientific' notation output,
use \indexc{scientific}.

\begin{block}{Scientific notation}
  \label{sl:io-sci}
  Combining width and precision:
  
  \snippetwithoutput{precisionwidth}{io}{iofsci}
\end{block}

\begin{comment}
\begin{verbatim}
cout << "Combine width and precision:" << "\n";
x = 1.234567;
cout << scientific;
for (int i=0; i<10; i++) {
  cout << setw(10) << setprecision(4) << x << "\n";
  x *= 10;
}
\end{verbatim}
\begin{block}{Output}
  \label{sl:io-sci-out}
\begin{verbatim}
Combine width and precision:
1.2346e+00
1.2346e+01
1.2346e+02
1.2346e+03
1.2346e+04
1.2346e+05
1.2346e+06
1.2346e+07
1.2346e+08
1.2346e+09
\end{verbatim}
\end{block}
\end{comment}

\Level 1 {Saving and restoring settings}

\begin{verbatim}
ios::fmtflags old_settings = cout.flags();
\end{verbatim}

\begin{verbatim}
cout.flags(old_settings);
\end{verbatim}

\begin{verbatim}
int old_precision = cout.precision();
\end{verbatim}

\begin{verbatim}
cout.precision(old_precision);
\end{verbatim}

\Level 0 {File output}

The \lstinline{iostream} is just one example of a \indexterm{stream},
which is a general mechanism for converting entities to exportable form.

In particular, file output works the same as screen output:
after you create a stream variable, you can `lessless` to it.
\begin{lstlisting}
mystream << "x: " << x << '\n';
\end{lstlisting}

The following example uses an \indexc{ofstream}:
an output file stream.
This has an \indexc{open} method
to associate it with a file, and a corresponding \indexc{close} method.

\begin{block}{Text output to file}
  \label{sl:io-file}

  Use:
  \def\snippetcodefraction{.5}
  \def\snippetanswfraction{.5}
  \snippetwithoutput{fio}{io}{fio}

  Compare: \lstinline{cout} is a stream that has already been opened
  to your terminal `file'.
\end{block}

The \indexc{open} call can have flags, for instance for appending:
\begin{lstlisting}
file.open(name,std::fstream::out | std::fstream::app);
g\end{lstlisting}

\begin{block}{Binary output}
  \label{sl:io-bin}
  Binary output: write your data byte-by-byte from memory to file.\\
  (Why is that better than a printable representation?)
  
  \snippetwithoutput{fiobin}{io}{fiobin}
\end{block}

\Level 0 {Output your own classes}
\label{sec:lessless}

You have used statements like:
\begin{verbatim}
cout << "My value is: " << myvalue << "\n";
\end{verbatim}
How does this work? The `double less' is an operator with a left
operand that is a stream, and a right operand for which output is
defined; the result of this operator is again a stream. Recursively,
this means you can chain any number of applications of~\verb+<<+
together.

  If you want to output a class that you wrote yourself, you have to
  define how the \n{<<} operator deals with your class.

Here we solve this in two steps:

\begin{block}{String an object, 1}
  \label{sl:class-cout1}
 Define a function that yields a string representing the object, and 

  \lstset{style=snippetcode}
 \verbatimsnippet{pointfuncout1}
\end{block}

\begin{block}{String an object, 2}
  \label{sl:class-cout2}
  Redefine the less-less operator to use this.

  \lstset{style=snippetcode}
  \verbatimsnippet{pointfuncout2}
\end{block}

(See section~\ref{sec:stringstream} for \indexc{stringstream} and the
\indextermheader{sstream} header.)

%% \begin{block}{Redefine less-less}
%%   \label{sl:lessless-def}
%%   \verbatimsnippet{classostream}
%% \end{block}

If you don't want to write that accessor function, you can declare
the lessless operator as a \indexc{friend}:
\begin{lstlisting}
class container {
private: double x;
public:
  friend ostream& operator<<( ostream& s,const container& c ) {
    s << c.x; /* no accessor */
    return s; };
};
\end{lstlisting}

\Level 0 {Output buffering}
\label{sec:to-endl-or-not}

In C, the way to get a newline in your output was to include the
character~\verb+\n+ in the output. This still works in~C++, and at
first it seems there is no difference with using \indexc{endl}. However,
\indexc{endl} does more than breaking the output line: it
performs a \indexcstd{flush}.

\Level 1 {The need for flushing}

Output is usually not immediately written to screen or disc or
printer: it is saved up in buffers. This can be for efficiency,
because output a single character may have a large overhead, or it may
be because the device is busy doing something else, and you don't want
your program to hang waiting for the device to free up.

However, a problem with buffering is the output on the screen may lag
behind the actual state of the program. In particular, if your program
crashes before it prints a certain message, does it mean that it
crashed before it got to that line, or does it mean that the message
is hanging in a buffer.

This sort of output, that absolutely needs to be handled when the
statement is called, is often called \indexterm{logging} output.
The fact that \lstinline{endl} does a flush would mean that it would be good
for logging output. However, it also flushes when not strictly
necessary. In fact there is a better solution:
\indexcstd{cerr} works just like \lstinline{cout}, except it
doesn't buffer the output.

\Level 1 {Performance considerations}

If you want a newline in your output (whether screen or more general stream),
using \lstinline+endl+ may slow down your program because of the flush
it performs.
More efficiently, you would add a newline character to the output directly:
\begin{lstlisting}
somestream << "Value: " << x << '\n';
otherstream << "Total " << nerrors << " reported\n";
\end{lstlisting}
In other words, use \lstinline{cout} for regular output, \lstinline{cerr} for logging
output, and use \verb+\n+ instead of \lstinline{endl}.

\Level 0 {Input}
\label{sec:termin}

The \indexc{cin} command can be used to read integers
and floating point formats.
%
\snippetwithoutput{cinfloat}{io}{cinfloat}
%
As is illustrated with the last number in this example,
\lstinline{cin} will read until the first character that does not
fit the format of the variable, in this case the second period.
On the other hand, the \n{e} in the number before it is 
interpreted as the exponent of a floating point representation.

\begin{block}{Better terminal input}
  \label{sl:getline}
  It is better to use \indexc{getline}. This returns a string,
  rather than a value, so you need to convert it with the following bit
  of magic:
  %
  \verbatimsnippet{readin}

  You can not use \lstinline{cin} and \lstinline{getline} in the same program.

  More info:
  \url{http://www.cplusplus.com/forum/articles/6046/}.

\end{block}

\Level 1 {File input}

\begin{block}{File input streams}
  \label{sl:filein}
  Input file stream, method \indexc{open}, then use
  \indexc{getline} to read one line at a time:
  %
  \verbatimsnippet{filein}
\end{block}

\begin{block}{End of file test}
  There are several ways of testing for the end of a file
  \begin{itemize}
  \item For text files, the \indexc{getline} function returns
    \lstinline{false} if no line can be read.
  \item The \indexc{eof} function can be used after you have done
    a read.
  \item \indexcdef{EOF} is a return code of some library
    functions; it is not true that a file ends with an \n{EOT}
    character.
    Likewise you can not assume a
    \n{Control-D} or \n{Control-Z} at the end of the file.
  \end{itemize}
\end{block}

\begin{exercise}
  \label{ex:foxcount}
  Put the following text in a file:
\begin{verbatim}
the quick brown fox
jummps over the
lazy dog.
\end{verbatim}
  Open the file, read it in, and count how often each letter in the
  alphabet occurs in it
\end{exercise}

\begin{advanced}
  You may think that \lstinline{getline} always returns a \lstinline{bool}, but that's
  not true. If actually returns an \indexc{ifstream}. However, a conversion operator
\begin{verbatim}
explicit operator bool() const;
\end{verbatim}
  exists for anything that inherits from \indexc{basic_ios}.
\end{advanced}

\Level 1 {Input streams}

Tests, mostly for file streams: \indexc{is_eof} \indexc{is_open}

\Level 1 {C-style file handling}

The old \indexc{FILE} type should not be used anymore.

\Level 0 {Fmtlib}
\label{sec:fmtlib}

\Level 1 {basics}

\begin{block}{Fmtlib basics}
  \label{sl:fmtlib-basic}
  \begin{itemize}
  \item \lstinline{print} for printing,\\
    \lstinline{format} gives \lstinline+std::string+;
  \item Arguments indicated by curly braces;
  \item braces can contain numbers (and modifiers, see next)
  \end{itemize}
  \snippetwithoutput{fmtbasic}{io}{fmtbasic}
  API documentation:
  \url{https://fmt.dev/latest/api.html}
\end{block}

\Level 1 {Align and padding}

In \n{fmtlib}, the `greater than' sign plus a number indicates
right aligning and the width of the field.

\begin{block}{Right aligned in fmtlib}
  \label{sl:fmtwidth}
  \snippetwithoutput{fmtwidth}{io}{fmtwidth}
\end{block}

\begin{block}{Padding characters in fmtlib}
  \label{sl:fmtleftpad}
  \snippetwithoutput{fmtleftpad}{io}{fmtleftpad}
\end{block}

\Level 1 {Construct a string}

If you want to construct a string piecemeal,
for instance because it involves a loop over something,
you can use a \indexc{memory_buffer}:
\begin{lstlisting}
  fmt::memory_buffer b;
  fmt::format_to(std::back_inserter(b),"[");
  for ( auto i : indices )
    fmt::format_to(std::back_inserter(b),"{}, ",i);
  fmt::format_to(std::back_inserter(b),"]");
  cout << to_string(b) << endl;
\end{lstlisting}

\Level 1 {Number bases}

In \n{fmtlib}, you can indicate the base with which to represent an integer
by specifying one of \n{box} for binary, octal, hex respectively.

\begin{block}{Number bases in fmtlib}
  \label{sl:fmtbase}
  \snippetwithoutput{fmtbase}{io}{fmtbase}
\end{block}

\Level 1 {Output your own classes}

With \indextermtt{fmtlib} this takes a different approach:
here you need to specialize the \indexc{formatter} struct/class.
%
\snippetwithoutput{fmtstream}{io}{fmtstream}

