% -*- latex -*-
%%%%%%%%%%%%%%%%%%%%%%%%%%%%%%%%%%%%%%%%%%%%%%%%%%%%%%%%%%%%%%%%
%%%%
%%%% This TeX file is part of the course
%%%% Introduction to Scientific Programming in C++11/Fortran2003
%%%% copyright 2017-2020 Victor Eijkhout eijkhout@tacc.utexas.edu
%%%%
%%%% infect-abstract.tex : abstract for the infectuous diseases project
%%%%
%%%%%%%%%%%%%%%%%%%%%%%%%%%%%%%%%%%%%%%%%%%%%%%%%%%%%%%%%%%%%%%%

Infectuous diseases have been an important issues in recent years.
From the `anti-vax' movement to the recent Covid-19 epidemic,
infectuous diseases are having a noticeable impact on society.
Of the many aspects that can be studied, here we take a look
at the statistics of the way a disease spreads through the population.

This project walks a student through the development
of a `network model' of a population.
After the full model has been implemented, there are questions
about the sensitivity of the disease propagation to various parameters
that students can explore.

Any programming language can be used here, but the project
description targets an object-oriented formulation.

This project can be done by a single undergraduate in about a week,
at the end of a first semester programming course.

\newpage

\begin{tabular}{|l|p{5in}|}
  \hline
  Summary&Model the spread of infectuous diseases, applied
  to diseases for which a vaccine exists.
  \\
  Topics&Arrays, classes
  \\
  Audience&Undergraduate or AP high school
  \\
  Difficulty&Moderate
  \\
  Strengths&Relatively easy to program, lots of opportunity for initiative
  \\
  Weaknesses&Open to sloppy programming and minimal exploration
  \\
  &No graphics output, so the interpretation of output requires some imagination
  \\
  Dependencies&None. In particular, no biology background is needed.
  \\
  Variants&Application to diseases without vaccine (Ebola, Covid-19);
  this requires extra programming and exploration.
  \\
  \hline
\end{tabular}
