% -*- latex -*-
%%%%%%%%%%%%%%%%%%%%%%%%%%%%%%%%%%%%%%%%%%%%%%%%%%%%%%%%%%%%%%%%
%%%%
%%%% This TeX file is part of the course
%%%% Introduction to Scientific Programming in C++/Fortran2003
%%%% copyright 2017-2021 Victor Eijkhout eijkhout@tacc.utexas.edu
%%%%
%%%% pascal_discussion.tex : discussion of the pascal homework
%%%%
%%%%%%%%%%%%%%%%%%%%%%%%%%%%%%%%%%%%%%%%%%%%%%%%%%%%%%%%%%%%%%%%

\heading{vector-of-vector solution}

\begin{lstlisting}
class triangle {
private:
  int rows, cols;
  vector<vector<int>> elements;
 public:
  triangle(int n){
    rows = n; cols = n;
    elements = vector<vector<int>>(n, vector<int>(n));};
  void set(int i, int j, double v){
    elements.at(i).at(j) = v;};
\end{lstlisting}

This is the basic design.\\
the vector-of-vector is not optimal, but good enough for now.

\heading{use of double}

\begin{lstlisting}
  vector<vector<double>> elements;
\end{lstlisting}

An integer problem should be solved with integers.\\
This probably uses `double' to deal with overflow: wrong way to solve it.\\
Subtract 1/2 point.

\heading{triangular vector-of-vector}

\begin{lstlisting}
  pascal(int n)
  {
    size = 0;
    elements.push_back({1});
    for (int i = 1; i <= n; i++)
    {
      vector<int> adding;
      for (int j = 0; j <= i; j++)
      {
        adding.push_back(fact(i) / (fact(j) * fact(i - j)));
      }
      elements.push_back(adding);
    }
  };
\end{lstlisting}

This is nifty: it gives you a triangular array\\
but dynamic solutions are generally not advisable for static problems\\
I would have preferred a solution without this much "push back".

\heading{one-dimensional storage, variant 1}

\begin{lstlisting}
  vector<int> array;
  [...]
  int get(int row,int col) {
    return array.at( (row-1)*size + (col-1) );
  };
  int set(int row,int col,int value) {
    array.at( (row-1)*size + (col-1) ) = value;
  };
\end{lstlisting}

This uses a single vector to store the element.\\
Could be slightly better by introducing an indexing function, but this is fine.

\heading{one vector, variant 2}

This is the optimized solution that stores a triangle,
with triangular indexing.
Bonus point.

\begin{lstlisting}
  class pascal {
    private:
    int rows;
    vector<long int>  coeffs;
    public:
    [...]
    long int get(int i,int j) {
      return coeffs.at(i*(i+1)/2+j);
    };
\end{lstlisting}

I would have preferred the indexing hidden somewhere.

\heading{elegant solution for triangles}

\begin{lstlisting}
  void set_triangle_coefficients(){
    for(int i = 0; i < dimension; i++){
      for(int j = 0; j <= i; j++){
        if(j == 0 || j == i){
          set(i, j, 1);
        }else{
          set(i, j, get(i-1,j-1) + get(i-1,j));
        }
      }
    }
  }
\end{lstlisting}

\heading{recursion in the readout}

Unnecessary recursion: wrong computational complexity.\\
Subtract point.

\begin{lstlisting}
  int get(int r, int c){
    if ( c == 0 || c == r)
    return 1;
    return get(r-1,c-1) + get(r-1,c);
  };
\end{lstlisting}

\heading{Factorials, part 1/3}

\begin{lstlisting}
  int get (int r, int c){
    int value;
    value = factorial(r-1) / factorial(c-1) / factorial(r-c);
    return value;
  };
\end{lstlisting}

This may give overflow.

\heading{Factorials, part 2/3}

\begin{lstlisting}
  int get (int r, int c){
    int value;
    value = round(round(factorial(r-1)/(factorial(c-1))/factorial(r-c)));
    return value;
  };
\end{lstlisting}

Use of doubles and rounding is dangerous.

\heading{Factorial, part 3/3}

This is clever, pity about the recursion

\begin{lstlisting}
  long get(int r, int c){
    if ( c == 0 || c == r)
    return 1;
    if (c > r/2)
    return get(r,r-c);
    return r*get(r-1,c-1)/c;
  };
\end{lstlisting}

\heading{direct printing of stars}

passes the tester, not what was asked

\begin{lstlisting}
  class Pascal {
    private:
    int rows;
    public:
    Pascal(int row) {rows = row;}
    int printPascal{
      for (int i = 0; i < rows; i++)
      {
        int val = 1;
        for (int j = 1; j < (rows - i); j++)
        {
          cout << " ";
        }
        for (int k = 0; k <= i; k++)
        {
          cout << " " << val;
          val = val * (i - k) / (k + 1);
        }
        cout << endl;
      }
      cout << endl;
    };
    return 0;
  };
\end{lstlisting}

\heading{I don't like it}

The function `get' does not get at all:\\
it uses implicit knowledge of in what sequence it will be called.

\begin{lstlisting}
  long int get(long int &val, int i, int j){
    val = val * (i-j) / (j+1);
    return val;
  }
  void print(long int m){
    for (int i = 0; i < rows; i++)
    {
      long int val = 1;
      for (int k = 1; k < (rows - i); k++)
      {
        cout << " ";
      }
      for (int j = 0; j <= i; j++)
      {
        if (val % m != 0){
        cout <<"*"<<" ";
      } else {
        cout <<" "<<" ";
      }
      get(val,i,j);
    }
    cout << endl;
  }
  }
\end{lstlisting}

\heading{too much stackoverflow}

\begin{lstlisting}
  void print_row(vector<int> dummy){
    for (vector<int>::iterator i = dummy.begin(); i != dummy.end(); ++i)
    cout<<*i<<" ";
    cout<<endl;
  }
\end{lstlisting}

\heading{too much stackoverflow}

Never use "malloc".

\begin{lstlisting}
  class pascal {
    private:
    int n;
    long long **arr; 

    arr = (long long **)malloc(n * sizeof(long long *));
    for (int i.....) {
      arr[i] = (long long *)malloc((i + 1) * sizeof(long long));
\end{lstlisting}

\heading{too much stackoverflow}

Yes, `new' is C++, no this is still bad programming.\\
Never use `new' either.

\begin{lstlisting}
  class pascal {
    private:
    int **arr;
    int n;
    public:
    pascal(int rows) {
      n = rows;
      *arr = new int[n];
\end{lstlisting}

\heading{too much stackoverflow, patched variant}

Attempt to fix the memory leak\\
Incorrect: should first have freed all the \lstinline+arr[i]+.

\begin{lstlisting}
  ~pascal(){
    free(arr);
  }
\end{lstlisting}

