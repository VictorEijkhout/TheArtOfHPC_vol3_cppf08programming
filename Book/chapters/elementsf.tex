% -*- latex -*-
%%%%%%%%%%%%%%%%%%%%%%%%%%%%%%%%%%%%%%%%%%%%%%%%%%%%%%%%%%%%%%%%
%%%%
%%%% This TeX file is part of the course
%%%% Introduction to Scientific Programming in C++/Fortran2003
%%%% copyright 2017-2023 Victor Eijkhout eijkhout@tacc.utexas.edu
%%%%
%%%% elementsf.tex : basic language elements of Fortran
%%%%
%%%%%%%%%%%%%%%%%%%%%%%%%%%%%%%%%%%%%%%%%%%%%%%%%%%%%%%%%%%%%%%%

Fortran is an old programming language, dating back to the 1950s, and
the first `high level programming language' that was widely used.
In a way, the fields of programming language design and compiler
writing started with Fortran, rather than this language being based on
established fields. Thus, the design of Fortran has some
idiosyncrasies that later designed languages have not adopted. Many of
these are now `deprecated' or simply inadvisable. Fortunately, it is
possible to write Fortran in a way that is every bit as modern and
sophisticated as other current languages.

In this part of the book, you will learn safe practices for
writing Fortran. Occasionally we will not mention practices that you
will come across in old Fortran codes, but that we would not advise
you taking up. While this exposition of Fortran can stand on its own,
we will in places point out explicitly differences with~C++.

\Level 0 {Source format}

Fortran started in the era when programs were stored on
\indexterm{punch card}s. Those had 80 columns, so a line of Fortran
source code could not have more than 80 characters. Also, the first 6
characters had special meaning. This is referred to as \emph{fixed
  format}\index{source!format!fixed}. However, starting with
\indextermbus{Fortran}{90} it became possible to have \emph{free
  format}\index{source!format!free}, which allowed longer lines
without special meaning for the initial columns.

There are further differences between the two formats (notably
continuation lines) but we will only discuss free format in this course.

Many compilers have a convention for indicating the source format by
the file name extension:
\begin{itemize}
\item \n{f} and \n{F} are the extensions for old-style fixed format;
  and
\item \n{f90} and \n{F90} are the extensions for new free format.
\end{itemize}
Capital letters indicate that the \indextermbus{C}{preprocessor}
is applied to the file.
For this course we will use the \indextermtt{F90} extension.

\Level 0 {Compiling Fortran}

The minimal Fortran program is:
%
\verbatimsnippet{emptyf}

You would compile this:
\begin{verbatim}
yourfortrancompiler -o myprogram myprogram.F90
\end{verbatim}
and then execute with 
\begin{verbatim}
./myprogram
\end{verbatim}
For Fortran programs, the compiler is \indextermtt{gfortran} for the
GNU compiler, and \indextermtt{ifort} for Intel. 

\begin{exercise}
  Add the line
\begin{lstlisting}
print *,"Hello world!"
\end{lstlisting}
to the empty program, and compile and run it.
\end{exercise}

Fortran ignores \emph{case}\index{Fortran!case ignores}
in both keywords and identifiers.
Keywords such as \indexf{Program} in the above program
can thus just as well be written as
\n{PrOgRaM}.

A program optionally has a \indexfdef{stop} statement, which can
return a message to the \ac{OS}.
%
\snippetwithoutput{stopf}{basicf}{stop}

Additionally, a numeric code returned by \indexf{stop}
\begin{lstlisting}
stop 1
\end{lstlisting}
can be queried with the \verb+$?+ shell parameter:
%
\snippetwithoutput{stopret}{basicf}{return}

\Level 0 {Main program}

Fortran does not use curly brackets to delineate blocks, instead you
will find \indextermtt{end} statements. The very first one appears
right when you start writing your program:
a~Fortran program needs to start with a \indexf{Program} line, and end with
\indexf{End Program}. The program needs to have a name on both lines:
%
\verbatimsnippet{emptyf}
%
and you can not use that name for any entities in the program.

\begin{remark}
  The \indexterm{emacs} editor will supply the block type and name if
  you supply the `end' and hit the TAB or RETURN key; see
  section~\ref{sec:editor-mode}.
\end{remark}

\Level 1 {Program structure}

Unlike C++, Fortran can not mix variable declarations and executable
statements, so both the main program and any subprograms have roughly
a
structure:
\begin{lstlisting}
Program foo
  < declarations >
  < statements >
End Program foo
\end{lstlisting}

\begin{slide}{Program structure}
  \label{sl:programf}
Structure of your program file:
\begin{lstlisting}
Program foo
  < declarations >
  < statements >
End Program foo
\end{lstlisting}
\end{slide}

Another thing to note is that there are no include directives.
Fortran does not have a `standard library' such as C++
that needs to be explicitly included.
Or you could say that the Fortran standard library is always by default included.

\begin{slide}{Things to note}
  \label{sl:programfnotes}
  \begin{itemize}
  \item No includes before the program.
  \item Program has a name (emacs tip: type \n{end<TAB>})
  \item There is an \lstinline{End}, rather than curly braces.
  \item Declarations first, not interspersed.
  \end{itemize}
\end{slide}

\Level 1 {Statements}

Let's say a word about layout. Fortran has a `one line, one statement'
principle, stemming from its \indexterm{punch card} days.
\begin{itemize}
\item As long as a statement fits on one line, you don't have to
  terminate it explicitly with something like a semicolon:
\begin{lstlisting}
x = 1
y = 2
\end{lstlisting}
\item If you want to put two statements on one line, you have to
  terminate the first one:
\begin{lstlisting}
x = 1; y = 2
\end{lstlisting}
But watch out for the line length: this is often limited to 132 characters.
\item If a statement spans more than one line, all but the first line
  need to have an explicit \indexterm{continuation character}, the ampersand:
\begin{lstlisting}
x = very &
  long &
  expression
\end{lstlisting}
\end{itemize}

\begin{slide}{Statements}
  \label{sl:fstatement}
  \begin{itemize}
  \item One line, one statement
\begin{lstlisting}
x = 1
y = 2
\end{lstlisting}
(historical reasons)
\item semicolon to separate multiple statements per line
\begin{lstlisting}
x = 1; y = 2
\end{lstlisting}
\item Continuation of a line
\begin{lstlisting}
x = very &
  long &
  expression
\end{lstlisting}
  \end{itemize}
\end{slide}

\Level 1 {Comments}

Fortran knows only single-line
\emph{comments}\index{Fortran!comments},
indicated by an exclamation point:
\begin{lstlisting}
x = 1 ! set x to one
\end{lstlisting}
Everything from the exclamation point to the end of the line is ignored.

Maybe not entirely obvious: you can have a comment after a
continuation character:
\begin{lstlisting}
x = f(a) & ! term1 
  + g(b)   ! term2
\end{lstlisting}

\begin{slide}{Comments}
  \label{sl:fcomment}
  \begin{itemize}
  \item Ignore to end of line
\begin{lstlisting}
x = 1 ! set x to one
\end{lstlisting}
\item comment after continuation
\begin{lstlisting}
x = f(a) & ! term1 
  + g(b)   ! term2
\end{lstlisting}
  \item No multi-line comments.
  \end{itemize}
\end{slide}

\begin{remark}
  In \fstandard{77}, 19 continuation lines were allowed.
  In \fstandard{95} this number was~40.
  As of the \fstandard{2003} standard, a~line can be continued 256 times.
\end{remark}

\Level 0 {Variables}

Unlike in C++, where you can declare a variable right before you need
it, Fortran wants its variables declared near the top of the program
or subprogram:
\begin{lstlisting}
Program YourProgram
  implicit none
  ! variable declaration
  ! executable code
End Program YourProgram
\end{lstlisting}
The \lstinline{implicit none} should always be included;
see section~\ref{sec:f-implicit} for an explanation.

A variable declaration looks like:
\begin{lstlisting}
type [ , attributes ] :: name1 [ , name2, .... ]
\end{lstlisting}
where
\begin{itemize}
\item we use the common grammar shorthand that \n{[ something ]}
  stands for an optional `something';
\item \textit{type} is most commonly \indexf{integer}, \indexf{real(4)}, \indexf{real(8)},
  \indexf{logical}. See below; section~\ref{sec:ftype}.
\item the optional \textit{attributes} are things such as
  \indexf{dimension}, \indexf{allocatable},
  \indexf{intent}, \indexf{parameter} et cetera.
\item \textit{name} is something you come up with. This has to start
  with a letter. Unusually, variable names are case-insensitive.
  Thus, 
  \begin{lstlisting}
    Integer :: MYVAR
    MyVar = 2
    print *,myvar
  \end{lstlisting}
  is perfectly legal.
\end{itemize}

\begin{slide}{Variable declarations}
  \label{sl:fvars}
  \begin{itemize}
  \item Variable declarations at the top of the program unit,\\
    before any executable statements.
  \item declaration
\begin{lstlisting}
type, attributes :: name1, name2, ....
\end{lstlisting}
where
\begin{itemize}
\item \textit{type} is most commonly \indexf{integer},
  \indexf{real(4)}, \indexf{real(8)},
  \indexf{logical}
\item \textit{attributes} can be \indexf{dimension},
  \indexf{allocatable}, \indexf{intent},
  \indexf{parameter} et cetera.
\end{itemize}
\item Keywords and variables are
  \emph{case-insensitive}\index{Fortran!case-insensitive}
\end{itemize}
\end{slide}

\begin{remark}
  In \fstandard{66} there was a limit of six characters to the
  \emph{length} of a \emph{variable name}\index[f90]{variable!length of name},
  though many compilers had extensions to this.
  As of the \fstandard{2003} standard, a variable name can be 63 characters long.
\end{remark}
The built-in data types of Fortran:

\begin{block}{Data types}
  \label{sl:ftypes}
  \begin{itemize}
  \item Numeric: \indexf{Integer}, \indexf{Real},
    \indexf{Complex}
    \item precision control:
\begin{lstlisting}
Integer :: i
Integer(4) :: i4
Integer(8) :: i8
\end{lstlisting}
This usually corresponds to number of bytes; see textbook for full story.
  \item Logical: \indexf{Logical}.
  \item Character: \indexf{Character}. Strings are realized as
    arrays of characters.
  \item Derived types (like C++ structures or classes): \indexf{Type}
  \end{itemize}  
\end{block}

Some variables are not intended ever to change,
such as if you introduce a variable \lstinline+pi+ with value~$3.14159$.
You can mark this name as being a synonym for the value,
rather than a variable you can assign to,
with the \indexfdef{parameter} keyword.
\begin{lstlisting}
real,parameter :: pi = 3.141592    
\end{lstlisting}

\begin{slide}{Parameter}
  \label{sl:fparameter}
  Variable has a name, and a value that can change\\
  What if the value should never change?
  Use the \indexf{parameter} attribute
\begin{lstlisting}
real,parameter :: pi = 3.141592    
\end{lstlisting}
This can not be changed like an ordinary variable.\\
Like \lstinline{const} in C++.
\end{slide}

In chapter~\ref{ch:arrayf} you will see that
\indexf{parameter}s are often used for defining the size of an
array.

\begin{slide}{Strings}
  \label{sl:fchar20}
  \begin{lstlisting}
    character*20 :: prompt
    prompt = "up to 20 characters"
  \end{lstlisting}
\end{slide}

Further specifications for numerical
precision are discussed in section~\ref{sec:fprecision}.
Strings are discussed in chapter~\ref{ch:stringf}.

\Level 1 {Declarations}
\label{sec:ftype}

\Level 2 {Implicit declarations}
\label{sec:f-implicit}

Fortran has a somewhat unusual treatment of variable types:
if you don't
specify what data type a variable is, Fortran will deduce it from
a simple rule, based on the first character of the name.
This is a very dangerous practice, so we
advocate putting a line
\begin{lstlisting}
implicit none
\end{lstlisting}
immediately after any program or subprogram header.
Now every variable needs to be given a type explicitly
in a declaration.

\begin{slide}{Implicit typing}
  \label{sl:fimplicit}
  Fortran strictly does not need variable declarations (like python):\\
  variables have a type that is determined by name.\\
  This is legal:
\begin{lstlisting}
Program danger
  i = 3
  x = 2.5
  j = i * x
  print *,y
end Program danger
\end{lstlisting}
 But:\\
  This is \textbf{very dangerous}. Use \lstinline{implicit none}
  in every program unit.
\begin{lstlisting}
Program myprogram
  implicit none
  integer :: i
  real :: x
  ! more stuff
End Program myprogram
\end{lstlisting}
\end{slide}

\Level 2 {Variable `kind's}
\label{sec:fprecision}

Fortran has several mechanisms for indicating the precision of a numerical type.
%
\verbatimsnippet{fstorage}

This often corresponds to the number of bytes used, \textbf{but not always}.
It is technically a numerical \indextermdef{kind selector},
and it is nothing more than an identifier for a specific type.

\Level 1 {Initialization}

Variables can be initialized in their declaration:
\begin{lstlisting}
integer :: i=2
real(4) :: x = 1.5
\end{lstlisting}

That this is done at compile time, leading to a common error:
\begin{lstlisting}
subroutine foo()
  implicit none
  integer :: i=2
  print *,i
  i = 3
end subroutine foo
\end{lstlisting}
On the first subroutine call \n{i} is printed with its initialized
value, but on the second call this initialization is not repeated, and
the previous value of \n{3} is remembered.

\Level 0 {Complex numbers}
\label{sec:fcomplex}

A complex number is a pair of real numbers.
Complex constants can be written with a parenthesis notation,
but to form a complex number from two real variables requires
the \indexf{CMPLX} function.
You can also use this function to force a real number
to complex, so that subsequent computations are done
in the complex realm.

Real and imaginary parts can be extracted with the function
\indexf{real} and \indexf{aimag}.

\begin{block}{Complex}
  \label{sl:fcomplex}
  Complex constants are written as a pair of reals in parentheses.\\
  There are some basic operations.
  %
  \snippetwithoutput{fcomplex}{basicf}{complex}
\end{block}

The imaginary root~$i$ is not predefined. Use
\begin{lstlisting}
Complex,parameter :: i = (0,1)
\end{lstlisting}

In Fortran2008, \indexf{Complex} is a derived type,
and the real/imaginary parts can be extracted as
\begin{lstlisting}
print *,rotated%re,rotated%im
\end{lstlisting}
\snippetwithoutput{fcomplexf08}{basicf}{complexf08}

\begin{exercise}
  Write a program to compute the complex roots of the quadratic
  equation \[ ax^2+bx+c=0 \]
  The variables \lstinline+a,b,c+ have to be real,
  but use the \lstinline+cmplx+ function to force the computation
  of the roots to happen in the complex domain.
\end{exercise}

\Level 0 {Expressions}

Fortran has arithmetic, logical, and string expressions.
\begin{itemize}
\item Arithmetic expressions look more or less the way you would expect them to.
  The only unusual operator is the power operator: \lstinline+x**.5+ is the
  square root of variable~\lstinline{x}.
\item For boolean expressions there are constants \lstinline+.true.+ and \lstinline+.false.+;
  operators are likewise enclosed in dots: \lstinline+.and.+ and such.
  Boolean variables are of type \indexf{Logical}.
  See section~\ref{sl:fbool} for more.
\item String handling will be discussed in chapter~\ref{ch:stringf}.
\end{itemize}

\Level 0 {Bit operations}
\label{sec:bitf}

As of \fstandard{95} there are functions for bitwise operations:
\begin{itemize}
\item \indexf{btest}\lstinline+(word,pos)+
  returns a \lstinline{logical}
  if bit \lstinline{pos} in \lstinline{word} is set.
\item \indexf{ibits}\lstinline+(word,pos,len)+
  returns an integer (of the same kind as \lstinline{word}
  with bits $p,\cdots p+\ell-1$ (extending leftward)
  right-adjusted.
\item \indexf{ibset}\lstinline+(i,pos)+
  takes an integer and
  returns the integer resulting from setting bit \lstinline{pos} to~1.
  Likewise, \indexf{ibclr} clears that bit.
\item \indexf{iand}, \indexf{ior}, \indexf{ieor}
  all operate on two integers, returning the bitwise and/or/xor result.
\item \indexf{mvbits}\lstinline+(from,frompos,len,to,tpos)+
  copies a range of bits between two integers.
\end{itemize}

\Level 0 {Commandline arguments}

Modern Fortran has functions for querying
\indexterm{commandline arguments}.
First of all \indexfdef{command_argument_count} queries the number
of arguments.
This does not include the command itself,
so this is one less than the C/C++ \indexc{argc} argument to main.
\verbatimsnippet{fargcount}

The command can be retrieved with \indexfdef{get_command}.

The commandline arguments are retrieved with
\indexfdef{get_command_argument}.
These are strings as in~C/C++,
but you have to specify their length in advance:
\verbatimsnippet{fargcstring}

Converting this string to an integer or so takes a little format trickery:
\verbatimsnippet{fargcstringread}
(see section~\ref{sec:fformat}.)

\Level 0 {Fortran type kinds}

\Level 1 {Kind selection}

Kinds can be used to ask for a type with specified precision.
\begin{itemize}
\item For integers you can specify the number of decimal digits with
  \indexfdef{selected_int_kind}\n{($n$)}.
\item For floating point numbers can specify the number of
  significant digits, and optionally the decimal exponent range with
  \indexfdef{selected_real_kind}\n{($p$[,$r$])}.
  of significant digits.
\end{itemize}

Conversely, the properties of such types can be retrieved again
with the functions \indexf{precision} (not for integers),
\indexf{range}, \indexf{storage_size}.

\begin{block}{Real kind declaration}
  Declaration of precision and/or range:
  \snippetwithoutput{frealkind}{basicf}{kind}
\end{block}

Likewise, you can specify the precision of a constant.
Writing \n{3.14} will usually be a single precision
real. 

\begin{block}{Single/double precision constants}
  \label{sl:fsingledouble}
  Adding single/double precision constants, print as double:
  %
  \snippetwithoutput{f0const}{basicf}{e0}
\end{block}

You can query how many bytes a data type takes with
\indexfdef{kind}.

\begin{block}{Numerical precision}
  \label{sl:fprecision48}
  Number of bytes determines numerical precision:
  \begin{itemize}
  \item Computations in 4-byte have relative error $\approx 10^{-6}$
  \item Computations in 8-byte have relative error $\approx 10^{-15}$
  \end{itemize}
  Also different exponent range: max $10^{\pm 50}$ and $10^{\pm 300}$ respectively.
\end{block}

\begin{block}{Storage size}
  F08: \indexf{storage_size} reports number of bits.

  F95: \indexf{bit_size} works on integers only.

  \indexf{c_sizeof} reports number of bytes, requires
  \indexf{iso_c_binding} module.
  %
  \snippetwithoutput{isocbind}{basicf}{binding}
\end{block}

Force a constant to be \indexf{real(8)}:
\begin{block}{Double precision constants}
  \label{sl:fdouble}
\begin{lstlisting}
real(8) :: x,y
x = 3.14d0
y = 6.022e-23
\end{lstlisting}
  \begin{itemize}
  \item Use a compiler flag such as \n{-r8} to force all reals to be 8-byte.
  \item Write \n{3.14d0}
  \item \verb+x = real(3.14, kind=8)+
  \end{itemize}
\end{block}

\Level 1 {Range}

You can use the function \indexf{huge} to query the maximum value of a type.

\snippetwithoutput{fdeftypes}{typef}{def}

With \indextermbus{ISO}{bindings} there is a more systematic approach.

Integers:
\verbatimsnippet{fisokinds}

\snippetwithoutput{fisosizes}{typef}{int}

Floating point numbers:
\verbatimsnippet{isoreal3264}

\begin{comment}
  \url{https://docs.oracle.com/cd/E19957-01/805-4940/6j4m1u7pk/index.html}
  \url{https://fortranwiki.org/fortran/show/ieee_arithmetic}
\end{comment}


\Level 0 {Quick comparison Fortran vs C++}

\Level 1 {Statements}

\begin{block}{Fortran statements}
  \label{sl:fstatements}
  Some of it is much like C++:
  \begin{itemize}
  \item Assignments:
\begin{lstlisting}
x = y
x = 2*y / (a+b)
z1 = 5; z2 = 6
\end{lstlisting}
(Note the lack of semicolons at the end of statements.)
\item I/O
\item conditionals and loops
  \end{itemize}
Different:
\begin{itemize}
\item function definition and calls
\item array syntax
\item object oriented programming
\item modules
\end{itemize}
\end{block}

\Level 1 {Input/Output, or I/O as we say}
\label{sec:fio}

\begin{block}{Simple I/O}
  \label{sl:frw}
  \begin{itemize}
  \item Input: 
\begin{lstlisting}
READ *,n
\end{lstlisting}
\item Output:
\begin{lstlisting}
PRINT *,n
\end{lstlisting}
  \end{itemize}
  There is also \indexf{Write}.

  The `star' indicates that default formatting is used.\\
  Other syntax for read/write with files and formats.
\end{block}

\Level 1 {Expressions}
\label{sec:fexpr}

\begin{block}{Arithmetic expressions}
  \label{sl:farith}
  \begin{itemize}
  \item Pretty much as in C++
  \item Exception: \n{r**a} for power $r^a$.
  \item Modulus (the \lstinline+%+ operator in C++)
    is a function: \lstinline{MOD(7,3)}.
  \end{itemize}
\end{block}

\begin{block}{Boolean expressions}
  \label{sl:fbool}
  \begin{itemize}
  \item 
    Long form:\\
    \n{.and. .not. .or.}\\
    \n{.lt. .le. .eq. .ne. .ge. .gt.}\\
    \n{.true. .false.}
  \item Short form:\\
    \verb+< <= == /= > >=+
  \end{itemize}
\end{block}

\begin{block}{Conversion and casting}
  Conversion is done through functions.
  \begin{itemize}
  \item \indexf{INT}: truncation; \indexf{NINT} rounding
  \item \indexf{REAL}, \indexf{FLOAT}, \indexf{SNGL}, \indexf{DBLE}
  \item \indexf{CMPLX}, \indexf{CONJG}, \indexf{AIMIG}
  \end{itemize}
\url{http://userweb.eng.gla.ac.uk/peter.smart/com/com/f77-conv.htm}
\end{block}

\begin{block}{Complex}
  Complex numbers exist; section~\ref{sec:fcomplex}.
\end{block}

\begin{block}{Strings}
  Strings are delimited by single or double quotes.

  For more, see chapter~\ref{ch:stringf}.
\end{block}

\Level 0 {Review questions}

\begin{exercise}
  What is the output for this fragment, assuming \n{i,j} are integers?
  %
  \verbatimsnippet{idiv}
\end{exercise}

\begin{exercise}
  What is the output for this fragment, assuming \n{i,j} are integers?
  %
  \verbatimsnippet{fdiv}
\end{exercise}

\begin{exercise}
  \label{ex:f-elt-rev1}
  In declarations
\begin{lstlisting}
real(4) :: x
real(8) :: y
\end{lstlisting}
what do the \n{4} and \n{8} stand for?

What is the practical implication of using the one or the other?
\end{exercise}

\begin{exercise}
  \label{ex:read-writet3np1}
  Write a program that :
  \begin{itemize}
  \item displays the message \n{Type a number},
  \item accepts an integer number from you (use~\n{Read}),
  \item makes another variable that is three times that integer plus one,
  \item and then prints out the second variable.
  \end{itemize}
\end{exercise}

\begin{exercise}
  \label{ex:f-elt-rev2}
  In the following code, if \n{value} is nonzero, what do expect about
  the output?
  %
  \verbatimsnippet{d0}
\end{exercise}
