% -*- latex -*-
%%%%%%%%%%%%%%%%%%%%%%%%%%%%%%%%%%%%%%%%%%%%%%%%%%%%%%%%%%%%%%%%
%%%%
%%%% This TeX file is part of the course
%%%% Introduction to Scientific Programming in C++/Fortran2003
%%%% copyright 2017-2022 Victor Eijkhout eijkhout@tacc.utexas.edu
%%%%
%%%% obscure.tex : other stuff
%%%%
%%%%%%%%%%%%%%%%%%%%%%%%%%%%%%%%%%%%%%%%%%%%%%%%%%%%%%%%%%%%%%%%

\Level 0 {Auto}

This is not actually obscure, but it intersects many other topics, so
we put it here for now.

\Level 1 {Declarations}

Sometimes the type of a variable is obvious:
\begin{lstlisting}
std::vector< std::shared_ptr< myclass >>*
  myvar = new std::vector< std::shared_ptr< myclass >>
                ( 20, new myclass(1.3) );
\end{lstlisting}
(Pointer to vector of 20 shared pointers to \n{myclass}, initialized
with unique instances.)  You can write this as:
\begin{lstlisting}
auto myvar =
  new std::vector< std::shared_ptr< myclass >>
            ( 20, new myclass(1.3) );
\end{lstlisting}

\begin{slide}{Type deduction}
  \label{sl:auto-deduct}
In:
\begin{lstlisting}
std::vector< std::shared_ptr< myclass >>*
  myvar = new std::vector< std::shared_ptr< myclass >>
                ( 20, new myclass(1.3) );
\end{lstlisting}
the compiler can figure it out:
\begin{lstlisting}
auto myvar =
  new std::vector< std::shared_ptr< myclass >>
            ( 20, new myclass(1.3) );
auto result = someobject.somemethod();
\end{lstlisting}
\end{slide}

\begin{block}{Type deduction in functions}:
  \label{sl:auto-fun}
  Return type can be deduced in C++17:
  \verbatimsnippet{autofun}  
\end{block}

\begin{block}{Type deduction in methods}:
  \label{sl:auto-method}
  Return type of methods can be deduced in C++17:
  \verbatimsnippet{autoclass}  
\end{block}

\begin{block}{Auto and references, 1}
  \label{sl:auto-ref1}
  \n{auto} discards references and such:
  %
  \snippetwithoutput{autoplain}{auto}{plainget}
\end{block}

\begin{block}{Auto and references, 2}
  \label{sl:auto-ref2}
  Combine \n{auto} and references:
  %
  \snippetwithoutput{autoref}{auto}{refget}
\end{block}

\begin{block}{Auto and references, 3}
  \label{sl:auto-ref3}
  For good measure:
  %
  \verbatimsnippet{constrefget}
\end{block}

\begin{comment}
  \begin{block}{Auto plus}
    \label{sl:auto-plus-const}
    Keywords like \n{const} and the reference character~\n{\&} can be
    added:
\begin{lstlisting}
// class member
  some_object my_object;
// class method:
  some_object &get_some_object() { return my_object; };
// main program:
auto object_copy  = thing.get_some_object();
auto &object_mutable  = thing.get_some_object();
const auto &object_immutable  = thing.get_some_object();
\end{lstlisting}
  \end{block}
\end{comment}

\Level 1 {Auto and function definitions}

The return type of a function can be given with a
\emph{trailing return type}\index{type!return!trailing}
definition:
\begin{lstlisting}
auto f(int i) -> double { /* stuff */ };
\end{lstlisting}
This notations is more common for lambdas, chapter~\ref{ch:lambda}.

\Level 1 {\texttt{decltype}: declared type}

There are places where you want the compiler to deduce the type of a
variable, but where this is not immediately possible. Suppose that in
\begin{lstlisting}
auto v = some_object.get_stuff();
f(v);
\end{lstlisting}
you want to put a \lstinline{try ... catch} block around just the
creation of~\lstinline{v}. This does not work:
\begin{lstlisting}
try { auto v = some_object.get_stuff(); 
} catch (...) {}
f(v);
\end{lstlisting}
because the \lstinline{try} block is a scope. It also doesn't work to write
\begin{lstlisting}
auto v;
try { v = some_object.get_stuff(); 
} catch (...) {}
f(v);
\end{lstlisting}
because there is no indication what type \lstinline{v} is created
with.

Instead, it is possible to query the type of the expression that
creates~\lstinline{v} with \indexcdef{decltype}:
\begin{lstlisting}
decltype(some_object.get_stuff()) v;
try { auto v = some_objects.get_stuff(); 
} catch (...) {}
f(v);
\end{lstlisting}

\ref{sec:stack-heap}

\Level 0 {Casts}
\label{sec:cast}

In C++, constants and variables have clear types. For cases where you
want to force the type to be something else, there is the
\indextermdef{cast} mechanism. With a cast you tell the compiler:
treat this thing as such-and-such a type, no matter how it was
defined.

In C, there was only one casting mechanism:
\begin{lstlisting}
sometype x;
othertype y = (othertype)x;
\end{lstlisting}
This mechanism is still available as the
\indexc{reinterpret_cast}, which does `take this byte and pretend
it is the following type':
\begin{lstlisting}
sometype x;
auto y = reinterpret_cast<othertype>(x);
\end{lstlisting}

The inheritance mechanism necessitates another casting mechanism.
An object from a derived class contains in it all the information of
the base class. It is easy enough to take a pointer to the derived
class, the bigger object, and cast it to a pointer to the base object.
The other way is harder.

Consider:
\begin{lstlisting}
class Base {};
class Derived : public Base {};
Base *dobject = new Derived;
\end{lstlisting}
Can we now cast dobject to a pointer-to-derived ?
\begin{itemize}
\item \indexc{static_cast} assumes that you know what you are
  doing, and it moves the pointer regardless.
\item \indexc{dynamic_cast} checks whether \n{dobject} was
  actually of class \n{Derived} before it moves the pointer, and
  returns \indexc{nullptr} otherwise.
\end{itemize}

\begin{remark}
  One further problem with the C-style casts is that their syntax is
  hard to spot, for instance by searching in an editor.
  Because C++ casts have a unique keyword, they are easier to recognize
  in a text editor.
\end{remark}

\begin{slide}{C++ casts}
  \label{sl:cpp-casts}
  \begin{itemize}
  \item \n{reinterpret_cast}: 
    Old-style `take this byte and pretend it is XYZ'; very dangerous.\\
    Instead:
  \item \n{static_cast}: simple scalar stuff
  \item \n{static_cast}: cast base to derived without check.
  \item \n{dynamic_cast:} cast base to derived with check.
  \item \n{const_cast}: Adding/removing const
  \end{itemize}

  Also: syntactically clearly recognizable.\\
  no reason for using the old `paren' cast
\end{slide}

Further reading \url{https://www.quora.com/How-do-you-explain-the-differences-among-static_cast-reinterpret_cast-const_cast-and-dynamic_cast-to-a-new-C++-programmer/answer/Brian-Bi}

\Level 1 {Static cast}

One use of casting is to convert to constants to a `larger' type. For
instance, allocation does not use integers but \indexc{size_t}.

\verbatimsnippet{longintcast}

However, if the
conversion is possible, the result may still not be `correct'.
%
\snippetwithoutput{longcast}{cast}{intlong}
%
There are no runtime tests on static casting.

Static casts are a good way of casting back void pointers to what they
were originally.

\begin{slide}{Static cast}
  \label{sl:const-cast}
  \verbatimsnippet{longintcast}
  \snippetwithoutput{longcast}{cast}{intlong}  
\end{slide}

\Level 1 {Dynamic cast}

Consider the case where we have a base class and derived classes.
%
\verbatimsnippet{polybase}
%
Also suppose that we have a function that takes a pointer to the base
class:
%
\verbatimsnippet{polybasefunc}
%
The function can discover what derived class the base pointer refers
to:
%
\verbatimsnippet{polycast}

\begin{slide}{Pointer to base class}
  \label{sl:dyn-base-ptr}

  Class and derived:
  %
  \verbatimsnippet{polybase}
\end{slide}

If we have a pointer to a derived object, stored in a pointer to a
base class object, it's possible to turn it safely into a derived
pointer again:
%
\snippetwithoutput{polycast}{cast}{deriveright}

On the other hand, a \indexc{static_cast} would not do the job:
%
\snippetwithoutput{polywrong}{cast}{derivewrong}

\begin{slide}{Cast to derived class}
  \label{sl:dyn-cast}
  This is how to do it:
  %
  \snippetwithoutput{polycast}{cast}{deriveright}
\end{slide}

\begin{slide}{Cast to derived class, the wrong way}
  \label{sl:dyn-cast-wrong}
  Do not use this function \n{g}:
  %
  \snippetwithoutput{polywrong}{cast}{derivewrong}
\end{slide}

Note: the base class needs to be polymorphic, meaning that that pure
virtual method is needed. This is not the case with a static cast,
but, as said, this does not work correctly in this case.

\Level 1 {Const cast}

With \indexc{const_cast} you can add or remove const from a
variable. This is the only cast that can do this.

\Level 1 {Reinterpret cast}

The \indexcdef{reinterpret_cast} is the crudest cast, and it
corresponds to the C~mechanism of `take this byte and pretend it of
type whatever'. There is a legitimate use for this:
\begin{lstlisting}
void *ptr;
ptr = malloc( how_much );
auto address = reinterpret_cast<long int>(ptr);
\end{lstlisting}
so that you can do arithmetic on the address. For this particular
case, \indexcdef{intptr_t} is actually better.

\Level 1 {A word about void pointers}

A traditional use for casts in~C was the treatment of
\indextermsub{void}{pointer}s. The need for this is not as severe in
C++ as it was before.

A typical use of void pointers appears in the
PETSc~\cite{petsc-efficient,petsc-home-page} library. Normally when
you call a library routine, you have no further access to what happens
inside that routine. However, PETSc has the functionality for you to
specify a monitor so that you can print out internal quantities.
\begin{lstlisting}
int KSPSetMonitor(KSP ksp,
  int (*monitor)(KSP,int,PetscReal,void*),
  void *context,
  // one parameter omitted
  );
\end{lstlisting}
Here you can declare your own monitor routine that will be called
internally: the library makes a \indexterm{call-back} to your code.
Since the library can not predict whether your monitor routine may
need further information in order to function, there is the
\n{context} argument, where you can pass a structure as void pointer.

This mechanism is no longer needed in C++ where you would use a
\indexterm{lambda} (chapter~\ref{ch:lambda}):
\begin{lstlisting}
KSPSetMonitor( ksp,
  [mycontext] (KSP k,int ,PetscReal r) -> int {
    my_monitor_function(k,r,mycontext); } );
\end{lstlisting}


\Level 0 {lvalue vs rvalue}
\label{sec:lrvalue}

The terms `lvalue' and `rvalue' sometimes appear in compiler error
messages.
\begin{lstlisting}
int foo() {return 2;}

int main()
{
    foo() = 2;

    return 0;
}

# gives:
test.c: In function 'main':
test.c:8:5: error: lvalue required as left operand of assignment
\end{lstlisting}

See the `lvalue' and `left operand'? To first order of approximation
you're forgiven for thinking that an \indextermdef{lvalue} is something
on the left side of an assignment. The name actually means `locator
value': something that's associated with a specific location in
memory. Thus an lvalue is, also loosely, something that can be modified.

An \indextermdef{rvalue} is then something that appears on the right
side of an assignment, but is really defined as everything that's not
an lvalue. Typically, rvalues can not be modified.

The assignment \n{x=1} is legal because a variable \n{x} is at some specific
location in memory, so it can be assigned to. On the other hand,
\n{x+1=1} is not legal, since \n{x+1} is at best a temporary,
therefore not at a specific memory location, and thus not an lvalue.

Less trivial examples:
\begin{lstlisting}
int foo() { x = 1; return x; }
int main() {
  foo() = 2;
}
\end{lstlisting}
is not legal because \n{foo} does not return an lvalue. However,
\begin{lstlisting}
class foo {
private:
  int x;
public:
  int &xfoo() { return x; };
};
int main() {
  foo x;
  x.xfoo() = 2;
\end{lstlisting}
is legal because the function \n{xfoo} returns a reference to the
non-temporary variable \n{x} of the \n{foo} object.

Not every lvalue can be assigned to: in
\begin{lstlisting}
const int a = 2;
\end{lstlisting}
the variable \n{a} is an lvalue, but can not appear on the left hand
side of an assignment.

\Level 1 {Conversion}

Most lvalues can quickly be converted to rvalues:
\begin{lstlisting}
int a = 1;
int b = a+1;
\end{lstlisting}
Here \n{a} first functions as lvalue, but becomes an rvalue in the
second line.

The ampersand operator takes an lvalue and gives an rvalue:
\begin{lstlisting}
int i;
int *a = &i;
&i = 5; // wrong
\end{lstlisting}

\Level 1 {References}

The ampersand operator yields a reference. It needs to be assigned
from an lvalue, so
\begin{lstlisting}
std::string &s = std::string(); // wrong
\end{lstlisting}
is illegal. The type of \n{s} is an `lvalue reference' and it can not
be assigned from an rvalue.

On the other hand
\begin{lstlisting}
const std::string &s = std::string();
\end{lstlisting}
works, since \n{s} can not be modified any further.

\Level 1 {Rvalue references}
\label{sec:rvalue-ref}

A new feature of C++ is
intended to minimize the amount of data copying through
\indexterm{move semantics}.

Consider a copy assignment operator
\begin{lstlisting}
BigThing& operator=( const BigThing &other ) {
  BigThing tmp(other); // standard copy
  std::swap( /* tmp data into my data */ );
  return *this;
};
\end{lstlisting}
This calls a copy constructor and a destructor on \n{tmp}. (The use of
a temporary makes this safe under exceptions. The \indexc{swap}
method never throws an exception, so there is no danger of half-copied
memory.)

However, if you assign
\begin{lstlisting}
thing = BigThing(stuff);
\end{lstlisting}
Now a constructor and destructor is called for the temporary rvalue object on
the right-hand side.

Using a syntax that is new in \indexterm{C++}, we create an
\indextermbus{rvalue}{reference}:
\begin{lstlisting}
BigThing& operator=( BigThing &&other ) {
  swap( /* other into me */ );
  return *this;
}
\end{lstlisting}

\Level 0 {Move semantics}

With an
\emph{overloaded operator}\index{operator!overloading!and copies},
such as addition,
on matrices (or any other big object):
\begin{lstlisting}
Matrix operator+(Matrix &a,Matrix &b);
\end{lstlisting}
the actual addition will involve a copy:
\begin{lstlisting}
Matrix c = a+b;
\end{lstlisting}

Use a move constructor:
\begin{lstlisting}
class Matrix {
private:
  Representation rep;
public:
  Matrix(Matrix &&a) {
    rep = a.rep;
    a.rep = {};
  }
};
\end{lstlisting}

\Level 0 {Graphics}

C++ has no built-in graphics facilities, so you have to use external
libraries such as \indexterm{OpenFrameworks},
\url{https://openframeworks.cc}.

\Level 0 {Standards timeline}
\label{sec:cpp-standards}

Each standard has many changes over the previous.

If you want to detect what language standard you are compiling with,
use the \indexc{__cpluscplus} macro:
%
\snippetwithoutput{cppversion}{basic}{version}
%
This returns a \lstinline{long int} with possible values
\n{199711}, \n{201103}, \n{201402}, \n{201703}, \n{202002}.

Here are some of the highlights of the various standards.

\Level 1 {C++98/C++03}

Of the \cppstandard{03} standard
we only highlight deprecated features.
\begin{itemize}
\item \indexc{auto_ptr} was an early attempt at smart
  pointers. It is deprecated, and C++17 compilers will actually issue
  an error on it. For current smart pointers see chapter~\ref{ch:pointer}.
\end{itemize}

\Level 1 {C++11}
\index{C++!C++11|(textbf}

\begin{itemize}
\item \indexc{auto}
\begin{lstlisting}
const auto count = std::count
    (begin(vec),end(vec),value);
\end{lstlisting}
The \lstinline{count} variable now gets the type of whatever
\lstinline{vec} contained.

\item  Range-based for. We have been treating this as the base case,
  for instance in section~\ref{sec:arrayrange}. The C++11 mechanism,
  using an \indexterm{iterator} (section~\ref{sec:iterator-class}) is
  largely obviated.

\item Lambdas. See chapter~\ref{ch:lambda}.

\item Chrono.
\item Variadic templates.

\item Smart pointers. 
\begin{lstlisting}
unique_ptr<int> iptr( new int(5) );    
\end{lstlisting}
This fixes problems with \indexc{auto_ptr}.

\item \indexc{constexpr}
\begin{lstlisting}
constexpr int get_value() {
  return 5*3;
}
\end{lstlisting}
\end{itemize}

\index{C++!C++11|)}

\Level 1 {C++14}
\index{C++!C++14|(textbf}

C++14\index{C++!C++14|textbf} can be considered a bug fix on C++11.
It simplifies a number of things and makes them more elegant.

\begin{itemize}
\item
  Auto return type deduction:
\begin{lstlisting}
  auto f() {
    SomeType something;
    return something;
  }
\end{lstlisting}

\item Generic lambdas (section~\ref{sec:lambda-generic})
\begin{lstlisting}
const auto count = std::count(begin(vec),end(vec),
    [] ( const auto i ) { return i<3; }    
);
\end{lstlisting}
Also more sophisticated capture expressions.

\item \indexc{constexpr}
\begin{lstlisting}
constexpr int get_value() {
  int val = 5;
  int val2 = 3;
  return val*val2
}
\end{lstlisting}
\end{itemize}

\index{C++!C++14|)}

\Level 1 {C++17}

\index{C++!C++17|(textbf}
\begin{itemize}
\item  Optional; section~\ref{sec:std-optional}.
\item Structured binding declarations as an easier way of dissecting
  tuples; section~\ref{sec:tuple}.
\item Init statement in conditionals; section~\ref{sec:if-init}.
\end{itemize}

\index{C++!C++17|)}

\Level 1 {C++20}
\index{C++!C++20|(textbf}
\label{sec:cpp20}

%% https://www.reddit.com/r/cpp/comments/cfk9de/201907_cologne_iso_c_committee_trip_report_the/

\begin{itemize}
\item   \indexterm{modules}: these offer a better interface
  specification than using
  \emph{header files}\index{header file!vs modules}.
\item   \indexterm{coroutines}, another form of parallelism.
\item   \indexterm{concepts} including in the standard library via
  ranges; section~\ref{sec:cpp-concepts}.
\item   \indextermsub{spaceship}{operator} including in the standard library
\item   broad use of normal C++ for direct compile-time programming, without
  resorting to template metaprogramming (see last trip reports)
\item \indexterm{ranges}
\item   \indexterm{calendars} and \indexterm{time zones}
\item   \indextermbus{text}{formatting}
\item \indexc{span}. See section~\ref{sec:gsl-span}.
\item \indextermheader{numbers}. Section~\ref{sec:std-numbers}.
\item Safe integer/unsigned comparison; section~\ref{sec:unsigned-cmp}.
\end{itemize}

Here is a summary with examples:
\url{https://oleksandrkvl.github.io/2021/04/02/cpp-20-overview.html}.

\index{C++!C++20|)}
