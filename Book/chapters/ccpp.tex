% -*- latex -*-
%%%%%%%%%%%%%%%%%%%%%%%%%%%%%%%%%%%%%%%%%%%%%%%%%%%%%%%%%%%%%%%%
%%%%
%%%% This TeX file is part of the course
%%%% Introduction to Scientific Programming in C++/Fortran2003
%%%% copyright 2017-2023 Victor Eijkhout eijkhout@tacc.utexas.edu
%%%%
%%%% ccpp.tex : differences between C++ and C
%%%%
%%%%%%%%%%%%%%%%%%%%%%%%%%%%%%%%%%%%%%%%%%%%%%%%%%%%%%%%%%%%%%%%

\Level 0 {I/O}

There is little employ for \indexc{printf} and
\indexc{scanf}. Use \indexc{cout} (and~\indexc{cerr})
and \indexc{cin} instead. There is also the \indexterm{fmtlib} library.

Chapter~\ref{ch:io}.

\Level 0 {Arrays}

Arrays through square bracket notation are unsafe. They are basically
a pointer, which means they carry no information beyond the memory location.

It is much better
to use \indexc{vector}. Use range-based loops, even if you use
bracket notation.

Chapter~\ref{ch:array}.

Vectors own their data exclusively, so having multiple C~style pointers into
the same data act like so many arrays does not work. For that, use the
\indexc{span}; section~\ref{sec:gsl-span}.

\Level 1 {Vectors from C arrays}

Suppose you have to interface to a C~code that uses
\indexc{malloc}.  Vectors have advantages, such as that they know
their size, you may want to wrap these C~style arrays in a vector
object. This can be done using a \indextermsub{range}{constructor}:
\begin{lstlisting}
vector<double> x( pointer_to_first, pointer_after_last );
\end{lstlisting}

Such vectors can still be used dynamically, but this may give a memory
leak and other possibly unwanted behavior:

\snippetwithoutput{cvector1}{array}{cvector}

\begin{slide}{Vectors from C arrays}
  \label{sl:cvector}
  Use a range constructor to make a vector from a C array:
\begin{lstlisting}
vector<double> x( pointer_to_first, pointer_after_last );
\end{lstlisting}
Note subtleties:
%
\snippetwithoutput{cvector1}{array}{cvector}
\end{slide}

\Level 0 {Dynamic storage}

Another advantage of vectors and other containers is the \indextermdef{RAII}
mechanism, which implies that dynamic storage automatically gets
deallocated when leaving a scope. Section~\ref{sec:stack-heap}. (For
safe dynamic storage that transcends scope, see smart pointers
discussed below.)

RAII stands for `Resource Allocation Is Initialization'. This means
that it is no longer possible to write
\begin{verbatim}
double *x;
if (something1) x = malloc(10);
if (something2) x[0];
\end{verbatim}
which may give a memory error.
Instead, declaration of the
name and allocation of the storage are one indivisible action.

On the other hand:
\begin{lstlisting}
vector<double> x(10);
\end{lstlisting}
declares the variable \n{x}, allocates the dynamic storage, and
initializes it.

\Level 0 {Strings}

A \indextermbus{C}{string} is a character array with a \indexterm{null
  terminator}. On the other hand, a~\indexc{string} is an object
with operations defined on it.

Chapter~\ref{ch:string}.

\Level 0 {Pointers}

Many of the uses for \indextermbus{C}{pointer}s, which are
really addresses, have gone away.
\begin{itemize}
\item Strings are done through \n{std::string}, not character arrays;
  see above.
\item Arrays can largely be done through \n{std::vector}, rather than
  \indexc{malloc}; see above.
\item Traversing arrays and vectors can be done with ranges;
  section~\ref{sec:arrayrange}.
\item To pass an argument
  \emph{by reference}\index{parameter!passing!by reference},
  use a \emph{reference}\index{reference!argument}.
  Section~\ref{sec:passing}.
\item Anything that obeys a scope should be created through a
  \indexterm{constructor}, rather than using \indexc{malloc}.
\end{itemize}

There are some legitimate needs for pointers, such as Objects on the
heap. In that case, use \indexc{shared_ptr} or \indexc{unique_ptr};
section~\ref{sec:shared_ptr}.
The C~pointers are now called \indextermsub{bare}{pointer}s, and they
can still be used for `non-owning' occurrences of pointers.

\Level 1 {Parameter passing}

No longer by address: now true references! Section~\ref{sec:passing}.

\Level 1 {Addresses}

C~programmers are accustomed to using the ampersand for getting the address of
variables, and in C++: objects.
C++~programmers need to beware that it's possible to redefine the ampersand:
\begin{lstlisting}
T operator&(){ /* stuff */ };
\end{lstlisting}
If you absolutely need an address, there is the \indexc{addressof} function.

\Level 0 {Objects}

Objects are structures with functions attached to
them. Chapter~\ref{ch:object}.

\Level 0 {Namespaces}

No longer name conflicts from loading two packages: each can have its
own namespace. Chapter~\ref{ch:namespace}.

\Level 0 {Templates}

If you find yourself writing the same function for a number of types,
you'll love templates. Chapter~\ref{ch:template}.

\Level 0 {Obscure stuff}

\Level 1 {Lambda}

Function expressions.
Chapter~\ref{ch:lambda}.

\Level 1 {Const}

Functions and arguments can be declared const. This helps the
compiler. Section~\ref{sec:constparam}.

\Level 1 {Lvalue and rvalue}

Section~\ref{sec:lrvalue}.

\endinput

You’ll have no destructors, so cleanup is manual. This is most fun
with early-return functions, but it can keep you entertained for all
cases. File handles, memory, and other resources (thread locks,
anyone) are all waiting patiently and silently for you to forget them.

Initialization has be be explicitly called. No constructors either.

Want inheritance? Sure. Write your own vtable (often done with function pointers in a struct).
Instead of templates, you’ll need to abandon type safety and cast back and forth to (void*). Don’t explicitly cast to (void *), because the compiler never warns about explicit or implicit casts to and from (void *).

You’ll also need to make sure you’re using the right library calls - snprintf versus sprintf etc. Hopefully an existing project will be using the right ones.

On the plus side, you’re moving to Linux, and a lot of the tooling available is - while very command-line oriented - very good indeed.

For an IDE, I’d recommend CLion from JetBrains, but I’m told that with sufficient patience, Atom can be encouraged into doing useful stuff.

You’ll find that while the command-line of GDB, the debugger, isn’t very easy to learn to begin with, it’s very powerful, allowing you to do conditional breakpoints with comparative ease.

Valgrind is amazing. Voodoo. It’ll find uninitialized memory, allocation errors, overflows, and leaks - all common and hard to debug issues in C.

The CLang static analyzer is pretty impressive, too.

(copied from
\url{https://www.quora.com/How-should-a-C++-programmer-learn-Linux-C/answer/Dave-Cridland})
