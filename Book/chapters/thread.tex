% -*- latex -*-
%%%%%%%%%%%%%%%%%%%%%%%%%%%%%%%%%%%%%%%%%%%%%%%%%%%%%%%%%%%%%%%%
%%%%
%%%% This TeX file is part of the course
%%%% Introduction to Scientific Programming in C++/Fortran2003
%%%% copyright 2021-2022 Victor Eijkhout eijkhout@tacc.utexas.edu
%%%%
%%%% thread.tex : about threading
%%%%
%%%% THIS FILE IS NO LONGER USE: see concur.tex
%%%%
%%%%%%%%%%%%%%%%%%%%%%%%%%%%%%%%%%%%%%%%%%%%%%%%%%%%%%%%%%%%%%%%

\Level 0 {Threading}

Create a thread:
\begin{lstlisting}
#include <thread>
  
auto somefunc = [] (int i,int j) { /* stuff */ };
std::thread mythread( somefunc, arg1,arg2 );
mythread.join()
\end{lstlisting}

Multiple threads:
\begin{lstlisting}
vector<thread> mythreads;
for ( i=/* stuff */ )
  mythreads.push_back( thread( somefunc,someargs ) );
for ( i=/* stuff */ )
  mythreads[i].join();
\end{lstlisting}

\Level 1 {The current thread}
\label{sec:this-thread}

\begin{lstlisting}
std::this_thread::get_id();
\end{lstlisting}
This is a unique ID, but not like an MPI rank.

There is also a \indexcdef{sleep_for} function for a thread.

The \cppstandard{20} \indexc{jthread} launches a thread
which will join when its destructor is called.
With the creation loop:
\begin{lstlisting}
{
  vector<thread> mythreads;
  for ( i=/* stuff */ )
    mythreads.push_back( thread( somefunc,someargs ) );
}
\end{lstlisting}
the joins happen when the vector goes out of scope.

\Level 1 {Simple example}

Here is a simple hello world example.
Because there is no guarantee on the ordering of when threads
start or end, the output can look messy:
%
\snippetwithoutput{threadsmess}{thread}{hellomess}

We bring order in this message by including a wait.
Note that the thread now executes a function
given by a lambda expression:
%
\snippetwithoutput{threadsnice}{thread}{hellonice}

Of course, in practice you don't synchronize threads through waits.
Read on.

\Level 1 {Data races}

\begin{lstlisting}
std::mutex alock;
alock.lock();
/* critical section */
alock.unlock();
\end{lstlisting}
This has a bunch of problems, for instance if the critical section can throw
an exception.

One solution is \indexcstd{lock_guard}:
\begin{lstlisting}
std::mutex alock;
thread( [] () {
    std::lock_guard<std::mutex> myguard(alock);
    /* critical stuff */
    } );
\end{lstlisting}
The lock guard locks the mutex when it is created,
and unlocks it when it goes out of scope.

For \cppstandard{17}, \indexcstd{scoped_lock}
can do this with multiple mutexes.

Atomic variables exist:
\begin{lstlisting}
std::atomic<int> shared_int;
shared_int++;
\end{lstlisting}

Communication between threads:
\begin{lstlisting}
std::condition_variable somecondition;
// thread 1:
std::mutex alock;
std::unique_lock<std::mutex> ulock(alock);
somecondition.wait(ulock)
// thread 2:
somecondition.notify_one();
\end{lstlisting}

Similar but different:
\begin{lstlisting}
#include <future>
std::future<int> future_computation =
  std::async( [] (int x) { return f(x); },
              100 );
future_computation.get();
\end{lstlisting}

\begin{lstlisting}
std::future_status comp_status;
comp_status = future_computation.wait_for( /* chrono duration */ );
if (comp_status==std::future_status::ready)
  /* computation has finished */
\end{lstlisting}

Waiting for a future to be ready:
%
\snippetwithoutput{cpppromisewait}{thread}{promisewait}
