% -*- latex -*-
%%%%%%%%%%%%%%%%%%%%%%%%%%%%%%%%%%%%%%%%%%%%%%%%%%%%%%%%%%%%%%%%
%%%%
%%%% This TeX file is part of the course
%%%% Introduction to Scientific Programming in C++/Fortran2003
%%%% copyright 2017-2022 Victor Eijkhout eijkhout@tacc.utexas.edu
%%%%
%%%% scalar.tex : fine points of scalar types
%%%% THIS IS INCLUDED IN obscure.tex
%%%%
%%%%%%%%%%%%%%%%%%%%%%%%%%%%%%%%%%%%%%%%%%%%%%%%%%%%%%%%%%%%%%%%

\Level 1 {Integers}
\label{sec:more-ints}

There are several integer types.
First of all they can differ by how many bytes they take up;
section~\ref{sec:shortintlong}.
Next, there are signed and unsigned types;
see section~\ref{sec:unsigned-cmp}.

\Level 2 {Integer precision}
\label{sec:shortintlong}

In addition to \lstinline{int},
there are also \indexcdef{short} and \indexcdef{long} integers.
\verbatimsnippet{intshortlong}

\begin{itemize}
\item A short int is at least 16 bits;
\item An integer is at least 16 bits, which was the case
  in the old days of the \indexterm{DEC PDP-11},
  but nowadays they are commonly 32 bits;
\item A long integer is at least 32 bits, but often~64;
\item A `long long' integer is at least 64 bits.
\item If you need only one byte for your integer, you can use a \lstinline{char};
  see section~\ref{sec:charint}.
\end{itemize}

There are a number of generally accepted \indextermp{data model}
for the definition of these types; see \HPSCref{sec:data-model}.

If you want to determine precisely what the range of
integers or real numbers is that is stored in an \lstinline{int} or \lstinline{float}
variable, you can use \indextermheader{limits}; see section~\ref{sec:limits}.

If you want to dictate how many bits to use, there is
the \indextermheader{cstdint} header, which defines such types
as \indexc{int16_t}, \indexc{int32_t}, \indexc{int64_t}.

\Level 2 {Integer overflow}

From the limited space that an integer takes, it is clear
that there have to be a smallest and largest integer.
Querying such limits on integers is discussed in section~\ref{sec:limits}.

Computations that exceed those limits have an undefined result;
however, since \cppstandard{20} integers are guaranteed to be stored as
`two's complement'; see \HPSCref{sec:int-rep}.

\Level 2 {Unsigned types}
\label{sec:unsigned-cmp}

For the integer types \indexc{int}, \indexc{short}, \indexc{long}
there are unsigned types
\begin{lstlisting}
unsigned int i;
unsigned short s;
unsigned long l;
\end{lstlisting}
which contain nonnegative values.
Consequently they have twice the range:
%
\snippetwithoutput{unsignedlimit}{int}{limit}

(For the mechanism used here, see section~\ref{sec:limits}.)

Unsigned values are fraught with danger. For instance,
comparing them to integers gives counter-intuitive results:
%
\snippetwithoutput{unsignedcomp}{int}{cmp}

For this reason, \cppstandard{20} has introduced utility functions
\lstinline+cmp_equal+ and such (in the \indexc{utility} header)
that do these comparisons correctly.

\Level 1 {Floating point types}

Truncation and precision are tricky things.
As a small illustration, let's do the same computation
in single and double precision.
While the results show the same with the default \lstinline{cout}
formatting,
if we subtract them we see a non-zero difference.

\snippetwithoutput{truncvsf}{basic}{point3}

You can actually explain the size of this difference,
however, we defer discussion of the details
of floating point arithmetic to
\HPSCref[chapter]{ch:arithmetic}.

\Level 1 {Not-a-number}
\label{sec:naninf}

The \indextermbus{IEEE}{754} standard for floating point numbers
states that certain bit patterns correspond to the value
\indextermtt{NaN}: `not a number'.
This is the result of such computations as the square root
of a negative number, or zero divided by zero;
you can also explicitly generate it
with \indexc{quiet_NaN} or \indexc{signalling_NaN}.

\begin{itemize}
\item \n{NaN} is only defined for floating point types:
  the test \indexc{has_quiet_NaN} is false for other types
  such as \indexc{bool} or \indexc{int}.
\item Even through \indexc{complex} is built on top of
  floating point types, there is no \n{NaN} for it.
\end{itemize}

\snippetwithoutput{nanhasnan}{limits}{nan}

\Level 2 {Tests}

There are tests for detecting whether a number is \indextermtt{Inf} or
\indextermtt{NaN}. However, using these may slow a computation down.

\begin{block}{Detection of Inf and NaN}
  The functions \indexc{isinf} and \indexc{isnan} are
  defined for the floating point types (\n{float}, \n{double}, \n{long
    double}), returning a \n{bool}.
\begin{lstlisting}
#include <math.h>
isnan(-1.0/0.0);   // false
isnan(sqrt(-1.0)); // true
isinf(-1.0/0.0);   // true
isinf(sqrt(-1.0)); // false
\end{lstlisting}
\end{block}

\Level 1 {Common numbers}
\label{sec:std-numbers}

\begin{lstlisting}
#include <numbers>
static constexpr float pi = std::numbers::pi;
\end{lstlisting}

