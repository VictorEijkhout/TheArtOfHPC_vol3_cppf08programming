% -*- latex -*-
%%%%%%%%%%%%%%%%%%%%%%%%%%%%%%%%%%%%%%%%%%%%%%%%%%%%%%%%%%%%%%%%
%%%%
%%%% This TeX file is part of the course
%%%% Introduction to Scientific Programming in C++/Fortran2003
%%%% copyright 2017-2022 Victor Eijkhout eijkhout@tacc.utexas.edu
%%%%
%%%% scalar.tex : fine points of scalar types
%%%% THIS IS INCLUDED IN obscure.tex
%%%%
%%%%%%%%%%%%%%%%%%%%%%%%%%%%%%%%%%%%%%%%%%%%%%%%%%%%%%%%%%%%%%%%

%% \input intdetails
\Level 1 {Integers}
%%packtsnippet intdetails
\label{sec:more-ints}

There are several integer types,
differing by the number of bytes they take up,
and by whether they are signed or not.
Here is a systematic discussion.

\Level 2 {The long and short of it}
\label{sec:shortintlong}

In addition to \lstinline{int},
there are also \indexcdef{short}, \indexcdef{long}
and \indexcdef{long long} integers.
These names give some, imperfect, indication
as to their size, and therefore range.
The actual sizes and ranges of these types are implementation defined.
The next section will discuss types with a more precise definition.

\begin{itemize}
\item A short int is at least 16 bits;
\item An integer is at least 16 bits, which was the case
  in the old days of the \indexterm{DEC PDP-11},
  but nowadays they are commonly 32 bits;
\item A long integer is at least 32 bits, but often~64;
  the big exception being the \indexterm{Windows} \acs{OS},
  where \lstinline{long} integers are 32~bits.
\item A \lstinline{long long} integer is at least 64 bits.
\item If you need only one byte for your integer, you can use a \lstinline{char};
  see section~\ref{sec:charint}.
\end{itemize}

\begin{nopackt}
  There are a number of generally accepted \indextermp{data model}
  for the definition of these types; see \HPSCref{sec:data-model}.
\end{nopackt}

All these types are signed integers:
a~$k$-bit integer accomodates a range $-2^{k-1}\ldots 0\ldots 2^{k-1}-1$.
Prefixing these types with the keyword \lstinline{unsigned}
gives nonnegative types with a range~$0\ldots 2N-1$.

If you want to determine precisely the range of
integers (or real numbers, which we discuss later)
that is stored in an \lstinline{int} variable and such,
you can use the \indextermheader{limits} header; see section~\ref{sec:limits}.

\Level 2 {Byte by byte}
\label{sec:unsigned-cmp}

If you want to dictate how many bits to use, there is
the \indextermheader{cstdint} header
containing \emph{fixed width integer types}\index{integer!fixed width types}.
It defines such types
as \indexc{int16_t} and \indexc{uint16_t}:
\begin{lstlisting}
int8_t   // 8 bits
uint8_t  // 8 bits, unsigned
int16_t  // 16 bits
uint16_t // 16 bits, unsigned
int32_t  // 32 bits
uint32_t // 32 bits, unsigned
int64_t  // 64 bits
uint64_t // 64 bits, unsigned
\end{lstlisting}

The following example confirms these sizes by using the \indexc{sizeof} function:
%
\snippetwithoutput{sizeoffixed}{int}{sizeof}

Since \cppstandard{20} integers are guaranteed to be stored as
`two's complement'; see \HPSCref{sec:int-rep}.

Unsigned values are fraught with danger. For instance,
comparing them to integers gives counter-intuitive results:
%
\snippetwithoutput{unsignedcomp}{int}{cmp}

For this reason, \cppstandard{20} has introduced utility functions
\lstinline+cmp_equal+ and such (in the \indexc{utility} header)
that do these comparisons correctly.

\Level 2 {Why different integers?}

Historically, short integers were motivated by a need to save space.
These days, that argument is largely irrelevant.
The argument for short integers now derives from
the limited \indexterm{bandwidth} of processors:
`streaming' type applications
have a performance that is determined by the available bandwidth,
and using short integers effectively doubles that bandwidth.
\begin{packt}
  See section~\ref{sec:bandwidth} for a discussion.
\end{packt}

Long integers are also motivated by the current abundance of
memory and disc space.
Modern applications have
large amounts of memory available to them,
more than can be addressed with a 32-bit integer.
This phenomenon is exacerbated by the existence of
\indexterm{multicore} processors, that can handle
many times the memory of earlier processors in the same time.

%%packtsnippet end

\Level 1 {Floating point types}

Truncation and precision are tricky things.
As a small illustration, let's do the same computation
in single and double precision.
While the results show the same with the default \lstinline{cout}
formatting,
if we subtract them we see a non-zero difference.

\snippetwithoutput{truncvsf}{basic}{point3}

You can actually explain the size of this difference,
however, we defer discussion of the details
of floating point arithmetic to
\HPSCref[chapter]{ch:arithmetic}.

\Level 1 {Not-a-number}
\label{sec:naninf}

The \indextermbus{IEEE}{754} standard for floating point numbers
states that certain bit patterns correspond to the value
\indextermtt{NaN}: `not a number'.
This is the result of such computations as the square root
of a negative number, or zero divided by zero;
you can also explicitly generate it
with \indexc{quiet_NaN} or \indexc{signalling_NaN}.

\begin{itemize}
\item \n{NaN} is only defined for floating point types:
  the test \indexc{has_quiet_NaN} is false for other types
  such as \indexc{bool} or \indexc{int}.
\item Even through \indexc{complex} is built on top of
  floating point types, there is no \n{NaN} for it.
\end{itemize}

\snippetwithoutput{nanhasnan}{limits}{nan}

\Level 2 {Tests}

There are tests for detecting whether a number is \indextermtt{Inf} or
\indextermtt{NaN}. However, using these may slow a computation down.

\begin{block}{Detection of Inf and NaN}
  The functions \indexc{isinf} and \indexc{isnan} are
  defined for the floating point types (\n{float}, \n{double}, \n{long
    double}), returning a \n{bool}.
\begin{lstlisting}
#include <math.h>
isnan(-1.0/0.0);   // false
isnan(sqrt(-1.0)); // true
isinf(-1.0/0.0);   // true
isinf(sqrt(-1.0)); // false
\end{lstlisting}
\end{block}

\Level 1 {Common numbers}
\label{sec:std-numbers}

\begin{lstlisting}
#include <numbers>
static constexpr float pi = std::numbers::pi;
\end{lstlisting}

