% -*- latex -*-
%%%%%%%%%%%%%%%%%%%%%%%%%%%%%%%%%%%%%%%%%%%%%%%%%%%%%%%%%%%%%%%%
%%%%
%%%% This TeX file is part of the course
%%%% Introduction to Scientific Programming in C++/Fortran2003
%%%% copyright 2017-2022 Victor Eijkhout eijkhout@tacc.utexas.edu
%%%%
%%%% infect.tex : exercises for infectious disease simulation
%%%%
%%%%%%%%%%%%%%%%%%%%%%%%%%%%%%%%%%%%%%%%%%%%%%%%%%%%%%%%%%%%%%%%

This section contains a sequence of exercises that builds up to a
somewhat realistic simulation of the spread of infectious
diseases. 

\Level 0 {Model design}

It is possible to model disease propagation statistically, but here we
will build an explicit simulation: we will maintain an explicit
description of all
the people in the population, and track for each of them their status.

We will use a simple model where a person can be:
\begin{itemize}
\item sick: when they are sick, they can infect other people;
\item susceptible: they are healthy, but can be infected;
\item recovered: they have been sick, but no longer carry the disease,
  and can not be infected for a second time;
\item vaccinated: they are healthy, do not carry the disease, and can
  not be infected.
\end{itemize}
In more complicated models a person could be infectious during only
part of their illness, or there could be secondary infections with
other diseases, et cetera. We keep it simple here:
any sick person is can infect others while they are sick.

In the exercises below we will gradually develop a somewhat realistic
model of how the disease spreads from an infectious person. We always
start with just one person infected.
The program will then track the population from day to day,
running indefinitely until none of the population
is sick. Since there is no re-infection, the run will always end.

\Level 1 {Other ways of modeling}

Instead of capturing every single person in code, a~`contact network'
model,
it is possible to use an
\ac{ODE} approach to disease modeling. You would then model the
percentage of infected persons by a single scalar, and derive
relations for that and other scalars~\cite{Anderson:population1979}.

\url{http://mathworld.wolfram.com/Kermack-McKendrickModel.html}

This is known as a `compartmental model', where each of the three SIR
states is a compartment: a~section of the population.
Both the contact network and
the compartmental model capture part of the truth. In fact, they can
be combined. We can consider a country as a set of cities, where
people travel between any pair of cities. We then use a compartmental
model inside a city, and a contact network between cities.

In this project we will only use the network model.

\Level 0 {Coding}

\Level 1 {The basics}

We start by writing code that models a single person. The main methods
serve to infect a person, and to track their state. We need to have
some methods for inspecting that state. 

The intended output looks something like:
\begin{verbatim}
On day 10, Joe is susceptible
On day 11, Joe is susceptible
On day 12, Joe is susceptible
On day 13, Joe is susceptible
On day 14, Joe is sick (5 to go)
On day 15, Joe is sick (4 to go)
On day 16, Joe is sick (3 to go)
On day 17, Joe is sick (2 to go)
On day 18, Joe is sick (1 to go)
On day 19, Joe is recovered
\end{verbatim}

\begin{exercise}
  \label{ex:infect:person}
  Write a \n{Person} class with methods:
  \begin{itemize}
  \item \n{status_string()} : returns a description of the
    person's state as a \n{string};
  \item \n{update()} : update the person's status to the next day;
  \item \n{infect(n)} : infect a person, with the disease to run for
    $n$ days;
  \item \n{is_stable()} : return a \n{bool} indicating whether the
    person has been sick and is recovered.
  \end{itemize}
\end{exercise}

Your main program could for instance look like:
\verbatimsnippet{infect0}

Here is a suggestion how you can model 
disease status. Use a single integer with the following interpretation:
\begin{itemize}
\item healthy but not vaccinated, value~$0$,
\item recovered, value~$-1$,
\item vaccinated, value~$-2$,
\item and sick, with $n$ days to go before recovery,  modeled by value~$n$.
\end{itemize}
The \n{Person::update} method then updates this integer.

\begin{remark}
  Consider a proint of programming style.
  Now that you've modeled the state of a person with an integer,
  you can use that as
\begin{lstlisting}
void infect(n) {
    if (state==0)
        state = n;
}
\end{lstlisting}
But you can also write
\begin{lstlisting}
bool is_susceptible() {
  return state==0;
}
void infect(n) {
    if (is_susceptible())
        state = n;
}
\end{lstlisting}
Which do you prefer and why?
\end{remark}

\Level 1 {Population}

Next we need a \n{Population} class. Implement a population as a \n{vector}
consisting of \n{Person} objects. Initially we only infect one person, and there
is no transmission of the disease.

The trace output should look something like:
\begin{verbatim}
Size of
population?
In step   1 #sick:    1 :  ? ? ? ? ? ? ? ? ? ? + ? ? ? ? ? ? ? ? ?
In step   2 #sick:    1 :  ? ? ? ? ? ? ? ? ? ? + ? ? ? ? ? ? ? ? ?
In step   3 #sick:    1 :  ? ? ? ? ? ? ? ? ? ? + ? ? ? ? ? ? ? ? ?
In step   4 #sick:    1 :  ? ? ? ? ? ? ? ? ? ? + ? ? ? ? ? ? ? ? ?
In step   5 #sick:    1 :  ? ? ? ? ? ? ? ? ? ? + ? ? ? ? ? ? ? ? ?
In step   6 #sick:    0 :  ? ? ? ? ? ? ? ? ? ? - ? ? ? ? ? ? ? ? ?
Disease ran its course by step 6
\end{verbatim}

\begin{remark}
  Such a display is good for a sanity check on your program behavior.
  If you include such displays
  in your writeup, make sure to use a monospace font, and don't use a population
  size that needs line wrapping. In further testing, you should use large populations,
  but do not include these displays.
\end{remark}

\begin{exercise}
  \label{ex:infect:notransfer}
  Program a population without infection.
  \begin{itemize}
  \item Write the \n{Population} class. The constructor takes the number of people:
\begin{lstlisting}
Population population(npeople);  
\end{lstlisting}
  \item Write a method that infects a random person:
\begin{lstlisting}
population.random_infection();
\end{lstlisting}
For the `random' part you can use the C~language random number generator
(section~\textbookref{sec:crand}),
or the new \ac{STL} one in section~\textbookref{sec:stl:random}.
  \item Write a method \n{count_infected} that counts how many people are infected.
  \item Write an \n{update} method that updates all persons in the population.
  \item Loop the \n{update} method until no people are infected: the
    \n{Population::update} method should apply \n{Person::update} to
    all person in the population.
  \end{itemize}
\item Write a routine that displays the state of the popular, using
  for instance: \n{?}~for susceptible, \n{+}~for infected, \n{-}~for recovered.
\end{exercise}

\Level 1 {Contagion}

This past exercise was too simplistic: the original patient zero was
the only one who ever got sick.
Now let's incorporate contagion, and investigate the spread of the disease
from a single infected person.

We start with a very simple model of infection.

\begin{exercise}
  \label{ex:infect:1}
  Read in a number $0\leq p \leq 1$ representing the probability of
  disease transmission upon contact.
\begin{lstlisting}
population.set_probability_of_transfer(probability);  
\end{lstlisting}
  Incorporate this into the
  program: in each step the direct neighbours of an infected person
  can now get sick themselves.
  Run a number of simulations with population sizes and contagion
  probabilities. Are there cases where people escape getting sick?
\end{exercise}

\begin{exercise}
  \label{ex:infect:2}
  Incorporate vaccination: read another number representing the
  percentage of people that has been vaccinated. Choose those members
  of the population randomly.

  Describe the effect of vaccinated people on the spread of the
  disease. Why is this model unrealistic?
\end{exercise}

\Level 1 {Spreading}
\label{sec:infect-spread}

To make the simulation more realistic, we let every sick person come
into contact with a fixed number of random people every day. This
gives us more or less the \indexterm{SIR model};
\url{https://en.wikipedia.org/wiki/Epidemic_model}.

Set the number of people that a person comes into contact with, per
day, to~6 or so. (You can also let this be an upper bound for a random
value, but that
does not essentially change the simulation.) You have already programmed the probability that a
person who comes in contact with an infected person gets sick themselves.
Again start the simulation with a single infected person.

\begin{exercise}
  \label{ex:infect:3}
  Code the random interactions. Now run a number of simulations varying
  \begin{itemize}
  \item The percentage of people vaccinated, and
  \item the chance the disease is transmitted on contact.
  \end{itemize}
  Record how long the disease runs through the population. With a
  fixed number of contacts and probability of transmission, how is
  this number of function of the percentage that is vaccinated?

  Report this function as a table or graph. Make sure you have enough
  data points for a meaningful conclusion. Use a realistic population size.
  You can also do multiple runs and report the average, to even out
  the effect of the random number generator.
\end{exercise}
\begin{comment}
\begin{verbatim}
for p in 0 10 25 50 75 90 95 98 ; do (echo 400 ; echo .2 ; echo $p ) 
    | ./infect3 ; done


An vaccination percentage of 0 leads to a maximum number of infected
of 17052 or 42.6325 percent
The disease ran its course in 38 days
susceptible people unaffected: 17295 out of 40000

An vaccination percentage of 10 leads to a maximum number of infected
of 14719 or 36.8 percent
The disease ran its course in 36 days
susceptible people unaffected: 15797 out of 36043

An vaccination percentage of 25 leads to a maximum number of infected
of 10823 or 27.06 percent
The disease ran its course in 37 days
susceptible people unaffected: 13576 out of 29993

An vaccination percentage of 50 leads to a maximum number of infected
of 5075 or 12.69 percent
The disease ran its course in 52 days
susceptible people unaffected: 9925 out of 20017

An vaccination percentage of 75 leads to a maximum number of infected
of   0 or 0.0025 percent
The disease ran its course in 6 days
susceptible people unaffected: 9922 out of 9922

An vaccination percentage of 90 leads to a maximum number of infected
of   0 or 0.0025 percent
The disease ran its course in 6 days
susceptible people unaffected: 3971 out of 3971

An vaccination percentage of 95 leads to a maximum number of infected
of   0 or 0.0025 percent
The disease ran its course in 6 days
susceptible people unaffected: 1990 out of 1990

An vaccination percentage of 98 leads to a maximum number of infected
of   0 or 0.0025 percent
The disease ran its course in 6 days
susceptible people unaffected: 843 out of 843
\end{verbatim}
\end{comment}

\begin{exercise}
  Investigate the matter of `herd immunity': if enough people are
  vaccinated, then some people who are not vaccinated will still never get
  sick. Let's say you want to have this probability over 95
  percent. Investigate the percentage of vaccination that is needed
  for this as a function of the contagiousness of the disease.

  As in the previous exercise, make sure your data set is large enough.
\end{exercise}

\begin{remark}
  The screen output you used above is good for sanity checks on small problems.
  However, for realistic simulations you have to think what is a realistic population size.
  If your university campus is a population where random people are likely to meet each other,
  what would be a population size to model that? How about the city where you live?

  Likewise, if you test different vaccination rates, what granularity do you use?
  With increases of 5 or 10 percent you can print all results to you screen,
  but you may miss things. Don't be afraid to generate large amount of data
  and feed them directly to a graphing program.
\end{remark}

\Level 1 {Mutation}

The Covid years have shown how important mutations of an original virus can be.
Next, you can include mutation in your project. We model this as follows:
\begin{itemize}
\item Every so many transmissions, a~virus will mutate into a new variant.
\item A~person who has recovered from one variant is still susceptible to other variants.
\item For simplicity assume that each variant leaves a person sick the same number of days, and
\item Vaccination is all-or-nothing: one vaccine is enough to protect against all variant;
\item On the other hand, having recovered from one variant is not protection against others.
\end{itemize}

Implementation-wise speaking, we model this as follows:
\begin{itemize}
\item There is a global \lstinline{variants} counter for new virus variants,
  and a global \lstinline{transmissions} counter.
\item Every time a person infects another, the newly infected person gets
  a new \lstinline{Disease} object, with the current variant,
  and the transmissions counter is updated.
\item There is a parameter that determines after how many transmissions the disease mutates.
  If there is a mutation, the global \lstinline{variants} counter is updated,
  and from that point on, every infection is with the new variant.
  (Note: this is not very realistic. You are free to come up with a better model.)
\item A each \lstinline{Person} object has a vector of variants that they are recovered from;
  recovery from one variant only makes them immune from that specific variant, not from others.
\end{itemize}

\begin{exercise}
  Add mutation to your model. Experiment with the mutation rate:
  as the mutation rate increases, the disease should stay in the population longer.
  Does the relation with vaccination rate change that you observed before?
\end{exercise}

\Level 1 {Diseases without vaccine: Ebola and Covid-19}

\emph{This section is optional, for bonus points}

The project so far applies to diseases for which a vaccine is available,
such as MMR for measles, mumps and rubella.
The analysis becomes different if no vaccine exists, such
as is the case for \indexterm{ebola} and \indexterm{covid-19},
as of this writing.

Instead, you need to incorporate `social distancing' into your code:
people do not get in touch with random others anymore,
but only those in a very limited social circle.
Design a model distance function, and explore various settings.

The difference between ebola and covid-19 is how long an
infection can go unnoticed: the \indexterm{incubation period}.
With ebola, infection is almost
immediately apparent, so such people are removed from the
general population and treated in a hospital.
For covid-19, a person can be infected, and infect others,
for a number of days before they are sequestered from the population.

Add this parameter to your simulation and explore the behavior
of the disease as a function of it.

\Level 0 {Ethics}

The subject of infectious diseases and vaccination
is full of ethical questions.
The main one is \emph{The chances of something happening to me
are very small, so why shouldn't I bend the rules a little?}.
This reasoning is most often applied to vaccination,
where people for some reason or other refuse to get vaccinated.

Explore this question and others you may come up with:
it is clear that everyone bending the rules will have disastrous
consequences, but what if only a few people did this?

\Level 0 {Project writeup and submission}

\Level 1 {Program files}

In the course of this project you have written more than one main
program, but some code is shared between the multiple programs.
Organize your code with one file for each main program, and a single
`library' file with the class methods.

You can do this two ways:
\begin{enumerate}
\item You make a `library' file, say \n{infect_lib.cc},
  and your main programs each have a line
\begin{lstlisting}
#include "infect_lib.cc"
\end{lstlisting}
This is not the best solution, but it is acceptable for now.
\item The better solution requires you to use
  \indextermsub{separate}{compilation} for building the program, and you
  need a \indexterm{header} file.
  You would now have \n{infect_lib.cc} which is compiled separately,
  and \n{infect_lib.h} which is included both in the library file and the main program:
\begin{lstlisting}
#include "infect_lib.h"
\end{lstlisting}
\begin{inbook}
  See section~\textbookref{sec:hfile} for more information.
\end{inbook}
\end{enumerate}

Submit all source files with instructions on how to build all the main
programs. You can put these instructions in a file with a descriptive
name such as \n{README} or \n{INSTALL}, or you can use a
\indexterm{makefile}.

\Level 1 {Writeup}

In the writeup, describe the `experiments' you have performed and the
conclusions you draw from them. The exercises above give you a number
of questions to address.

For each main program, include some sample output, but note that this
is no substitute for writing out your conclusions in full sentences.

The exercises in section~\textbookref{sec:infect-spread}
ask you to explore the program behavior as a
function of one or more parameters. Include a table to report on the
behavior you found. You can use Matlab or Matplotlib in Python (or
even Excell) to plot your data, but that is not required.

\Level 0 {Bonus: mathematical analysis}

The SIR model can also be modeled through coupled difference or differential
equations. 
\begin{enumerate}
\item The number $S_i$ of susceptible people at time~$i$ decreases by a fraction
  \[ S_{i+1} = S_i(1-\lambda_i\,dt)\]
  where $\lambda_i$ is the product of the number of infected people
  and a constant that reflects the number of meetings and the infectiousness
  of the disease. We write:
  \[ S_{i+1} = S_i(1-\lambda I_i\,dt)\]
\item The number of infected people similarly increases by $\lambda S_iI_i$,
  but it also decreases by people recovering (or dying):
  \[ I_{i+1} = I_i(1+\lambda S_i\,dt-\gamma\,dt). \]
\item Finally, the number of `removed' people equals that last term:
  \[ R_{i+1} = R_i(1+\gamma I_i). \]
\end{enumerate}

\begin{exercise}
  Code this scheme. What is the effect of varying $dt$?
\end{exercise}

\begin{exercise}
  For the disease to become an epidemic, the number of newly infected
  has to be larger than the number of recovered. That is,
  \[ \lambda S_iI_i - \gamma I_i>0 \Leftrightarrow S_i>\gamma/\lambda. \]
  Can you observe this in your simulations?
\end{exercise}

The parameter $\gamma$ has a simple interpretation. Suppose that a person
stays ill for $\delta$ days before recovering.
If $I_t$ is relatively stable, that means every day the same number of people
get infected as recover, and therefore a $1/\delta$ fraction of people
recover each day. Thus, $\gamma$~is the reciprocal of the duration of the infection
in a given person.

\endinput

\begin{exercise}
  You can make the model more realistic by letting vaccination be
  only partly effective. For instance, 50\% of people got the flu
  vaccine, but it was only 40\% effective; 90\%~of people have the
  measles vaccine, and it is about 97\%
  effective. (\url{https://www.cdc.gov/nchs/fastats/immunize.htm}) How
  does your model function in this case? Keep in mind that different
  diseases have different degrees of infectiousness
  (\url{https://en.wikipedia.org/wiki/Basic_reproduction_number}).
\end{exercise}
