% -*- latex -*-
%%%%%%%%%%%%%%%%%%%%%%%%%%%%%%%%%%%%%%%%%%%%%%%%%%%%%%%%%%%%%%%%
%%%%
%%%% This TeX file is part of the course
%%%% Introduction to Scientific Programming in C++/Fortran2003
%%%% copyright 2017-2023 Victor Eijkhout eijkhout@tacc.utexas.edu
%%%%
%%%% stl.tex : about the standard template library
%%%%
%%%%%%%%%%%%%%%%%%%%%%%%%%%%%%%%%%%%%%%%%%%%%%%%%%%%%%%%%%%%%%%%

Concurrency is a difficult subject.
It is both like and unlike parallelism,
which is a prime concern on modern scientific computing,
so it is both relevant, and not-so relevant,
to the target audience of this book.

In a way, parallelism is clear: it's about things happening at the same time,
for instance on the multiple cores of your modern processor.
The benefit of parallelism is that your code runs faster,
and the challenge is how to write your program
to express that there are indeed things that can be done simultaneously.

Etymologically, `concurrency' also seems to be about things
happening simultaneously. However, it predates actual parallel processors:
concurrency is important in \acp{OS} where even on a single processor
you have a program running, simultaneously with processes that are printing,
checking mail, et cetera.

Thus, concurrency can be defined as the study of activities that have no clear
temporal relation. The main issue of study is then how to deal with `shared state':
objects that are accessed in no clear order by more than one process.

Concurrency is important in such areas as \acp{OS}, \indexterm{database}s,
and \indexterm{web server}s.
The main mechanism for concurrency is threading, which we cover in this chapter.
The \cppstandard{20} standard also added \indexterm{co-routine}s,
which we do not cover here.

For scientific computing, parallelism is more important than concurrency;
for this we refer to Volume~2 of this series,
which covers the \indexterm{MPI} and \indexterm{OpenMP} systems,
both of which can easily be used from~C++.

\Level 0 {Thread creation}
\label{sec:cpp-thread}

This uses the \indexheader{thread} header:
\begin{lstlisting}
#include <thread>
\end{lstlisting}

A thread is an object of class \lstinline+std::thread+,
and creating it you immediately begin its execution.
The thread constructor takes at least one argument,
a \indexterm{callable}
(a~function pointer or a lambda),
and optionally the arguments of that function-like object.

The environment that calls the thread needs to call \indexc{join}
to ensure the completion of the thread.
Until you do so, there is no guarantee that the thread's
function has run.

As a very simple example, here we have a thread that sleeps for a second.
Note: this uses the \ac{OS} \indexunix{sleep} function;
there are better mechanisms; see section~\ref{sec:this-thread}.
%
\snippetwithoutput{waitthreadblock}{thread}{block}
%
An example with a function that takes arguments:

\begin{lstlisting}
#include <thread>
  
auto somefunc = [] (int i,int j) { /* stuff */ };
std::thread mythread( somefunc, arg1,arg2 );
mythread.join()
\end{lstlisting}

\Level 1 {Multiple threads}

Creating a single thread is not very useful.
Often you will create multiple threads
that will subdivide work,
and then wait until they all finish.
\begin{lstlisting}
vector<thread> mythreads;
for ( i=/* stuff */ )
  mythreads.push_back( thread( somefunc,someargs(i) ) );
for ( i=/* stuff */ )
  mythreads[i].join();
\end{lstlisting}
Note that we create the vector of threads with size zero,
and use \lstinline{push_back} to fill it,
since a thread starts executing.
Creating the vector immediately with the right size
would have created a bunch of no-op threads.

\begin{exercise}
  If you are very concerned about the cost of allocation,
  how can you create the space for the threads, without
  creating the threads.
\end{exercise}

Here is a simple hello world example.
Because there is no guarantee on the ordering of when threads
start or end, the output can look messy:
%
\snippetwithoutput{threadsmess}{thread}{hellomess}

(Note the call to \indexc{emplace_back}:
because of \indexterm{perfect forwarding}
it can invoke the constructor on the \lstinline{thread} arguments.)

We bring order in this message by including a wait.
As as separate issue, the thread now executes a function
given by a lambda expression:
%
\snippetwithoutput{threadsnice}{thread}{hellonice}

Of course, in practice you don't synchronize threads through waits.
Read on.

\Level 1 {Asynchronous tasks}

One problem with threads is how to return data.
You could solve that with capturing a variable by reference,
but that is not very elegant.
A~better solution would be if you could ask a thread
`what is the thing you just calculated'.

For this we have the \indexcstd{future},
from the header \indextermheader{future},
templated with the return type. Thus, a
\begin{lstlisting}
std::future<int>
\end{lstlisting}
will be an \lstinline{int}, somewhere in the future.
You retrieve the value with \indexc{get}:

\verbatimsnippet{cppasyncone}

An example with multiple futures:

\verbatimsnippet{cppasyncvec}

One problem with \lstinline{async} is that the task need not execute on a new thread:
the runtime can decide to execute it on the calling thread,
only when the \lstinline{get} call is made.
To force a new thread to be spawned immediately, use
\begin{lstlisting}
auto fut = std::async
    ( std::launch:async, fn, arg1, arg2 );
\end{lstlisting}
%% Lazy evaluation on the calling thread can be explicitly
%% specified with \lstinline+std::launch:async+.

\Level 1 {Return results: futures and promises}

Explicit use of promises and futures is on a lower level than \indexc{async}.

\verbatimsnippet{cpppromise}

\begin{multicols}{2}
  \verbatimsnippet{cpppromisevec}
\end{multicols}

\Level 1 {The current thread}
\label{sec:this-thread}

\begin{lstlisting}
std::this_thread::get_id();
\end{lstlisting}
This is a unique ID, but you can not derive any further information from it.
For instance, it is not related to \ac{OS} process IDs,
or the number of cores of your processor.

There is also a \indexcdef{sleep_for} function for a thread:
%
\verbatimsnippet{threadsnice}

\Level 1 {More thread stuff}

The \cppstandard{20} \indexc{jthread} launches a thread
which will join when its destructor is called.
With the creation loop:
\begin{lstlisting}
{ // start a scope
  vector<thread> mythreads;
  for ( i=/* stuff */ )
    mythreads.push_back( thread( somefunc,someargs ) );
} // end of scope
\end{lstlisting}
The joins now happen without an explicit call when the vector goes out of scope.
This also solves the problem that explicit \indexc{join} calls are ordered.

\Level 0 {Data races}

An important topic in concurrency is that of
\indextermp{data race}:
the phenomenon that multiple accesses to a single data item
are not temporally or causally ordered,
for instance because the accesses are from threads
that are simultaneously active.

\begin{lstlisting}
std::mutex alock;
alock.lock();
/* critical section */
alock.unlock();
\end{lstlisting}
This has a bunch of problems, for instance if the critical section can throw
an exception.

One solution is \indexcstd{lock_guard}:
\begin{lstlisting}
std::mutex alock;
thread( [] () {
    std::lock_guard<std::mutex> myguard(alock);
    /* critical stuff */
    } );
\end{lstlisting}
The lock guard locks the mutex when it is created,
and unlocks it when it goes out of scope.

For \cppstandard{17}, \indexcstd{scoped_lock}
can do this with multiple mutexes.

Atomic variables exist in the \indexheader{atomic} header:
\begin{lstlisting}
#include <atomic>  
std::atomic<int> shared_int;
shared_int++;
\end{lstlisting}

Communication between threads:
\begin{lstlisting}
std::condition_variable somecondition;
// thread 1:
std::mutex alock;
std::unique_lock<std::mutex> ulock(alock);
somecondition.wait(ulock)
// thread 2:
somecondition.notify_one();
\end{lstlisting}

Similar but different:
\begin{lstlisting}
#include <future>
std::future<int> future_computation =
  std::async( [] (int x) { return f(x); },
              100 );
future_computation.get();
\end{lstlisting}

\begin{lstlisting}
std::future_status comp_status;
comp_status = future_computation.wait_for( /* chrono duration */ );
if (comp_status==std::future_status::ready)
  /* computation has finished */
\end{lstlisting}

\Level 0 {Synchronization}

Threads run concurrent with the environment (or thread)
that created them.
That means that there is no temporal ordering between actions
prior to synchronization with \lstinline+std::thread::join+.

In this example, the concurrent updates of \n{counter}
form a \indexterm{data race}:
%
\snippetwithoutput{threadracevar}{thread}{race}
%
Formally, this program has \acf{UB},
which you see reflected in the different final values.

The final value can be fixed by declaring the counter
as \lstinline+std::atomic+:
%
\snippetwithoutput{threadatomicvar}{thread}{atomic}
%
Note that the accesses by the main and the thread
are still not predictable, but that is a feature,
not a bug, and definitely not \ac{UB}.
