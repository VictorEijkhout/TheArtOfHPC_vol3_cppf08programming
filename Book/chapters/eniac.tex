% -*- latex -*-
%%%%%%%%%%%%%%%%%%%%%%%%%%%%%%%%%%%%%%%%%%%%%%%%%%%%%%%%%%%%%%%%
%%%%
%%%% This TeX file is part of the course
%%%% Introduction to Scientific Programming in C++/Fortran2003
%%%% copyright 2017-2022 Victor Eijkhout eijkhout@tacc.utexas.edu
%%%%
%%%% eniac.tex : ballistics project
%%%%
%%%%%%%%%%%%%%%%%%%%%%%%%%%%%%%%%%%%%%%%%%%%%%%%%%%%%%%%%%%%%%%%

\textbf{THIS PROJECT IS NOT READY FOR PRIME TIME}

\Level 0 {Introduction}

From \url{https://encyclopedia2.thefreedictionary.com/ballistics}

\begin{quotation}
Ballistics 

the science of the movement of artillery shells, bullets, mortar
shells, aerial bombs, rocket artillery projectiles and missiles,
harpoons, and so on. Ballistics is a technical military science based
on a set of physics and mathematics disciplines. Interior ballistics
is distinguished from exterior ballistics.

Interior ballistics is concerned with the movement of a projectile (or
other body whose mechanical freedom is restricted by certain
conditions) in the bore of a gun under the influence of powder gases
as well as the rules governing other processes occurring in the bore
or in the chamber of a powder rocket during firing. Interior
ballistics views the firing as a complex process of rapid
transformation of the powder’s chemical energy into heat energy and
then into the mechanical work of displacing the projectile, charge,
and recoil parts of the gun. In interior ballistics the different
periods that are distinguished in the firing are the preliminary
period, which is from the start of the powder combustion until the
projectile begins to move; the first (primary) period, which is from
the start of projectile movement until the end of powder combustion;
the second period, which is from the end of powder combustion until
the moment that the projectile leaves the bore (the period of
adiabatic expansion of the gases); and the period of the aftereffect
of the powder gases on the projectile and barrel. The laws governing
the processes related to this last period are dealt with in a special
division of ballistics, known as intermediate ballistics. The end of
the period of aftereffect on the projectile divides the phenomena
studied by interior and exterior ballistics.

The main divisions of interior ballistics are “pyrostatics,”
“pyrodynamics,” and ballistic gun design. Pyrostatics is the study of
the laws of powder combustion and gas formation during the combustion
of powder in a constant volume in which the effect of the chemical
composition of the powder and its forms and dimensions on the laws of
combustion and gas formation is determined. Pyrodynamics is concerned
with the study of the processes and phenomena that take place in the
bore during firing and the determination of the relationships between
the design features of the bore, the conditions of loading, and
various physical-chemical and mechanical processes that occur during
firing. On the basis of a consideration of these processes and also of
the forces operating on the projectile and barrel, a system of
equations is established that describes the firing process, including
the basic equation of interior ballistics, which relates the magnitude
of the burned part of the charge, the pressure of powder gases in the
bore, the velocity of the projectile, and the length of the path it
has traveled. The solution of this system and the discovery of the
relationship between change in the pressure $\rho$ of the powder
gases, the velocity v of the projectile, and other parameters on path
l of the projectile and the time it has moved along the bore is the
first main (direct)

to solve this problem the analytic method, numerical integration
methods (including those based on computers), and tabular methods are
used. In view of the complexity of the firing process and insufficient
study of particular factors, certain assumptions are made. The
correction formulas of interior ballistics are of great practical
significance; they make it possible to determine the change in muzzle
velocity of the projectile and maximum pressure in the bore when there
are changes made in the loading conditions.

Ballistic gun design is the second main (correlative) problem of
interior ballistics. By it are determined the design specifications of
the bore and the loading conditions under which a projectile of given
caliber and mass will obtain an assigned (muzzle) velocity in
flight. The curves of change in the pressure of the gases in the bore
and of the velocity of the projectile along the length of the barrel
and in time are calculated for the variation of the barrel selected
during designing. These curves are the initial data in designing the
artillery system as a whole and the ammunition for it. Internal
ballistics also includes the study of the firing process in the rifle,
in cases when special and combined charges are used, in systems with
conical barrels, and in systems in which gases are exhausted during
powder combustion (high-low pressure guns and recoilless guns,
infantry mortars). Another important division is the interior
ballistics of powder rockets, which has developed into a special
science. The main divisions of the interior ballistics of powder
rockets are pyrostatics of a semiclosed space, which consider the laws
of powder combustion at comparatively low and constant pressure; the
solution of the basic problem of the interior ballistics of powder
rockets, which is to determine (under set loading conditions) the
rules of variation in pressure of the powder gases in the chamber with
regard to time and to determine the rules of variation in thrust
necessary to ensure the required rocket velocity; the ballistic design
of powder rockets, which involves determining the energy-producing
characteristics of the powder, the weight and form of the charge, and
the design parameters of the nozzle which ensure, with an assigned
weight of the rocket’s warhead, the necessary thrust force during its
operation.

Exterior ballistics is concerned with the study of the movement of
unguided projectiles (mortar shells, bullets, and so on) after they
leave the bore (or launcher) and the factors that affect this
movement. It includes basically the study of all the elements of
motion of the projectile and of the forces that act upon it in flight
(the force of air resistance, the force of gravity, reactive force,
the force arising in the aftereffect period, and so on); the study of
the movement of the center of mass of the projectile for the purpose
of calculating its trajectory (see Figure 2) when there are set
initial and external conditions (the basic problem of exterior
ballistics); and the determination of the flight stability and
dispersion of projectiles. Two important divisions of exterior
ballistics are the theory of corrections, which develops methods of
evaluating the influence of the factors that determine the
projectile’s flight on the nature of its trajectory, and the technique
for drawing up firing tables and of finding the optimal exterior
ballistics variation in the designing of artillery systems. The
theoretical solution of the problems of projectile movement and of the
problems of the theory of corrections amounts to making up equations
for the projectile’s movement, simplifying these equations, and
seeking methods of solving them. This has been made significantly
easier and faster with the appearance of computers. In order to
determine the initial conditions—that is, initial velocity and angle
of departure, shape and mass of the projectile—which are necessary to
obtain a given trajectory, special tables are used in exterior
ballistics. The working out of the technique for drawing up a firing
table involves determining the optimal combination of theoretical and
experimental research that makes it possible to obtain firing tables
of the required accuracy with the minimal expenditure of time. The
methods of exterior ballistics are also used in the study of the laws
of movement of spacecraft (during their movement without the influence
of controlling forces and moments). With the appearance of guided
missiles, exterior ballistics played a major part in the formation and
development of the theory of flight and became a particular instance
of this theory.

The appearance of ballistics as a science dates to the 16th
century. The first works on ballistics are the books by the Italian
N. Tartaglia, A New Science (1537) and Questions

and Discoveries Relating to Artillery Fire (1546). In the 17th century
the fundamental principles of exterior ballistics were established by
Galileo, who developed the parabolic theory of projectile movement, by
the Italian E. Torricelli, and by the Frenchman M. Mersenne, who
proposed that the science of the movement of projectiles be called
ballistics (1644). I. Newton made the first investigations of the
movement of a projectile, taking air resistance into consideration
(Mathematical Principles of Natural Philosophy,1687). During the 17th
and 18th centuries the movement of projectiles was studied by the
Dutchman C. Huygens, the Frenchman P. Varignon, the Englishman
B. Robins, the Swiss D. Bernoulli, the Russian scientist L. Eiler, and
others. The experimental and theoretical foundations of interior
ballistics were laid in the 18th century in works of Robins,
C. Hutton, Bernoulli, and others. In the 19th century the laws of air
resistance were established (the laws of N. V. Maievskii and
N. A. Zabudskii, Havre’s law, and A. F. Siacci’s law).
\end{quotation}

The numerical analysis of ballistics calculations on the \indexterm{ENIAC}
are described in \cite{eniac-firing}.

\Level 1 {Physics}

These are the governing equations:
\begin{equation}
  \begin{array}{ll}
    x''&= -E(x'-w_x) + 2\Omega \cos L \sin \alpha y'\\
    y''&= -Ey' -g - 2\Omega \cos L \sin \alpha x'\\
    z''&= -E(z'-w_z) +2\Omega \sin L x' + 2\Omega \cos L \cos \alpha y'
  \end{array}
\end{equation}
where 
\begin{itemize}
\item $x,y,z$ are the quantities of interest: distance, altitude, sideways displacement;
\item $w_x,w_z$ are wind speed;
\item $E$ is a complicated function of~$y$, involving air density and speed of sound;
\item $\alpha$ is the azimuth, that is, angle of firing;
\item All other quantities are needed for physical realism, but will be set $\equiv1$
  in this coding exercise.
\end{itemize}

\Level 1 {Numerical analysis}

This uses an Euler-MacLaurin scheme of third order:
\begin{equation}
  f_1-f_0 = \frac{1}{2} (f'_0+f'_1)h + \frac{1}{12}(f'_0-f'_1)h^2 + O(h^5)
\end{equation}
which works out to
\begin{equation}
  \begin{array}{ll}
    \bar x'_1 &= x'_0+x''_0\Delta t\\
    x_1       &= x_0+x'_0\Delta t\\
    x'_1      &= x'_0 + (x''_0+\bar x''_1) \frac{\Delta t}{2}\\
    x_1       & = x_0 ( x'_0 + \bar x'_1 ) \frac{\Delta t}{2} +
                  (x''_0-\bar x''_1) \frac{\Delta t^2}{12}\\
  \end{array}
\end{equation}
