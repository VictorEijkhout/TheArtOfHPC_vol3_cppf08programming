% -*- latex -*-
%%%%%%%%%%%%%%%%%%%%%%%%%%%%%%%%%%%%%%%%%%%%%%%%%%%%%%%%%%%%%%%%
%%%%
%%%% This TeX file is part of the course
%%%% Introduction to Scientific Programming in C++/Fortran2003
%%%% copyright 2017-2022 Victor Eijkhout eijkhout@tacc.utexas.edu
%%%%
%%%% cpp.tex : about the C preprocessor
%%%%
%%%%%%%%%%%%%%%%%%%%%%%%%%%%%%%%%%%%%%%%%%%%%%%%%%%%%%%%%%%%%%%%

In your source files you have seen lines starting with a hash sign,
like
\begin{lstlisting}
#include <iostream>
\end{lstlisting}
Such lines are interpreted by the
%
\emph{C~preprocessor}\index{C preprocessor|see{preprocessor}}.

We will see some of its more common uses here.

\Level 0 {Include files}

The \indexpragma{include} pragma causes the named file to be included.
That file is in fact an ordinary text file, stored somewhere in your system.
As a result,
your source file is transformed to another source file, in a
source-to-source translation stage,
and only that second file is actually compiled by the
%
\emph{compiler}\index{compiler!and preprocessor}.

Normally, this intermediate file with all included literally included
is immediately destroyed again, but
in rare cases you may want to dump it for debugging.
See your compiler documentation for how to generate this file.

\Level 1 {Kinds of includes}

While you can include any kind of text file,
normally you include a
\indexterm{header file}
at the start of your source.

There are two kinds of includes
\begin{enumerate}
\item The file name can be included in angle brackets,
\begin{lstlisting}
#include <vector>
\end{lstlisting}
  which is typically used for system headers that are part of the
  compiler infrastructure;
\item the name can also be in double quotes,
\begin{lstlisting}
#include "somelib.h"
\end{lstlisting}
  which is typically used for files that are part of your code,
  or for libraries that you have downloaded and installed yourself.
\end{enumerate}

\Level 1 {Search paths}

System headers can usually be found by the compiler because they
are in some standard location.
Including other headers may require additional action by you.
If you 
\begin{lstlisting}
#include "foo.h"
\end{lstlisting}
the compiler only looks for that file in the current directory
where you are compiling.

If the include file is part of a library that you have downloaded
and installed yourself, say it is in
\begin{verbatim}
/home/yourname/mylibrary/include/foo.h
\end{verbatim}
then you could of course
\begin{lstlisting}
#include "/home/yourname/mylibrary/include/foo.h"
\end{lstlisting}
but that does not make your code very portable to other systems
and other users.
So how do you make
\begin{lstlisting}
#include "foo.h"
\end{lstlisting}
be understood on any system?

The answer is that you can give your compiler one or more \indextermbusp{include}{path}.
This is done with the \n{-I} flag.
\begin{verbatim}
icpc -c yourprogram.cxx -I/home/yourname/mylibrary/include
\end{verbatim}
You can specify more than one such flag, and the compiler will try to find
the \n{foo.h} file in all of them.

Are you now thinking that you have to type that path every time you compile?
Now is the time to learn \indextermp{makefile} and the \indexterm{Make} utility.

\Level 0 {Textual substitution}
\index{preprocessor!macros|(}

Suppose your program has a number of arrays and loop bounds that are
all identical. To make sure the same number is used, you can create a
variable, and pass that to routines as necessary.
\begin{lstlisting}
void dosomething(int n) {
  for (int i=0; i<n; i++) ....
}

int main() {
  int n=100000;

  double array[n];
   
  dosomething(n);
\end{lstlisting}
You can also use \indexpragma{define} to define a
\emph{preprocessor macro}:
\begin{lstlisting}
#define N 100000
void dosomething() {
  for (int i=0; i<N; i++) ....
}

int main() {
  double array[N];
   
  dosomething();
\end{lstlisting}
It is traditional to use all uppercase for such macros.

\Level 1 {Dynamic definition of macros}

Having numbers defined in the source takes away some flexibility.
You can regain some of that flexibility by letting the macro
be defined by the compiler, using the \n{-D} flag:
\begin{verbatim}
icpc -c yourprogram.cxx -DN=100000
\end{verbatim}

Now what if you want a default value in your source,
but optionally refine it with the compiler?
You can solve this by testing for definition
in the source with \indexpragma{ifndef} `if not defined':
\begin{lstlisting}
#ifndef N
#define N 10000
#endif
\end{lstlisting}

\Level 1 {A new use for `using'}

The \indexc{using} keyword that you saw in section~\ref{sec:usingio}
can also be used as a replacement for the \indexpragma{typedef} pragma
if it's used to introduce synonyms for types.
\begin{lstlisting}
using Matrix = vector<vector<float>>;
\end{lstlisting}

\Level 1 {Parameterized macros}
\label{sec:cpp-define-fun}

Instead of simple text substitution, you can have
%
\emph{parameterized preprocessor macros}\index{preprocessor!macro!parameterized}
\begin{lstlisting}
#define CHECK_FOR_ERROR(i) if (i!=0) return i
...
ierr = some_function(a,b,c); CHECK_FOR_ERROR(ierr);
\end{lstlisting}

When you introduce parameters, it's a good idea to use lots of parentheses:
\begin{lstlisting}
// the next definition is bad!
#define MULTIPLY(a,b) a*b
...
x = MULTIPLY(1+2,3+4);
\end{lstlisting}
Better
\begin{lstlisting}
#define MULTIPLY(a,b) (a)*(b)
...
x = MULTIPLY(1+2,3+4);
\end{lstlisting}

Another popular use of macros is to simulate multi-dimensional indexing:
\begin{lstlisting}
#define INDEX2D(i,j,n) (i)*(n)+j
...
double array[m,n];
for (int i=0; i<m; i++)
  for (int j=0; j<n; j++)
    array[ INDEX2D(i,j,n) ] = ...
\end{lstlisting}

\begin{exercise}
  Write a macro that simulates 1-based indexing:
\begin{lstlisting}
#define INDEX2D1BASED(i,j,n)  ????
...
double array[m,n];
for (int i=1; i<=m; i++)
  for (int j=n; j<=n; j++)
    array[ INDEX2D1BASED(i,j,n) ] = ...
\end{lstlisting}
\end{exercise}

\index{preprocessor!macros|)}

\Level 0 {Conditionals}

\index{preprocessor!conditionals|(}

There are a couple of \emph{preprocessor conditions}.

\Level 1 {Check on a value}

The \n{#if} macro tests on nonzero. A common application is to
temporarily remove code from compilation:
\begin{lstlisting}
#if 0
  bunch of code that needs to
  be disabled
#endif
\end{lstlisting}

You can also test on numerical equality:
\begin{lstlisting}
#if VARIANT == 1
  some code
#elif VARIANT == 2
  other code
#else
#error No such variant
#endif
\end{lstlisting}

\Level 1 {Check for macros}

The \indexpragma{ifdef}
%\n{#ifdef}\index{#ifdef@{\texttt{\#ifdef}}}
test tests for a macro being defined. Conversely,
\indexpragma{ifndef}
%\n{#ifndef}\index{#ifndef@{\texttt{\#ifndef}}}
tests for a macro not being defined. For instance,
\begin{lstlisting}
#ifndef N
#define N 100
#endif
\end{lstlisting}
Why would a macro already be defined? Well you can do that on the
compile line:
\begin{lstlisting}
  icpc -c file.cc -DN=500
\end{lstlisting}

Another application for this test is in preventing recursive inclusion
of header files; see section~\ref{ex:globalvar}.

\index{preprocessor!conditionals|)}

\Level 1 {Including a file only once}

It is easy to wind up including a file such as \n{iostream} more than
once, if it is included in multiple other header files. This adds to
your compilation time, or may lead to subtle problems. A~header file
may even circularly include itself. To prevent this, header files
often have a structure%
\index{#ifndef@{\texttt{\#ifndef}}}
\begin{lstlisting}
// this is foo.h
#ifndef FOO_H
#define FOO_H

// the things that you want to include

#endif
\end{lstlisting}
Now the file will effectively be included only once: the second time
it is included its content is skipped.

Many compilers support the pragma
\indexpragma{once}
that has the same effect:
\begin{lstlisting}
// this is foo.h
#pragma once

// the things you want to include only once
\end{lstlisting}
However, this is not standardized, and the precise meaning is unclear
(what if this is placed halfway a file)
so the \indexterm{Core Guidelines} recommend the explicit guards.

\Level 0 {Other pragmas}

Packing data structure without padding bytes by \indexpragma{pack}
  \begin{lstlisting}
    #pragma pack(push, 1)
    // data structures
    #pragma pack(pop)
  \end{lstlisting}

If you have too many \indexpragma{ifdef} cases, you may
get combinations that don't work.
There is a convenient pragma to exit compilations that don't make sense:
\indexpragma{error}.
\begin{verbatim}
#ifdef __vax__
#error "Won't work on VAXen."
#endif
\end{verbatim}

Less serious is \indexpragma{warning} which will only
put a warning string on your screen.
